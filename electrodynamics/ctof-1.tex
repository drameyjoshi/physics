\chapter{The Special Theory of Relativity}\label{c1}
\begin{enumerate}
\item A frame of reference is a set of cartesian coordinate axes and a clock.

\item An inertial frame is the one in which Newton's first law is valid. The 
existence of such frames was inferred from the experimental observations of the
seventeenth and the eighteenth centuries.

\item Experiments also showed that a frame fixed to the earth is not inertial
but the one fixed with respect to the distant stars was. If $K$ is an inertial
frame then so is $K^\op$ if it moves at a constant relative velocity with respect
to $K$. Thus, there are an infinite number of inertial frames.

\item Suppose that two events happen in an inertial frame $K$: one at a point
$\vec{a}$, and at time $t$ and another one at a point $\vec{b}$ and at time 
$t + \delta t$. When viewed from another inertial frame $K^\op$ moving at a 
velocity $\vec{v}$ with respect to $K$, the events occured at points $\vec{r}_0 
+ \vec{a} + \vec{v}t$ and $\vec{r}_0 + \vec{b} +\vec{v}(t + \delta t)$. Here 
$\vec{r}_0$ is the point in $K$ where $K^\op$'s origin was at $t=0$. An observer 
in $K$ reports that the two events occured at times $\delta t$ apart and at points 
$\vec{b} - \vec{a}$ apart. An observer in $K^\op$ reports that the two events 
happened at a distance $\vec{b} - \vec{a} + \vec{v}\delta t$ apart. However, 
he does not dispute that the same time $\delta t$ elapsed between the two events. 
If $\delta t$ were zero then the two events will be simultaneous in \emph{all} 
inertial frames of reference although they could have happened at different 
points as seen by different observers. This is a consequence of the implicit 
assumption of an absolute time in classical physics.

\item This idea can be corroborated by a simple example. An person on a platform may
see a passenger lifting a tea cup as the coach entered the platform and take it
to his lips as the coach exited it. The two events thus occurred, from his 
perspective, at different points. However, a co-passenger will report them to
be happening at the same spot. The observers will agree on the duration between
the events.

\item The special theory of relativity has its roots in two facts borne out 
of experiments:
\begin{itemize}
\item The laws of physics are the same in all inertial frames of reference and
\item Changes in the state of a system are propagated with a finite speed.
\end{itemize}

\item It is also confirmed by experiments that the changes in the state of system 
are propagated at a speed not exceeding
\begin{equation}\label{c1e1}
c = 2.998 \times 10^8 \;\text{cm/s}.
\end{equation}
No material body can move at a speed exceeding $c$ because if it could then it
can be used to signal the change of state of another body. Further, $c$ is the
maximum speed in \emph{all} inertial frames. For if it were not then the laws of
physics will not be identical across frames. Thus, $c$ is a universal constant.
It is also the speed of light in vacuum.

\item The idea of an absolute, universal time is in conflict with the experimental 
observation of a finite speed of propagation of light in vacuum. For if the
speed is $c$ in a frame $K$ then its speed will be $3c/2$ in a frame $K^\op$
approaching $K$ with a speed $c/2$ in a direction opposite to that of propagation
of light. That this is not true was confirmed by Michelson and Morley's experiment.
Time elapses differently in different systems and therefore a value of a time
difference must be accompanied by a specification of the frame in which it was 
measured.

\item Consider a frame $K$ with a source of light at its origin and two detectors at
points $(-a, 0)$ and $(a, 0)$. A spherical light front will reach the two 
detectors simultaneously when observed from $K$. Let us consider the experiment
replicated in another frame $K^\op$ that moves with a velocity $v\uv{x}$ with
respect to $K$. Suppose that their origins coincided then the light signal was 
emitted in $K$. Since the speed of light is the same in $K$ and $K^\op$ an observer
in $K^\op$ also sees a spherical wavefront propagating isotropically. However, he
will observe the wavefront reaching $(a, 0)$ sooner than it reaches $(-a, 0)$.
Events that are simultaneous in $K$ are not so in $K^\op$.

\item An event is described by the point at which occurred and the time when it
occurred. The four numbers describing an event can be interpreted as points in a
four-dimensional space. They are called the \emph{world points}. World points of
a system move along curves in the four-dimensional space called the \emph{world 
lines}.

\item A material body at a point $\vec{r}$ to which nothing happens travels along
a world line that is parallel to the $t$ axis. In the four-dimensional space, nothing
is still. If a material body moves along the $x$ axis with a uniform speed $v$ then
its world line is a straight line making an angle $\tan^{-1}(v)$ with the $t$ axis.
If it accelerates (decelerates) then the world line will be a curve turning towards 
(away from) the $x$-axis. 
\begin{figure}
\includegraphics[scale=0.8]{ex1}
\caption{World lines}
\label{c1f1}
\end{figure}

\item Consider an experiment in an inertial frame in which light took time 
$\delta t$ to travel between points $(x, y, z)$ and $(x + \delta x, y + 
\delta y, z + \delta z)$. The same experiment was observed from another inertial 
frame in which the light pulse travelled from $(x^\op, y^\op, z^\op)$ to $(x^\op 
+ \delta x^\op, y^\op + \delta y^\op, z^\op + \delta z^\op)$ in time $\delta t^\op$.
Since the speed of light is the same in two frames,
\begin{eqnarray}
c^2\delta t^2 &=& \delta x^2 + \delta y^2 + \delta z^2 \label{c1e2} \\
c^2\delta {t^\op}^2 &=& \delta {x^\op}^2 + \delta {y^\op}^2 + \delta {z^\op}^2 \label{c1e3}
\end{eqnarray}
From these equations, we conclude that
\begin{equation}\label{c1e4}
c^2\delta t^2 - \delta x^2 - \delta y^2 - \delta z^2 = 
c^2\delta {t^\op}^2 - \delta {x^\op}^2 - \delta {y^\op}^2 - \delta {z^\op}^2
\end{equation}
This suggests that the quantity 
\begin{equation}\label{c1e5}
\delta s^2 = c^2\delta t^2 - \delta x^2 - \delta y^2 - \delta z^2
\end{equation}
is invariant across all inertial frame references. The quantity $\delta s$ is
called an interval between two world points.

\item If $\delta t = 0$ then equation \eqref{c1e4} suggests that $\delta {t^\op}^2$
need not be zero. Events simultaneous in $K$ need not be so in $K^\op$. Further, if
$\delta t = 0$ then $\delta s^2 < 0$. Intervals between world points for which 
$\delta s^2 < 0$ are called \emph{space like}. Two points separated by a space-
like interval will always have different ``space'' coordinates.

\item If $\delta x^2 + \delta y^2 + \delta z^2 = 0$ then $\delta s^2 > 0$. In this
case $\delta t$ can never be zero. Intervals between world points for which 
$\delta s^2 > 0$ are called \emph{time like}. Two points separated by a time-
like interval will always have different ``time'' coordinates.

\item It follows from the definitions of time-like and space-like intervals that
\begin{itemize}
\item If two world points are separated by a time-like interval then there exists
an inertial frame in which the events occur at the same space-point, that is, at
same values of $(x, y, z)$.
\item If two world points are separated by a space-like interval then there exists
an inertial frame in which the events occur at the same time.
\end{itemize}

\item An interval for which $\delta s = 0$ is called ``light-like''.

\item Since the nature of an interval depends on an invariant quantity like
$\delta s^2$, an interval that is space-like (or time-like or light-like) in one
inertial frame is space-like (or time-like or light-like) in all inertial frames.

\item A causal relationship between events can exists only their world-points
are separated by a time-like interval.

\item In the four-dimensional space with coordinates $x, y, z$ and $t$, the equation
\begin{equation}\label{c1e6}
c^2t^2 = x^2 + y^2 + z^2
\end{equation}
defines a double cone with origin as the vertex and the $t$ axis as its principle
axis. It is called the \emph{light cone}. 
World points inside the cone are separated by a time-like interval. World
points outside it are separated by space-like interval. If two world points $A$
and $B$ are inside the cone then in every inertial frame the difference between
their time coordinates will be non-zero. Likewise, if they were outside the cone
then in every inertial frame their space coordinates will not all be the same.
The part of the cone above (below) the origin is called the absolute future 
(past) of the origin.

\item A time interval measured with a single clock is called
\emph{proper time}. Let an interval $\delta t$ be measured in an inertial frame
$K$ with a clock at point $(x, y, z)$. In another inertial frame, $K^\op$ it will
be measured by two (synchronised) clocks at two different locations. One of the
clocks will measure start of the interval and by the time the event at the end 
of the interval happens, a different clock will go past what is $(x, y, z)$ in
$K$. The relation between the two measurements follow from equation \eqref{c1e4}
with $\delta x = \delta y = \delta z = 0$. Thus,
\begin{equation}\label{c1e7}
\delta t^2 = \delta {t^\op}^2 - \frac{\delta {x^\op}^2 + \delta {y^\op}^2 + \delta {z^\op}^2}{c^2}
= \delta {t^\op}^2\left(1 - \frac{1}{c^2}\left(\frac{\delta r^\op}{\delta t^\op}\right)^2\right),
\end{equation}
where
\begin{equation}\label{c1e8}
\delta {r^\op}^2 = \delta {x^\op}^2 + \delta {y^\op}^2 + \delta {z^\op}^2.
\end{equation}
Now $\delta r^\op/\delta t^\op$ is the speed at which and observer in $K^\op$
moves with respect to the point $(x, y, z)$ in $K$. It is precisely the relative
speed of $K^\op$ with respect to $K$. If we call it $u$, we can write equation
\eqref{c1e7} as
\begin{equation}\label{c1e9}
\delta t^2 = \delta {t^\op}^2\left(1 - \frac{u^2}{c^2}\right).
\end{equation}
Since $u < c$, $\delta t < \delta t^\op$. Thus, the proper time interval is the
smallest among the intervals measured in all inertial frames.

\item In the previous point, an observer in $K^\op$ will notice that the clock
in $K$ is moving relative to him and it reports an interval $\delta t$ while his
clocks reported $\delta t^\op$. From equation \eqref{c1e9} the moving clock
reported a shorter time interval. That is why we say that moving clocks go slow.

\item Among all clocks travelling between world points $A$ and $B$ that differ 
only in their $t$-coordinate, the clock that is stationary measures the maximum
time. Consider two clocks, one of which remains stationary and the other whose
world line curves arbitrarily between $A$ and $B$. The time measured by each of
them is their proper time. For the stationary clock, the time interval is
\begin{equation}\label{c1e10}
\delta t_1 = \frac{1}{c}\int_A^B ds_1
\end{equation}
while that measured by the moving clock is
\begin{equation}\label{c1e11}
\delta t_2 = \frac{1}{c}\int_A^B ds_2.
\end{equation}
Now, $ds_1 = cdt_1$ while $ds_2 = \sqrt{c^2dt_2^2 - dx_2^2 - dy_2^2 - dz_2^2}$.
If for all intervals along the second clock's path, $ds_1$ will be greater
than or equal to $ds_2$, then $t_1 \ge t_2$.

\item We will now derive the transformation between the coordinates of world
points in two inertial frames of reference. Let $(x, y, z, t)$ be the coordinates
of a world point in $K$. Let the coordinates of the same points be $(x^\op, y^\op,
z^\op, t^\op)$ in $K^\op$. If we consider a pulse of light travelling from the 
origin to the point being considered then from \eqref{c1e4} we have
\begin{equation}\label{c1e12}
c^2t^2 - x^2 - y^2 - z^2 = c^2{t^\op}^2 - {x^\op}^2 - {y^\op}^2 - {z^\op}^2.
\end{equation}
We can consider each side of \eqref{c1e12} the ``distance'' of the two world 
points from the origin in the respective inertial frame. 

One way in which equation \eqref{c1e12} will be valid is when the two inertial
frames have their origins displaced by a constant. However, this is not what
happens when the frames are moving relative to each other.

We next check if we can treat the motion as a ``rotation'' in the four-dimensional
space. There are six mutually orthogonal planes in four dimensions. If $K$ and
$K^\op$ are such that their $x$-axes coincide while $y$ and $z$ axes remain parallel
then the ``rotation'' can happen only in the $xt$-plane. Further, $y = y^\op$ and
$z = z^\op$ so that
\begin{equation}\label{c1e13}
c^2t^2 - x^2 = c^2{t^\op}^2 - {x^\op}^2.
\end{equation}
We can readily verify that a transformation of the form
\begin{eqnarray}
 x  &=&  x^\op\cosh\psi + c t^\op\sinh\psi \label{c1e14} \\
c t &=&  x^\op\sinh\psi + c t^\op\cosh\psi \label{c1e15}
\end{eqnarray}
satisfies \eqref{c1e12}. If we focus only on the motion of the origin of $K^\op$
in $K$ then we have $x^\op = 0$ and equations \eqref{c1e14} and \eqref{c1e15}
give
\begin{equation}\label{c1e16}
\frac{x}{ct} = \tanh\psi.
\end{equation}
If the relative speed of $K^\op$ with respect to $K$ is $u$ then $u = x/t$ and
hence
\begin{equation}\label{c1e17}
\tanh\psi = \frac{u}{c}.
\end{equation}
The factor $u/c$ occurs quite frequently in the theory of relativity. We denote 
it by
\begin{equation}\label{c1e18}
\frac{u}{c} = \beta.
\end{equation}
Since $1 - \tanh^2\psi = \sech^2\psi$ we have
\begin{eqnarray}
\cosh\psi &=& \frac{1}{\sqrt{1 - \beta^2}} \label{c1e19} \\
\sinh\psi &=& \frac{\beta}{\sqrt{1 - \beta^2}} \label{c1e20}
\end{eqnarray}
Substituting these in \eqref{c1e14} and \eqref{c1e15} we get 
\begin{eqnarray}
x &=& \frac{x^\op + ut^\op}{\sqrt{1 - \beta^2}} \label{c1e21} \\
t &=& \frac{t^\op + ux^\op/c^2}{\sqrt{1 - \beta^2}}. \label{c1e22} 
\end{eqnarray}
We get a more symmetrical form if we define $\tau = ct$ so that
\begin{eqnarray}
x &=& \frac{x^\op + \beta\tau^\op}{\sqrt{1 - \beta^2}} \label{c1e23} \\
\tau &=& \frac{\tau^\op + \beta x^\op}{\sqrt{1 - \beta^2}}. \label{c1e24} 
\end{eqnarray}
Equations \eqref{c1e21} and \eqref{c1e22} (or equivalently \eqref{c1e23}
and \eqref{c1e24}) are called \emph{Lorentz transformation}.

\item It is not clear from the above analysis why a Lorentz transformation is
called a ``rotation'' in the 4-dimensional space-time. We can write equations 
\eqref{c1e14} and \eqref{c1e15} in matrix form as
\begin{equation}\label{c1e25}
\begin{bmatrix}x \\ ct \end{bmatrix} = \begin{bmatrix} \cosh\psi & \sinh\psi \\
\sinh\psi & \cosh\psi \end{bmatrix}\begin{bmatrix}x^\op \\ ct^\op \end{bmatrix}.
\end{equation}
We can as well write it as
\begin{equation}\label{c1e26}
\begin{bmatrix}x \\ ict \end{bmatrix} = \begin{bmatrix} \cosh\psi & -i\sinh\psi \\
i\sinh\psi & \cosh\psi \end{bmatrix}\begin{bmatrix}x^\op \\ ict^\op \end{bmatrix}.
\end{equation}
Since $\cos(i\psi) = \cosh\psi$ and $\sin(i\psi) = i\sinh\psi$, we have
\begin{equation}\label{c1e27}
\begin{bmatrix}x \\ ict \end{bmatrix} = \begin{bmatrix} \cos(i\psi) & -\sin(i\psi) \\
\sin(i\psi) & \cos(i\psi) \end{bmatrix}\begin{bmatrix}x^\op \\ ict^\op \end{bmatrix}.
\end{equation}
In the early years of special relativity, the world points were considered to have
coordinates $(x, y, z, ict)$ and the distance between two world points mimicked the
usual Euclidean formula. In such a space, a Lorentz transformation did indeed look
like a ``rotation'' by an imaginary angle in the $tx$ plane.

\item From equation \eqref{c1e24}
\begin{eqnarray}
\tau_1 &=& \frac{\tau_1^\op + \beta x_1^\op}{\sqrt{1 - \beta^2}} \label{c1e28} \\
\tau_2 &=& \frac{\tau_2^\op + \beta x_2^\op}{\sqrt{1 - \beta^2}} \label{c1e29}
\end{eqnarray}
so that
\begin{equation}\label{c1e30}
\tau_2 - \tau_1 = \frac{\tau_2^\op - \tau_1^\op + \beta(x_2^\op - x_1^\op)}{\sqrt{1 - \beta^2}}.
\end{equation}
If $x_2^\op = x_1^\op$ then $\delta\tau^\op = \tau_2^\op - \tau_1^\op$ is the 
proper time. Equation \eqref{c1e30} becomes
\begin{equation}\label{c1e31}
\delta\tau = \frac{\delta\tau^\op}{\sqrt{1 - \beta^2}}.
\end{equation}
Thus, $\delta\tau^\op < \delta\tau$ as we could have expected.

\item Likewise, from equation \eqref{c1e23},
\begin{eqnarray}
x_1 &=& \frac{x^\op_1 + u\tau^\op_1}{\sqrt{1 - \beta^2}} \label{c1e32} \\
x_2 &=& \frac{x^\op_2 + u\tau^\op_2}{\sqrt{1 - \beta^2}} \label{c1e33}
\end{eqnarray}
so that
\begin{equation}\label{c1e34}
x_2 - x_1 = \frac{x^\op_2 - x^\op_1 + u(\tau^\op_2 - \tau^\op_1)}{\sqrt{1 - \beta^2}}.
\end{equation}
If a rod with ends at $x_1$ and $x_2$ is at rest in a frame $K$ then to
measure its length in the moving frame we have to measure its ends at the same
time. That is $\tau^\op_2 = \tau^\op_1$ so that equation \eqref{c1e34} becomes
\begin{equation}\label{c1e35}
\delta x = \frac{\delta x^\op}{\sqrt{1 - \beta^2}}.
\end{equation}
The length $\delta x$ of the rod measured in the frame in which it was at rest
is called its \emph{proper length}. From \eqref{c1e35} we conclude that $\delta x^\op
< \delta x$. The rod was at rest in $K$ but appeared to move in $K^\op$. Since
$\delta x^\op < \delta x$ we conclude that moving rods appear to be shortened.

\item From equations \eqref{c1e21} and \eqref{c1e22}, along with the fact that
$y = y^\op$ and $z = z^\op$, we get
\begin{eqnarray}
dx &=& \frac{dx^\op + udt^\op}{\sqrt{1 - \beta^2}} \label{c1e36} \\
dy &=& dy^\op \label{c1e37} \\
dz &=& dz^\op \label{c1e38} \\
dt &=& \frac{dt^\op + udx^\op/c^2}{\sqrt{1 - \beta^2}} \label{c1e39}
\end{eqnarray}
so that
\begin{eqnarray}
\frac{dx}{dt} &=& \frac{dx^\op + udt^\op}{dt^\op + udx^\op/c^2} \label{c1e40} \\
\frac{dy}{dt} &=& \frac{dy^\op}{dt^\op + udx^\op/c^2} \label{c1e41} \\
\frac{dz}{dt} &=& \frac{dz^\op}{dt^\op + udx^\op/c^2}. \label{c1e42}
\end{eqnarray}
We also have
\[
v_x = \frac{dx}{dt}\;;\;v_y = \frac{dy}{dt}\;;\;v_z = \frac{dz}{dt}\;;\;
v_x^\op = \frac{dx^\op}{dt^\op}\;;\;v_y^\op = \frac{dy^\op}{dt^\op}
\;;\;v_z^\op = \frac{dz^\op}{dt^\op}
\]
so that equations \eqref{c1e40} to \eqref{c1e42} can be written as
\begin{eqnarray}
v_x &=& \frac{v_x^\op + u}{1 + uv_x^\op/c^2} \label{c1e43} \\
v_y &=& \frac{v_y^\op}{1 + uv_x^\op/c^2} \label{c1e44} \\
v_z &=& \frac{v_z^\op}{1 + uv_x^\op/c^2}. \label{c1e45} 
\end{eqnarray}
These are the formulae for transformation of velocity components. We
quickly confirm that if $(v_x^\op, v_y^\op, v_z^\op) = (c, 0, 0)$ then
equations \eqref{c1e43} to \eqref{c1e44} give $(v_x, v_y, v_z) = (c, 0, 0)$.

\item The coordinates of an event $(ct, x, y, z)$ can be considered to be
components of a four-dimensional vector, or a 4-vector. The components are
denoted by $x^i$ and we have $x^0 = ct, x^1 = x, x^2 = y, x^3 = z$. The 
components of the same vector in another inertial frame travelling with a 
relative velocity $u\uv{x}$ are, by equations \eqref{c1e23} and \eqref{c1e24},
\[
x^0 = \frac{\bar{x}^0 + \beta\bar{x}^1}{\sqrt{1 - \beta^2}};\;
x^1 = \frac{\bar{x}^1 + \beta\bar{x}^0}{\sqrt{1 - \beta^2}};\;
x^2 = \bar{x}^2; x^3 = \bar{x}^3.
\]
Any set of four quantities $A^0, A^1, A^2, A^3$ which transform in a similar
way form a 4-vector. Thus, the transformation equations for these components
are
\begin{equation}\label{c1e46}
A^0 = \frac{\bar{A}^0 + \beta\bar{A}^1}{\sqrt{1 - \beta^2}};\;
A^1 = \frac{\bar{A}^1 + \beta\bar{A}^0}{\sqrt{1 - \beta^2}};\;
A^2 = \bar{A}^2; A^3 = \bar{A}^3.
\end{equation}
The numbers $A^0, A^1, A^2, A^3$ are called the \emph{contravariant} components
of the vector. The four related quantities
\begin{equation}\label{c1e47}
A_0 = A^0; A_1 = -A^1; A_2 = -A^2; A_3 = -A^3
\end{equation}
are called the \emph{covariant} components of the same vector. The magnitude of
the vector is $A^\mu A_\mu$, where we have used the summation convention. We will 
sometimes write a 4-vector as $A^\mu = (A^0, \vec{A})$ and $A_\mu = (A^0, -\vec{A})$.

\item If $A^\mu$ and $B^\mu$ are two 4-vectors then their scalar product is $A^\mu 
B_\mu$, which is same as $A_\mu B^\mu$. The resulting quantity is called a 4-scalar. 

\item The relation between contravariant and covariant components of a tensor
can be expressed as
\begin{equation}\label{c1e48}
A^\mu = g^{\mu\nu}A_\nu\;\text{ and }\; A_\mu = g_{\mu\nu}A^\nu,
\end{equation}
where $g^{\mu\nu}$ and $g_{\mu\nu}$ are components of the \emph{metric tensor}. 
They are identical and are given by
\begin{equation}\label{c1e49}
g^{\mu\nu} = g_{\mu\nu} = \begin{bmatrix}1 & 0 & 0 & 0 \\
0 & -1 & 0 & 0 \\
0 & 0 & -1 & 0 \\
0 & 0 & 0 & -1
\end{bmatrix}.
\end{equation}

\item Right now we will only introduce 4-tensors of second order in terms of their
contravariant components $\tensor{T}{^{\mu\nu}}$ or covariant components 
$\tensor{T}{_{\mu\nu}}$ or mixed components $\tensor{T}{^\mu_\nu}$ or 
$\tensor{T}{_\mu^\nu}$. We will not bother to specify their transformation properties.
These components are related to each other through the multiplication by the metric
tensor. Thus, 
\begin{equation}\label{c1e50}
\tensor{T}{_{\mu\nu}} = g_{\mu\alpha}\tensor{T}{^\alpha_\nu} =
g_{\mu\alpha}\tensor{T}{_\nu^\alpha} = g_{\mu\alpha}g_{\nu\beta}T^{\alpha\beta}.
\end{equation}
Note that the components $\tensor{T}{^\mu_\nu}$ and $\tensor{T}{_\mu^\nu}$ are 
different.

\item A unit 4-tensor is defined as
\begin{equation}\label{c1e51}
\tensor{\delta}{^\mu_\nu} = \tensor{\delta}{_\mu^\nu} = \begin{cases}
1 \text{  if  } \mu = \nu \\
0 \text{  if  } \mu \ne \nu.
\end{cases}
\end{equation}

\item The tensors $g_{\mu\nu}, g^{\mu\nu}, \tensor{\delta}{^\mu_\nu}$ and 
$\tensor{\delta}{_\mu^\nu}$ have the same components in all inertial frames. The 
competely asymmetric unit tensor of rank 4 also has the same property. It is
defined as
\begin{equation}\label{c1e52}
e^{\mu\nu\rho\sigma} = \begin{cases}
1 \text{  if  } \mu\nu\rho\sigma \text{  is an even permutation.} \\
-1 \text{  if  } \mu\nu\rho\sigma \text{  is an odd permutation.} \\
0 \text {  otherwise.}
\end{cases}
\end{equation}
If we set
\begin{equation}\label{c1e53}
e_{\mu\nu\rho\sigma} = -e^{\mu\nu\rho\sigma}
\end{equation}
then $e_{\mu\nu\rho\sigma}e^{\mu\nu\rho\sigma}$ is just the negative of the total 
number of non-zero components of $e_{\mu\nu\rho\sigma}$, which is the same as the
number of permutation of the four indices. Thus,
\begin{equation}\label{c1e54}
e_{\mu\nu\rho\sigma}e^{\mu\nu\rho\sigma} = -4! = -24.
\end{equation}
We get negative sign becaue $e^{0123} = 1, e_{0123} = -1$ so that their product 
is $-1$.

\item A polar vector is the one which changes sign upon reflection of the coordinate
axes. An axial vector is the one whose sign remains unchanged. If $\vec{A}$ and 
$\vec{B}$ are polar vectors then $\vec{C} = \vec{A} \times \vec{B}$ is an axial
vector. The components of an axial vector can be viewed as the components of an
anti-symmetric tensor. The two are related as 
\begin{eqnarray}
C_i &=& \epsilon_{ijk}A_jB_k \label{c1e55} \\
C_{ij} &=& A_iB_j - A_jB_i \label{c1e56}
\end{eqnarray}
Thus, $C_1 = A_2B_3 - A_3B_2, C_2 = A_3B_1 - A_1B_3, C_3 = A_1B_2 - A_2B_1$ while
\begin{equation}\label{c1e57}
(C^{ij}) = \begin{bmatrix}0 & A_1B_2 - A_2B_1 & A_1B_3 - A_3B_1 \\
-(A_1B_2 - A_2B_1) & 0 & A_2B_3 - A_3B_2 \\
-(A_1B_3 - A_3B_1) & -(A_2B_3 - A_3B_2) & 0
\end{bmatrix}
\end{equation}
or
\begin{equation}\label{c1e58}
(C^{ij}) = \begin{bmatrix}0 & C_3 & -C_2 \\ -C_3 & 0 & C_1 \\ C_2 & -C_1 & 0
\end{bmatrix}
\end{equation}
Furthermore, from the properties of $\epsilon_{ijk}$ we readily infer that
\begin{equation}\label{c1e59}
C_i = \frac{1}{2}\epsilon_{ijk}C_{jk}.
\end{equation}

\item The dot product of two polar vectors is a scalar. That of a polar vector
and an axial vector is a pseudo-scalar. Thus, the vector triple product is a
pseudo-scalar.

\item The components of $\epsilon_{ijk}$ remain the same even if we invert the
coordinate axes. Therefore, it is really a pseudo-tensor. That is why we cannot 
express it in terms of $\delta_{ij}$, the ``unit tensors''. On the other hand
$\epsilon_{ijk}\epsilon_{lmn}$ is a true tensor (of rank $6$) and can be expressed
in terms of the Kronecker delta as
\begin{equation}\label{c1e60}
\epsilon_{ijk}\epsilon_{lmn} = \begin{vmatrix}
\delta_{il} & \delta_{im} & \delta_{in} \\
\delta_{jl} & \delta_{jm} & \delta_{jn} \\
\delta_{kl} & \delta_{km} & \delta_{kn}
\end{vmatrix}
\end{equation}

\item Likewise, $\epsilon_{\alpha\beta\gamma\delta}$ is a pseudo, anti-symmetric
4-tensor whose components cannot be written in terms of $\delta_{\mu\nu}$. However,
$\epsilon_{\alpha\beta\gamma\delta}\epsilon_{\mu\nu\rho\sigma}$ is a true tensor
of rank $8$ whose components can be expressed in terms of $\delta_{\mu\nu}$.
\begin{equation}\label{c1e61}
\epsilon_{\alpha\beta\gamma\delta}\epsilon_{\mu\nu\rho\sigma} = \begin{vmatrix}
\delta_{\alpha\mu} & \delta_{\alpha\nu} & \delta_{\alpha\rho} & \delta_{\alpha\sigma} \\
\delta_{\beta\mu} & \delta_{\beta\nu} & \delta_{\beta\rho} & \delta_{\beta\sigma} \\
\delta_{\gamma\mu} & \delta_{\gamma\nu} & \delta_{\gamma\rho} & \delta_{\gamma\sigma} \\
\delta_{\delta\mu} & \delta_{\delta\nu} & \delta_{\delta\rho} & \delta_{\delta\sigma}
\end{vmatrix}
\end{equation}

\item If $A_{\rho\sigma}$ is an anti-symmetric tensor then
\begin{equation}\label{c1e62}
{A^\ast}^{\mu\nu} = \epsilon^{\mu\nu\rho\sigma}A_{\rho\sigma}
\end{equation}
is said to be \emph{dual} to $A^{\rho\sigma}$. Their product, ${A^\ast}^{\mu\nu}
A_{\mu\nu}$ is a pseudo-scalar. Likewise, 
\begin{equation}\label{c1e63}
{A^\ast}^{\mu\nu\rho} = \epsilon^{\mu\nu\rho\sigma}A_{\sigma}
\end{equation}
is said to be \emph{dual} to $A^\sigma$. ${A^\ast}^{\mu\nu}$ and ${A^\ast}^{\mu\nu\rho}$
are pseudo-tensors of rank $2$ and $3$ respectively.

\item The space components of an anti-symmetric 4-tensor form a 3-tensor of the kind 
described by equation \eqref{c1e58}. The time components form a polar 3-vector. Thus,
\begin{equation}\label{c1e64}
(A^{ij}) = \begin{bmatrix} 0 & p_x & p_y & p_z \\
-p_x & 0 & a_z & -a_y \\
-p_y & -a_z & 0 & a_x \\
-p_z & a_y & -a_x & 0
\end{bmatrix}
\end{equation}
A short notation for an anti-symmetric 4-tensor is $(A^{ij}) = (\vec{p}, \vec{a})$.
Its covariant form is $(A_{ij}) = (-\vec{p}, \vec{a})$.

\item The gradient operator in space-time is defined by
\begin{equation}\label{c1e65}
\frac{\partial\phi}{\partial x^i} = 
\left(\frac{1}{c}\frac{\partial\phi}{\partial t}, \grad\phi\right).
\end{equation}
These components are to be taken as covariant components of the gradient vector.
The left hand side has $x^\mu$ in the ``denominator'' so the index $\mu$ is assumed
to be ``below'' making the componet covariant. Furthermore,
\begin{equation}\label{c1e66}
d\phi = \frac{\partial\phi}{\partial x^\mu}dx^\mu
\end{equation}
is a scalar. Similary, the divergence of a 4-vector $A^\mu$ is also a scalar
\begin{equation}\label{c1e67}
\dive(A^\mu) = \frac{\partial A^\mu}{\partial x^\mu}.
\end{equation}

\item In three dimensions we deal with integrals over a volume, a surface or
along a line. In four dimensions we have integrals of four kinds - over the
4-volume, a 3-surface, a 2-surface and along a curve.
\begin{enumerate}
\item In the case of a line integral, the element of integration is the line-element
$dx^\mu$ in four dimensions and $dx^i$ in three.

\item In three dimensions, integral over a surface is over the area elements 
$d\vec{f} = d\vec{r} \times d\vec{r}^\op$. An integral of the form
\begin{eqnarray*}
\int \vec{A}\cdot d\vec{f} &=& \int A^i df_i \\
 &=& \int A^1(dx_2dx_3^\op - dx_3dx_2^\op) + \int A^2(dx_3dx_1^\op - dx_1dx_3^\op) + \\
 & & \int A^3(dx_3dx_2^\op - dx_2dx_3^\op).
\end{eqnarray*}
The components of $d\vec{f}$ are expressible in terms of an anti-symmetric tensor 
using \eqref{c1e59}, that is
\begin{equation}\label{c1e68}
df_i = \frac{1}{2}\epsilon_{ijk}df^{jk} = \frac{1}{2}\epsilon_{ijk}(dx^jd{x^\op}^k - 
dx^kd{x^\op}^j).
\end{equation}
In three dimensions it is more common to use $d\vec{f}$ than the tensor $df^{jk}$.
The vector is indeed a dual to the tensor. In four dimensions, the dual to the
tensor $df^{\mu\nu}$ is also a tensor, given by the analogoue of equation \eqref{c1e62}. 
That is,
\begin{equation}\label{c1e69}
df^\ast_{\mu\nu} = \frac{1}{2}\epsilon_{\mu\nu\rho\sigma}df^{\rho\sigma}.
\end{equation}
Here
\begin{equation}\label{c1e70}
df^{\mu\nu} = dx_1^\mu dx_2^\nu - dx_1^\nu dx_2^\mu.
\end{equation}

\item In four dimensions, there is yet another surface, or rather a hyper-surface,
described by three vectors $dx^\mu, dx^{\op\mu}, dx^{\tp\mu}$. Its ``area'' is given
by
\begin{equation}\label{c1e71}
dS^{\mu\nu\rho} = \begin{vmatrix}
dx^\mu & dx^{\op\mu} & dx^{\tp\mu} \\
dx^\nu & dx^{\op\nu} & dx^{\tp\nu} \\
dx^\rho & dx^{\op\rho} & dx^{\tp\rho}
\end{vmatrix}
\end{equation}
The numbers $dS^{\mu\nu\rho}$ are components of an anti-symmetric tensor 
$(dS^{\mu\nu\rho})$. In this case, the dual is a 4-vector
\begin{equation}\label{c1e72}
dS^\mu = -\frac{1}{6}\epsilon^{\mu\nu\rho\sigma}dS_{\nu\rho\sigma}.
\end{equation}
The magnitude of $dS^\mu$ is the ``area'' of the hypersurface and it direction
is normal to it. It is not possible to visualise the normal. The way $\uv{z}$ is
normal to all lines in the $xy$ plane, $dS^\mu$ is normal to all lines on the
hypersurface.

\item The volume integral is over the element
\begin{equation}\label{c1e73}
d\Omega = dx^0 dx^1 dx^2 dx^3 = cdtdv.
\end{equation}
It is a scalar.

\item The analogue of Gauss' theorem is
\begin{equation}\label{c1e74}
\oint A^\mu dS_\mu = \int \pdt{A^\mu}{x^\mu}d\Omega.
\end{equation}

\item The analogue of Stokes' theorem is
\begin{equation}\label{c1e75}
\oint A_\mu dx^\mu = \int df^{\mu\nu}\pdt{A_{\nu}}{x^\mu}.
\end{equation}
We manipulate the right hand side as
\begin{eqnarray*}
\int df^{\mu\nu}\pdt{A_{\nu}}{x^\mu} &=& \frac{1}{2}\left(\int df^{\mu\nu}\pdt{A_{\nu}}{x^\mu} +
\int df^{\mu\nu}\pdt{A_{\nu}}{x^\mu}\right) \\
&=& \frac{1}{2}\left(\int df^{\mu\nu}\pdt{A_{\nu}}{x^\mu} + \int df^{\nu\mu}\pdt{A_{\mu}}{x^\nu}\right) \\
&=& \frac{1}{2}\left(\int df^{\mu\nu}\pdt{A_{\nu}}{x^\mu} - \int df^{\mu\nu}\pdt{A_{\mu}}{x^\nu}\right),
\end{eqnarray*}
where used the anti-symmetric nature of $df^{\mu\nu}$ as it is obvious from \eqref{c1e70}.
Therefore,
\[
\int df^{\mu\nu}\pdt{A_{\nu}}{x^\mu} = 
\frac{1}{2}\int df^{\mu\nu}\left(\pdt{A_{\nu}}{x^\mu} - \pdt{A_{\mu}}{x^\nu}\right)
\]
so that we can as well write \eqref{c1e74} as
\begin{equation}\label{c1e76}
\oint A_\mu dx^\mu = \frac{1}{2}\int df^{\mu\nu}\left(\pdt{A_{\nu}}{x^\mu} - \pdt{A_{\mu}}{x^\nu}\right)
\end{equation}

\item We also have a ``hyper''-Stokes theorem that deals with integrals of the 
type
\begin{equation}\label{c1e77}
\int A^{\mu\nu}df^\ast_{\mu\nu} = \int\left(dS_\mu\pdt{A^{\mu\nu}}{x^\nu} - dS_\nu\pdt{A^{\mu\nu}}{x^\mu}\right)
\end{equation}
\end{enumerate}

\item The 4-velocity of a particle is
\begin{equation}\label{c1e78}
u^\mu = \td{x^\mu}{s},
\end{equation}
where
\[
ds = \sqrt{c^2dt^2 - dx^2 - dy^2 - dz^2} = cdt\sqrt{1 - \frac{v^2}{c^2}},
\]
and
\[
v = \left(\td{x}{t}, \td{y}{t}, \td{z}{t}\right)
\]
is the ordinary velocity of the particle. Thus,
\begin{eqnarray}
u^0 &=& \frac{cdt}{ds} = \left(1 - \frac{v^2}{c^2}\right)^{-1/2} \label{c1e79} \\
u^1 &=& \frac{dx}{ds} = \frac{v_x}{c\sqrt{1 - v^2/c^2}} \label{c1e80} \\
u^2 &=& \frac{dy}{ds} = \frac{v_y}{c\sqrt{1 - v^2/c^2}} \label{c1e81} \\
u^3 &=& \frac{dz}{ds} = \frac{v_z}{c\sqrt{1 - v^2/c^2}}. \label{c1e82}
\end{eqnarray}
We can as well write
\begin{equation}\label{c1e83}
u^\mu = \left(\frac{1}{\sqrt{1 - v^2/c^2}}, \frac{\vec{v}}{c\sqrt{1 - v^2/c^2}}\right).
\end{equation}
It is interesting to note that the 4-velocity is a dimensionless quantity. Furthermore,
\begin{equation}\label{c1e84}
u^\mu u_\mu = \td{x^\mu}{s}\td{x_mu}{s} = \frac{dx^\mu dx_\mu}{ds^2} = 1,
\end{equation}
because by definition $ds^2 = dx^\mu dx_\mu$. The magnitude of $u^\mu$ is always
unity. We also define the 4-acceleration as
\begin{equation}\label{c1e85}
w^\mu = \td{u^\mu}{s}
\end{equation}
Differentiating \eqref{c1e83} with respect to $s$ we get
\begin{equation}\label{c1e86}
w^\mu u_\mu + u^\mu w_\mu = 0 \Rightarrow w^\mu u_\mu + u_\mu w^\mu = 0 \Rightarrow w^\mu u_\mu = 0.
\end{equation}
Thus, the 4-acceleration is always perpendicular to the 4-velocity.
\end{enumerate}

\section{Problems}
\begin{enumerate}
\item Lorentz transformation of a symmetric tensor. 
\item[Solution:] Let $(C^{\mu\nu})$ be a symmetric tensor. We can consider it to
be a 4-dyad of the form $C^{\mu\nu} = A^\mu B^\nu$ for two 4-vectors $(A^\mu)$ 
and $(B^\nu)$. Written in matric form
\begin{equation}\label{c1e87}
(C^{\mu\nu}) = \begin{bmatrix}A^0 \\ A^1 \\ A^2 \\ A^3\end{bmatrix}
\begin{bmatrix}B^0 & B^1 & B^2 & B^3 \end{bmatrix}
\end{equation}
so that $C^{\mu\nu} = A^\mu B^\nu$. Symmetry of the tensor requires $C^{\mu\nu}
= C^{\nu\mu}$. The Lorentz transformation of the vectors $(A^\mu)$ and $(B^\nu)$
is given by equations \eqref{c1e46}. Therefore,
\begin{eqnarray*}
C^{00} &=& A^0 B^0 \\
 &=& \frac{(\bar{A}^0 + \beta\bar{A}^1)(\bar{B}^0 + \beta\bar{B}^1)}{1 - \beta^2} \\
 &=& \frac{(\bar{A}^0\bar{B}^0 + \beta(\bar{A}^1\bar{B}^0 + \bar{A}^0\bar{B}^1) + \beta^2\bar{A}^1\bar{B}^1}{1 - \beta^2} \\
 &=& \frac{\bar{C}^{00} + \beta(\bar{C}^{10} + \bar{C}^{01}) + \beta^2\bar{C}^{11}}{1 - \beta^2}
\end{eqnarray*}
Using the symmetry of the tensor,
\begin{equation}\label{c1e88}
C^{00} = \frac{\bar{C}^{00} + 2\beta\bar{C}^{10} + \beta^2\bar{C}^{11}}{1 - \beta^2}.
\end{equation}
\end{enumerate}

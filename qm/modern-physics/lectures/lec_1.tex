\documentclass{beamer}
\usefonttheme[onlymath]{serif}
\usepackage{graphicx}
\usetheme{Boadilla}
\title{Quantum Physics - dance of the atoms}
\author{Amey Joshi}
\date{\today}
\begin{document}
\begin{frame}
\titlepage
\end{frame}

\begin{frame}
\frametitle{Welcome to the course}
This is probably your first introduction to physics that was developed in the
20th century. It is the start of a fascinating journey in which you will learn
about phenomena beyond your senses.
\begin{itemize}
\item Objects moving at speeds close to that of light and
\item Objects that so delicate that even watching them disturbs them.
\end{itemize}
It is a story of mankind's triumph over the limitations of human senses. I
enjoyed it ever since I learnt about it. I hope you too feel the same.

I assume that
\begin{itemize}
\item You are comfortable with mathematatics up to the level of class 12.
\item You are well-versed with Newton's laws, basic optics, electromagnetism
and thermodynamics.
\item You have a boundless curiosity to find out how Nature works.
\end{itemize}
\end{frame}

\begin{frame}
\frametitle{What will we cover?}
\begin{enumerate}
\item Special theory of relativity
\item Wave particle duality
\item Early attempts to understand the structure of an atom
\item Wave mechanics
\item Atomic structure
\item Molecular structure
\item Solids
\item Nuclei and elementary particles
\item Quantum mechanics of spins and photons and
\item If time permits, a sneak peek at quantum logic.
\end{enumerate}
\end{frame}

\begin{frame}
\frametitle{What's expected from you?}
\begin{enumerate}
\item Be curious, explore, make mistakes, participate in the class, enjoy the 
learning.
\item Don't hesitate to ask questions. Don't hold back even if you are not 
comfortable with English.
\item There will be plenty of problems and assignments. Do them sincerely.
\item The purpose of this course is to trigger your interest. Think beyond
grades.
\item I do not get to see you in an on-line lecture. Let's make an attempt to 
communicate nevertheless.
\item Be respectful and courteous to each other. Make this a safe and enjoyable
learning environment.
\end{enumerate}
\end{frame}

\begin{frame}
\frametitle{Exams etc.}
\begin{enumerate}
\item There will be $40\%$ weightage for mid-term, another $40\%$ for end-term
and $20\%$ for assignments.
\item You will be examined for your understanding, not memory. 
\item Collaborate with your peers but be ready to explain your work. Copying
solutions without understanding is cheating.
\item Do your assignments and your exams will be a cake walk.
\item Use the internet, use the calculator, or better still use Python to solve
computational problems and plotting curves.
\item A sincere attempt, an honest mistake, an incomplete or even wrong solution
will fetch you grades. Cheating is a guaranteed to lead to failure.
\end{enumerate}
\end{frame}

\begin{frame}
\frametitle{Notes etc.}
\begin{enumerate}
\item Don't split your attention by taking down notes. Lecture slides and study
material will be available for you. Focus on the conversation.
\item Yet, keep a paper and pen handy to draw diagrams, calculate, solve 
problems in the class.
\item I will keep office hours over the weekend.
\item If you want to go beyond the lecture notes, you may refer to these books
\begin{enumerate}
\item Concepts of Modern Physics by Arthur Beiser,
\item Feynman's Lectures on Physics, volumes 1, 2, 3,
\end{enumerate}
\item Unfortunately, we don't have a lab accompanying this course. Yet, we will
watch a few experiments on YouTube.
\item Without further delay, let's begin the journey.
\end{enumerate}
\end{frame}

\begin{frame}
\frametitle{Setting the stage - Newton's laws}
The three laws are
\begin{enumerate}
\item Every body continues to be in a state of rest or of uniform rectilinear
motion unless acted by an external force.
\item The rate of change of momentum of a body is equal to the net force acting
on it.
\item Every action has an equal and opposite reaction.
\end{enumerate}
A \emph{frame of reference} is just a set of coordinate axes and a clock. A
frame of reference in which Newton's first law is valid is called an \emph{
inertial frame of reference}. Which of these are inertial frame of reference?
\begin{itemize}
\item Your desk.
\item An accelerating train.
\item An aeroplane flying at a uniform speed and a fixed height.
\end{itemize}
\end{frame}

\begin{frame}
\frametitle{Frames of reference}
\begin{enumerate}
\item Is earth an inertial frame of reference? Have you seen Focault's pendulum?
\item Solar system? Our galaxy?
\item Newton thought that the frame of reference of the fixed stars is a
universal inertial frame of reference.
\item But why fuss over an inertial frame of reference? Imagine you're in a
train moving one a straight, level track at uniform speed and there's a ball
kept on the floor. Now suppose that engine driver applies brakes. What happens?
Can you explain what you see using Newton's laws?
\item Let's do it together. Draw a free body diagram of the ball ...
\end{enumerate}
\end{frame}

\begin{frame}
\frametitle{Frame of reference}
\begin{enumerate}
\item Newton's first law is surely not valid when the train decelerates.
\item How about Newton's second law? 
\item How about the third law?
\item So you see that Newton's framework requires an inertial frame of 
reference. There are other ways to look at classical mechanics where frames of
reference are not as important. But let's stick to Newton's laws.
\end{enumerate}
\end{frame}
\end{document}

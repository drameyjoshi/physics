\documentclass{report}
\usepackage{amsmath}
\begin{document}
\begin{enumerate}
\item[1.1] Levi-Civita tensor.
\begin{enumerate}
\item[(a)] $\hat{e}_1, \hat{e}_2, \hat{e}_3$ are unit vectors along three 
Cartesian axes. Show that
\[
\epsilon_{ijk} = \hat{e}_i\cdot(\hat{e}_j \times \hat{e}_k).
\]
If $j = k$ or $i = j$ the triple product vanishes. The magnitude of the
triple product is $\pm 1$. Its sign depends on whether $i, j, k$ are in
even or odd permutation. Thus, the properties of rhs are the same as that of
the Levi-Civita tensor.

\item[(b)] Quite straightforward.

\item[(c)] Show that $\epsilon_{ijk}\epsilon_{ist} = \delta_{js}\delta{kt} 
- \delta_{jt}\delta_{ks}$.
We start from lhs,
\begin{equation}\label{e1}
\epsilon_{ijk}\epsilon{ist} = \epsilon_{1jk}\epsilon_{1st} +
                              \epsilon_{2jk}\epsilon_{2st} +
                              \epsilon_{3jk}\epsilon_{3st} 
\end{equation}
In the first term on the rhs, $j, k$ can be either $2, 3$ or $3, 2$ for it
to be non-zero. Thus, there are four possibilities:
\begin{itemize}
\item $j = 2, k = 3, s = 2, t = 3$ when the term is $1$.
\item $j = 3, k = 2, s = 2, t = 3$ when the term is $-1$.
\item $j = 3, k = 2, s = 3, t = 2$ when the term is $1$.
\item $j = 2, k = 3, s = 3, t = 2$ when the term is $-1$.
\end{itemize}
These are also the possible values of $\delta_{js}\delta_{kt} - \delta_{jt}
\delta_{ks}$. Further, when $j, k$ take these values, the other terms on
the rhs of equation \eqref{e1} are zero.

Identical analysis can be used to evaluate the remaining terms.

\item[(d)] Given that $\vec{a}$ and $\vec{b}$ are constant vectors. If
$\vec{L}^{op}$ is the angular momentum operator then show that
\[
[\vec{L}^{op}\cdot\vec{a}, \vec{L}^{op}\cdot\vec{b}] = i\hbar\vec{L}^{op}
\cdot (\vec{a} \times \vec{b}).
\]
We start with the left hand side,
\begin{eqnarray*}
[\vec{L}^{op}\cdot\vec{a}, \vec{L}^{op}\cdot\vec{b}] 
    &=& [L^{op}_i a_i, L^{op}_j b_j] \\
    &=& L^{op}_i a_i L^{op}_j b_j - L^{op}_j b_i L^{op}_i a_i \\
    &=& a_ib_j(L^{op}_i L^{op}_j - L^{op}_j L^{op}_i) i\\
    &=& a_ib_j[L^{op}_i, L^{op}_j] \\
    &=& a_ib_j\epsilon_{ij}L^{op}_k \\
    &=& \vec{L}^{op}\cdot(\vec{a} \times \vec{b}).
\end{eqnarray*}
\end{enumerate}

\item[1.2] Levi-Civita tensor.
\begin{enumerate}
\item[(a)] $\delta_{ii} = \delta_{11} + \delta_{22} + \delta_{33} = 3$.
\item[(b)] $\delta_{ij}\epsilon_{ijk} = 0$ because when $i = j$ the first
term is $1$ but the second term is $0$ and when $i \ne j$ it is the other
way round.
\item[(c)] $\epsilon_{ijk}\epsilon_{ljk} = \delta_{il}$. Note that $j$ and
$k$ must be different for the two terms on the lhs to be non-zero. That
leaves only one possibility for both $i$ and $l$. Further, the product will
be $1$ irrespective of the sign of the individual terms because they are
identical when they are non-zero.
\item[(d)] $\epsilon_{ijk}\epsilon_{ijk} = 1$.
\end{enumerate}

\item[1.3] Vector identities.
\begin{enumerate}
\item[(a)] $\vec{A} \times \vec{B} = \epsilon_{ijk}\hat{e}_iA_jB_k$ and
$\vec{C} \times \vec{D} = \epsilon_{lmn}\hat{e}_lC_mD_n$ so that
$(\vec{A} \times \vec{B})\cdot(\vec{C} \times \vec{D}) = \epsilon_{ijk}
\hat{e}_iA_jB_k \cdot \epsilon_{lmn}\hat{e}_lC_mD_n = \epsilon_{ijk}
\epsilon_{lmn}A_jB_kC_mD_n\hat{e}_i\cdot\hat{e}_l = \epsilon_{ijk}
\epsilon_{lmn}A_jB_kC_mD_n\delta_{il} = \epsilon_{ijk}\epsilon_{imn}
A_jB_kC_mD_n = (\delta_{jm}\delta_{kn} - \delta_{jn}\delta_{km})A_jB_kC_m
D_n = A_jC_j B_kD_k - A_jD_j - B_kC_k = \vec{A}\cdot\vec{C}\vec{B}\cdot
\vec{D} - \vec{A}\cdot\vec{D}\vec{B}\cdot\vec{C}$.

\item[(b)] \begin{eqnarray*}
\nabla\cdot(\vec{f}\times\vec{g}) &=& (\hat{e}_l\partial_l)
\cdot(\epsilon_{ijk}\hat{e}_if_jg_k) \\
 &=& \delta_{il}\partial_l\epsilon_{ijk}f_jg_k \\
 &=& \epsilon_{ijk}\partial_i(f_jg_k) \\
 &=& \epsilon_{ijk}(g_k\partial_if_j + f_j\partial_ig_k) \\
 &=& g_k\epsilon_{ijk}\partial_if_j + f_j\epsilon_{ijk}\partial_i g_k \\
 &=& g_k\epsilon_{kij}\partial_i f_j + f_j\epsilon_{jki}\partial_i g_k \\
 &=& g_k\epsilon_{kij}\partial_if_j - f_j\epsilon_{jik}\partial_i g_k \\
 &=&\vec{g}\cdot\nabla\times\vec{f} - \vec{f}\cdot\nabla\times\vec{g}
\end{eqnarray*}

\item[(c)] Let $\vec{A} \times \vec{B} = \vec{E}$ and $\vec{C} \times 
\vec{D} = \vec{F}$ so that  $(\vec{A} \times \vec{B}) \times (\vec{C} 
\times \vec{D}) = \vec{E} \times \vec{F} = \epsilon_{ijk}\hat{e}_iE_jF_k
= \epsilon_{ijk}\hat{e}_i\epsilon_{jlm}A_lB_m\epsilon_{krs}C_rD_s$. Now,
\begin{eqnarray*}
\epsilon_{ijk}\epsilon_{jlm} &=& -\epsilon_{jki}\epsilon_{jlm} \\
 &=& -(\delta_{kl}\delta_{im} - \delta_{km}\delta_{li}) \\
 &=& \delta_{km}\delta_{li} - \delta_{kl}\delta_{im}
\end{eqnarray*}
so that 
\begin{eqnarray*}
\vec{E} \times \vec{F} &=& lB_m\epsilon_{krs}C_rD_s \\
 &=& \hat{e}_i(A_iB_k\epsilon_{krs}C_rD_S - A_kB_i\epsilon_{krs}C_rD_s\\
 &=& (\epsilon_{krs}B_kC_rD_S)A_i\hat{e}_i - 
     (\epsilon_{krs}A_kC_rD_s)B_i\hat{e}_i \\
 &=& (\vec{B}\cdot(\vec{C} \times \vec{D}))\vec{A} - 
     (\vec{A}\cdot(\vec{C} \times \vec{D}))\vec{B}.
\end{eqnarray*}

\item[(d)] Show that $(\vec{\sigma}\cdot\vec{a})(\vec{\sigma}\cdot\vec{b}) = 
\vec{a}\cdot\vec{b} + i\vec{\sigma}\cdot(\vec{a}\times\vec{b})$.
$(\vec{\sigma}\cdot\vec{a})(\vec{\sigma}\cdot\vec{b}) = \sigma_i a_i \sigma_jb_j
= a_ib_j\sigma_i\sigma_j = a_ib_j(\delta_{ij} + i\epsilon_{ijk}\sigma_k)
= a_ib_i + i\epsilon_{ijk}a_ib_j\sigma_k = \vec{a}\cdot\vec{b} + i\epsilon_{kij}
\sigma_ka_ib_j = \vec{a}\cdot\vec{b} + i\vec{\sigma}\cdot(\vec{a}\times\vec{b})$.
\end{enumerate}

\item[(1.4)] Vector derivative identities.
\begin{enumerate}
\item[(a)] $\nabla\cdot(f\vec{g}) = \hat{e}_i\partial_i(f\hat{e}_jg_j) =
\hat{e}_i\cdot\hat{e}_j\partial_i(fg_j) = \delta_{ij}\partial_i(fg_j) =
\partial_i(fg_i) = \nabla f \cdot\vec{g} + f\nabla\cdot\vec{g}$.

\item[(b)] $\nabla\times(f\vec{g}) = \hat{e}_i\epsilon_{ijk}\partial_j(fg_k)
= \hat{e}_i\epsilon_{ijk}(\partial_j f)g_k + \hat{e}_i\epsilon_{ijk}f
(\partial_jg_k) = \nabla f\times \vec{g} + f\nabla\times\vec{g} = 
f\nabla\times\vec{g} - \vec{g}\times\nabla f$.

\item[(c)] 
\begin{eqnarray*}
\nabla\times(\vec{g}\times\vec{r}) &=& 
    \hat{e}_i\epsilon_{ijk}\partial_j(\epsilon_{klm}g_lr_m) \\
 &=& \hat{e}_i\epsilon_{ijk}\epsilon_{klm}\partial_j(g_lr_m) \\
 &=& \hat{e}_i\epsilon_{kij}\epsilon_{klm}\partial_j(g_lr_m) \\
 &=& \hat{e}_i(\delta_{il}\delta_{jm} - \delta_{im}\delta_{jl})
     \partial_j(g_lr_m) \\
 &=& \hat{e}_i(\partial_j(g_ir_j) - \partial_j(g_jr_i)) \\
 &=& \hat{e}_i(r_j\partial_j g_i + g_i\partial_jr_j) -
     \hat{e}_i(r_i\partial_jg_j + g_j\partial_jr_i) \\
 &=& \vec{r}\cdot\nabla\vec{g} + 3\vec{g} - \vec{r}\nabla\cdot{g} 
     - \hat{e}_ig_j\delta_{ij} \\
 &=& \vec{r}\cdot\nabla\vec{g} + 3\vec{g} - \vec{r}\nabla\cdot{g} - \vec{g} \\
 &=& \vec{r}\cdot\nabla\vec{g} + 2\vec{g} - \vec{r}\nabla\cdot{g}
\end{eqnarray*}

\end{enumerate}
\end{enumerate}
\end{document}

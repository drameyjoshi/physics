\documentclass{article}
\usepackage{amsmath, graphicx, xcolor, hyperref}
\usepackage{mathabx}
\newcommand{\op}{{\;\prime}}
\newcommand{\grad}{\nabla\;}
\newcommand{\dive}{\nabla\cdot}
\newcommand{\curl}{\nabla\times}
\newcommand{\td}[2]{\frac{d{#1}}{d{#2}}}
\newcommand{\pdt}[2]{\frac{\partial{#1}}{\partial{#2}}}
\newcommand{\uv}[1]{\hat{e}_{#1}}
\newcommand{\un}{\hat{n}}

\begin{document}
\title{The Electric Potential}
\author{Amey Joshi}
\date{15-Jun-2024}
\maketitle
\begin{enumerate}
\item Given that $\vec{E} = 6xy\uv{x} + 3(x^2 - y^2)\uv{y}$. Denote the origin 
$(0,0,0)$ by $O$ and the point $(x_1, y_1, 0)$ by $A$. In the first case,
\begin{equation}\label{e1}
d\vec{s} = \uv{x}dx + \uv{y}dy
\end{equation}
so that the line integral is
\begin{eqnarray}
L_1 &=& \int_O^A\vec{E}\cdot d\vec{s} \nonumber \\
 &=& \int_0^{x_1}3x^2\uv{y}\cdot\uv{x}dx + 
     \int_0^{y_1}(6x_1y\uv{x} + 3(x_1^2 - y^2)\uv{y}dy)\cdot\uv{y}dy \nonumber \\
 &=& 0 + \int_0^{y_1}(x_1^2 - y^2)dy \nonumber \\
 &=& 3x_1^2y - y_1^3 \label{e2}
\end{eqnarray}
In the second case, the equation of the straight line is 
\begin{equation}\label{e3}
y = \frac{y_1}{x_1}x
\end{equation}
so that
\[
dy = \frac{y_1}{x_1}dx
\]
and
\begin{equation}\label{e4}
d\vec{s} = \uv{x}dx + \uv{y}dy = \uv{x}dx + \uv{y}\frac{y_1}{x_1}dx.
\end{equation}
Further, along the straight line,
\begin{equation}\label{e5}
\vec{E} = 6\frac{y_1}{x_1}x^2\uv{x} + 3\left(x^2 - \frac{y_1^2}{x_1^2}x^2\right)\uv{y}.
\end{equation}
so that the line integral is
\begin{eqnarray}
L_2 &=& \int_O^A\vec{E}\cdot d\vec{s} \nonumber \\
 &=& \int_O^A6\frac{y_1}{x_1}x^2dx + 
 3\left(x^2 - \frac{y_1^2}{x_1^2}x^2\right)\frac{y_1}{x_1}dx \nonumber \\
 &=& 2\frac{y_1}{x_1}x_1^3 + \frac{y_1}{x_1}\left(1 - \frac{y_1^2}{x_1^2}\right)x_1^3 \nonumber \\
 &=& 3x_1^2y_1 - y_1^3 \label{e6}
\end{eqnarray}
In the third case, $d\vec{s} = \uv{x}dx + \uv{y}dy$ and
\begin{eqnarray}
L_3 &=& -\int_0^{y_1}3y^2dy + \int_0^{x_1}6xy_1dx \nonumber \\
 &=& -y_1^3 + 3x_1^2y_1 \label{e7}
\end{eqnarray}
Equations \eqref{e2}, \eqref{e6} and \eqref{e7} show that the line integrals are
the same. The potential function is
\begin{equation}\label{e8}
\varphi(x, y, 0) = -\int_O^{(x, y, 0)}\vec{E}\cdot d\vec{s} = -(3x^2y - y^3).
\end{equation}
However,
\begin{equation}\label{e9}
\varphi(x, y, z) = -\int_O^{(x, y, z)}\vec{E}\cdot d\vec{s} = -(3x^2y - y^3).
\end{equation}
is also true (why?) so that
\[
-\grad\varphi = 6xy\uv{x} + 3(x^2-y^2)\uv{y} = \vec{E}.
\]

\item The potential of the charge configuration is
\begin{equation}\label{e10}
\varphi(0, 0, z) = 12/z - 6/(z - 3)
\end{equation}
so that the variation of potential is as shown in figure \ref{c2f2}. It was 
generated using
\begin{verbatim}
import numpy as np
import matplotlib.pyplot as plt

def varphi(z: float) -> float:
    return 12/z - 6/(z - 3)
    
Z1 = np.linspace(-6, 6)
Z2 = np.linspace(-0.1, 0.1)
Z3 = np.linspace(2.9, 3.1)

Z = np.concatenate((Z1, Z2, Z3))
Z.sort()
V = [varphi(z) for z in Z]

plt.plot(Z, V)
plt.xlabel(r'$z$')
plt.ylabel(r'$\varphi(0, 0, z)$')
plt.title('Potential due to two charges on the z-axis')
plt.show()
\end{verbatim}
\begin{figure}[!ht]
\center
\includegraphics[scale=0.6]{purcell_c2p2}
\caption{Problem 2}
\label{c2f2}
\end{figure}
A few points to consider:
\begin{itemize}
\item Examine its behaviour in the regions $z < 0, z \in (0, 3)$ and $z > 3$.
\item Why did we create three arrays $Z1, Z2, Z3$ in the code?
\item Why did we sort the array $Z$?
\end{itemize}

\item The potential at a point in the $y$ axis is
\begin{equation}\label{e11}
\varphi(y) = -\frac{2}{y} + \frac{1}{\sqrt{1 + (y - 1)^2}} + \frac{1}{\sqrt{1 + (y - 1)^2}} 
= -2\left(\frac{1}{y} - \frac{1}{\sqrt{1 + (y - 1)^2}}\right)
\end{equation}
Electric field $\vec{E} = -\grad\varphi$ will be zero when $d\varphi/dy = 0$. 
A few points to note about the potential:
\begin{itemize}
\item The function blows up at the origin.
\item For $y \rightarrow \pm \infty$, $\varphi(y) \rightarrow 0$.
\item For $y < 0$, $\varphi(y) > 0$.
\item $\varphi(0^+) = \infty, \varphi(1) = 0, \varphi(y) > 0$ for $y \in (0, 1) \cup (1, \infty)$.
\end{itemize}
Now,
\[
\td{\varphi}{z} = 2\left(\frac{1}{y^2} - \frac{(y - 1)}{(y^2 - 2y + 2)^{3/2}}\right)
\]
and $d\varphi/dz = 0$ implies $(y^2 - 2y + 2)^3 = y^4(y - 1)^2$. The roots of
\[
4 y^5 - 17 y^4 + 32 y^3 - 36 y^2 + 24 y - 8 = 0
\]
can be found using
\begin{verbatim}
np.roots([4, -17, 32, -36, 24, -8]).
\end{verbatim}
They are $y \approx 1.62072, 0.427249 \pm 0.974453 i, 0.887391 \pm 0.550058$. The only 
real root is $1.62072$.
The variation of the potential along the $y$ axis is shown in figure \ref{c2f3a} to 
\ref{c2f3e}.
\begin{figure}[!ht]
\center
\includegraphics[scale=0.75]{purcell_c2p3a}
\caption{In range $(-2, 2)$.}
\label{c2f3a}
\end{figure}
\begin{figure}[!ht]
\center
\includegraphics[scale=0.75]{purcell_c2p3b}
\caption{In range $(1.5, 1.7)$.}
\label{c2f3b}
\end{figure}
\begin{figure}[!ht]
\center
\includegraphics[scale=0.75]{purcell_c2p3c}
\caption{In range $(0.1, 2)$.}
\label{c2f3c}
\end{figure}
\begin{figure}[!ht]
\center
\includegraphics[scale=0.75]{purcell_c2p3d}
\caption{In range $(-5, -0.1)$.}
\label{c2f3d}
\end{figure}
\begin{figure}[!ht]
\center
\includegraphics[scale=0.75]{purcell_c2p3e}
\caption{In range $(1.5, 5)$.}
\label{c2f3e}
\end{figure}

The figures were generated using
\begin{verbatim}
import numpy as np
import matplotlib.pyplot as plt

def varphi(y: float) -> float:
    return -2 * (1/y - 1/np.sqrt(1 + (y - 1)**2))

def plot(ymin: float,
         ymax: float,
         fname: str,
         loc: str='best',
         show_x_axis: bool=True) -> None:
    Y = np.linspace(ymin, ymax)
    V = [varphi(y) for y in Y]
    label = f'In range [{ymin}, {ymax}]'
    plt.plot(Y, V, label=label)
    if show_x_axis:
        plt.hlines(y=0, xmin=ymin, xmax=ymax, color='red')
        
    plt.xlabel(r'$y$')
    plt.ylabel(r'$\varphi(y)$')
    plt.legend(loc=loc)
    plt.savefig(fname)
    plt.close()


def main():
    plot(-2, 2, 'course/purcell_c2p3a.png')
    plot(1.5, 1.7, 'course/purcell_c2p3b.png', show_x_axis=False)   
    plot(0.1, 2, 'course/purcell_c2p3c.png', loc='lower right')
    plot(-5, -0.1, 'course/purcell_c2p3d.png')
    plot(1, 5, 'course/purcell_c2p3e.png')


if __name__ == '__main__':
    main()

\end{verbatim}

\item Given the potential
\begin{equation}\label{e11}
\varphi(x, y, z) = \begin{cases}
x^2 + y^2 + z^2 \text{ for } x^2 + y^2 + z^2 < a^2 \\
-a^2 + \frac{2a^3}{\sqrt{x^2 + y^2 + z^2}} \text{ for } a^2 < x^2 + y^2 + z^2
\end{cases}
\end{equation}
the electric field $\vec{E} = -\grad\varphi$ is
\begin{equation}\label{e12}
\vec{E} = \begin{cases} -2(x\uv{x} + y\uv{y} + z\uv{z}) \text{ for } x^2 + y^2 + z^2 < a^2 \\
2a^3\frac{x\uv{x} + y\uv{y} + z\uv{z}}{(x^2 + y^2 + z^2)^{3/2}} \text{ for } a^2 < x^2 + y^2 + z^2
\end{cases}
\end{equation}
Therefore 
\begin{equation}\label{e13}
\dive\vec{E} = \begin{cases} -6 \text{ for } x^2 + y^2 + z^2 < a^2 \\
0 \text{ for } a^2 < x^2 + y^2 + z^2
\end{cases}
\end{equation}
so that, by Gauss law, the charge density is,
\begin{equation}\label{e14}
\rho = \begin{cases} -\frac{3}{2\pi} \text{ for } x^2 + y^2 + z^2 < a^2 \\
0 \text{ for } a^2 < x^2 + y^2 + z^2
\end{cases}
\end{equation}

\item Let $a$ be the radius of the sphere. If $Q$ is the total charge on it then
the potential at its surface is
\[
\varphi = -\frac{Q}{a}.
\]
If $\sigma$ is the surface charge density then $Q = 4\pi a^2\sigma$ so that
\[
\varphi = -4\pi a\sigma
\]
that is,
\[
\sigma = -\frac{\varphi}{4\pi a}.
\]
Given that $\varphi = 1000/300$ stat volt and electronic charge is $4.8 \times
10^{-10}$ esu, 
\[
\sigma = \frac{10}{3}\frac{1}{4\pi a}\frac{10^{10}}{4.8} \approx 
\frac{5.5}{a} \times 10^{-3} \times 10^{10} \approx \frac{5.5}{a} \times 10^7.
\]

\item The electric field on the surface of the earth is
\[
\vec{E} = \frac{1}{4\pi\epsilon_0}\frac{Q}{R^2}\uv{r}.
\]
Given that $R = 6.378 \times 10^6$ m,
\[
E \approx 9 \times 10^9 \times \frac{1}{4.078 \times 10^{13}} 
\approx 2.21 \times 10^{-4} \text{V/m}.
\]
The potential at the surface is
\[
\varphi = \frac{1}{4\pi\epsilon_0}\frac{Q}{R} \approx 1411 \text{V}.
\]

\item The integral,
\[
\int_C^D\vec{E}\cdot d\vec{s} = \varphi(D) - \varphi(C).
\]
It is easy to get the terms on the rhs.
\begin{eqnarray}
\varphi(C) &=& -\frac{2}{2\sqrt{5}} + \frac{3}{5} \label{e16} \\
\varphi(D) &=& -\frac{2}{5} + \frac{3}{2\sqrt{5}} \label{e17}
\end{eqnarray}
so that
\[
\int_C^D\vec{E}\cdot d\vec{s} = \frac{\sqrt{5} - 2}{2}.
\]

\item For the infinitely long cylinder of radius $a$ whose cross section is shown
in figure 2.17 of the book, the electric field is normal to the curved surface of
the cylinder. In order to get the field at a point which is at a distance $r$ from
the cylinder's axis, align the coordinates so that $z$ axis coincides with the 
cylinder's axis. Consider a pillbox coaxial with the cylinder and of radius $r$.
If $r < a$, the pillbox encloses a charge $\pi r^2 h \rho$, where $h$ is the
pillbox's height. The flux through the pillbox is $2\pi rh E$. Therefore, by
Gauss' law,
\[
2\pi rh E = 4\pi (\pi r^2h \rho)
\]
so that
\begin{equation}\label{e18}
\vec{E} = 2\pi\rho \vec{r}, \text{ if } r < a.
\end{equation}
If $r \ge a$, a similar analysis yields
\[
2\pi rh E = 4\pi (\pi a^2 h \rho)
\]
so that
\begin{equation}\label{e19}
\vec{E} = 2\pi\rho \frac{a^2}{r}\uv{r}, \text{ if } r \ge a.
\end{equation}
We are asked to find the potential assuming $\varphi(0) = 0$. Thus, if $r < a$,
\begin{equation}\label{e20}
\varphi(r) = \int_0^r\vec{E}\cdot d\vec{s} = 
\int_0^r 2\pi\rho \vec{r}^\op\cdot d\vec{r}^\op = \pi\rho r^2 \text{ if } r < a.
\end{equation}
If $r \ge a$,
\begin{eqnarray}
\varphi(r) &=& \int_0^r\vec{E}\cdot d\vec{s} \nonumber \\
 &=& \int_0^a\vec{E}\cdot d\vec{s} + \int_a^r\vec{E}\cdot d\vec{s} \nonumber \\
 &=& \pi\rho a^2 + 2\pi\rho a^2 \int_a^r \frac{dr^\op}{r^\op} \nonumber \\
 &=& \pi\rho a^2 + 2\pi\rho a^2\ln\left(\frac{r}{a}\right) \nonumber \\
 &=& \pi\rho a^2\left(1 + \ln\left(\frac{r^2}{a^2}\right)\right) \label{e21}
\end{eqnarray}
Note that $\varphi$ is continuous at $r = a$. So is the electric field.

\item The potential at the rim is $4\sigma a$ and that on the $y$ axis is 
\[
\varphi(0, y, 0) = 2\pi\sigma(\sqrt{y^2 + a^2} \pm y)
\]
so that the point on $y$ axis where the said equipotential cuts the $y$ axis
is given by the equation
\[
\pi(\sqrt{y^2 + a^2} \pm y) = 2 a.
\]
As expected, this equation is independent of $\sigma$. It is a purely geometrical
fact. I cannot sketch the equipotential surface without knowing its functional 
form. Finding $\varphi$ at an arbitrary point from the dissc is not a trivial 
problem.

\item The potential at a point on the $z$ axis is
\[
d\varphi(2d) = \frac{dq}{2d - z} = \lambda\frac{dz}{2d - z}
\]
so that
\begin{equation}\label{e22}
\varphi(2d) = \int_{-d}^d d\varphi(z) = -\int_{-d}^d\frac{dz}{z - 2d} = \lambda\ln 3
\end{equation}
For a point on the $x$ axis,
\[
d\varphi(x) = \frac{dq}{\sqrt{x^2 + z^2}}
\]
so that
\begin{eqnarray}
\varphi(x) &=& \lambda\int_{-d}^d\frac{dz}{\sqrt{x^2 + z^2}} \nonumber \\
 &=& \lambda\int_{-\alpha}^\alpha \sec\theta d\theta\;\text{ where }\alpha = \tan^{-1}(d/x) \nonumber \\
 &=& \lambda\ln|\sec\theta + \tan\theta|\Big|_{-\alpha}^\alpha \nonumber \\
 &=& \lambda\ln\left|\frac{\sqrt{x^2 + d^2} + d}{\sqrt{x^2 + d^2} - d}\right| \label{e23}
\end{eqnarray}
$\varphi(z = 2d) = \varphi(x)$ implies $\sqrt{x^2 + d^2} + d = 3(\sqrt{x^2 + d^2} - d)
\Rightarrow x = d\sqrt{3}$.

\item The distance between the point $P = (3d/2, 0, d)$ and a point $(0, 0, z)$ on
the rod is
\begin{equation}\label{e24}
r = \sqrt{\frac{9d^2}{4} + 0 + (d - z)^2}
\end{equation}
so that the potential at it due to a small charge element $dq$ at $z$ is
\[
d\varphi(P) = \frac{dq}{r} = 2\lambda\frac{dz}{\sqrt{9d^2 + 4(d - z)^2}}
\]
and 
\[
\varphi(P) = 2\lambda\int_{-d}^d\frac{dz}{\sqrt{9d^2 + 4(d - z)^2}}
\]
Let $u = 2(d - z)$ so that $du = -2dz$ and the limits of integral are from $u=4d$
to $z = 0$ and
\[
\varphi(P) = -\lambda\int_{4d}^0\frac{du}{\sqrt{9d^2 + u^2}}.
\]
If $u = 3d\tan\theta$ then $du = 3d\sec^2\theta d\theta$ and the limit of the 
integral are from $\alpha = \tan^{-1}(4/3)$ to $\theta = 0$.
\[
\varphi(P) = -\lambda\int_{\alpha}^0\sec\theta d\theta = 
-\lambda\ln|\sec\theta + \tan\theta|\Big|_{-\alpha}^0 =
-(0 - \lambda\ln 3) = \lambda\ln 3.
\]

We will repeat this analysis for an arbitrary point $Q = (X, 0, Z)$. Its distance 
from a point $(0, 0, z)$ on the rod is
\begin{equation}\label{e25}
r = \sqrt{X^2 + (Z - z)^2}
\end{equation}
so that the potential at it us
\[
\varphi(Q) = \lambda\int_{-d}^d\frac{dz}{\sqrt{X^2 + (Z - z)^2}}
\]
Let $u = Z - z$ so that $du = -dz$ and the limits of the integral go from
$Z + d$ to $Z - d$ and
\[
\varphi(Q) = -\lambda\int_{Z+d}^{Z-d}\frac{du}{\sqrt{X^2 + u^2}} = 
\lambda\int_{Z-d}^{Z+d}\frac{du}{\sqrt{X^2 + u^2}}.
\]
If $u = X\tan\theta$, $du = X\sec^2\theta d\theta$ and the limits of the
integral become 
\[
\alpha = \tan^{-1}\left(\frac{Z - d}{X}\right) \;\text{and}\;
\beta = \tan^{-1}\left(\frac{Z + d}{X}\right).
\]
We then have,
\[
\varphi(Q) = \lambda\int_\alpha^\beta \sec\theta d\theta = 
\lambda\ln|\sec\theta + \tan\theta|\Big|_\alpha^\beta.
\]
Now,
\[
\tan\alpha = \frac{Z-d}{X} \Rightarrow \sec\alpha=\frac{\sqrt{X^2 + (Z-d)^2}}{X}
\]
so that
\[
\varphi(Q) = \lambda\ln\left|\frac{\sqrt{X^2 + (Z+d)^2} + Z+d}{X}\right| - 
\lambda\ln\left|\frac{\sqrt{X^2 + (Z-d)^2} + Z-d}{X}\right|
\]
or
\begin{equation}\label{e26}
\varphi(Q) = \lambda\ln\left|\frac{\sqrt{X^2 + (Z+d)^2} + Z+d}{\sqrt{X^2 + (Z-d)^2} + Z-d}\right|
\end{equation}
$\varphi(Q)$ will be equal to $\varphi(P)$ when
\[
\frac{\sqrt{X^2 + (Z+d)^2} + Z+d}{\sqrt{X^2 + (Z-d)^2} + Z-d} = \sqrt{3}
\]
that is
\[
\sqrt{X^2 + (Z+d)^2} + Z+d = 3\sqrt{X^2 + (Z-d)^2} + 3Z-3d
\]
\end{enumerate}

\section{Potential of a uniformly charged disc}
The textbook mentions that the potential at arbitrary points due to a uniformly
charged disc can be expressed in terms of elliptic functions. We will derive the
expresions here following the paper (\href{https://arxiv.org/pdf/2004.04540}{pre-print})\cite{bs}.
\subsection{Arbitrary point on the disc}
\subsection{Arbitrary point not on the disc}

\begin{thebibliography}{1}
\bibitem{bs} On the electrostatic potential and electric field of a uniformly charged disk.
published in the European Journal of Physics 41.4 (2020): 045201
\end{thebibliography}
\end{document}
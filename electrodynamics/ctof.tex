\documentclass{report}
\usepackage{amsmath, amssymb}
\usepackage{graphicx}
\usepackage{tensor}
\usepackage{hyperref}
\usepackage{xcolor}
\usepackage{mathabx}
\usepackage{bm}

\renewcommand{\vec}{\mathbf}
\newcommand{\uv}[1]{\hat{\mathbf{e}}_{#1}}
\newcommand{\un}{\hat{\mathbf{n}}}

%\newcommand{\uv}[1]{\hat{e}_{#1}}
%\newcommand{\un}{\hat{n}}

\newcommand{\abs}[1]{\left\vert{#1}\right\vert}
\newcommand{\td}[2]{\frac{d{#1}}{d{#2}}}
\newcommand{\pdt}[2]{\frac{\partial{#1}}{\partial{#2}}}
\newcommand{\spdt}[2]{\frac{\partial^2{#1}}{\partial{#2}^2}}
\newcommand{\op}{{\,\prime}}
\newcommand{\tp}{{\,\prime\prime}}
\newcommand{\ta}[1]{\left\langle{#1}\right\rangle}
\newcommand{\tav}[1]{\ta{\vec{#1}}}
\DeclareMathOperator{\grad}{grad}
\DeclareMathOperator{\dive}{div}
\DeclareMathOperator{\curl}{curl}
\DeclareMathOperator{\Arg}{arg}
\DeclareMathOperator{\re}{Re}
\DeclareMathOperator{\im}{Im}
\DeclareMathOperator{\sech}{sech}

%\newcommand{\fint}{\mbox{--}\mkern-16mu\int}

\begin{document}
\tableofcontents
\chapter{The Special Theory of Relativity}\label{c1}
\begin{enumerate}
\item A frame of reference is a set of cartesian coordinate axes and a clock.

\item An inertial frame is the one in which Newton's first law is valid. The 
existence of such frames was inferred from the experimental observations of the
seventeenth and the eighteenth centuries.

\item Experiments also showed that a frame fixed to the earth is not inertial
but the one fixed with respect to the distant stars was. If $K$ is an inertial
frame then so is $K^\op$ if it moves at a constant relative velocity with respect
to $K$. Thus, there are an infinite number of inertial frames.

\item Suppose that two events happen in an inertial frame $K$: one at a point
$\vec{a}$, and at time $t$ and another one at a point $\vec{b}$ and at time 
$t + \delta t$. When viewed from another inertial frame $K^\op$ moving at a 
velocity $\vec{v}$ with respect to $K$, the events occured at points $\vec{r}_0 
+ \vec{a} + \vec{v}t$ and $\vec{r}_0 + \vec{b} +\vec{v}(t + \delta t)$. Here 
$\vec{r}_0$ is the point where the origin of $K$ and $K^\op$ coincided. An observer 
in $K$ reports that the two events occured at times $\delta t$ apart and at points 
$\vec{b} - \vec{a}$ apart. An observer in $K^\op$ reports that the two events 
happened at a distance $\vec{b} - \vec{a} + \vec{v}\delta t$ apart. However, 
he does not dispute that the same time $\delta t$ elapsed between the two events. 
If $\delta t$ were zero then the two events will be simultaneous in \emph{all} 
inertial frames of reference although they could have happend at different 
points as seen by different observers. This is a consequence of the implicit 
assumption of an absolute time in classical physics.

\item This idea can be corroborated by a simple example. An person on a platform may
see a passenger lifting a tea cup as the coach entered the platform and take it
to his lips as the coach exited it. The two events thus occurred, from his 
perspective, at different points. However, a co-passenger will report them to
be happening at the same spot. The observers will agree on the duration between
the events.

\item The special theory of relativity has its roots in two facts borne out 
of experiments:
\begin{itemize}
\item The laws of physics are the same in all inertial frames of reference and
\item Changes in the state of a system are propagated with a finite speed.
\end{itemize}

\item It is also confirmed by experiments that the changes in the state of system 
are propagated at a speed not exceeding
\begin{equation}\label{c1e1}
c = 2.998 \times 10^8 \;\text{cm/s}.
\end{equation}
No material body can move at a speed exceeding $c$ because if it could then it
can be used to signal the change of state of another body. Further, $c$ is the
maximum speed in \emph{all} inertial frames. For if it were not then the laws of
physics will not be identical across frames. Thus, $c$ is a universal constant.
It is also the speed of light in vacuum.

\item The idea of an absolute, universal time is in conflict with the experimental 
observation of a finite speed of propagation of light in vacuum. For if the
speed is $c$ in a frame $K$ then its speed will be $3c/2$ in a frame $K^\op$
approaching $K$ with a speed $c/2$ in a direction opposite to that of propagation
of light. That this is not true was confirmed by Michelson and Morley's experiment.
Time elapses differently in different systems and therefore a value of a time
difference must be accompanied by a specification of the frame in which it was 
measured.

\item Consider a frame $K$ with a source of light at its origin and two detectors at
points $(-a, 0)$ and $(a, 0)$. A spherical light front will reach the two 
detectors simultaneously when observed from $K$. Let us consider the experiment
replicated in another frame $K^\op$ that moves with a velocity $v\hat{e}_x$ with
respect to $K$. Suppose that their origins coincided then the light signal was 
emitted in $K$. Since the speed of light is the same in $K$ and $K^\op$ an observer
in $K^\op$ also sees a spherical wavefront propagating isotropically. However, he
will observe the wavefront reaching $(a, 0)$ sooner than it reaches $(-a, 0)$.
Events that are simultaneous in $K$ are not so in $K^\op$.

\item An event is described by the point at which occurred and the time when it
occurred. The four numbers describing an event can be interpreted as points in a
four-dimensional space. They are called the \emph{world points}. They move along
curves in the four-dimensional space called the \emph{world lines}.

\item A material body at a point $\vec{r}$ to which nothing happens travels along
a world line that is parallel to the $t$ axis. In the four-dimensional space, nothing
is still. If a material body moves along the $x$ axis with a uniform speed $v$ then
its world line is a straight line making an angle $\tan^{-1}(v)$ with the $t$ axis.
If it accelerates (decelerates) then the world line will be a curve turning towards 
(away from) the $x$-axis. 
\begin{figure}
\includegraphics[scale=0.8]{ex1}
\caption{World lines}
\label{c1f1}
\end{figure}

\item Consider an experiment in an inertial frame in which light took time 
$\delta t$ to travel between points $(x, y, z)$ and $x + \delta x, y + 
\delta y, z + \delta z)$. The same experiment was observed from another inertial 
frame in which the light pulse travelled from $(x^\op, y^\op, z^\op)$ to $x^\op 
+ \delta x^\op, y^\op + \delta y^\op, z^\op + \delta z^\op)$ in time $\delta t^\op$.
Since the speed of light is the same in two frames,
\begin{eqnarray}
c^2\delta t^2 &=& \delta x^2 + \delta y^2 + \delta z^2 \label{c1e2} \\
c^2\delta {t^\op}^2 &=& \delta {x^\op}^2 + \delta {y^\op}^2 + \delta {z^\op}^2 \label{c1e3}
\end{eqnarray}
From these equations, we conclude that
\begin{equation}\label{c1e4}
c^2\delta t^2 - \delta x^2 - \delta y^2 - \delta z^2 = 
c^2\delta {t^\op}^2 - \delta {x^\op}^2 - \delta {y^\op}^2 - \delta {z^\op}^2
\end{equation}
This suggests that the quantity 
\begin{equation}\label{c1e5}
\delta s^2 = c^2\delta t^2 - \delta x^2 - \delta y^2 - \delta z^2
\end{equation}
is invariant across all inertial frame references. The quantity $\delta s$ is
called an interval between two world points.

\item If $\delta t = 0$ then equation \eqref{c1e4} suggests that $\delta {t^\op}^2$
need not be zero. Events simultaneous in $K$ need not be so in $K^\op$. Further, if
$\delta t = 0$ then $\delta s^2 < 0$. Intervals between world points for which 
$\delta s^2 < 0$ are called \emph{space like}. Two points separated by a space-
like interval will always have different ``space'' coordinates.

\item If $\delta x^2 + \delta y^2 + \delta z^2 = 0$ then $\delta s^2 > 0$. In this
case $\delta t$ can never be zero. Intervals between world points for which 
$\delta s^2 > 0$ are called \emph{time like}. Two points separated by a time-
like interval will always have different ``time'' coordinates.

\item It follows from the definitions of time-like and space-like intervals that
\begin{itemize}
\item If two world points are separated by a time-like interval then there exists
an inertial frame in which the events occur at the same space-point, that is, at
same values of $(x, y, z)$.
\item If two world points are separated by a space-like interval then there exists
an inertial frame in which the events occur at the same time.
\end{itemize}

\item An interval for which $\delta s = 0$ is called ``light-like''.

\item Since the nature of an interval depends on an invariant quantity like
$\delta s^2$, an interval that is space-like (or time-like or null-like) in one
inertial frame is space-like (or time-like or null-like) in all inertial frames.

\item A causal relationship between events can exists only their world-points
are separated by a time-like interval.

\item In the four-dimensional space with coordinates $x, y, z$ and $t$, the equation
\begin{equation}\label{c1e6}
c^2t^2 = x^2 + y^2 + z^2
\end{equation}
defines a double cone with origin as the vertex and the $t$ axis as its principle
axis. It is called the \emph{light cone}. 
World points inside the cone are separated by a time-like interval. World
points outside it are separated by space-like interval. If two world points $A$
and $B$ are inside the cone then in every inertial frame the difference between
their time coordinates will be non-zero. Likewise, if they were outside the cone
then in every inertial frame their space coordinates will not all be the same.
The part of the cone above (below) the origin is called the absolute future 
(past) of the origin.
\end{enumerate}

\chapter{Relativistic Mechanics}\label{c2}
\begin{enumerate}
\item In the previous chapter, we argued that the expression
\begin{equation}\label{c2e1}
\delta t = \frac{1}{c}\int_A^B ds
\end{equation}
has a maximum for a particle if it is stationary. In this expression, $A$ and 
$B$ are world points and 
\[
ds = \sqrt{dx^\nu dx_\nu}.
\]
It follows immediately that, among all particles that start their journey at 
$A$ and end it at $B$, the expression
\begin{equation}\label{c2e2}
-\frac{1}{c}\int_A^B ds,
\end{equation}
has a minimum for the particle that is stationary.

\item We also know that for a classical system, there exists an action integral
$S$ whose value is an extremum for a path followed by the system in the 
configuration space. It is a minimum over an infinitesiml length of the path.

\item If we have to extend this idea to relativistic mechanics then the integral
must be invariant under Lorentz transformation. Therefore, it must be a true scalar.
One example of such an integral is given by \eqref{c2e2}. We can mildly generalise
it to
\begin{equation}\label{c2e3}
S = -\alpha \int_A^B ds,
\end{equation}
where $\alpha$ is a constant. From the discussion around equations \eqref{c2e1}
and \eqref{c2e3}, we infer that $\alpha > 0$.

\item We would like to write the action integral for relativistic systems as
\begin{equation}\label{c2e4}
S = \int_{t_1}^{t_2}Ldt,
\end{equation}
in analogy for the classical systems. From equations \eqref{c2e3} and 
\eqref{c2e4}, we get
\begin{equation}\label{c2e5}
L = -\alpha c\sqrt{1 - \frac{v^2}{c^2}},
\end{equation}
where
\[
v^i = d{x^i}{t}
\]
is the 3-velocity of the particle. Can we guess $\alpha$?

\item As $v/c \rightarrow 0$, $L$ of \eqref{c2e5} should go over to the classical
Lagrangian for a free particle, which is just $mv^2/2$. For small $v/c$, we can
write \eqref{c2e5} as
\[
L = -\alpha c\left(1 - \frac{1}{2}\frac{v^2}{c^2}\right) = 
-\alpha c + \frac{\alpha}{2}\frac{v^2}{c}.
\]
A constant term in a Lagrangian can always be ignored and we infer that
\begin{equation}\label{c2e6}
\alpha = mc.
\end{equation}
Therefore, we guess the correct form of the Lagrangian for a free relativistic
particle to be
\begin{equation}\label{c2e7}
L = -mc^2\sqrt{1 - \frac{v^2}{c^2}}.
\end{equation}

\item The generalised momentum corresponding to this Lagrangian is
\begin{equation}\label{c2e8}
p^i = \pdt{L}{v^i} = (-mc^2)\frac{1}{2}\left(1 - \frac{v^2}{c^2}\right)^{-1/2}\frac{-2v^i}{c^2}
= \frac{mv^i}{\sqrt{1 - v^2/c^2}}.
\end{equation}
If
\begin{eqnarray}
\beta &=& \frac{v}{c} \\ \label{c2e9}
\gamma &=& \frac{1}{\sqrt{1 - \beta^2}} \label{c2e10}
\end{eqnarray}
then
\begin{equation}\label{c2e11}
\td{p^i}{t} = m\gamma \td{v^i}{t} + m\td{\gamma}{t}v^i.
\end{equation}
This equation is quite different from the one used in classical mechanics.

\item The energy of the particle is
\[
E = p^i v_i - L = mv^2\gamma + \frac{mc^2}{\gamma} = m\gamma\left(v^2 + \frac{c^2}{\gamma^2}\right)
\]
so that
\begin{equation}\label{c2e12}
E = m\gamma\left(v^2 + c^2\left(1 - \frac{v^2}{c^2}\right)\right) 
= mc^2\gamma
\end{equation}

\item From \eqref{c2e12} we see that the energy of the particle is not zero in
the relativistic framework even if the particle is resting. When $v = 0$, $\gamma
= 1$ and the particle's energy is
\begin{equation}\label{c2e13}
E = mc^2.
\end{equation}
For small velocities, we can approximate \eqref{c2e12} to
\begin{equation}\label{c2e14}
E = mc^2\left(1 + \frac{1}{2}\frac{v^2}{c^2} + O(\beta^4)\right) = 
mc^2 + \frac{1}{2}mv^2 + O(\beta^4).
\end{equation}

\item Since $\beta < 1$ for material bodies, $\gamma > 1$ and hence, from 
\eqref{c2e12} $E \ge mc^2 > 0$. Thus energy of a free particle is always positive.
This is also true in classical mechanics for an elementary particle, that is a
particle without internal degrees of freedom. For a composite body, in classical
mechanics, energy can be negative and is determined to within a constant. Howwever,
in the relativistic regime this is not so, $E > 0$ is always true.

\item For a composite body at rest, the energy $E = Mc^2$ is not the same as 
\[
\sum_{i=1}^N m_ic^2,
\]
where $m_i, i = 1, \ldots, N$ are the rest masses of the constituent particles.
The rest energy will also include the energy of interaction between the particles.
Since 
\[
M \ne \sum_{i=1}^N m_ic
\]
conservation of mass is not true in relativistic mechanics.

\item From \eqref{c2e8},
\[
p^2 = p_ip^i = mv^2\gamma^2
\]
and from \eqref{c2e12}
\[
E^2 = m^2c^4\gamma^2 \Rightarrow E^2 - E^2\beta^2 = m^2c^4 \Rightarrow 
E^2 - p^2c^2  = m^2c^4,
\]
so that
\begin{equation}\label{c2e15}
E^2 = p^2c^2 + m^2c^4.
\end{equation}
Since $E$ is expressed in terms of $p$, we can as well write the Hamiltonian as
\begin{equation}\label{c2e16}
H = c\sqrt{p^2 + m^2c^2}
\end{equation}

\item Since $p^i = mv^i\gamma$ and $E = mc^2\gamma$, we also have
\begin{equation}\label{c2e17}
p^i = \frac{E}{c^2}v^i.
\end{equation}
If $v \rightarrow c$, $\gamma \rightarrow \infty$. In this limit, both $E$ and
$p^i$ blow up unless $m = 0$. In that case, the two are related by
\begin{equation}\label{c2e18}
E = pc.
\end{equation}

\item From equations \eqref{c2e4} and \eqref{c2e6}, we can write the principle of
least action as
\begin{equation}\label{c2e19}
\delta S = -mc\delta\int_{t_1}^{t_2} ds = 0.
\end{equation}
Since $ds = \sqrt{dx^\mu dx_\mu}$, 
\begin{eqnarray*}
\delta ds &=& \frac{1}{2}\frac{(\delta dx^\mu)dx_\mu + dx^\mu (\delta dx_\mu)}{ds} \\
 &=& \frac{1}{2}\frac{(\delta dx^\mu)dx_\mu + dx_\mu (\delta dx^\mu)}{ds} \\
 &=& \frac{dx_\mu(\delta dx^\mu)}{ds} \\
 &=& u_\mu \delta(dx^\mu) \\
 &=& u_\mu d(\delta x^\mu),
\end{eqnarray*}
where we used \eqref{c1e78}. Thus, equation \eqref{c2e19} becomes
\begin{equation}\label{c2e20}
\delta S = -mc\int_{t_1}^{t_2}u_\mu d(\delta x^\mu)
\end{equation}
Integrating by parts,
\begin{equation}\label{c2e21}
\delta S = -mc u_\mu \delta x^\mu\Big|_{t_1}^{t_2} + 
mc\int_{t_1}^{t_2}\delta x^\mu \td{u_\mu}{s}ds.
\end{equation}
Since variations in trajectories vanish at the end points, the first term on the
right hand side is zero and hence
\begin{equation}\label{c2e22}
\delta S = mc\int_{t_1}^{t_2}\delta x^\mu \td{u_\mu}{s}ds.
\end{equation}
$\delta S = 0$ for all possible variations $\delta x^\mu$ therefore implies that 
for a free particle,
\begin{equation}\label{c2e23}
\td{u_\mu}{s} = 0.
\end{equation}
The particle travels with a uniform velocity, unsurprisingly.

\item Now fix $t_1$ and let $t_2$ be varied for true trajectories, that the ones
which occur in reality and for which \eqref{c2e23} is valid. Then we have from
\eqref{c2e21},
\begin{equation}\label{c2e24}
\delta S = -mcu_\mu \delta (x^\mu)_{t_2},
\end{equation}
so that
\begin{equation}\label{c2e25}
mcu_\mu = -\pdt{S}{x^\mu}.
\end{equation}
The 4-vector $mc^\mu$, from equations \eqref{c1e79} to \eqref{c1e82} is
\[
u^\mu = \left(\gamma, \frac{v^i}{c}\gamma\right)
\]
so that
\[
mcu^\mu = \left(mc\gamma, mv^i\gamma\right) = \left(\frac{E}{c}, p^i\right)
\]
We define the 4-momemtum $p^\mu$ as
\begin{equation}\label{c2e26}
p^\mu = \left(\frac{E}{c}, p^i\right)
\end{equation}
so that \eqref{c2e25} becomes
\begin{equation}\label{c2e27}
p_\mu = -\pdt{S}{x^\mu}.
\end{equation}
Note that if $p^\mu$ is given by \eqref{c2e26} then
\begin{equation}\label{c2e28}
p_\mu = \left(\frac{E}{c}, -p^i\right).
\end{equation}
The transformation equations for $p^\mu$ are given by
\begin{equation}\label{c2e29}
\frac{E}{c} = \gamma\left(\frac{\bar{E}}{c} + \beta\bar{p}^1\right);
p^1 = \gamma\left(\bar{p}^1 + \beta\frac{\bar{E}}{c}\right); p^2 = \bar{p}^2;
p^3 = \bar{p}^3.
\end{equation}
We also have
\begin{equation}\label{c2e30}
p^\mu p_\mu = \frac{E^2}{c^2} - p^2 = m^2c^2.
\end{equation}
We now define the 4-force as 
\begin{equation}\label{c2e31}
g^\mu = \td{p^\mu}{s}.
\end{equation}
Since $ds = \sqrt{c^2(dt)^2 - dx_i dx^i}$,
\begin{equation}\label{c2e32}
ds = cdt\sqrt{1 - \frac{v^2}{c^2}} = \frac{c}{\gamma}dt.
\end{equation}
Therefore,
\[
g^\mu = \frac{\gamma}{c}\left(\frac{1}{c}\td{E}{t}, \td{\vec{p}}{t}\right).
\]
Since $E = \vec{f}\cdot\vec{v}$, we can write this as
\begin{equation}\label{c2e33}
g^\mu = \left(\frac{\gamma}{c^2}\vec{f}\cdot\vec{v}, \frac{\gamma}{c}\td{\vec{p}}{t}\right).
\end{equation}

\item From equations \eqref{c2e27} and \eqref{c2e30},
\[
\left(-\pdt{S}{x^\mu}\right)\left(-\pdt{S}{x_\mu}\right) = m^2c^2
\]
or
\begin{equation}\label{c2e34}
\pdt{S}{x^\mu}\pdt{S}{x_\mu} = g^{\mu\nu}\pdt{S}{x^\mu}\pdt{S}{x^\nu} = m^2c^2
\end{equation}
is the relativistic Hamilton-Jacobi equation. Expanding the sum
\begin{equation}\label{c2e35}
\frac{1}{c^2}\left(\pdt{S}{t}\right)^2 - \left(\pdt{S}{x}\right)^2 
- \left(\pdt{S}{y}\right)^2 - \left(\pdt{S}{z}\right)^2 = m^2c^2.
\end{equation}
\end{enumerate}
\chapter{Charges in Electromagnetic Fields}\label{c3}
\begin{enumerate}
\item Unless they are in touch with each other, particles interact with each
other via fields.

\item Disturbances in fields propagate information. It is a postulate of the 
special theory of relativity that the changes in the state of a system are
progagated at a finite speed not exceeding $c$, the speed of light in vacuum.

\item This restriction rejects the hypothesis of an ideal rigid body in relativity.
For suppose a force is applied to one end of a rigid body. Then in order that the
inter-particle distance is unchanged, the force has to be propagated instantaneously
to the other end. Therefore, in elementary analysis of the theory of relativity
all particles are assumed to be points.

\item The principle of action for a free particle is given by equation \eqref{c2e4}
and \eqref{c2e7},
\begin{equation}\label{c3e1}
S = -\int_{t_1}^{t_2}mcds.
\end{equation}
For a particle in an electromagnetic field, the right hand side of this equation
should be augmented with another terms that represents the interaction of the
particle with the field. It should comprise of quantities describing the particle
as well as the field.

The particle's interaction with the field is described by a single quantity, its
charge $e$ and the field by its 4-potential $A_\mu$. The complete action is
\begin{equation}\label{c3e2}
S = -\int_{t_1}^{t_2} \left(mcds + \frac{e}{c}A_\mu dx^\mu\right).
\end{equation}

The 4-potential can be written as 
\begin{equation}\label{c3e3}
A_\mu = (\varphi, \vec{A})
\end{equation}
so that we can write \eqref{c3e2} as
\begin{equation}\label{c3e4}
S = \int_{t_1}^{t_2}\left(-mcds - e\varphi dt + 
\frac{e}{c}\vec{A}\cdot d\vec{x}\right).
\end{equation}
Since $ds = c\sqrt{1 - \beta^2}dt$ and $d\vec{x} = \vec{v}dt$,
\begin{equation}\label{c3e5}
S = \int_{t_1}^{t_2}\left(-mc^2\sqrt{1 - \beta^2} - e\varphi + 
\frac{e}{c}\vec{A}\cdot\vec{v}\right)dt.
\end{equation}
Thus, the Lagrangian for a particle in an electromagnetic field is
\begin{equation}\label{c3e6}
L = -mc^2\sqrt{1 - \beta^2} - e\varphi + \frac{e}{c}\vec{A}\cdot\vec{v}
= -mc^2\sqrt{1 - \beta^2} - e\varphi + \frac{e}{c}A^i v_i.
\end{equation}
From the Lagrangian, we get the generalised momentum
\begin{equation}\label{c3e7}
p^i = \pdt{L}{v^i} = m\gamma v^i + \frac{e}{c}A^i.
\end{equation}
From equation \eqref{c2e8}, the first term is just the generalised momentum for
the Lagrangian of \eqref{c2e7}. The Hamiltonian is given be the Legendre
transform
\begin{eqnarray*}
H &=& p^iv_i - L \\
  &=& m\gamma v^iv_i + \frac{e}{c}A^iv_i + mc^2\sqrt{1 - \beta^2} + e\varphi - 
  \frac{e}{c}A^iv_i \\
  &=& \frac{mv^2}{\sqrt{1 - \beta^2}} + mc^2\sqrt{1 - \beta^2} + e\varphi \\
  &=& \frac{mc^2}{\sqrt{1 - \beta^2}} + e\varphi
\end{eqnarray*}
or that
\begin{equation}\label{c3e8}
H = mc^2\gamma + e\varphi.
\end{equation}
Stricly speaking \eqref{c3e9} is not the correct form of the Hamiltonian because 
it is expressed in terms of the generalised velocity. To remedy this defect, we 
write
\eqref{c3e9} as
\[
\frac{(H - e\varphi)^2}{c^2} = m^2c^2 \gamma^2.
\]
Now,
\begin{equation}\label{c3e9}
\gamma^2 = \frac{1}{1 - \beta^2} = 1 + \frac{\beta^2}{1 - \beta^2} = 
1 + \beta^2\gamma^2
\end{equation}
so that
\[
\frac{(H - e\varphi)^2}{c^2} = m^2c^2(1 + \beta^2\gamma^2) 
\]
or that
\[
\frac{(H - e\varphi)^2}{c^2} = m^2c^2 + m^2\gamma^2 v^2,
\]
From \eqref{c3e7} the last term on the rhs can be written as
\begin{equation}\label{c3e10}
\frac{(H - e\varphi)^2}{c^2} = m^2c^2 + \left(p^i - \frac{e}{c}A^i\right)
\left(p_i - \frac{e}{c}A_i\right)
\end{equation}
Thus,
\begin{equation}\label{c3e11}
H = e\varphi + \sqrt{m^2c^4 + c^2 
\left(p^i - \frac{e}{c}A^i\right)\left(p_i - \frac{e}{c}A_i\right)}
\end{equation}
is the correct form of the Hamiltonian. For speeds small as compared to $c$,
\[
\sqrt{1 - \beta^2} = 1 - \frac{1}{2}\beta^2
\]
so that the Lagrangian of \eqref{c3e6} can be written as
\begin{equation}\label{c3e12}
L = \frac{1}{2}mv^2 - e\varphi + \frac{e}{c}\vec{A}\cdot\vec{v},
\end{equation}
where we ignored the constant terms $-mc^2$. At non-relativistic speeds, $\gamma$
can be well-approximated by $1$ so that the generalised momentum is
\begin{equation}\label{c3e13}
\vec{p} = m\vec{v} + \frac{e}{c}\vec{A}
\end{equation}
and the Hamiltonian is
\begin{equation}\label{c3e14}
H = \frac{1}{2m}\left(\vec{p} - \frac{e}{c}\vec{A}\right)^2 + e\varphi.
\end{equation}

From \eqref{c2e27},
\[
p^\nu = g^{\mu\nu}p_\mu = -g^{\mu\nu}\pdt{S}{x^\mu}
\]
so that
\[
p^0 = -\frac{1}{c}\pdt{S}{t}, \vec{p} = \grad S
\]
and hence the Hamilton-Jacobi equation is, from \eqref{c3e10},
\[
\frac{1}{c^2}\left(\pdt{S}{t} + e\varphi\right)^2 = m^2c^2 + 
\left(\grad S - \frac{e}{c}\vec{A}\right)^2
\]
where we replaced $H$ in \eqref{c3e10} with $-(1/c)\partial_t S$ because it is
the energy of the particle. We can reexpress the relativistic Hamilton-Jacobi
equation as 
\begin{equation}\label{c3e15}
\left(\grad S - \frac{e}{c}\vec{A}\right)^2 - 
\frac{1}{c^2}\left(\pdt{S}{t} + e\varphi\right)^2 + m^2c^2 = 0.
\end{equation}

\item The Lagrangian of \eqref{c3e12} allows us to derive the equation of motion.
\begin{eqnarray*}
\grad L &=& -e\grad\varphi + \frac{e}{c}(\vec{A}\cdot\grad\vec{v} + 
 \vec{v}\cdot\grad\vec{A} + \vec{A}\times\curl\vec{v} + \vec{v}\times\curl\vec{A})\\ 
 &=& -e\grad\varphi + \frac{e}{c}(\vec{v}\cdot\grad\vec{A} + \vec{v}\times\curl\vec{A})\\ 
\grad_{\vec{v}}L &=& m\vec{v} + \frac{e}{c}\vec{A}.
\end{eqnarray*}
The equation for $\grad L$ simplifies because $\vec{v}$ is independent of $\vec{x}$.
\begin{equation}\label{c3e16}
\frac{d}{dt}\left(m\vec{v} + \frac{e}{c}\vec{A}\right) = 
-e\grad\varphi + \frac{e}{c}(\vec{v}\cdot\grad\vec{A} + \vec{v}\times\curl\vec{A})
\end{equation}
is the equation of motion. Now.
\begin{equation}\label{c3e17}
\td{\vec{A}}{t} = \pdt{\vec{A}}{t} + \vec{v}\cdot\vec{A}
\end{equation}
so that \eqref{c3e16} is simplified to
\begin{equation}\label{c3e18}
\td{\vec{p}}{t} = e\left(-\grad\varphi - \frac{1}{c}\pdt{\vec{A}}{t}\right)
+ \frac{e}{c}\vec{v}\times\curl\vec{A}
\end{equation}
We define
\begin{eqnarray}
\vec{E} &=& -\grad\varphi - \frac{1}{c}\pdt{\vec{A}}{t} \label{c3e19} \\
\vec{B} &=& \curl\vec{A} \label{c3e20}
\end{eqnarray}
and call $\vec{E}$ as the electric field and $\vec{B}$ as the magnetic field. The
quantity
\begin{equation}\label{c3e21}
\vec{F} = e\vec{E} + \frac{\vec{v}}{c}\times\vec{B}
\end{equation}
is called the Lorentz force.

\item Recall that 
\begin{equation}\label{c3e22}
p^\mu = mcu^\mu = (mc\gamma, mv^i\gamma) = \left(\frac{E}{c}, \vec{p}\right)
\end{equation}
so that the energy of the particle is
\begin{equation}\label{c3e23}
E = mc^2\gamma
\end{equation}
so that
\begin{equation}\label{c3e24}
\td{E}{t} = mc^2\td{\gamma}{t} = 
-\frac{mc^2}{2}\gamma^3\left(\frac{-2\vec{v}}{c^2}\cdot\td{\vec{v}}{t}\right)
= m\gamma^3 \vec{v}\cdot\td{\vec{v}}{t}
\end{equation}
and
\[
\td{\vec{p}}{t} = m\gamma\td{\vec{v}}{t} + m\gamma^3\frac{\vec{v}}{c^2}\vec{v}\cdot\td{\vec{v}}{t} = 
m\gamma^3\left(m\gamma^{-2}\td{\vec{v}}{t} + \frac{\vec{v}}{c^2}\vec{v}\cdot\td{\vec{v}}{t}\right)
\]
or, since
\[
\gamma = \frac{1}{\sqrt{1 - v^2/c^2}},
\]
we have,
\[
\td{\vec{p}}{t} = m\gamma^3\left(\left(1 - \frac{v^2}{c^2}\right)\td{\vec{v}}{t} + 
\frac{\vec{v}}{c^2}\vec{v}\cdot\td{\vec{v}}{t}\right)
\]
Therefore,
\begin{equation}\label{c3e25}
\vec{v}\cdot\td{\vec{p}}{t} = m\gamma^3\left(\left(1 - \frac{v^2}{c^2}\right)\vec{v}\cdot\td{\vec{v}}{t}
+ \frac{v^2}{c^2}\vec{v}\cdot\td{\vec{v}}{t}\right) = m\gamma^3\vec{v}\cdot\td{\vec{v}}{t}
\end{equation}
From equations \eqref{c3e24} and \eqref{c2e25},
\begin{equation}\label{c3e26}
\td{E}{t} = \vec{v}\cdot\td{\vec{p}}{t}.
\end{equation}
If we apply it to the equation of motion \eqref{c3e18} and use the definitions 
of the fields, we get
\begin{equation}\label{c3e27}
\td{E}{t} = e\vec{v}\cdot\vec{E}.
\end{equation}

Here we have used the fact that
\begin{equation}\label{c3e28}
\td{v^2}{t} = \td{(\vec{v}\cdot\vec{v})}{t^2} = 2\vec{v}\cdot\td{\vec{v}}{t}.
\end{equation}

\item The equation of motion
\begin{equation}\label{c3e29}
\td{\vec{p}}{t} = -e\vec{E} + \frac{\vec{v}}{c}\times\vec{B}
\end{equation}
is invariant under time reversal if we also reverse the direction of the magnetic
field.
\end{enumerate}


\chapter{The Electromagnetic Field Equations}\label{c4}
\begin{enumerate}
\item From equations \eqref{c3e19} and \eqref{c3e20} it follows that
\begin{eqnarray}
\curl\vec{E} &=& -\frac{1}{c}\pdt{\vec{H}}{t} \label{c4e1} \\
\dive\vec{H} &=& 0 \label{c4e2}
\end{eqnarray}
These equations can be easily cast in integral form as 
\begin{eqnarray}
\oint\vec{E}\cdot d\vec{l} &=& -\frac{1}{c}\frac{\partial}{\partial t}\int\vec{H}\cdot d\vec{f} \label{c4e3} \\
\oint\vec{H}\cdot d\vec{f} &=& 0 \label{c4e4}
\end{eqnarray}
Note that, in equation \eqref{c4e3}, the surface integral on the rhs is over the
surface bounded by the countour along which the line integral of lhs is calculated.
The line integral is over a closed curve but the surface integral is not over a 
closed surface. The equation states that the circulation of the electric field
is equal to $-1/c$ times the flux of $\vec{H}$ through the surface bound by the
countour. On the other hand, \eqref{c4e4} tells that flux of $\vec{H}$ through
any closed surface is zero.

\item Equations \eqref{c4e1} and \eqref{c4e2} form the first pair of Maxwell
equations. They do not describe the electromagnetic field completely because
they involve the time derivative of $\vec{H}$ alone and not $\vec{E}$.

\item It is possible to write equations \eqref{c4e1} and \eqref{c4e2} in terms
of the electromagnetic field tensor $F_{\mu\nu}$. Since
\[
F_{\mu\nu} = \pdt{A_\nu}{x^\mu} - \pdt{A_\mu}{x^\nu},
\]
we have
\begin{eqnarray*}
\pdt{F_{\mu\nu}}{x^\rho} &=& \pdt{A^2_\nu}{x^\mu x^\rho} - \pdt{A^2_\mu}{x^\nu x^\rho} \\
\pdt{F_{\nu\rho}}{x^\mu} &=& \pdt{A^2_\rho}{x^\nu x^\mu} - \pdt{A^2_\nu}{x^\rho x^\mu} \\
\pdt{F_{\rho\mu}}{x^\nu} &=& \pdt{A^2_\mu}{x^\rho x^\nu} - \pdt{A^2_\rho}{x^\mu x^\nu}
\end{eqnarray*}
from which we get
\begin{equation}\label{c4e5}
\pdt{F_{\mu\nu}}{x^\rho} + \pdt{F_{\nu\rho}}{x^\mu} + \pdt{F_{\rho\mu}}{x^\nu} = 0.
\end{equation}

\item We will examine \eqref{c4e5} in greater details.
\begin{enumerate}
\item $\mu = 0, \nu = 1, \rho = 2$:
\[
\pdt{F_{01}}{x^2} + \pdt{F_{12}}{x^0} + \pdt{F_{20}}{x^1} = 0 \Rightarrow
\pdt{E_y}{x} - \pdt{E_x}{y} = - \frac{1}{c}\pdt{H_z}{t}.
\]

\item $\mu = 0, \nu = 1, \rho = 3$:
\[
\pdt{F_{01}}{x^3} + \pdt{F_{13}}{x^0} + \pdt{F_{30}}{x^1} = 0 \Rightarrow
\pdt{E_x}{z} - \pdt{E_z}{x} = - \frac{1}{c}\pdt{H_y}{t}.
\]

\item $\mu = 0, \nu = 2, \rho = 3$:
\[
\pdt{F_{02}}{x^3} + \pdt{F_{23}}{x^0} + \pdt{F_{30}}{x^2} = 0 \Rightarrow
\pdt{E_z}{y} - \pdt{E_y}{z} = - \frac{1}{c}\pdt{H_x}{t}.
\]

\item $\mu = 1, \nu = 2, \rho = 3$:
\[
\pdt{F_{12}}{x^3} + \pdt{F_{23}}{x^1} + \pdt{F_{31}}{x^2} = 0 \Rightarrow
-\pdt{H_z}{z} - \pdt{H_x}{x} - \pdt{H_y}{y} = 0.
\]
\end{enumerate}
Thus, \eqref{c4e5} encodes the two Maxwell equations \eqref{c4e1} and \eqref{c4e2}.
We also note that
\begin{enumerate}
\item If any two indices in \eqref{c4e5} are equal then we get the identity $0=0$.
\item If all three indices are equal then each term on lhs of \eqref{c4e5} is
zero.
\item Let us examine what happens when we consider other combinations of indices.
If we swap the values of $\mu$ and $\nu$ in case (a), that is, if $\mu = 1, \nu = 0,
\rho = 2$ then we have
\[
\pdt{F_{10}}{x^2} + \pdt{F_{02}}{x^1} + \pdt{F_{21}}{x^0} = 0 \Rightarrow
-\pdt{E_x}{y} + \pdt{E_y}{x} - \frac{1}{c}\pdt{H_z}{t} = 0,
\]
which is same as the conclusion of case (a). We can similarly show that all swaps
give 
\end{enumerate}
\end{enumerate}
\chapter{Constant electromagnetic fields}\label{c5}
\begin{enumerate}
\item If the fields are independent of time then they obey
\begin{eqnarray}
\dive\vec{E} &=& 4\pi\rho \label{c5e1} \\
\curl\vec{E} &=& 0 \label{c5e2} \\
\dive\vec{H} &=& 0 \label{c5e3} \\
\curl\vec{H} &=& 4\pi\frac{\vec{J}}{c}. \label{c5e4}
\end{eqnarray}
The symmetry of the situation reveals itself if we note the correspondence $\vec{E}
\mapsto \vec{H}$ and $\rho \mapsto \vec{J}/c$ and $\dive \mapsto \curl$. Equations
\eqref{c5e2} and \eqref{c5e3} suggest that
\begin{eqnarray}
\vec{E} &=& -\grad\phi \label{c5e5} \\
\vec{H} &=& \curl\vec{A} \label{c5e6}
\end{eqnarray}
which is understandable if we note that only the time component of $J^\mu$ 
determines $\vec{E}$ and its space component determines $\vec{H}$. Therefore,
only the time component of $\vec{A}^\mu$ suffices to describe the electric
field and its space component suffices to describe the magnetic field.

\item From equations \eqref{c5e1} and \eqref{c5e5} we have
\begin{equation}\label{c5e7}
\nabla^2\phi = -4\pi\rho.
\end{equation}
In region where there are no charges, the potential satisfies Laplace's equation
\begin{equation}\label{c5e8}
\nabla^2\phi = 0.
\end{equation}
This equation immediately implies that the potential function cannot have a local
minimum or a maximum. The nature of extremum of a function of three variables is
determined by its Hessian matrix. It is a minimum (maximum) if the Hessian matrix
is positive (negative) definite. Since the trace of a matrix is the sum of its
eigenvalues, we have an equivalent condition that the trace of the Hessian matrix
is positive (negative) for an extremum to be a minimum (maximum). But the trace
of Hessian is $\nabla^2\phi$ evaluated at the extremum. If $\phi$ is a solution
of Laplace's equation \eqref{c5e8}, then the Hessian can never be positive or
negative definite.

\item The field of a point charge can be derived using Gauss' law. Let the point
charge be at $\vec{r}^\op$. In order to find the field at $\vec{r}$, we consider
a sphere centred at $\vec{r}^\op$ and passing through $\vec{r}$. As the space is
isotropic, we expect the magnitude of $\vec{E}$ to be the same at all points on the
sphere and its direction along $\vec{r} - \vec{r}^\op$. Therefore,
\[
4\pi |\vec{r} - \vec{r}^\op|^2 = q
\]
or
\begin{equation}\label{c5e9}
\vec{E} = \frac{q}{|\vec{r} - \vec{r}^\op|^3}(\vec{r} - \vec{r}^\op).
\end{equation}
If we define
\begin{equation}\label{c5e10}
\vec{R} = \vec{r} - \vec{r}^\op
\end{equation}
then we can write \eqref{c5e9} simply as
\begin{equation}\label{c5e11}
\vec{E} = \frac{q}{R^3}\vec{R} = \frac{q}{R^2}\uv{R}.
\end{equation}
Equation \eqref{c5e11} is \emph{Coulomb's law}. The potential of this field is
\begin{equation}\label{c5e12}
\phi = \frac{q}{R}.
\end{equation}
If there are discrete, point charges at $\vec{r}_a$ then the field due to all of 
them is
\begin{equation}\label{c5e13}
\vec{E} = \sum_a \frac{q_a}{R_a^2}\uv{R_a},
\end{equation}
where $\vec{R}_a = \vec{r}^\op - \vec{r}$. The potential is
\begin{equation}\label{c5e14}
\phi = \sum_a\frac{q_a}{R_a}.
\end{equation}
For a continuum of charges, \eqref{c5e11} and \eqref{c5e12} generalise to
\begin{eqnarray}
\phi(\vec{r}) &=& \int\frac{\rho(\vec{r}^\op)}{R}dV \label{c5e15} \\
\vec{E}(\vec{r}) &=& \int\frac{\rho(\vec{r}^\op)}{R^2}\uv{R}dV  \label{c5e16}
\end{eqnarray}

\item For a point charge, $\rho = q\delta(\vec{R})$. In this case, \eqref{c5e7}
becomes,
\[
\nabla^2\phi = 4\pi q\delta(\vec{R}).
\]
From \eqref{c5e12}, we also have
\begin{equation}\label{c5e17}
\nabla^2\left(\frac{1}{R}\right) = -4\pi\delta(\vec{R}).
\end{equation}

\item In the absence of magnetic field, the energy density is
\begin{equation}\label{c5e18}
W = \frac{1}{8\pi}\int E^2dV.
\end{equation}
From \eqref{c5e5},
\[
W = -\frac{1}{8\pi}\int\vec{E}\cdot\grad\phi dV.
\]
Since $\dive(\phi\vec{E}) = \grad\phi\cdot\vec{E} + \phi\dive\vec{E}$, we have
\begin{eqnarray*}
W &=& \frac{1}{8\pi}\int\phi\dive\vec{E}dV - \frac{1}{8\pi}\int\dive(\phi\vec{E})dV \\
  &=& \frac{1}{8\pi}\int\phi\dive\vec{E}dV - \frac{1}{8\pi}\oint \phi\vec{E}\cdot d\vec{f}.
\end{eqnarray*}
The potential goes as $O(r^{-1})$ and the field as $O(r^{-2})$ so that the surface
integral goes as $O(r^{-1})$. If we choose a large enough surface the second 
integral can be made as small as we desire. Thus, we have
\[
W = \frac{1}{8\pi}\int\phi\dive\vec{E}dV
\]
or using \eqref{c5e1},
\begin{equation}\label{c5e19}
W = \frac{1}{2}\int\rho\phi dV.
\end{equation}
If the charge density consists od point charges, equation \eqref{c5e19} becomes
\begin{equation}\label{c5e20}
W = \sum_a q_a\phi(\vec{r}_a),
\end{equation}
where $\phi(\vec{r}_a)$ is the potential due to all charges at the point 
$\vec{r}_a$ where the charge $q_a$ is located. This formula immediately runs 
into a tricky situation.

\item The potential $\phi$ due to a charge $q$ at $\vec{r}^\op$ at the same point
is infinity according to \eqref{c5e12}. As a result, from \eqref{c5e20}, a point
charge has infinite energy and therefore an infinite mass. To avoid such an absurd
result, we say that the laws of classical electrodynamics are not applicable at
extremely short distances. In fact, we cannot even ask the question if the mass of
a charge has electrodynamic origin, that is, it exists because of electrodynamic
energy.

The energy of a uniformly charged sphere of radius $s$ can be computed as follow.
We first note that its charge density is
\begin{equation}\label{c5e21}
\rho = \frac{Q}{\frac{4\pi}{3}s^3}.
\end{equation}
The field due to a sphere of radius $r$ and a uniform charge density $\rho$ at 
points on and outside it is as if the entire charge was concentrated at its 
centre. If $q(r)$ is the charge in a sphere of radius $r$ then the energy
of a charge $dq$ at a distance $r$ from it is
\[
dU = \frac{qdq}{r},
\]
where 
\[
q = \frac{4\pi}{3}r^3 \rho
\]
and $dq = 4\pi r^2 dr$. Thus,
\[
dU = \frac{16\pi^2}{3}\rho^2 r^4dr
\]
and the energy of the entire sphere is
\begin{equation}\label{c5e22}
U = \int_0^s dU = \frac{3}{5}\frac{Q^2}{s}.
\end{equation}
If we consider an elementary charge $q$ to have an infinitesimal radius $r_0$ and
mass $m$ then if the mass has electrodynamic origin,
\[
mc^2 = \frac{3}{5}\frac{q^2}{s}.
\]
Ignoring the factor of $3/5$, the ``classical radius'' of the charge $q$ is defined
to be
\begin{equation}\label{c5e23}
s = \frac{q}{m} c^2.
\end{equation}

\item Quantum effects become important at distance of the order of $\hslash/mc$. The
ratio of this distance to the classical radius of an electron is
\[
\frac{\hslash}{mc}\frac{mc^2}{e^2} = \frac{\hslash c}{e^2} = \frac{1}{\alpha}
\approx 137,
\]
where $\alpha$ is the fine-structure constant. Thus, quantum effects must be taken
into account at distances much larger than the ``classical radius'' of the electron. 
The lower limit of the applicability of classical electrodynamics is much larger 
than the classical electron radius.

\item To avoid the tricky questions of self-energy of elementary charges, we write
the potential $\phi(\vec{r}_a)$ in equation \eqref{c5e20} as
\begin{equation}\label{c5e24}
\phi(\vec{r}_a) = \sum_{b \ne a} \frac{q_b}{R_{ab}},
\end{equation}
where $R_{ab} = |\vec{r}_a - \vec{r}_b|$.

\item We next consider the field of a charge $q$ moving with a uniform velocity
$\vec{v}$. We align the axes such that $\vec{v} = v\uv{x}$ and call the frame
moving with the charge $K^\op$. Let $K$ be the laboratory frame in with axes 
parallel to those of $K^\op$ and let their origins coincide at $t = 0$. Let $P$
be a field point with coordinates $(x^\op, y^\op, z^\op)$ in $K^\op$. Then the
scalar potential at $P$ in $K^\op$ is
\begin{equation}\label{c5e25}
\phi^\op = \frac{q}{R^\op},
\end{equation}
where 
\begin{equation}\label{c5e26}
R^\op = \sqrt{{x^\op}^2 + {y^\op}^2 + {z^\op}^2}
\end{equation}
and the vector potential is
\begin{equation}\label{c5e27}
\vec{A}^\op = 0.
\end{equation}
Thus,
\begin{equation}\label{c5e28}
{A^\op}^\mu = \left(\frac{q}{R^\op}, 0, 0, 0\right).
\end{equation}
From \eqref{c1e46}, the 4-potential in $K$ is
\begin{equation}\label{c5e29}
A^\mu = \left(\gamma\frac{q}{R^\op}, \beta\gamma\frac{q}{R^\op}, 0, 0\right),
\end{equation}
where
\begin{eqnarray*}
\vec{\beta} &=& \frac{\vec{v}}{c} \\
\gamma &=& \frac{1}{\sqrt{1 - \beta^2}}.
\end{eqnarray*}
However, \eqref{c5e29} is not a correct expression because we continue to have
$R^\op$ in it. The relation between $(x, y, z)$ and $(x^\op, y^\op, z^\op)$ is
given the Lorentz transformation
\[
x^\op = \gamma(x - vt) \;;\; y^\op = y \;;\; z^\op = z
\]
so that
\begin{equation}\label{c5e30}
{R^\op}^2 = \gamma^2(x - vt)^2 + (y^2 + z^2).
\end{equation}
We define a function $R^\ast$ of the coordinates $x, y, z$ as
\begin{equation}\label{c5e31}
R^\ast(x, y, z) = \sqrt{(x - vt)^2 + (1 - \beta^2)(y^2 + z^2)}
\end{equation}
so that
\begin{equation}\label{c5e32}
\frac{R^\op(x^\op, y^\op, z^\op)}{\gamma} = R^\ast(x, y, z)
\end{equation}
and the partial Lorentz transformation of \eqref{c5e29} can be completed to
\begin{equation}\label{c5e33}
A^\mu = \left(\frac{q}{R^\ast}, \beta\frac{q}{R^\ast}, 0, 0\right)
\end{equation}
or
\begin{eqnarray}
\phi(x, y, z) &=& \frac{q}{R^\ast} \label{c5e34} \\
\vec{A}(x, y, z) &=& \frac{q\vec{v}}{cR^\ast} \label{c5e35}
\end{eqnarray}

\item The electric and magnetic fields in $K^\op$ frame are
\begin{eqnarray}
\vec{E}^\op &=& \frac{q}{{R^\op}^3}\vec{R}^\op \label{c5e36} \\
\vec{H}^\op &=& 0 \label{c5e37}
\end{eqnarray}
We use equations \eqref{c3e81} to \eqref{c3e83} to get the electric field in the
$K$ frame.
\begin{eqnarray}
E_x &=& E_x^\op \label{c5e38} \\
E_y &=& \gamma E_y^\op \label{c5e39} \\
E_z &=& \gamma E_z^\op \label{c5e40}
\end{eqnarray}
Using \eqref{c5e36} we get
\begin{eqnarray}
E_x &=& \frac{q}{{R^\op}^3}x^\op \label{c5e41} \\
E_y &=& \gamma\frac{q}{{R^\op}^3}y^\op \label{c5e42} \\
E_z &=& \gamma\frac{q}{{R^\op}^3}z^\op \label{c5e43}
\end{eqnarray}
These equations are not satisfactory because their rhs are written in terms of
the primed coordinates. Using $R^\op = \gamma R^\ast$ from \eqref{c5e32} and the
Lorentz transformation formulae
\begin{eqnarray}
E_x &=& \frac{q}{{R^\ast}^3}\frac{x - vt}{\gamma^2} \label{c5e44} \\
E_y &=& \frac{q}{{R^\ast}^3}\frac{y}{\gamma^2} \label{c5e45} \\
E_z &=& \frac{q}{{R^\ast}^3}\frac{z}{\gamma^2} \label{c5e46}
\end{eqnarray}
so that
\begin{equation}\label{c5e47}
\vec{E} = (1 - \beta^2)\frac{q}{{R^\ast}^3}\vec{R},
\end{equation}
where
\begin{equation}\label{c5e48}
\vec{R} = (x - vt)\uv{x} + y\uv{y} + z\uv{z}.
\end{equation}
If $\theta$ is the angle between $\vec{R}$ and $\vec{v}$ then
\begin{equation}\label{c5e49}
\vec{v}\cdot\vec{R} = v(x - vt) = vR\cos\theta \Rightarrow x - vt = R\cos\theta
\end{equation}
so that,
\begin{equation}\label{c5e50}
y^2 + z^2 = R^2 - (x - vt)^2 = R^2\sin^2\theta.
\end{equation}
From \eqref{c5e31}, \eqref{c5e49} and \eqref{c5e50} we get
\begin{equation}\label{c5e51}
{R^\ast}^2 = R^2\cos^2\theta + (1 - \beta^2)R^2\sin^2\theta = 
R^2(1 - \beta^2\sin^2\theta).
\end{equation}
We can now write \eqref{c5e47} solely in terms of $\vec{R}$ as
\begin{equation}\label{c5e52}
\vec{E} = \frac{q\vec{R}}{R^3} \frac{1 - \beta^2}{(1 - \beta^2\sin^2\theta)^{3/2}}.
\end{equation}
Note that $\vec{R}$ is the position of the (moving) charge in $K$ frame.

\item Lorentz transformation of $\vec{E}$ and $\vec{H}$ are a consequence of the
transformation properties of the field tensor. One should use \eqref{c3e81} to
\eqref{c3e84} instead of evaluating $-\grad\phi$ with $\phi$ given by \eqref{c5e34}.
That is, although
\[
\phi^\op = \frac{q}{R^\op} \mapsto \phi = \frac{q}{R^\ast},
\]
the mapping
\[
-\grad^\op\phi^\op \mapsto -\grad\phi
\]
does not hold.

\item In \eqref{c5e52}, $\theta$ is the angle between the radius vector of the
field point $\vec{R}$ and the direction of motion. The magnitude of the field
is
\begin{equation}\label{c5e53}
E = \frac{q}{R^2}\frac{1 - \beta^2}{(1 - \beta^2\sin^2\theta)^{3/2}}
\end{equation}
Figure \ref{c5f1} shows how the magnitude of the electric field varies with 
$\theta$ for different values of $\beta$. As the speed of the charge approaches
$c$, the electric field gets increasingly concentrated along the directions
perpendicular to the motion of the particle.
\begin{figure}[!ht]
\includegraphics[scale=0.8]{ex2}
\caption{$E(\theta)$}
\label{c5f1}
\end{figure}
For an observer in the $K$ frame, the charge in motion creates an electric current
and therefore expects to observe magnetic field. In fact, equations \eqref{c3e84}
to \eqref{c3e86} give
\begin{eqnarray}
H_x &=& 0 \label{c5e54} \\
H_y &=& -\gamma\beta E^\op_z \label{c5e55} \\
H_z &=& \gamma\beta E^\op_y \label{c5e56}
\end{eqnarray}
Since $\beta = v/c = v_x/c$, using \eqref{c5e39} and \eqref{c5e40}, we get
\begin{eqnarray}
H_x &=& 0 \label{c5e57} \\
H_y &=& -\frac{1}{c}v_xE_z \label{c5e58} \\
H_z &=& \frac{1}{c}v_xE_y. \label{c5e59}
\end{eqnarray}
These three equations can be combined as
\begin{equation}\label{c5e60}
\vec{H} = \frac{1}{c}\vec{v} \times \vec{E} = \vec{\beta} \times \vec{E}.
\end{equation}

\item We now consider the motion of a charge $q$ in the electric field produced
by another one $Q$. Let their masses by $m$ and $M$ respectively. Assume that
$m \ll M$ so that the charge $Q$ is almost stationary. The problem now reduces to
studying the motion of $q$ in a potential $\phi = Q/r$.

The energy of the charged particle is
\begin{equation}\label{c5e61}
\mathcal{E} = \sqrt{p^2c^2 + m^2c^4} + \frac{\alpha}{r},
\end{equation}
where
\begin{equation}\label{c5e62}
\alpha = qQ.
\end{equation}
The motion in a central field is confined to a plane. Align the coordinate axes
such that $Q$ is at the origin and the motion happens in $xy$ plane. Furthermore,
since the torque on the particle is zero, its angular momentum is a constant of
motion. The particle's velocity in polar coordinates is
\begin{equation}\label{c5e63}
\vec{v} = \dot{r}\uv{r} + r\dot{\theta}\uv{\theta}
\end{equation}
so that $v^2 = \dot{r}^2 + r^2\dot{\theta}^2$. Since $\vec{L} = \vec{r} \times
\vec{p} = m\gamma\vec{r} \times \vec{p}$, (using \eqref{c2e8}) we have $L = 
\gamma mr^2\dot{\theta}$. We can express $v^2$ in terms of $L$ as
\[
v^2 = \dot{r}^2 + \frac{L^2}{m^2\gamma^2 r^2}
\]
so that
\begin{equation}\label{c5e64}
p^2 = m^2\gamma^2\dot{r}^2 + \frac{L^2}{r^2} = p_r^2 + \frac{L^2}{r^2}.
\end{equation}
Therefore, \eqref{c5e61} becomes
\begin{equation}\label{c5e65}
\mathcal{E} = c\sqrt{p_r^2 + \frac{L^2}{r^2} + m^2c^2} + \frac{\alpha}{r}.
\end{equation}

\item If $\alpha > 0$ then as $r$ increases, the rhs of \eqref{c5e64} also
increases even if $p_r \rightarrow 0$. Since $\mathcal{E}$ is a constant of
motion, a drop in $r$ can be compensated by a drop in $p_r$ only to a limited
extent. This prevents the two charges from coming arbitrarily close to each 
other.

To analyse the condition $\alpha < 0$, write \eqref{c5e65} as
\[
\mathcal{E} = \frac{Lc}{r}\sqrt{\frac{r^2(p_r^2 + m^2c^2)}{L^2} + 1} - \frac{|\alpha|}{r}.
\]
so that we can approximate
\[
\mathcal{E} \approx \frac{Lc}{r} + \frac{cr}{2L}(p_r^2 + m^2c^2) - \frac{|\alpha|}{r}.
\]
for small $r$. If $Lc > |\alpha|$ then the rhs blows up as $r \rightarrow 0$. 
Therefore, to enforce constancy of $\mathcal{E}$, $r$ cannot be permitted to
decrease indefinitely. If $Lc < |\alpha|$, if $p_r = m\gamma\dot{r}$ can become
indefinitely large, $r$ may be allowed to become arbitrarily small while still
keeping $\mathcal{E}$ constant.

In the non-relativistic case, we would have had
\[
\mathcal{E} = \frac{1}{2}mv_r^2 + \frac{L^2}{2mr^2} - \frac{|\alpha|}{r}.
\]
If $L \ne 0$ then the first two terms, which are always positive, rise much faster
than the third term falls. Therefore, energy conservation does not allow the two
charges to come arbitrarily close to each other. On the other hand, if $L = 0$, 
then $v_r$ can rise enough to compensate the drop in the third term and still
keep  $\mathcal{E}$ constant. Thus, the two charges can come arbitrarily close only
if they approach to each other head-on.

\item The Hamilton-Jacobi equation \eqref{c3e15} for this problem is
\[
(\grad S)^2 - \frac{1}{c^2}\left(\pdt{S}{t} + \frac{\alpha}{r}\right)^2 + m^2c^2 = 0.
\]
Since $Q$ is stationary, $q$ experiences only the electric field and therefore 
$\vec{A} = 0$. The gradient in plane-polar coordinates is
\begin{equation}\label{c5e66}
\grad S = \pdt{S}{r}\uv{r} + \frac{1}{r}\pdt{S}{\phi}
\end{equation}
so that the Hamilton-Jacobi equation becomes
\begin{equation}\label{c5e67}
\left(\pdt{S}{r}\right)^2 + \frac{1}{r^2}\left(\pdt{S}{\phi}\right)^2 - \frac{1}{c^2}
\left(\pdt{S}{t} + \frac{\alpha}{r}\right)^2 + m^2c^2 = 0.
\end{equation}
or
\[
\left(\pdt{S}{r}\right)^2 + \frac{1}{r^2}\left(\pdt{S}{\phi}\right)^2 - \frac{1}{c^2}
\left(\pdt{S}{t}\right)^2 + 2\frac{\alpha}{rc^2}\pdt{S}{t} + \frac{\alpha^2}{c^2r^2}
+ m^2c^2 = 0.
\]
This is a non-linear, first-order pde. We try a solution of the form
\begin{equation}\label{c5e68}
S(r, \phi, t) = -\mathcal{E}t + L\phi + f(r)
\end{equation}
to get
\[
{f^\op(r)}^2 + \frac{L^2}{r^2} - \frac{\mathcal{E}^2}{c^2} - 2\frac{\alpha\mathcal{E}}{rc^2}
+ \frac{\alpha^2}{c^2r^2} + m^2c^2 = 0.
\]
or
\[
f^\op = \pm\sqrt{\frac{1}{c^2}\left(\mathcal{E} - \frac{\alpha}{r}\right)^2 - \frac{L^2}{r^2} 
-m^2c^2}
\]
The solution, thus, is
\begin{equation}\label{c5e69}
S(r, \phi, t) = -\mathcal{E}t + L\phi \pm 
\frac{1}{c}\int\sqrt{\left(\mathcal{E}-\frac{\alpha}{r}\right)^2-\frac{L^2c^2}{r^2} -m^2c^4}dr
\end{equation}
The trajectories are given by (find out why)
\begin{equation}\label{c5e70}
\pdt{S}{L} = \text{constant.}
\end{equation}
Differentiating \eqref{c5e69},
\begin{equation}\label{c5e71}
\pdt{S}{L} = \phi \mp Lc\int\frac{dr}{r\sqrt{A^2r^2 - 2Br + C^2}},
\end{equation}
where
\begin{eqnarray}
A^2 &=& \mathcal{E}^2 - m^2c^4 \label{c5e72} \\
B   &=& \alpha\mathcal{E} \label{c5e73} \\
C^2 &=& \alpha^2 - L^2c^2 \label{c5e74}
\end{eqnarray}
We now use the result
\begin{equation}\label{c5e75}
\int\frac{dx}{x\sqrt{a^2x^2 - 2bx + c^2}} = 
-\frac{1}{c}\tanh^{-1}\left(\frac{c^2 - bx}{c\sqrt{a^2x^2 - 2bx + c^2}}\right) +
\text{const}.
\end{equation}
in equation \eqref{c5e71} to get
\begin{equation}\label{c5e76}
\phi_0 = \phi \pm \frac{Lc}{C}\tanh^{-1}\left(\frac{C^2 - Br}{C\sqrt{A^2r^2 - 2Br + C^2}}\right),
\end{equation}
where $\phi_0$ is a constant of integration.

We consider the following cases:
\begin{enumerate}
\item $\alpha^2 > L^2c^2$, that is $C$ is real. Then \eqref{c5e71} can be written as
\[
\tanh\left(\frac{C}{Lc}(\phi_0 - \phi)\right) = \frac{C^2 - Br}{C\sqrt{A^2r^2 - 2Br + C^2}}.
\]
Let
\begin{equation}\label{c5e77}
X = \frac{C}{Lc}(\phi_0 - \phi) = \sqrt{\frac{\alpha^2}{L^2c^2} - 1}(\phi_0 - \phi)
\end{equation}
so that
\begin{eqnarray*}
\tanh^2 X &=& \frac{(C^2 - Br)^2}{C^2(A^2r^2 - 2Br + C^2)} \\
C^2(A^2r^2 + C^2 - 2Br)\tanh^2X &=& C^4 - 2BC^2r + B^2r^2 \\
C^2A^2r^2\tanh^2X &=& (C^4 - 2BC^2r)\sech^2X + B^2r^2 \\
C^2A^2r^2\sinh^2X &=& C^2(C^2 - 2B^2r) + B^2r^2\cosh^2X \\
C^2A^2r^2(\cosh^2X - 1) &=& C^2(C^2 - 2B^2r) + B^2r^2\cosh^2X \\
(C^2A^2 - B^2)r^2\cosh^2X &=& C^2(A^2r^2 - 2B^2r + C^2)
\end{eqnarray*}
Now,
\[
A^2C^2-B^2 = (L^2c^2 -\alpha^2)m^2c^4 -\mathcal{E}^2L^2c^2
\]
Since we assumes $\alpha^2 > L^2c^2$, the rhs of the above equation is negative.
Therefore, before taking the square-root, we flip the signs to get
\begin{eqnarray*}
(B^2 - C^2A^2)r^2\cosh^2X &=& C^2(2B^2r - A^2r^2 - C^2) \\
\sqrt{B^2 - C^2A^2}\cosh X &=& \frac{C^2}{r}\sqrt{2\frac{B^2}{C^2}r - \frac{A^2}{C^2}r^2 - 1}
\end{eqnarray*}
Thus,
\[
\pm c\sqrt{(L\mathcal{E})^2 + m^2c^2(\alpha^2 - L^2c^2)}\cosh X = 
\frac{C^2}{r}\sqrt{2\frac{B^2}{C^2}r - \frac{A^2}{C^2}r^2 - 1}
\]
Since $\cosh$ is an even function,
\begin{eqnarray}
\pm c\sqrt{(L\mathcal{E})^2 + m^2c^2(\alpha^2 - L^2c^2)}
\cosh\left((\phi - \phi_0)\sqrt{\frac{\alpha^2}{L^2c^2} - 1}\right) &=& \nonumber\\
\frac{C^2}{r}\sqrt{2\frac{B^2}{C^2}r - \frac{A^2}{C^2}r^2 - 1} \label{c5e78}
\end{eqnarray}
We can choose $\phi_0$ such that this equation reduces to
\begin{equation}\label{c5e79}
\pm c\sqrt{(L\mathcal{E})^2 + m^2c^2(\alpha^2 - L^2c^2)}
\cosh\left((\phi - \phi_0)\sqrt{\frac{\alpha^2}{L^2c^2} - 1}\right) = \frac{C}{r}.
\end{equation}
This can always be done because \eqref{c5e78} is of the form
\begin{equation}\label{c5e80}
A\cosh(B(x_0-x)) = Cf(x).
\end{equation}
We can write it as
\begin{equation}\label{c5e81}
x_0 = x + \frac{1}{B}\cosh^{-1}\left(\frac{Cf(x)}{A}\right).
\end{equation}
Choose $x_0$ such that if $f(x^\op) = 1$ then
\begin{equation}\label{c5e82}
x_0 = x^\op + \frac{1}{B}\cosh^{-1}\left(\frac{C}{A}\right).
\end{equation}
\end{enumerate}

\end{enumerate}
\chapter{Electromagnetic Waves}\label{c6}
\begin{enumerate}
\item Maxwell equations in vacuum are:
\begin{eqnarray}
\dive\vec{E} &=& 0 \label{c6e1} \\
\curl\vec{E} &=& -\frac{1}{c}\pdt{\vec{H}}{t} \label{c6e2} \\
\dive\vec{H} &=& 0 \label{c6e3} \\
\curl\vec{H} &=& \frac{1}{c}\pdt{\vec{E}}{t} \label{c6e4}
\end{eqnarray}
If the fields are also time independent, then we will have $\dive\vec{E} = 0,
\curl\vec{E} = 0, \dive\vec{H} = 0$ and $\curl\vec{H} = 0$. From equations
\eqref{c5e16} (Coulomb's law) and \eqref{c5e140} (Biot-Savart's law), we get
$\vec{E} = 0$ and $\tav{H} = 0$. Therefore, we assume that the fields depend
on time.

\item We will soon show that equations \eqref{c6e1} to \eqref{c6e4} have a
non-zero solution in vacuum. That is, electromagnetic fields can exist in vacuum.
We will also show that they satisfy the wave equation. Solutions of Maxwell
equations in vacuum are called electromagnetic waves.

\item As usual, it is easier to analyse the potentials. Let us choose them such
that $\varphi = 0$ and $\dive\vec{A} = 0$. Equation \eqref{c6e2} becomes, after
expressing the electric field in terms of the potentials,
\[
-\frac{1}{c}\curl\pdt{\vec{A}}{t} = 
-\frac{1}{c}\frac{\partial}{\partial t}\curl\vec{A},
\]
an identity. While, equation \eqref{c6e4} becomes
\[
\curl\curl\vec{A} = -\frac{1}{c^2}\spdt{\vec{A}}{t}.
\]
The lhs of this equation is $\grad\dive\vec{A} - \nabla^2\vec{A}$. Since we chose
$\dive{\vec{A}} = 0$, we get
\begin{equation}\label{c6e5}
\nabla^2\vec{A} - \frac{1}{c^2}\spdt{\vec{A}}{t} = 0.
\end{equation}
The choice $\dive{\vec{A}} = 0$ is called the Coulomb gauge and that it can be
exercised without violating physics is explained in the point leading to 
\eqref{c5e134}. Taking the curl of this equation gives
\begin{equation}\label{c6e6}
\nabla^2\vec{H} - \frac{1}{c^2}\spdt{\vec{H}}{t} = 0.
\end{equation}
Taking a partial derivative of \eqref{c6e5} and recalling that we chose $\varphi
= 0$, gives
\begin{equation}\label{c6e7}
\nabla^2\vec{E} - \frac{1}{c^2}\spdt{\vec{E}}{t} = 0.
\end{equation}

\item We can repeat this analysis using the Maxwell equations written in terms of
the electromagnetic field tensor. From \eqref{c4e37},
\[
\pdt{F^{\mu\nu}}{x^\nu} = -\frac{4\pi}{c}j^\mu.
\]
In the absence of sources, it becomes
\[
\pdt{F^{\mu\nu}}{x^\nu} = 0.
\]
We now use the definition 
\[
F^{\mu\nu} = \pdt{A^\nu}{x_\mu} - \pdt{A^\mu}{x_\nu}
\]
to get
\[
\frac{\partial}{\partial x^\nu}\pdt{A^\nu}{x_\mu} - 
\frac{\partial}{\partial x^\mu}\pdt{A^\mu}{x_\nu} = 0
\Rightarrow
\frac{\partial}{\partial x_\mu}\pdt{A^\nu}{x^\nu} - 
\frac{\partial}{\partial x^\mu}\pdt{A^\mu}{x_\nu} = 0.
\]
We now choose $A^\mu$ such that
\begin{equation}\label{c6e8}
\pdt{A^\nu}{x^\nu} = 0.
\end{equation}
In terms of coordinates, it is
\begin{equation}\label{c6e9}
\frac{1}{c}\pdt{\varphi}{t} - \dive\vec{A} = 0.
\end{equation}
Equations \eqref{c6e8} and \eqref{c6e9} are called the \emph{Lorenz gauge}. 
Unlike the condition of Coulomb gauge, that of the Lorenz gauge is invariant 
under Lorentz transformation. We are thus left with
\begin{equation}\label{c6e10}
\frac{\partial}{\partial x^\mu}\pdt{A^\mu}{x_\nu} = 
g^{\mu\rho}\frac{\partial}{\partial x^\nu}\pdt{A^\mu}{x^\rho} = 0.
\end{equation}
It is equivalent to \eqref{c6e5}.

\item The nine equations \eqref{c6e5}, \eqref{c6e6} and \eqref{c6e7} are all of
the form
\[
\nabla^2 f - \frac{1}{c^2}\spdt{f}{t} = 0 \Rightarrow 
\spdt{f}{t} - c^2\nabla^2 f = 0,
\]
where $f$ is one of the components of $\vec{A}, \vec{E}$ or $\vec{H}$. If the 
function $f$ depends only on $x$ and $t$, that is, if it is independent of $y$ 
and $z$, then the solution $f$ is called a plane wave. In this case, the 
equation simplifies to
\begin{equation}\label{c6e11}
\spdt{f}{t} - c^2\spdt{f}{x} = 0.
\end{equation}
To solve this equation, write it as
\begin{equation}\label{c6e12}
\left(\frac{\partial}{\partial t} - c\frac{\partial}{\partial x}\right)
\left(\frac{\partial}{\partial t} + c\frac{\partial}{\partial x}\right)f = 0.
\end{equation}
Introduce the variables
\begin{eqnarray}
\xi &=& t - \frac{x}{c} \label{c6e13} \\
\eta &=& t + \frac{x}{c} \label{c6e14}
\end{eqnarray}
so that
\begin{eqnarray}
\frac{\partial}{\partial t} &=& \frac{\partial}{\partial\xi} = 
 \frac{\partial}{\partial\eta} \label{c6e15} \\
\frac{\partial}{\partial x} &=& -\frac{1}{c}\frac{\partial}{\partial\xi} =
\frac{1}{c}\frac{\partial}{\partial\eta} \label{c6e16}
\end{eqnarray}
so that
\begin{eqnarray}
\frac{\partial}{\partial\eta} &=& \frac{1}{2}
	\left(\frac{\partial}{\partial t} + c\frac{\partial}{\partial x}\right) 
	\label{c6e17} \\
\frac{\partial}{\partial\xi} &=& \frac{1}{2}
	\left(\frac{\partial}{\partial t} - c\frac{\partial}{\partial x}\right) 
	\label{c6e18}
\end{eqnarray}
This pair of relations allows us to write \eqref{c6e12} as
\begin{equation}\label{c6e19}
\frac{\partial^2 f}{\partial\xi\partial\eta} = 0.
\end{equation}
Its solution can be written as
\begin{equation}\label{c6e20}
f(\xi, \eta) = f_1(\xi) + f_2(\eta).
\end{equation}
In terms of the original variables,
\begin{equation}\label{c6e21}
f(x, t) = f_1\left(t - \frac{x}{c}\right) + f_2\left(t + \frac{x}{c}\right).
\end{equation}

\item If $f_2 = 0$ then
\[
f(x, t) = f_1\left(t - \frac{x}{c}\right)
\]
Any reasonable function of $t - x/c$ is a solution of \eqref{c6e11}. For a fixed
$x$, the value of the function changes with $t$. Likewise, for a fixed $t$, the
function changes value with $x$. Further the value of the function is the same
for a given value of its argument, namely, $t - x/c$ or $ct - x$. If we focus 
our attention on one such value, we will see it travelling along the $x$ axis 
with a speed $x/t = c$, the speed of light.

If $f_1 = 0$ and $f = f_2$ then a similar interpretation is applicable except
that a point with a certain value of $f$ travels along the negative $x$ axis
with the speed $c$.

\item How do we use this information to derive the forms of $\vec{E}$ and 
$\vec{H}$ for plane waves? We will start by deriving an expression for the vector
potential. Since the components of $\vec{A}$ individually satisfy \eqref{c6e11}, 
none of them will be functions of $y$ and $z$. Further, $\dive\vec{A} = 0$ 
implies that
\[
\pdt{A_x}{x} = 0.
\]
Therefore,
\[
\spdt{A_x}{t} - c^2\spdt{A_x}{x} = 0 \Rightarrow \spdt{A_x}{t} = 0
\Rightarrow \pdt{A_x}{t} = \text{const.}
\]
This, in turn, implies that $E_x$ is a constant. If this constant is non-zero
then it results in a constant electric field in the direction of propagation
of the wave, also called the longitudnal direction. Being a constant, it is
not of the form $f(t \pm x/c)$ and therefore unrelated to the electromagnetic
wave. We can ignore it by setting $A_x = 0$. We are thus left with
\begin{equation}\label{c6e22}
\vec{A} = A_y(x)\uv{y} + A_z(x)\uv{z}
\end{equation}
and hence
\begin{eqnarray}
\vec{E} &=& -\frac{1}{c}\pdt{\vec{A}}{t} \label{c6e23} \\
\vec{H} &=& \curl\vec{A} \label{c6e24}
\end{eqnarray}
From equation \eqref{c6e15},
\[
\pdt{\vec{A}}{t} = \pdt{\vec{A}}{\xi}
\]
so that if we denote rhs of the above equation as $\vec{A}^\op$ then equation 
\eqref{c6e23} becomes
\begin{equation}\label{c6e25}
\vec{E} = -\frac{1}{c}\vec{A}^\op.
\end{equation}
Now,
\begin{equation}\label{c6e26}
\curl\vec{A} = -\uv{y}\pdt{A_z}{x} + \uv{z}\pdt{A_y}{x} = 
\frac{1}{c}\uv{y}\pdt{A_z}{\xi} - \frac{1}{c}\uv{z}\pdt{A_y}{\xi}
\end{equation}
where we used \eqref{c6e16}. If $\un$ is the direction of wave propagation then
$\un = \uv{x}$ and 
\begin{equation}\label{c6e27}
\un\times\vec{A}^\op = \uv{z}A^\op_y - \uv{y}A^\op_z
\end{equation}
and hence from \eqref{c6e26} and \eqref{c6e27} we get
\begin{equation}\label{c6e28}
\vec{H} = -\frac{1}{c}\un\times\vec{A}^\op.
\end{equation}
From \eqref{c6e25} and \eqref{c6e28} it immediately follows that
\begin{equation}\label{c6e29}
\vec{H} = \un\times\vec{E}.
\end{equation}
Thus both fields are perpendicular to $\un$, the direction of propagation. As a 
result, the plane electromagnetic waves in vacuum are transverse in nature. 
Equation \eqref{c6e29} also assures that magnitude of the two fields is the same.

\item The energy flux is
\begin{equation}\label{c6e30}
\vec{S} = \frac{c}{4\pi}\vec{E}\times\vec{H} = 
\frac{c}{4\pi}\vec{E}\times(\un\times\vec{E}) = 
\frac{c}{4\pi}E^2\un = \frac{c}{4\pi}H^2\un.
\end{equation}
We can as well write it as 
\begin{equation}\label{c6e31}
2W = \frac{c}{4\pi}(E^2 + H^2)\un
\end{equation}
so that
\begin{equation}\label{c6e32}
\vec{S} = cW\un
\end{equation}
where we used \eqref{c4e50}, where $W$ where is the energy density of the field.
The components of $\vec{S}/c$ are the components 
\[
T^{01}, T^{02}, T^{03}
\]
of the energy-momentum tensor \eqref{c4e90}. We also argued in point 23 of 
chapter \ref{c4} that $T^{01}/c, T^{02}/c, T^{03}/c$ are the components of 
momentum density of the system. Thus, we can interpret $\vec{S}/c^2$ as the 
momentum flux in the system. From \eqref{c6e32} we see that electromagnetic waves
carry momentum in the direction of propagation.

The relation between energy density and momentum density of electromagentic 
waves is similar to that of relativistic particle of zero mass.

\item If the $x$-axis is chosen to be along the direction of propagation of
the wave then $\vec{E} = E\uv{y}$ and $\vec{H} = H\uv{z}$. The components of the
energy momentum tensor evaluated below \eqref{c4e88} are
\begin{eqnarray}
T^{00} &=& \frac{E^2 + H^2}{8\pi} \label{c6e33} \\
T^{11} &=& -\frac{E^2 + H^2}{8\pi} = -\sigma_{xx} \label{c6e34},
\end{eqnarray}
All other components vanish.

\item Lorentz transformation for $W$ follows from \eqref{c1e88}.
\begin{equation}\label{c6e35}
W = \frac{W^\op + 2\beta S^\op_x/c + \beta^2T_{11}^\op}{1 - \beta^2}
 = \frac{W^\op + 2\beta S^\op_x/c - \beta^2\sigma_{xx}^\op}{1 - \beta^2}.
\end{equation}
Let $\alpha^\op$ is the angle between the $x^\op$ and the direction of 
propagation of the wave $\un$. Since $\vec{\beta}$ is along the $x^\op$ axis,
$\alpha^\op$ is also the angle between $\vec{\beta}$ and $\un$. Therefore,
using \eqref{c6e32}, we get
\[
S_x^\op = cW^\op\cos\alpha^\op
\]
and 
\begin{equation}\label{c6e36}
\sigma_{xx}^\op = -W^\op\cos^2\alpha^\op,
\end{equation}
so that
\begin{equation}\label{c6e37}
W = \frac{W^\op(1 + \beta\cos\alpha^\op)^2}{1 - \beta^2}.
\end{equation}

\item A wave whose time dependence is of the form $\cos(\omega t + \alpha)$, 
where $\omega$ and $\alpha$ are constants is called a mono-chromatic wave. If 
$f(\vec{r}, t)$ is the field quantity and if its time dependence is 
mono-chromatic then
\begin{equation}\label{c6e38}
\nabla^2 f = -\frac{\omega^2}{c^2} f.
\end{equation}
If the wave is propagating along the positive $x$ axis then $f$ is a function of
$t - x/c$ alone. If it is also mono-chromatic then,
\[
f(x, t) = f_0\cos\left(\omega t - \omega\frac{x}{c} + \alpha\right) = 
\re\left\{f_0\exp\left(-i\left(\omega t - \omega\frac{x}{c} + 
\alpha\right)\right)\right\}.
\]
The constant $\alpha$ can be absorbed in $f_0$ so that we can as well write
\begin{equation}\label{c6e39}
f(x, t) = \re\left\{f_0\exp\left(-i\left(\omega t - 
\omega\frac{x}{c}\right)\right)\right\}.
\end{equation}
Now,
\[
\cos\left(\omega t - \frac{\omega}{c}
\left(x + \frac{2\pi c}{\omega}\right)\right)
= \cos\left(\omega t - \frac{\omega}{c}x + 2\pi\right) = 
\cos\left(\omega t - \frac{\omega}{c}x\right)
\]
so that the quantity
\begin{equation}\label{c6e40}
\lambda = \frac{2\pi c}{\omega}
\end{equation}
is called the \emph{wavelength}. The vector,
\begin{equation}\label{c6e41}
\vec{k} = \frac{\omega}{c}\un
\end{equation}
is called the \emph{wave-vector}. In terms of $\vec{k}$ we can generalise 
\eqref{c6e39} to
\begin{equation}\label{c6e42}
f(\vec{r}, t) = \re\left\{f_0\exp\left(i\left(\vec{k}\cdot\vec{r} - 
\omega t\right)\right)\right\}.
\end{equation}
If we perform only linear operations, we can write \eqref{c6e42} as
\begin{equation}\label{c6e43}
f(\vec{r}, t) = f_0e^{i(\vec{k}\cdot\vec{r} - \omega t)}
\end{equation}
with the understanding that only the real part is physically significant.

\item If, for a plane, mono-chromatic wave,
\begin{equation}\label{c6e44}
\vec{A}(\vec{r}, t) = \vec{A}_0e^{i(\vec{k}\cdot\vec{r} - \omega t)}
\end{equation}
then
\begin{equation}\label{c6e45}
\vec{E} = -\frac{1}{c}\pdt{\vec{A}}{t} = i\frac{\omega}{c}\vec{A} = ik\vec{A},
\end{equation}
where we used \eqref{c6e41} in the last step. Further,
\begin{equation}\label{c6e46}
\vec{H} = \curl\vec{A} = i\vec{k}\times\vec{A}.
\end{equation}

\item Let 
\begin{equation}\label{c6e47}
\vec{E} = \vec{E}_0 e^{i(\vec{k}\cdot\vec{r} - \omega t)},
\end{equation}
where
\begin{equation}\label{c6e48}
\vec{E}_0 = \vec{b}e^{-i\alpha}
\end{equation}
and
\begin{equation}\label{c6e49}
\vec{b} = \vec{b}_1 + i\vec{b}_2,
\end{equation}
so that $\vec{E}_0$ is a complex vector and $\vec{b}_1, \vec{b}_2$ are real 
vectors. If we insist that
\[
|\vec{E}_0|^2 = \vec{E}_0\cdot\vec{E}_0^\ast = \vec{b}\cdot\vec{b}
\]
is real then
\[
\vec{b}\cdot\vec{b} = \vec{b}_1\cdot\vec{b}_1 - \vec{b}_2\cdot\vec{b}_2 + 
2i\vec{b}_1\cdot\vec{b}_2
\]
must be real, which requires us to have
\begin{equation}\label{c6e50}
\vec{b}_1\cdot\vec{b}_2 = 0.
\end{equation}
If the wave propagates along the $x$ direction, we can choose $\vec{b}_1$ to be
along the $y$ direction and $\vec{b}_2$ along the $z$ direction. We can then 
write
\begin{eqnarray}
E_y &=& b_1\cos(\vec{k}\cdot\vec{r} - \omega t - \alpha) \label{c6e51} \\
E_z &=& \pm b_2\sin(\vec{k}\cdot\vec{r} - \omega t - \alpha) \label{c6e52}
\end{eqnarray}
There is a $\sin$ factor in the second expression because of $i$ as a multiple
in the second term of \eqref{c6e49}. We can combine \eqref{c6e51} and 
\eqref{c6e52} to
\begin{equation}\label{c6e53}
\frac{E_y^2}{b_1^2} + \frac{E_z^2}{b_2^2} = 1.
\end{equation}
The vector $\vec{E} = \vec{E}_y\uv{y} + \vec{E}_z\uv{z}$ thus lies on the 
ellipse defined by \eqref{c6e53}. As time goes by, the tip of the vector moves 
along the ellipse by following equations \eqref{c6e51} and \eqref{c6e52}. Such 
a wave is called \emph{elliptically polarised}. It is called \emph{circularly 
polarised} if $b_1 = b_2$ and plane polarised of one of $b_1$ or $b_2$ is zero.

\item If we introduce a 4-vector
\begin{equation}\label{c6e54}
k^\mu = \left(\frac{\omega}{c}, \vec{k}\right)
\end{equation}
then $x^\mu = (ct, \vec{r})$ implies
\begin{equation}\label{c6e55}
k_\mu x^\mu = \left(\frac{\omega}{c}, -\vec{k}\right)\cdot(ct, \vec{r})
= \omega t - \vec{k}\cdot\vec{r}.
\end{equation}
This allows us to write the solution \eqref{c6e44} of the wave equation as 
\begin{equation}\label{c6e56}
\vec{A} = \vec{A}_0\exp(ik_\mu x^\mu).
\end{equation}
We also observe that
\begin{equation}\label{c6e57}
k_\mu k^\mu = \frac{\omega^2}{c^2} - k^2 = 0,
\end{equation}
by \eqref{c6e41}. In the pseudo-Euclidean geometry of the space-time a norm of a
vector can be zero without the vector being zero.

\item The energy-momentum tensor of the electromagnetic field of a plane wave 
has the components,
\begin{eqnarray*}
T^{00} &=& \frac{E^2 + H^2}{8\pi} = W \\
T^{01} &=& \frac{E_yH_z - E_zH_y}{4\pi} = k^2A^2\\
T^{02} &=& \frac{E_zH_x - E_xH_z}{4\pi} = 0 \\
T^{03} &=& \frac{E_xH_y - E_yH_x}{4\pi} = 0 \\
T^{11} &=& \frac{E_x^2 - E_y^2 - E_z^2 + H_x^2 - H_y^2 - H_z^2}{8\pi} = W \\
T^{12} &=& -\frac{E_xE_y + H_xH_y}{4\pi} = 0\\
T^{13} &=& -\frac{E_xE_z + H_xH_z}{4\pi} = 0\\
T^{22} &=& \frac{-E_x^2 + E_y^2 - E_z^2 - H_x^2 + H_y^2 - H_z^2}{8\pi} = 0\\
T^{23} &=& -\frac{E_yE_z + H_yH_z}{4\pi} = 0 \\
T^{33} &=& \frac{-E_x^2 - E_y^2 + E_z^2 - H_x^2 - H_y^2 + H_z^2}{8\pi} = 0
\end{eqnarray*}
where we used the fact that $\vec{k} = k\uv{x}$, $\vec{H} = 
i\vec{k}\times\vec{A} = i(-kA_z\uv{y} + kA_y\uv{z})$, $\vec{E} = 
\un\times\vec{H} = \uv{x}\times\vec{H}
= i(-kA_z\uv{z} - kA_y\uv{y})$, for a plane wave. Since, for these expressions,
\[
E^2 + H^2 = 2k^2A^2,
\]
we can write
\begin{equation}\label{c6e58}
T^{01} = k^2A^2 = \frac{E^2 + H^2}{8\pi} = W.
\end{equation}
Since $\vec{k} = k\uv{x}$, $k^\mu = (\omega/c, k, 0, 0)$ and
\begin{equation}\label{c6e59}
T^{00} = T^{01} = T^{11} = W \Rightarrow T^{\mu\nu} = 
\frac{Wc^2}{\omega^2}k^\mu k^\nu
\end{equation}

\item The Lorentz transformation for $k^\mu$ is, according to \eqref{c1e46},
\begin{eqnarray}
k^0 &=& \gamma(\bar{k}^0 + \beta\bar{k}^1) \label{c6e60} \\
k^1 &=& \gamma(\bar{k}^1 + \beta\bar{k}^0) \label{c6e61} \\
k^2 &=& \bar{k}^2 \label{c6e62} \\
k^3 &=& \bar{k}^3 \label{c6e63}
\end{eqnarray}
where $\bar{k}^\mu$ is the wave vector measured in a frame $\bar{K}$ moving
at velocity $v\uv{x}/c$ with respect to $K$. From \eqref{c6e54}, $k^0=\omega/c$
so that \eqref{c6e60} becomes
\begin{equation}\label{c6e64}
\omega = \gamma(\bar{\omega} + v\bar{k}^1)
\end{equation}
If the wave propagates at an angle $\alpha$ with respect to the $x$ axis in 
$\bar{K}$ frame then $\bar{k}^1 = k\cos\alpha$. Using \eqref{c6e57}, we have
\begin{equation}\label{c6e65}
\bar{k}^1 = \frac{\bar{\omega}}{c}\cos\alpha
\end{equation}
so that \eqref{c6e64} becomes
\[
\omega = \gamma(\bar{\omega} + \beta\bar{\omega}\cos\alpha) = 
\gamma\bar{\omega}(1 + \beta\cos\alpha)
\]
or
\begin{equation}\label{c6e66}
\bar{\omega} = \frac{\omega\sqrt{1 - \beta^2}}{1 + \beta\cos\alpha}.
\end{equation}
If $\alpha = \pi/2$, $\bar{\omega} = \omega\sqrt{1 - \beta^2} < \omega$. This 
is the relativistic red-shift. On the other hand, if 
\[
\alpha > \cos^{-1}(\sqrt{1 - \beta^2} - 1)
\]
then $\bar{\omega} > \omega$ and we have the relativistic blue shift.

\item The spectral resolution of a wave is an expression of the wave field as a
superposition of monochromatic waves. It can be done in two ways:
\begin{enumerate}
\item Express the periodic function as
\begin{equation}\label{c6e67}
f(t) = \sum_{n=-\infty}^\infty f_n e^{-i\omega_0 nt},
\end{equation}
where $\omega_0 = 2\pi/T$ and $T$ is such that $f(t + T) = f(t)$. The 
superposition is written as a sum of waves of frequency $\omega_0$ and its 
harmonics. We can get the amplitudes $f_n$ using
\begin{equation}\label{c6e68}
f_n = \frac{1}{T}\int_{-T/2}^{T/2} f(t)e^{i\omega_0 nt},
\end{equation}
from which it is evident that
\begin{equation}\label{c6e69}
f_n = f_{-n}.
\end{equation}
The squared modulus of \eqref{c6e67} is
\begin{equation}\label{c6e70}
|f|^2 = \sum_{m, n = -\infty}^\infty f_n f_m^\ast e^{i\omega_0(m - n)t}.
\end{equation}
Since,
\begin{equation}\label{c6e71}
\int_{-T/2}^{T/2}e^{i\omega_0(m - n)t}dt = T\delta_{mn},
\end{equation},
\begin{equation}\label{c6e71a}
\overline{|f|}^2 = \frac{1}{T}\int_{-T/2}^{T/2}|f|^2 dt = 
\sum_{m, n = -\infty}^\infty f_n f_m^\ast \delta_{mn} = 
\sum_{n=-\infty}^\infty |f_n|^2.
\end{equation}
In view of \eqref{c6e69} we also have
\begin{equation}\label{c6e72}
\overline{|f|}^2 = 2\sum_{n=0}^\infty |f_n|^2.
\end{equation}

\item If $f(t) \rightarrow 0$ as $t \rightarrow \infty$, one can also write
\begin{equation}\label{c6e73}
f(t) = \int_{-\infty}^\infty \hat{f}(\omega) e^{-i\omega t}\frac{d\omega}{2\pi}.
\end{equation}
The inverse of this relation is
\begin{equation}\label{c6e74}
\hat{f}(\omega) = \int_{-\infty}^\infty f(t)e^{i\omega t}dt.
\end{equation}
From \eqref{c6e74} it is immediately evident that
\begin{equation}\label{c6e75}
\hat{f}^\ast(\omega) = \hat{f}(-\omega).
\end{equation}
From \eqref{c6e74},
\[
|f(t)|^2 = \frac{1}{4\pi^2}
\iint_{-\infty}^\infty \hat{f}(\omega)\hat{f}^\ast(\omega^\op)
e^{-i(\omega-\omega^\op)t}d\omega d\omega^\op
\]
and
\[
\int_{-\infty}^\infty |f(t)|^2dt = \frac{1}{4\pi^2}
\int_{-\infty}^\infty\iint_{-\infty}^\infty \hat{f}(\omega)
\hat{f}^\ast(\omega^\op)e^{-i(\omega-\omega^\op)t}d\omega d\omega^\op dt
\]
Since
\begin{equation}\label{c6e77}
\frac{1}{2\pi}\int_{-\infty}^\infty e^{-i(\omega - \omega^\op)t}dt = 
\delta(\omega - \omega^\op)
\end{equation}
we get
\[
\int_{-\infty}^\infty |f(t)|^2dt = \frac{1}{2\pi}
\iint_{-\infty}^\infty \hat{f}(\omega)\hat{f}^\ast(\omega^\op)
\delta(\omega-\omega^\op)d\omega d\omega^\op
\]
or
\begin{equation}\label{c6e78}
\int_{-\infty}^\infty |f(t)|^2dt = 
\frac{1}{2\pi}\int_{-\infty}^\infty |\hat{f}(\omega)|^2 d\omega.
\end{equation}
\end{enumerate}

\item A purely monochromatic wave extends all the way to infinity. There are no
such waves. Almost monochromatic waves have a frequency in a small band around a
certain value, say $\omega$. Unlike the pure monochromatic wave, whose amplitude
is a constant $\vec{E}_0$, the amplitude of an approximate monochromatic wave
is a slow varying quantity $\vec{E}_0(t)$. A pure monochromatic wave is 
polarised because $\vec{E}_0$ is constant. An approximate monochromatic wave is 
said to be partially polarised.

\item Experiments studying polarisation involve measurement of intensities of
waves transmitted through media like a Nicol prism. Therefore, one studies the
quadratic functions of electric field. Functions like $E_iE_j$ or $E_i^\ast
E_j^\ast$ have a phase factor because
\begin{equation}\label{c6e79}
\vec{E}(t) = \vec{E}_0(t)e^{-i\omega t}.
\end{equation}
On the other hand, $E_iE_j^\ast$ or $E_i^\ast E_j$ will not have them. The 
long-time averages of these quantities will typically be non-zero and will be
easily observable in an experiment. Therefore, the polarisation properties
of electromagnetic waves are determined by the tensor
\begin{equation}\label{c6e80}
J_{ij} = \overline{E_{0i}{E_{0j}^\ast}},
\end{equation}
where $E_{0i}$ is the $i$-th component of the amplitude vector $\vec{E}_0(t)$.

\item The vector $\vec{E}$ is confined to a plane in the case of a plane 
electromagnetic wave. If the wave propagates along the $z$ axis then the
electric field is restricted to the $xy$-plane. Therefore, the tensor in
equation \eqref{c6e80} has only two dimensions. If we define,
\begin{equation}\label{c6e81}
J = J_{ii} = \overline{\vec{E}_0\cdot\vec{E}_0^\ast}
\end{equation}
then it is clear that $J$ is the intensity of the wave. It is not related
to its polarisation properties. We therefore factor it out from $J_{ij}$ and
define the polarisation tensor
\begin{equation}\label{c6e82}
\rho_{ij} = \frac{J_{ij}}{J}.
\end{equation}
From the definition of $J_{ij}$ it is clear that
\begin{equation}\label{c6e83}
J_{ij} = J_{ji}^\ast \text{ and } \rho_{ij} = \rho_{ji}^\ast,
\end{equation}
that is, the tensors $J_{ij}$ and $\rho_{ij}$ are hermitian and their eigen-
values are real. Note that $J$ defined in \eqref{c6e81} is also the trace of the
tensor. Therefore,
\begin{equation}\label{c6e84}
\rho_{11} + \rho_{22} = \frac{J_{11}}{J} + \frac{J_{22}}{J} = 1.
\end{equation}
Hermiticity of $\rho_{ij}$ requires that
\begin{equation}\label{c6e85}
\rho_{21} = \rho_{12}^\ast.
\end{equation}
Equations \eqref{c6e84} and \eqref{c6e85} suggest that the tensor $\rho_{ij}$ is
of the form
\begin{equation}\label{c6e86}
\rho_{ij} = \begin{bmatrix}a & b + ic \\ b - ic & 1 - a \end{bmatrix},
\end{equation}
where $a, b, c \in \mathbb{R}$. This means that the polarisation state of a 
plane electromagnetic wave can be described by three real numbers.

\item We will now examine the polarisation tensor for a variety of situations:
\begin{enumerate}
\item In the case of pure monochromatic light, $\vec{E}_0(t)$ is the constant
$\vec{E}_0$ and hence
\begin{equation}\label{c6e87}
\rho = \begin{bmatrix}|E_{01}|^2 & E_{01}E_{02}^\ast \\ 
                      E_{01}^\ast E_{02} & |E_{02}|^2 
       \end{bmatrix}
\end{equation}
In this case, 
\begin{equation}\label{c6e88}
\det\rho = |E_{01}|^2|E_{02}|^2 - E_{01}E_{02}^\ast E_{01}^\ast E_{02} = 0.
\end{equation}
This is also the case of complete polarisation.

\item The other extreme is completely unpolarised light. In this case, all
directions are equivalent and the polarisation tensor is isotropic. To ensure
that its trace is $1$, we must have
\begin{equation}\label{c6e89}
\rho_{ij} = \frac{1}{2}\delta_{ij}.
\end{equation}

\item These two extremes suggest that we can define a quantity $P$ called the
degree of polarisation as
\begin{equation}\label{c6e90}
\det\rho = \frac{1}{4}(1 - P^2)
\end{equation}
so that $P = 1$ for fully polarised, monochromatic light and $P = 0$ for 
unpolarised light.
\end{enumerate}

\item A tensor like $\rho_{ij}$ can be split into a symmetric and an 
anti-symmetric part such that
\begin{equation}\label{c6e91}
\rho_{ij} = S_{ij} + A_{ij},
\end{equation}
where
\begin{eqnarray}
S_{ij} &=& \frac{1}{2}(\rho_{ij} + \rho_{ji}) \label{c6e92} \\
A_{ij} &=& \frac{1}{2}(\rho_{ij} - \rho_{ji}). \label{c6e93}
\end{eqnarray} 
The hermiticity of $\rho$ results in 
\begin{equation}\label{c6e94}
A_{ij} = \begin{bmatrix}0 & \rho_{12} - \rho_{21} \\
-(\rho_{21} - \rho_{12}) & 0
\end{bmatrix} = (\rho_{12} - \rho_{12}^\ast)
\begin{bmatrix}0 & 1 \\ -1 & 0 \end{bmatrix}
\end{equation}
Now, $(\rho_{12} - \rho_{12}^\ast)$ is pure imaginary. Let 
\begin{equation}\label{c6e95}
A = i(\rho_{12} - \rho_{12}^\ast)
\end{equation}
so that we can write \eqref{c6e91} as
\begin{equation}\label{c6e96}
\rho_{ij} = S_{ij} - \frac{i}{2}Ae_{ij},
\end{equation}
where
\begin{equation}\label{c6e97}
e_{ij} = \begin{bmatrix}0 & 1 \\ -1 & 0 \end{bmatrix}
\end{equation}
is the fully anti-symmetric tensor.
\begin{enumerate}
\item For a circularly polarised wave, $b_1 = b_2$ in \eqref{c6e53} so that from
equations \eqref{c6e51} and \eqref{c6e52} we get $E_{02} = \pm i E_{01}$. The 
polarisation tensor is
\begin{equation}\label{c6e98}
\rho_{ij} = \begin{bmatrix}|E_{01}|^2 & \mp i |E_{01}|^2 \\
\pm i |E_{01}|^2 & |E_{01}|^2
\end{bmatrix}\frac{1}{|E_{01}|^2} = 
\frac{1}{2}\delta_{ij} - \frac{\pm i}{2}e_{ij}
\end{equation}
so that $S_{ij} = \delta_{ij}/2$ and $A = \pm 1$.

\item If the wave is linearly polarised, one of $b_1$ or $b_2$ in \eqref{c6e53}
is zero. From equations \eqref{c6e51} and \eqref{c6e52} it is evident that we 
can write $\vec{E}_0$ as a real vector. Therefore, $A = 0$.

\item This suggests that the constant $A$ can be viewed as a degree of 
circularity in the polarisation. $A \in [-1, 1]$, taking extreme values for 
left and right circular polarised waves and middle value for plane polarised 
ones.
\end{enumerate}

\item A real symmmetric matrix $S_{ij}$ can be diagonalised. Let $\lambda_1$ and
$\lambda_2$ be the two eigenvalues and let the corresponding eigenvectors 
correspond to directions $\un^{(1)}$ and $\un^{(2)}$. Then we can write
\begin{equation}\label{c6e99}
S_{ij} = \lambda_1 n^{(1)}_in^{(1)}_j + \lambda_2 n^{(2)}_in^{(2)}_j,
\end{equation}
and $0 \le \lambda_1, \lambda_1 \le 1$. This equation follows from the 
similarity transform, $S = N\Lambda N^T$, where $\Lambda = \diag(\lambda_1,
\lambda_2)$ and 
\[
N = [\un^{(1)}, \un^{(2)}]
\]
is the matrix whose columns are the (normalised) eigen-vectors. Each of the 
two terms is a product of quantities related to one of the eigen-vectors. If
$A = 0$, then $\rho_{ij} = S_{ij}$, which means that each term of $\rho_{ij}$ 
is a sum of two terms, each one of which being a product of one of the two 
eigen-vectors. The two parts can be treated as independent of each other. Such 
a wave is called \emph{incoherent}.
\item Let $\phi$ be the angle between $\un^{(1)}$ and the $x$-axis. Then we can 
write
\begin{eqnarray}
\un^{(1)} &=& (\cos\phi, \sin\phi) \label{c6e100} \\
\un^{(2)} &=& (-\sin\phi, \cos\phi) \label{c6e101}
\end{eqnarray}
If $\lambda_1 > \lambda_2$, let
\begin{equation}\label{c6e102}
l = \lambda_1 - \lambda_2 > 0.
\end{equation}
The matrix $S$ can now be written as
\begin{equation}\label{c6e103}
S = \frac{1}{2}\begin{bmatrix}
1 + l\cos(2\phi) & l\sin(2\phi) \\
l\sin(2\phi) & 1 - l\cos(2\phi)
\end{bmatrix}
\end{equation}
The three numbers $A, l$ and $\phi$ describe the polarisation state of the wave.
One can replace them with
\begin{eqnarray}
\xi_1 &=& l\sin(2\phi) \label{c6e104} \\
\xi_2 &=& A \label{c6e105} \\
\xi_3 &=& l\cos(2\phi) \label{c6e106}
\end{eqnarray}
so that the polarisation tensor can be written as
\begin{equation}\label{c6e107}
\rho = \frac{1}{2}\begin{bmatrix}
1 + \xi_3 & \xi_1 - i\xi_2 \\
\xi_1 + i\xi_2 & 1 - \xi_3
\end{bmatrix}.
\end{equation}
The three numbers $\xi_1, \xi_2, \xi_3$ are called Stokes parameters. In this form,
\begin{equation}\label{c6e108}
\det\rho = \frac{1}{4}(1 - \xi_1^2 - \xi_2^2 - \xi_3^2),
\end{equation}
so that the degree of polarisation $P$ defined in \eqref{c6e90} becomes
\begin{equation}\label{c6e109}
P = \sqrt{\xi_1^2 + \xi_2^2 + \xi_3^2}.
\end{equation}

\item Poisson's equation for a point charge $q$ at the origin is
\begin{equation}\label{c6e110}
\nabla^2\varphi = -4\pi q\delta(\vec{r}).
\end{equation}
If 
\begin{equation}\label{c6e111}
\varphi(\vec{r}) = \frac{1}{8\pi^3}\iiint_{-\infty}^\infty \hat{\varphi}(\vec{k})e^{-i\vec{k}\cdot\vec{r}}d^3k
\end{equation}
then
\begin{equation}\label{c6e112}
\nabla^2\varphi(\vec{r}) = -\frac{1}{8\pi^3}\iiint_{-\infty}^\infty k^2\hat{\varphi}(\vec{k})e^{-i\vec{k}\cdot\vec{r}}d^3k.
\end{equation}
Thus, the Fourier transform of $\nabla^2\varphi(\vec{r})$ is $-k^2\hat{\varphi}(\vec{k})$.
The Fourier transform of rhs of \eqref{c6e110} is $4\pi q$, where we used the three
dimensional analogue of \eqref{c6e77}
\begin{equation}\label{c6e113}
\frac{1}{8\pi^3}\iiint_{-\infty}^\infty e^{-i(\vec{k} - \vec{k}^\op)\cdot\vec{r}}d^3r
 = \delta^3(\vec{k} - \vec{k}^\op).
\end{equation}
From \eqref{c6e112} and \eqref{c6e113} we get
\begin{equation}\label{c6e114}
\hat{\varphi}(\vec{k}) = \frac{4\pi q}{k^2}.
\end{equation}
Given a potential $\varphi$, the electrostatic field is $\vec{E} = -\grad\varphi$
so that, from \eqref{c6e111},
\begin{equation}\label{c6e115}
\vec{E} = -\frac{1}{8\pi^3}\iiint_{-\infty}^\infty -i\vec{k}\varphi(\vec{k}) d^3k.
\end{equation}
The Fourier transform of the electrostatic field is
\begin{equation}\label{c6e116}
\hat{\vec{E}}(\vec{k}) = -i\vec{k}\varphi(\vec{k})  = -i\frac{4\pi q\vec{k}}{k^2},
\end{equation}
where we used \eqref{c6e114}.

\item It is possible to consider the electromagnetic field confined to a finite
volume to be an ensemble of oscillators. This is exactly how Dirac quantised the
electromagnetic field. Consider the field confined to a parallelepiped of sides
$L, M, N$. If $\vec{A}(\vec{r}, t)$ is the vector potential then one can express it
as a Fourier series
\begin{equation}\label{c6e117}
\vec{A}(\vec{r}, t) = \sum_{\vec{k}}\vec{A}_{\vec{k}}(t) 
e^{i\vec{k}\cdot\vec{r}} =
\sum_{k_x, k_y, k_z}\vec{A}_{k_x,k_y,k_z}(t) e^{i(k_xx + k_yy + k_zz)}.
\end{equation}
Note that there are three triple sums in this equation, one each for a component
of $\vec{A}_{\vec{k}}$. This expression is analogous to \eqref{c6e67}. The way we 
chose $\omega_0 = 2\pi/T$, we choose $\vec{k}_0$ such that
\begin{equation}\label{c6e118}
k_{0x} = \frac{2\pi}{L}; k_{0y} = \frac{2\pi}{M}; k_{0z} = \frac{2\pi}{N}
\end{equation}
and the components of vector $\vec{k}$ in the sum are integer multiples of 
$k_{0x}, k_{0y}, k_{0z}$. Thus,
\begin{equation}\label{c6e119}
k_{x} = \frac{2\pi n_x}{L}; k_{y} = \frac{2\pi n_y}{M}; k_{z} = \frac{2\pi n_z}{N},
\end{equation}
where $n_x, n_y, n_z \in \mathbb{Z}$. Since $\vec{A}$ is real,
\begin{equation}\label{c6e120}
\vec{A}_{-\vec{k}} = \vec{A}^\ast_{\vec{k}}.
\end{equation}
We have been choosing the Coulomb gauge so that $\dive\vec{A} = 0$. Applying it
to \eqref{c6e117} gives,
\[
\dive\vec{A} = \sum_{\vec{k}}\vec{A}_{\vec{k}}(t)\cdot(i\vec{k})e^{i\vec{k}\cdot\vec{r}},
\]
so that
\begin{equation}\label{c6e121}
\dive\vec{A} = 0 \Rightarrow \vec{k}\cdot\vec{A}_{\vec{k}} = 0.
\end{equation}
Since $\vec{A}$ also satisfies \eqref{c6e5},
\[
\nabla^2\vec{A} - \frac{1}{c^2}\spdt{\vec{A}}{t} = 0 \Rightarrow
-\sum_{\vec{k}}k^2\vec{A}_{\vec{k}}e^{i\vec{k}\cdot\vec{r}} - 
\frac{1}{c^2}\sum_{\vec{k}}\ddot{\vec{A}}_{\vec{k}}e^{i\vec{k}\cdot\vec{r}} = 0.
\]
That is,
\[
\sum_{\vec{k}}\left(\ddot{\vec{A}}_{\vec{k}} + c^2k^2\vec{A}_{\vec{k}}\right)e^{i\vec{k}\cdot\vec{r}} = 0
\]
from which it follows that
\begin{equation}\label{c6e122}
\ddot{\vec{A}}_{\vec{k}} + c^2k^2\vec{A}_{\vec{k}} = 0.
\end{equation}
From \eqref{c6e119}, we get
\[
n_x = \frac{L}{2\pi}k_x; n_y = \frac{M}{2\pi}k_y; n_z = \frac{N}{2\pi}k_z
\]
so that
\[
\delta n_x = \frac{L}{2\pi}\delta k_x; \delta n_y = \frac{M}{2\pi}\delta k_y; 
\delta n_z = \frac{N}{2\pi}\delta k_z
\]
and hence
\begin{equation}\label{c6e123}
\delta n = \frac{V}{8\pi^3}\delta k_x \delta k_y \delta k_z,
\end{equation}
where $\delta n = \delta n_x \delta n_y \delta n_z$. Introducing spherical 
coordinates in the $k$-space, equation \eqref{c6e123} becomes
\begin{equation}\label{c6e124}
\delta n = \frac{V}{8\pi^3}k^2 \delta k \delta o,
\end{equation}
where $\delta o$ is a solid angle. Integrating over the solid angle and after
taking appropriate limits, we get
\begin{equation}\label{c6e125}
\td{n}{k} = \frac{V}{2\pi^2}k^2.
\end{equation}
This expression is called the \emph{density of states} of the electromagnetic
field. It is the number of states of the field in the $k$-space.

\item We will now find the energy of the confined field. To do that, we first 
need an expression for the electric and the magnetic fields. If we choose the
scalar potential to be zero, as we did in point 3, then
\begin{equation}\label{c6e126}
\vec{E} = -\frac{1}{c}\dot{\vec{A}} = 
-\frac{1}{c}\sum_{\vec{k}}\dot{\vec{A}}_{\vec{k}}(t) e^{i\vec{k}\cdot\vec{r}}
\end{equation}
and
\begin{equation}\label{c6e127}
\vec{H} = \curl\vec{A} = \sum_{\vec{k}}(i\vec{k}\times\vec{A}_{\vec{k}})e^{i\vec{k}\cdot\vec{r}}
\end{equation}
Therefore,
\begin{equation}\label{c6e128}
E^2 = \frac{1}{c^2}\sum_{\vec{k}}\sum_{\vec{k}^\op}
\dot{\vec{A}}_{\vec{k}}(t)\cdot\dot{\vec{A}}_{\vec{k}^\op}^\ast(t)
e^{i(\vec{k}-\vec{k}^\op)\cdot\vec{r}}
\end{equation}
Likewise,
\begin{equation}\label{c6e129}
H^2 = \sum_{\vec{k}}\sum_{\vec{k}^\op}(\vec{k}\times\vec{A}_{\vec{k}})\cdot
(\vec{k}\times\vec{A}_{\vec{k}}^\ast)e^{i(\vec{k}-\vec{k}^\op)\cdot\vec{r}}
\end{equation}
Since the energy of the field is
\[
\mathcal{E} = \frac{1}{8\pi}\int_V (E^2 + H^2)dv
\]
where the integral is over the finite volume of the parallelepiped. Thus,
\begin{eqnarray*}
\mathcal{E} &=& \frac{1}{8\pi}\sum_{\vec{k}}\sum_{\vec{k}^\op}
\left(\frac{1}{c^2}\dot{\vec{A}}_{\vec{k}}(t)\cdot\dot{\vec{A}}_{\vec{k}^\op}^\ast(t) +
(\vec{k}\times\vec{A}_{\vec{k}})\cdot(\vec{k}^\op\times\vec{A}_{\vec{k}^\op}^\ast)\right)
\int_V e^{i(\vec{k}-\vec{k}^\op)\cdot\vec{r}} dv \\
&=& \frac{V}{8\pi}\sum_{\vec{k}}\sum_{\vec{k}^\op}
\left(\frac{1}{c^2}\dot{\vec{A}}_{\vec{k}}(t)\cdot\dot{\vec{A}}_{\vec{k}^\op}^\ast(t) +
(\vec{k}\times\vec{A}_{\vec{k}})\cdot(\vec{k}^\op\times\vec{A}_{\vec{k}^\op}^\ast)\right)
\delta_{\vec{k}, \vec{k}^\op} \\
&=& \frac{V}{8\pi}\sum_{\vec{k}}
\left(\frac{1}{c^2}\dot{\vec{A}}_{\vec{k}}(t)\cdot\dot{\vec{A}}_{\vec{k}}^\ast(t) +
(\vec{k}\times\vec{A}_{\vec{k}})\cdot(\vec{k}\times\vec{A}_{\vec{k}}^\ast)\right)
\end{eqnarray*}
Now,
\begin{eqnarray*}
(\vec{k}\times\vec{A}_{\vec{k}})\cdot(\vec{k}\times\vec{A}_{\vec{k}}^\ast)) 
&=& ((\vec{k}\times\vec{A}_{\vec{k}})\times\vec{k})\cdot\vec{A}_{\vec{k}}^\ast) \\
&=& (-\vec{k}\times(\vec{k}\times\vec{A}_{\vec{k}}))\cdot\vec{A}_{\vec{k}}^\ast) \\
&=& -(\vec{k}(\vec{k}\cdot\vec{A}_{\vec{k}} - \vec{A}_{\vec{k}}k^2)\cdot\vec{A}_{\vec{k}}^\ast) \\
&=& k^2\vec{A}_{\vec{k}}\cdot\vec{A}_{\vec{k}}^\ast,
\end{eqnarray*}
where we used \eqref{c6e121} in the last step. The energy of the field is thus,
\[
\mathcal{E} = \frac{V}{8\pi}\sum_{\vec{k}}
\left(\frac{1}{c^2}\dot{\vec{A}}_{\vec{k}}(t)\cdot\dot{\vec{A}}_{\vec{k}}^\ast(t) +
k^2\vec{A}_{\vec{k}}\cdot\vec{A}_{\vec{k}}^\ast\right)
\]
or
\begin{equation}\label{c6e130}
\mathcal{E} = \frac{V}{8\pi c^2}\sum_{\vec{k}}
\left(\dot{\vec{A}}_{\vec{k}}(t)\cdot\dot{\vec{A}}_{\vec{k}}^\ast(t) +
k^2c^2\vec{A}_{\vec{k}}\cdot\vec{A}_{\vec{k}}^\ast\right)
\end{equation}
Each term in the sum on the rhs of \eqref{c6e130} corresponds to a single term in
the expansion \eqref{c6e117} of $\vec{A}$.

\item The solution of \eqref{c6e122} is a harmonic function
\begin{equation}\label{c6e131}
\vec{A}_{\vec{k}} = \vec{a}_{\vec{k}}e^{-i\omega_k t} + \vec{a}_{\vec{k}}^\ast e^{i\omega_k t},
\end{equation}
where
\begin{equation}\label{c6e132}
\omega_k = ck
\end{equation}
is the frequency of the mode. From \eqref{c6e120} we also have
\begin{equation}\label{c6e133}
\vec{A}_{-\vec{k}} = \vec{a}_{\vec{k}}^\ast e^{i\omega_k t} + \vec{a}_{\vec{k}} e^{-i\omega_k t}
\end{equation}
so that
\begin{equation}\label{c6e134}
\vec{a}_{-\vec{k}} = \vec{a}_{\vec{k}}.
\end{equation}
These two equations immediately give
\begin{equation}\label{c6e135}
\dot{\vec{A}}_{\vec{k}} = -ick(\vec{a}_{\vec{k}}e^{-i\omega_k t} - \vec{a}_{\vec{k}}^\ast e^{i\omega_k t})
\end{equation}
Therefore,
\begin{eqnarray}
\dot{\vec{A}}_{\vec{k}}(t)\cdot\dot{\vec{A}}_{\vec{k}}^\ast(t) &=& c^2k^2
(\vec{a}_{\vec{k}}e^{-i\omega_k t} - \vec{a}_{\vec{k}}^\ast e^{i\omega_k t})\cdot
(\vec{a}_{\vec{k}}^\ast e^{i\omega_k t} - \vec{a}_{\vec{k}} e^{-i\omega_k t}) \nonumber \\
&=& 2c^2k^2|a_{\vec{k}}|^2 \label{c6e136}
\end{eqnarray}
Likewise,
\begin{eqnarray}
\vec{A}_{\vec{k}}\cdot\vec{A}_{\vec{k}}^\ast &=& 
(\vec{a}_{\vec{k}}e^{-i\omega_k t} + \vec{a}_{\vec{k}}^\ast e^{i\omega_k t})\cdot
(\vec{a}_{\vec{k}}^\ast e^{i\omega_k t} + \vec{a}_{\vec{k}} e^{-i\omega_k t}) \nonumber \\
&=& 2|a_{\vec{k}}|^2 \label{c6e137}
\end{eqnarray}
Putting \eqref{c6e136} and \eqref{c6e137} in \eqref{c6e130} we get
\begin{equation}\label{c6e138}
\mathcal{E} = \frac{V}{8\pi c^2}\sum_{\vec{k}} 4c^2k^2 a^2_{\vec{k}}
= \sum_{\vec{k}} \frac{Vk^2}{2\pi}|a_{\vec{k}}|^2.
\end{equation}
Thus the total energy of the confined field is the sum of energies corresponding
to harmonic oscillator described by \eqref{c6e122}.

\item Before proceeding, we summarise what we have done so far. We described the
electromagnetic field in Coulomb gauge with $\varphi = 0$. The fields $\vec{E}$ 
and $\vec{H}$ were described in terms of a continuous function $\vec{A}$ of $t$
and $\vec{r}$. We then expressed $\vec{A}$ as a triple Fourier series in with
Fourier coefficients $\vec{A}_{\vec{k}}$, which turned out to be harmonic functions
with frequenct $ck$ and amplitudes $\vec{a}_{\vec{k}}$. Effectively, we expressed
$\vec{E}$ and $\vec{H}$ in terms of discrete quantities $\vec{a}_{\vec{k}}$. We will
now transform $\vec{a}_{\vec{k}}$ into canonically conjugate variables.

\item Define variables $\vec{Q}_{\vec{k}}$ and $\vec{P}_{\vec{k}}$ such that
\begin{eqnarray}
\vec{Q}_{\vec{k}} &=& \sqrt{\frac{V}{4\pi c^2}}(\vec{a}_{\vec{k}} + \vec{a}_{\vec{k}}^\ast) \label{c6e139} \\
\vec{P}_{\vec{k}} &=& -i\omega_{\vec{k}}\sqrt{\frac{V}{4\pi c^2}}(\vec{a}_{\vec{k}} - \vec{a}_{\vec{k}}^\ast) \label{c6e140}
\end{eqnarray}
so that
\begin{eqnarray*}
Q_{\vec{k}}^2 &=& \vec{Q}_{\vec{k}} \cdot \vec{Q}_{\vec{k}}^\ast \\
 &=& \frac{V}{4\pi c^2}(\vec{a}_{\vec{k}} + \vec{a}_{\vec{k}}^\ast)\cdot(\vec{a}_{\vec{k}} + \vec{a}_{\vec{k}}^\ast) \\
 &=& \frac{V}{4\pi c^2}(a_{\vec{k}}^2 + 2\vec{a}_{\vec{k}}\cdot\vec{a}_{\vec{k}}^\ast + {a_{\vec{k}}^\ast}^2)
\end{eqnarray*}
and
\[
P_{\vec{k}}^2 = -\omega_{\vec{k}}^2\frac{V}{4\pi c^2}(a_{\vec{k}}^2 - 2\vec{a}_{\vec{k}}\cdot\vec{a}_{\vec{k}}^\ast + {a_{\vec{k}}^\ast}^2).
\]
Therefore,
\[
P_{\vec{k}}^2 + \omega_{\vec{k}}^2Q_{\vec{k}}^2 = \omega_{\vec{k}}^2\frac{V}{\pi c^2}\vec{a}_{\vec{k}}\cdot\vec{a}_{\vec{k}}^\ast
\]
and
\[
\frac{1}{2}(P_{\vec{k}}^2 + \omega_{\vec{k}}^2Q_{\vec{k}}^2) = \omega_{\vec{k}}^2\frac{V}{2\pi c^2}|\vec{a}_{\vec{k}}^2
\]
Comparing this with \eqref{c6e138} allows us to call this expression $\mathcal{E}_{\vec{k}}$,
the energy of the oscillator corresponding to the term $\vec{A}_{\vec{k}}$. Thus,
we can write the Hamiltonian of the field as
\begin{equation}\label{c6e141}
\mathcal{H} = \sum_{\vec{k}}\mathcal{H}_{\vec{k}} = 
\sum_{\vec{k}}\frac{1}{2}(P_{\vec{k}}^2 + \omega_{\vec{k}}^2Q_{\vec{k}}^2)
\end{equation}

\item The canonical equations are
\begin{eqnarray}
\pdt{\mathcal{H}}{\vec{P}_{\vec{k}}} &=& \dot{\vec{Q}}_{\vec{k}} \Rightarrow 
 \vec{P}_{\vec{k}} = \dot{\vec{Q}}_{\vec{k}} \label{c6e142} \\
\pdt{\mathcal{H}}{\vec{Q}_{\vec{k}}} &=& -\dot{\vec{P}}_{\vec{k}} \Rightarrow
 \omega_{\vec{k}}^2\vec{Q}_{\vec{k}} = -\dot{\vec{P}}_{\vec{k}} \label{c6e143}
\end{eqnarray}
Substituting \eqref{c6e142} in \eqref{c6e143} gives the equation of motion
\begin{equation}\label{c6e144}
\ddot{\vec{Q}}_{\vec{k}} + \omega_{\vec{k}}^2\vec{Q}_{\vec{k}} = 0.
\end{equation}

\item We can write the Hamiltonian in \eqref{c6e141} as
\[
\mathcal{H}_{\vec{k}} = \sum_j\frac{1}{2}(P_{\vec{k},j}^2 + \omega^2_{\vec{k}}Q_{\vec{k},j}^2)
\]
which is a sum of three Hamiltonians, one each along a cartesian axis. This 
allows us to imagine the field to be made up of independent harmonic oscillators.

\item If we know $\vec{Q}_{\vec{k}}$ we can get $\vec{P}_{\vec{k}}$ by differentiation
with respect to $t$. Knowing these immediately gives us
\begin{eqnarray*}
\vec{a}_{\vec{k}} &=& \frac{i}{k}\sqrt{\frac{\pi}{V}}(\vec{P}_{\vec{k}} - i\omega_{\vec{k}}\vec{Q}_{\vec{k}}) \label{c6e145} \\
\vec{a}_{\vec{k}}^\ast &=& \frac{-i}{k}\sqrt{\frac{\pi}{V}}(\vec{P}_{\vec{k}} + i\omega_{\vec{k}}\vec{Q}_{\vec{k}}) \label{c6e146} \\
\vec{A} &=& 2\sqrt{\frac{\pi}{V}}\sum_{\vec{k}}
     \frac{ck\vec{Q}_{\vec{k}}\cos(\vec{k}\cdot\vec{r}) - \vec{P}_{\vec{k}}\sin(\vec{k}\cdot\vec{r})}{k} \label{c6e147} \\
\vec{E} &=& -2\sqrt{\frac{\pi}{V}}\sum_{\vec{k}}
     \frac{ck\vec{Q}_{\vec{k}}\sin(\vec{k}\cdot\vec{r}) + \vec{P}_{\vec{k}}\cos(\vec{k}\cdot\vec{r})}{k} \label{c6e148} \\
\vec{H} &=& -2\sqrt{\frac{\pi}{V}}\sum_{\vec{k}}
     \frac{ck\vec{k} \times \vec{Q}_{\vec{k}}\sin(\vec{k}\cdot\vec{r}) + \vec{k} \times \vec{P}_{\vec{k}}\cos(\vec{k}\cdot\vec{r})}{k} \label{c6e149}
\end{eqnarray*}
\end{enumerate}

\nocite{*}
%\bibliographystyle{plain}
%\bibliography{ctof}
\end{document}  
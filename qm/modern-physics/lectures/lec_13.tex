\documentclass{beamer}
\usefonttheme[onlymath]{serif}
\usepackage{graphicx, hyperref}
\hypersetup{colorlinks, urlcolor=blue}
\usepackage{subcaption}
\usetheme{Boadilla}

\newcommand{\pvec}[1]{\vec{#1}\mkern2mu\vphantom{#1}}

\title{Statistical Mechanics}
\author{Amey Joshi}
\date{\today}

\begin{document}
\begin{frame}
\titlepage
\end{frame}

\begin{frame}
\frametitle{Why Statistical Mechanics?}
\begin{enumerate}
\item There is no analytical solution of the three body problem in classical
mechanics. 
\item Neil Armstrong's mission to the moon was extremely risky because of the
complexity of the three body problem - that involving the earth, the moon and
the spaceship - and the \href{https://galileo-unbound.blog/tag/three-body-problem/}{chaotic}
nature of some of the trajectories.
\item The problem is tougher in quantum mechanics. The only three body 
Schr\"{o}dinger equation is that of H$_2^+$ ion but it assumes that the nuclei
are at rest.
\item Barring a few exceptions, almost all solutions to the Schr\"{o}dinger 
equation for practically important problems are numerical.
\item The numerical solutions too are an approximation. Many ignore the 
interactions between the electrons and the nuclear motion.
\end{enumerate}
\end{frame}

\begin{frame}
\frametitle{The Nature of Statistical Mechanics}
\begin{enumerate}
\item Consider an ideal gas in a container. We are not interested in the 
position or velocity of an individual molecule. Knowing about it is impossible
but also futile.
\item We are interested in the average speed, the pressure and the energy.
\item If we have a crystal of radioactive element, we do not want to know when a
given nucleus will decay but in a certain time interval how many will decay.
\item A crystal has a very large number of electrons, we want to know how many
are in a given energy level (band).
\item Thermodynamics is a phenomenological theory of macroscopic systems.
Statistical mechanics provides the foundation of thermodynamics.
\item It has very wide applicability but we will study it in rather simplest
situations.
\end{enumerate}
\end{frame}

\begin{frame}
\frametitle{Ideal ``gases''}
\begin{enumerate}
\item Any system of a large number of weakly interacting entities is considered
a gas.
\begin{enumerate}
\item Air at room temperature and atmospheric pressure.
\item Photons in a blackbody.
\item Electrons in a crystals.
\item Galaxies in a cluster.
\end{enumerate}
\item Consider a system of $N$ weakly interacting particles which can have 
energies $\epsilon_1, \epsilon_2, \ldots$. Given that the total energy of the 
system is $E$, find the most probable distribution of numbers $n_1, n_2, \ldots$
such that
\begin{equation}\label{e1}
n_1\epsilon_1 + n_2\epsilon_2 + \cdots = E.
\end{equation}
\item The interaction is weak but not zero. It is needed to establish a thermal
equilibrium.
\item We assume that a particle can be each energy state with an equal 
likelihood. It is an axiom that is justified only by an experimental 
confirmation of its conclusions.
\end{enumerate}
\end{frame}

\begin{frame}
\frametitle{Ideal ``gases''}
\begin{enumerate}
\item The solution to the problem is of the form
\begin{equation}\label{e2}
n(\epsilon) = g(\epsilon)f(\epsilon),
\end{equation}
where
\begin{enumerate}
\item $n(\epsilon)$ is the number of particles in energy state $\epsilon$,
\item $g(\epsilon)$ is the number of states with energy $\epsilon$. Recall that 
in a hydrogen atom multiple values of the triple $(n, l, m_l)$ may correspond
to the same energy. This is called `degeneracy'.
\item $f(\epsilon)$ is the probability of occupancy of a state with energy
$\epsilon$.
\end{enumerate}
\item If the energy is not quantised then we have
\begin{equation}\label{e3}
n(\epsilon)d\epsilon = g(\epsilon)f(\epsilon)d\epsilon.
\end{equation}
$g$ is the number of states with energy between $\epsilon$ and $\epsilon + 
d\epsilon$. Likewise, $n$ is the number of particles with energy between 
$\epsilon$ and $\epsilon + d\epsilon$.
\end{enumerate}
\end{frame}

\begin{frame}
\frametitle{Maxwell-Boltzmann statistics}
\begin{enumerate}
\item The form of $f$ in equation \eqref{e3} depends on the nature of particles
in the ``gas''.
\item If the particles are identical but so far apart that their wavefunctions 
do not overlap then they can be distinguished from each other. For them,
\begin{equation}\label{e4}
f(\epsilon) = Ae^{-\epsilon/kT},
\end{equation}
where the constant $A$ depends on the number of particles, $k$ is the Boltzmann
constant and $T$ is the temperature.
\item An example of an ideal ``Maxwell gas'' is air at room temperature and 
atmospheric pressure. The total energy $E$ is distributed among the continuous
translational energy states.
\item A slightly non-trivial example is the rotational states of a diatomic gas
molecules. We consider energy distributed among the discrete rotational energy
states. 
\end{enumerate}
\end{frame}

\begin{frame}
\frametitle{Bose-Einstein statistics}
\begin{enumerate}
\item Identical, indistingushable particles with an integral spin.
\item The wavefunction of the entire system is symmetrical under exchange of 
particles.
\item It is possible for a single energy state to have more than one particle.
Even all of them can be in a single state.
\item Examples are photons in a black body cavity, phonons in a crystal, 
electron-pair (Cooper pair) in superconductors, He nuclei etc.
\item The distribution function is
\begin{equation}\label{e5}
f(\epsilon) = \frac{1}{e^\alpha e^{\epsilon/kT} - 1},
\end{equation}
where $k$ is the Boltzmann constant, $T$ is the temperature and $\alpha$ is a 
quantity that depends on the number of particles and temperature.
\item When the number of particles is not conserved $\alpha = 0$.
\end{enumerate}
\end{frame}

\begin{frame}
\frametitle{Fermi-Dirac statistics}
\begin{enumerate}
\item Identical, indistingushable particles with a half-integral spin.
\item The wavefunction of the entire system is anti-symmetrical under exchange
of particles.
\item It is not possible for a single energy state to have more than one 
particle. Pauli exclusion principle must be obeyed.
\item Examples are electrons in a crystal, nucleons in a nucleus, neutrons in
a neutron star, electrons in a white-dwarf.
\item The distribution function is
\begin{equation}\label{e6}
f(\epsilon) = \frac{1}{e^{(\epsilon - \epsilon_F)/kT} + 1},
\end{equation}
where $k$ is the Boltzmann constant, $T$ is the temperature and $\epsilon_F$
is called the Fermi-energy of the system. It is defined as that energy for which
the distribution function is $1/2$.
\end{enumerate}
\end{frame}

\end{document}
\documentclass{beamer}
\usepackage{bm}
\newcommand{\op}{{\;\prime}}
\newcommand{\grad}{\nabla\;}
\newcommand{\dive}{\nabla\cdot}
\newcommand{\curl}{\nabla\times}
\newcommand{\td}[2]{\frac{d{#1}}{d{#2}}}
\newcommand{\pdt}[2]{\frac{\partial{#1}}{\partial{#2}}}
\newcommand{\uv}[1]{\hat{e}_{#1}}
\newcommand{\un}{\hat{n}}

\usefonttheme[onlymath]{serif}
\title{Electricity and Magnetism}
\subtitle{The Electric Potential}
\author{Amey Joshi}
\begin{document}
\frame{\titlepage}

\begin{frame}
\frametitle{Line integral of the electric field}
\begin{enumerate}
\item The difference between
\begin{equation}\label{e1}
\int_a^b f(x)dx \text{ and } \int_C \vec{F}\cdot d\vec{s},
\end{equation}
where $a, b$ are real numbers and $C$ is a curve joining two points.
\item If 
\begin{equation}\label{e2}
\vec{F} = f(r)\uv{r}
\end{equation}
then the line integral
\begin{equation}\label{e3}
\int_A^B \vec{F}\cdot d\vec{s}
\end{equation}
where $A$ and $B$ are points, is independent of the curve joining them. In
particular, this is true for $f(r) = kr^{-2}$.
\item Solve problem 1,
\end{enumerate}
\end{frame}

\begin{frame}
\frametitle{Line integral of the electric field}
\begin{enumerate}
\item An immediate consequence of the path-independence of the line integral is
\begin{equation}\label{e4}
\oint\vec{F}\cdot d\vec{s} = 0.
\end{equation}
\item A force $\vec{F}$ with this property is called \emph{conservative}.
\item The line integral is a scalar that depends only on the two points.
We denote it by $W(A, B)$. If $\vec{F}$ denotes a force then $W(A, B)$ is the
work done in moving from $A$ to $B$.
\item In particular, if $\vec{F} = q\vec{E}$ then $W(A, B)$ is the work done
by the electric field in moving a charge from $A$ to $B$.
\end{enumerate}
\end{frame}

\begin{frame}
\frametitle{The electric potential}
\begin{enumerate}
\item If $q = 1$, we define a scalar function,
\begin{equation}\label{e5}
\varphi(B, A) = -\int_A^B\vec{E}\cdot d\vec{s}
\end{equation}
and call it the electric potential difference between the points $A$ and $B$.
\item One can define the gravitational potential difference analogously.
\item In CGS units, $\varphi$ is measured in statvolt and its dimension is same
as erg/esu or esu/cm. The definition is identical in SI and the unit is volt.
Its dimension is same as $J/C$. 1 statvolt is $\approx 300$ V.
\item Solve problems 2, 3, 5 and 6.
\end{enumerate}
\end{frame}

\begin{frame}
\frametitle{Field and potential}
From \eqref{e5} we see that
\begin{equation}\label{e6}
d\varphi = -\vec{E}\cdot d\vec{s}.
\end{equation}
We also have
\begin{equation}\label{e7}
d\varphi = \pdt{\varphi}{x}dx + \pdt{\varphi}{y}dy + \pdt{\varphi}{z}dz
\end{equation}
Now, $d\vec{s} = \uv{x}dx + \uv{y}dy + \uv{z}dz$. If we define a vector
\begin{equation}\label{e8}
\grad\varphi = \pdt{\varphi}{x}\uv{x} + \pdt{\varphi}{y}\uv{y} + \pdt{\varphi}{z}\uv{z}
\end{equation}
then \eqref{e7} can be written as $d\varphi = \grad\varphi\cdot d\vec{s}$. Comparing
it with \eqref{e6} gives us
\begin{equation}\label{e9}
\vec{E} = -\grad\varphi.
\end{equation}
\end{frame}

\begin{frame}
\frametitle{Field and potential}
\begin{enumerate}
\item A vector field, in general, needs three functions for its full specification.
\item In the special case of a field for which \eqref{e4} is valid, the three components
are not independent of each other. In fact, such a vector field can be described
by a single scalar field, its potential.
\item Often times, but not always, the potential is zero at great distance from the
charges.
\item Show that the potential of a single charge $q$ located at the origin is
\begin{equation}\label{e10}
\varphi(\vec{r}) = -\frac{q}{r}.
\end{equation}
Generalise it the case when $q$ is located at $\vec{r}^\op$.
\end{enumerate}
\end{frame}

\begin{frame}
\frametitle{Field and potential}
\begin{enumerate}
\item The superposition principle permits us to further generalise \eqref{e10} to
\begin{equation}\label{e11}
\varphi(\vec{r}) = -\int\frac{\rho(\vec{r}^\op)}{|\vec{r} - \vec{r}^\op|}dv^\op.
\end{equation}
\item Write this equation using the Cartesian coordinates for greater clarity.
\item \eqref{e11} is useful only if $\rho$ is localised. To confirm this, apply it
to an infinitely long wire with a uniform charge density $\lambda$.
\item Many times, but not always, it is easier to evaluate the integral in \eqref{e11}.
In the case of an infinitely long wire, it is easier to get $\vec{E}$!! (Why?)
\item Solve problems 4, 7, 8, 9, 10, 11, 12.
\end{enumerate}
\end{frame}

\end{document}
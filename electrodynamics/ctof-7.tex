\chapter{The propagation of light}\label{c7}
\begin{enumerate}
\item A plane wave is the one for which the fields are a function only of $t - 
x/c$ if they are propagating along the $x$ axis. Its direction of propagation 
and amplitude are the same everywhere.

\item Oftentimes, one can treat arbitrary waves as plane waves in a small enough
region of space. This is possible only if the amplitude and direction remain 
practically the same over a single wavelength. If this is indeed true then one 
can introduce the idea of a \emph{wave surface} or a \emph{wave front}. It is an
imaginary surface at all points of which the wave has a constant phase. Wave 
fronts of plane waves are indeed planes.

In a tiny region of space a wave propagates along the normal to the wave front.
The normals of consecutive wave fronts create a \emph{ray}. Alternatively, a ray
is a curve whose tangent at every point coincides with the normal to the wave
front.

\item When the dimensions of the region of interest are large compared to the
wavelength, we can study the propagation of waves by paying attention only to
the rays. This is the domain of \emph{geometrical optics}. It is an approximation
in which we ignore the wave properties and consider only rays. Geometrical optics
works best for waves whose wavelengths are very small.

\item If $f$ is any quantity describing the wave then for a plane monochromatic 
wave,
\begin{equation}\label{c7e1}
f = a\exp(-i(k_\mu x^\mu + \alpha)),
\end{equation}
where $\alpha$ is the constant phase of the wave. The 4-vector $k^\mu$ is defined
by equation \eqref{c6e54}. Strictly speaking, we should have written \eqref{c7e1}
as
\[
f = a\re{\exp(-i(k_\mu x^\mu + \alpha))},
\]
but we will now assume that every equation with complex exponentials is 
interpreted by ignoring its imaginary part.

If 
\begin{equation}\label{c7e2}
\Psi = -k_\mu x^\mu + \alpha
\end{equation}
then $f = a\exp(i\Psi)$. It is called the \emph{eikonal}.

\item If we are dealing with waves whose wave fronts are not planes then their
amplitude is a function of $x^\mu$ and the expression for $\Psi$ is not as 
simple as in \eqref{c7e2}.

\item Whatever be the form of $\Psi$, the approximation of geometrical optics
is applicable only if $\Psi$ is practically constant over a single wave-length, 
or
\begin{equation}\label{c7e3}
\frac{\lambda}{\Psi} \rightarrow 0.
\end{equation} 
In regions whose extent is of the order of magnitude of $\lambda$, one can write
the eikonal as 
\begin{equation}\label{c7e4}
\Psi = \Psi(0) + x^\mu\pdt{\Psi}{x^\mu}.
\end{equation}
Comparing with \eqref{c7e2}, we see that
\begin{equation}\label{c7e5}
k_\mu = -\pdt{\Psi}{x^\mu}.
\end{equation}
From \eqref{c6e57}, $k_\mu k^\mu = 0$ so that we also have
\begin{equation}\label{c7e6}
\pdt{\Psi}{x^\mu}\pdt{\Psi}{x_\mu} = 0.
\end{equation}
Equation \eqref{c7e6} is called the \emph{eikonal equation}.

\item Since $f = a\exp(i\Psi)$ satisfies the wave equation, 
\[
\frac{\partial^2 f}{\partial x_\mu \partial x^\mu} = 0.
\]
Now,
\[
\pdt{f}{x_\mu} = \pdt{a}{x_\mu}\exp(i\Psi) + ia\exp(i\Psi)\pdt{\Psi}{x^\mu} =
\pdt{a}{x_\mu}\exp(i\Psi) + if\pdt{\Psi}{x_\mu}
\]
so that
\begin{eqnarray*}
\frac{\partial^2 f}{\partial x_\mu \partial x^\mu} &=& 
    \frac{\partial^2 a}{\partial x_\mu \partial x^\mu}e^{i\Psi} + 
 	i\pdt{a}{x_\mu}\exp(i\Psi)\pdt{\Psi}{x^\mu} + 
	i\pdt{f}{x^\mu}\pdt{\Psi}{x^\mu} + \\
 & & if\frac{\partial^2 \Psi}{\partial x_\mu \partial x^\mu} \\
 &=& \frac{\partial^2 a}{\partial x_\mu \partial x^\mu}e^{i\Psi} + 
    2i\pdt{a}{x_\mu}\exp(i\Psi)\pdt{\Psi}{x^\mu} -
	f\pdt{\Psi}{x^\mu}\pdt{\Psi}{x_\mu} + \\
 & &  if\frac{\partial^2 \Psi}{\partial x_\mu \partial x^\mu}
\end{eqnarray*}
The real and imaginary parts of this quantity should vanish independently. 
Therefore,
\begin{eqnarray}
\frac{1}{a}\frac{\partial^2 a}{\partial x_\mu \partial x^\mu} &=& 
  \pdt{\Psi}{x^\mu}\pdt{\Psi}{x_\mu} \label{c7e7} \\
\frac{2}{a}\pdt{a}{x_\mu}\exp(i\Psi)\pdt{\Psi}{x^\mu} &=& 
\frac{\partial^2 \Psi}{\partial x_\mu \partial x^\mu} \label{c7e8} 
\end{eqnarray}
In the domain of geometrical optics we also assume that the second derivative of
$a$ with respect to $x^\mu$ can be ignored so that the first of the previous pair
of equations gives the eikonal equation.

The book mentions that largeness of $\Psi$ leads one to that conclusion but I 
unable to see it that way.

Alternatively, since a small patch of an arbitrary wave front can be considered
to be a plane, and since for a plane wave $a$ is constant, the eikonal equation
follows immediately from \eqref{c7e7}.

\item We can expand the eikonal equation \eqref{c7e6} as
\[
\frac{1}{c^2}\left(\pdt{f}{t}\right)^2 - \left(\pdt{f}{x}\right)^2 - 
\left(\pdt{f}{y}\right)^2 - \left(\pdt{f}{z}\right)^2 = 0
\]
and compare it with the Hamilton-Jacobi equation \eqref{c2e35}
\[
\frac{1}{c^2}\left(\pdt{S}{t}\right)^2 - \left(\pdt{S}{x}\right)^2 
- \left(\pdt{S}{y}\right)^2 - \left(\pdt{S}{z}\right)^2 = m^2c^2
\]
to suspect an analogy between the action $S$ of point particle with the eikonal
$f$ of a ray and the ray itself being a massless point particle. We also recall
that the action is related to the Hamiltonian and the momemtum as (refer to
\eqref{c2e27}),
\begin{eqnarray}
\mathcal{H} &=& -\pdt{S}{t} \label{c7e9} \\
\vec{p} &=& \pdt{S}{\vec{r}} \label{c7e10}
\end{eqnarray}
If we expand \eqref{c7e4} and \eqref{c7e2} we get
\begin{eqnarray*}
\Psi &=& \Psi(0) + t\pdt{\Psi}{t} - \vec{r}\cdot\pdt{\Psi}{\vec{r}} \\
\Psi &=& \alpha - \omega t + \vec{r}\cdot\vec{k}
\end{eqnarray*}
Comparing the rhs of previous two equations we get
\begin{eqnarray}
\omega &=& -\pdt{\Psi}{t} \label{c7e11} \\
\vec{k} &=& \pdt{\Psi}{\vec{r}} \label{c7e12}
\end{eqnarray}
Comparing the pair of equations \eqref{c7e9} and \eqref{c7e10} with the pair
\eqref{c7e11} and \eqref{c7e12} we extend the analogy $\Psi \leftrightarrow S$
to $\omega \leftrightarrow \mathcal{H}$ and $\vec{k} \leftrightarrow \vec{p}$.
It was left to the development of quantum mechanics to transform this analogy
into fundamental equations of physics.

\item Hamilton equations for the particles are
\begin{eqnarray*}
\td{\vec{p}}{t} &=& -\pdt{\mathcal{H}}{\vec{r}} \\
\td{\vec{r}}{t} &=& +\pdt{\mathcal{H}}{\vec{p}}.
\end{eqnarray*}
Their analogues to geometrical optics are
\begin{eqnarray}
\td{\vec{k}}{t} &=& -\pdt{\omega}{\vec{r}} \label{c7e13} \\
\td{\vec{r}}{t} &=& +\pdt{\omega}{\vec{k}} \label{c7e14}
\end{eqnarray}
In vacuum, $\omega = ck$ so that the first equation gives $\dot{\vec{k}} = 0$ 
while the second one gives $\vec{v} = \dot{\vec{r}} = c\un$, $\un$ being a unit
vector in the direction of propagation of the ray. Neither of these results are 
surprising but the analogy is quite effective when extended to propagation of 
light in material media.

\item From the analogy $\omega \leftrightarrow \mathcal{H}$ and $\vec{k} 
\leftrightarrow \vec{p}$, we can write the Lagrangian for a ray as
\[
\mathcal{L} = \vec{k}\cdot\pdt{\omega}{\vec{k}} - \omega.
\]
However, $\omega = ck$ gives $\mathcal{L} = 0$ identically. Therefore, we cannot
extend the principle of least action for particles in the form
\[
\delta\int Ldt = 0
\]
to geometrical optics. However, when the energy of a particle is constant the
principle of least action is equivalent to Maupertuis principle
\[
\delta\int\vec{p}\cdot d\vec{q} = 0.
\]
The analogue of Maupertuis principle to geometrical optics is
\begin{equation}\label{c7e15}
\delta\int\vec{k}\cdot d\vec{l} = 0.
\end{equation}
In vacuum, $\omega\un = c\vec{k}$ so that the equation simplifies to
\begin{equation}\label{c7e16}
\delta\int dl = 0,
\end{equation}
which is Euler's principle of least time.

\item Geometrical optics deals only with rays. It considers the propagation of a
bundle or a pencil of rays. The rays determine the direction of propagation and 
give no information about the intensity. If rays start from a certain region then
after a certain duration $\delta t$ the points $c\delta t$ away from the starting
region define a surface called the wave-front. An infinitesimal portion of the
wave-front has two principal radii of curvatures, say $R_1$ and $R_2$. In 
general, they are not the same. Therefore, the points on two principle circles 
of curvature appear to emerge from different points. This is the cause of the 
aberration of astigmatism.

The area of the wave-front is proportional to $R_1R_2$ while the intensity is
inversely proportional to it. Thus,
\begin{equation}\label{c7e17}
I \propto \frac{1}{R_1R_2}.
\end{equation}

Note that this formula, in particular the two radii of curvature, describe only 
the chosen infinitesimal surface. Therefore \eqref{c7e17} is not applicable to
areas illuminated by a different pencil of rays.

The intensity of light is proportional to the squared modulus of the field.
Therefore, the field itself is given by
\begin{equation}\label{c7e18}
f \propto \frac{1}{\sqrt{R_1R_2}} e^{ikR},
\end{equation}
where $R$ is any one of $R_1$ or $R_2$. Since $R_1$ and $R_2$ are fixed, the only
difference between the formulae using them is the constant phase 
$\exp(ik(R_1-R_2))$, which can be readily absorbed in the constant of 
proportionality.

In the special case of the two radii of curvature being the same, the wavefront
is a sector of a sphere and two centres of curvature coincide. 

$I$ blows up with either $R_1$ or $R_2$ vanish. Therefore the centres of 
curvature are points of infinite intensity. The locus of all centres of 
curvatures define a caustic surface. Refer to 
\href{https://physics.stackexchange.com/a/256561/10236}{a StackExchange answer} 
for an excellent explanation of the how caustics are formed and the rays are 
tangential to it.

\item \emph{Optical systems} are transparent bodies through which light travels.
In general, the direction of travel is changed when light passes through optical
systems. The laws governing the change of direction of rays are similar to those
governing the change of direction of particles when they pass through 
electromagnetic (and gravitational) fields.

\item We observed in point 7 above that the eikonal equation \eqref{c7e6} is
equivalent to \eqref{c7e7} with lhs zero (which is probably the reason why the 
differential equation $|\nabla u|^2 = 1$ is called the eikonal equation). Thus,
the eikonal $\Psi$ satisfies
\[
\pdt{\Psi}{x^\mu}\pdt{\Psi}{x_\mu} = 0
\]
or
\[
\frac{1}{c^2}\left(\pdt{\Psi}{t}\right)^2 - \left(\pdt{\Psi}{x}\right)^2 - 
\left(\pdt{\Psi}{y}\right)^2 - \left(\pdt{\Psi}{z}\right)^2 = 0.
\]
For a monochromatic wave, $\Psi = -\omega t + \psi_0(\vec{r})$ so that
\[
\pdt{\Psi}{t} = \omega
\]
and hence the eikonal equation becomes
\[
\frac{\omega^2}{c^2} = |\nabla\psi_0|^2.
\]
Define $\psi = c\psi_0/\omega$ so that the eikonal equation for monochromatic
ray becomes
\begin{equation}\label{c7e19}
|\nabla\psi|^2 = 1.
\end{equation}
The solution of \eqref{c7e19} gives the eikonal $\psi$. The negative gradient 
of the eikonal gives the direction of motion. This fact follows from the 
observation that the eikonal of a plane, monochromatic wave is $\omega t - 
\vec{k}\cdot\vec{r}$, whose gradient is $-\vec{k}$.

\item The eikonal is the phase of the rays passing through a definite point.
While studying optical systems one is more interested in the rays passing
through a pair of points. The corresponding eikonal is $\psi(\vec{r}, 
\vec{r}^\op)$ and it is the phase difference of the ray between the points
$\vec{r}$ and $\vec{r}^\op$. For a fixed $\vec{r}^\op$, the eikonal $\psi
(\vec{r}, \vec{r}^\op)$ describes all rays passing through $\vec{r}^\op$. Since
the eikonal must satisfy \eqref{c7e19}, we must have
\begin{equation}\label{c7e20}
|\nabla\psi|^2 = 1 \text{ and } |\nabla^\op\psi|^2 = 1,
\end{equation}
where $\nabla^\op$ denotes gradient with respect to the primed coordinates. 
The direction of rays getting out of $\vec{r}$ is $-\nabla\psi$ and that of
rays getting into $\vec{r}^\op$ is $-(-\nabla^\op\psi) = \nabla^\op\psi$. The
eikonal equation guarantees that these vectors have unit magnitude. They may,
therefore, be denoted as $\un$ and $\un^\op$ respectively.

\item The four vectors $\vec{r}, \vec{r}^\op, \un, \un^\op$ are related by
the requirement that $\un$ and $\un^\op$ are gradients of a certain function
of $\vec{r}$ and $\vec{r}^\op$ which also satisfy equations \eqref{c7e20}. In
order to obtain a relation between the four vectors it is convenient to 
introduce another function $\chi$, called the \emph{angular eikonal}, which is
a Legendre transform of $\psi$. Since $\psi$ is a function of $\vec{r}$ and
$\vec{r}^\op$,
\begin{eqnarray*}
d\psi &=& \nabla\psi\cdot d\vec{r} + \nabla^\op\psi\cdot d\vec{r}^\op \\
 &=& -\un\cdot d\vec{r} + \un\cdot d\vec{r}^\op \\
 &=& -d(\un\cdot d\vec{r}) + \vec{r}\cdot d\un + d(\un\cdot d\vec{r}^\op) -
     \vec{r}^\op d\un^\op
\end{eqnarray*}
so that
\[
d(\psi + \un\cdot\vec{r} - \un^\op\cdot\vec{r}^\op) = 
\vec{r}\cdot d\un - \vec{r}^\op d\un^\op.
\]
The function,
\begin{equation}\label{c7e21}
\chi(\un, \un^\op) = \psi + \un\cdot\vec{r} - \un^\op\cdot\vec{r}^\op
\end{equation}
is the angular eikonal. The function $\chi$ is not required to satisfy any
differential equation. However, its arguments have to be of unit magnitude.
Therefore, the six variables on which it depends are not independent and two
of them can be expressed in terms of others as
\begin{eqnarray}
n_x &=& \sqrt{1 - n_y^2 - n_z^2} \label{c7e22} \\
n_x^\op &=& \sqrt{1 - {n_y^\op}^2 - {n_z^\op}^2} \label{c7e23}.
\end{eqnarray}
From these equations we have
\begin{eqnarray}
dn_x &=& -\frac{n_y}{n_x}dn_y - \frac{n_z}{n_x}dn_z \label{c7e24} \\
dn_x^\op &=& -\frac{n_y^\op}{n_x^\op}dn_y^\op - 
              \frac{n_z^\op}{n_x^\op}dn_z^\op \label{c7e25}
\end{eqnarray}
so that 
\begin{eqnarray}
d\chi &=& -\left(y - \frac{n_y}{n_x}x\right) dn_y 
  -\left(z - \frac{n_z}{n_x}x\right) dn_z \nonumber \\
 & & -\left(y^\op - \frac{n_y^\op}{n_x^\op}x\right) dn_y^\op 
  -\left(z^\op - \frac{n_z^\op}{n_x^\op}x\right) dn_z^\op \label{c7e26}
\end{eqnarray}
From these equations, it follows that
\begin{eqnarray}
\pdt{\chi}{n_y} &=& -\left(y - \frac{n_y}{n_x}x\right) \label{c7e27} \\
\pdt{\chi}{n_z} &=& -\left(z - \frac{n_z}{n_x}x\right) \label{c7e28} \\
\pdt{\chi}{n_y^\op} &=& \left(y-\frac{n_y^\op}{n_x^\op}x\right) \label{c7e29}\\
\pdt{\chi}{n_z^\op} &=& \left(z-\frac{n_z^\op}{n_x^\op}x\right) \label{c7e30}
\end{eqnarray}
The angular eikonal $\chi$ depends on the properties of the optical system. For
fixed values of $\un$ and $\un^\op$ these are equations of straight lines. They
are the incident and refracted rays. Equations \eqref{c7e27} to \eqref{c7e29}
determine the path of light ray passing through an optical system described by
the angular eikonal $\chi$.

\item A bundle of rays passing through a single point is said to be 
homocentric. In general, a homocentric bundle ceases to be so after passing 
through an optical system except in the trivial case of a plane mirror. A
homocentric bundle remains approximately homocentric for sufficiently narrow
beam travelling close to a particular line for a given optical system. This 
line is called the \emph{optic axis} and the rays travelling close to it are
called \emph{paraxial rays}.

\item Even arbitrarily narrow bundles of rays are not homocentric, in general.
Consider an infinitesimally small wave surface of an arbitrarily narrow bundle
of rays. In general, it will have two radii of curvature and will be crossed by
two principle circles. Rays passing through each of them will pass through its
centre of curvature. Only when the wave surface is spherical will two centres of
curvature merge and the bundle of rays be homocentric.

\item A surface of revolution about the optic axis will have identical radii of
curvature for points on the axis. This follows from the smoothness of the
surface of revolution. A smooth surface will approximately a paraboloid in the 
neighbourhood of the optic axis. By symmetry, the two radii of curvature will be
equal. This fact can be rigorously proved using the techniques of differential
geometry.

\item We will now use equations \eqref{c7e27} to \eqref{c7e30} for determining
image formation by paraxial rays. For sake of definiteness, let us align the $x$
axis along the optic axis. Since the rays are almost along the optic axis, the
$x$ component of $\un$ and $\un^\op$ are very large as compared to the other
components. We can, therefore, expand $\chi$ around the origin. Since $\chi$ has
to describe a system symmetric about the $x$ axis, it must be even-powered in its
arguments. To lowest order in the arguments, we can express $\chi$ as
\begin{equation}\label{c7e31}
\chi = \text{const.} + \frac{g}{2}(n_y^2 + n_z^2) + f(n_yn_y^\op+n_zn_z^\op)
+ \frac{h}{2}({n_y^\op}^2 + {n_z^\op}^2).
\end{equation}
Substituting it in equations \eqref{c7e27} to \eqref{c7e30}, we get
\begin{eqnarray*}
y &=& n_y\left(\frac{x}{n_x} - g\right) - fn_y^\op \\
z &=& n_z\left(\frac{x}{n_x} - g\right) - fn_z^\op \\
y^\op &=& n_y^\op\left(\frac{x^\op}{n_x^\op} + h\right) + fn_y \\
z^\op &=& n_z^\op\left(\frac{x^\op}{n_x^\op} + h\right) + fn_z
\end{eqnarray*}
The component $n_x^\op \approx 1$ for a lens and $\approx -1$ for a 
mirror. If we consider the optical system to be a lens then the above equations 
become
\begin{eqnarray}
y &=& n_y(x - g) - fn_y^\op \label{c7e32} \\
z &=& n_z(x - g) - fn_z^\op \label{c7e33} \\
y^\op &=& n_y^\op(x^\op + h) + fn_y \label{c7e34} \\
z^\op &=& n_z^\op(x^\op + h) + fn_z \label{c7e35}
\end{eqnarray}

\item Consider a homocentric bundle of rays starting from $(x, y, z)$ and
converging at $(x^\op, y^\op, z^\op)$ after passing through the optical system.
If the group of equations \eqref{c7e32} to \eqref{c7e35} were independent then
we would get exactly one $n_y, n_z, n_y^\op, n_z^\op$ which would describe only
one ray and not a whole bundle of them. In order to accommodate the entire
bundle we assume that the equations are not independent. That is, we assume that
the ratio of coefficients of like terms are equal. This gives us,
\begin{equation}\label{c7e36}
\frac{x - g}{f} = -\frac{f}{x^\op + h} = \frac{y}{y^\op} = \frac{z}{z^\op}
\end{equation}
one consequence of which is
\begin{equation}\label{c7e37}
(x - g)(x^\op + h) = -f^2.
\end{equation}
The points $x = g$ and $x^\op = -h$ are called the principal foci and $f$ is 
called the principal focal length of the optical system. When $x^\op = h$, the 
first equation in \eqref{c7e36} tells that $x$ must be $\infty$. Likewise, when 
$x^\op = -h$, $x = g$. When either $x$ or $x^\op$ are at infinity, the incident 
or the refracted rays are effectively parallel to the optic axis. In this 
scheme of things, the light ray travels from right to left.

We now choose the respective foci as the origins for the object and the image 
regions. The new coordinates $(X, Y, Z)$ and $(X^\op, Y^\op, Z^\op)$ are related
to the old coordinates as
\begin{equation}\label{c7e38}
X = x - g, Y = y, Z = z, X^\op = x^\op + h, Y^\op = y^\op, Z^\op = z^\op.
\end{equation}
In terms of these coordinates, equations \eqref{c7e36} become
\begin{equation}\label{c7e39}
\frac{X}{f} = -\frac{f}{X^\op} = \frac{Y}{Y^\op} = \frac{Z}{Z^\op}
\end{equation}
and \eqref{c7e37} simplifies to
\begin{equation}\label{c7e40}
XX^\op = -f^2.
\end{equation}
The ratio $Y^\op/Y$ is called the \emph{lateral magnification} of the optical
system. The variables $X$ and $X^\op$ are not proportional to each other in
the set of equations \eqref{c7e39}. Therefore, we define longitudinal 
magnification as $dX^\op/dX$. From \eqref{c7e40},
\[
X^\op = -\frac{f^2}{X}
\]
so that
\begin{equation}\label{c7e41}
\td{X^\op}{X} = \frac{f^2}{X^2} = \left(\frac{Y^\op}{Y}\right)^2.
\end{equation}
where we used the first and the third terms in \eqref{c7e39} to get the last
equality. This shows that it is impossible to get an undistorted image through
an optical system. There is always a distortion in the way the image is 
magnified in the two directions.

\item If $X = f$ then from equation \eqref{c7e40}, $X^\op = -f$. Thus, a bundle
of rays crossing the optic axis at $X = f$ crosses it again at $X^\op = -f$.
The pair of points $X$ and $X^\op$ are called the principal points. In the
old coordinate system, these points are $x = f + g$ and $x^\op = -f + h$ (refer
to \eqref{c7e38}) respectively.

\item If we measure the coordinates of the object and the image from the
principal points, that is if $\xi = X - f$ is the position of the object and
$\xi^\op = X^\op + f$ that of the image then from \eqref{c7e40},
\[
(\xi + f)(\xi^\op - f) = -f^2 \Rightarrow \xi\xi^\op - f\xi + f\xi^\op = 0
\]
from which we get
\begin{equation}\label{c7e42}
\frac{1}{\xi} - \frac{1}{\xi^\op} = -\frac{1}{f}.
\end{equation}

\item We continue to work in a setup in which light travels from right to left.
If $f > 0$ and if $X > f$, that is the object is in front of the focus so that
$\xi > 0$ then from the portion
\[
\frac{X}{f} = \frac{Y}{Y^\op}
\]
of \eqref{c7e39} we see that $Y^\op/Y > 0$, that is, the image is erect. Such
an optical system is said to be \emph{converging}. Under the same setup, if
$f < 0$ then $Y^\op/Y < 0$, that is, the image is inverted and the optical 
system is said to be diverging.

\item We start with \eqref{c7e37} and approximate it when $f, g, h$ are very 
large.
\[
x(x^\op +h) - g(x^\op + h) = -f^2
\]
When $h$ is very large, $x^\op + h \approx h$ so that
\[
hx - gx^\op - gh \approx -f^2
\]
so that
\begin{equation}\label{c7e43}
x^\op \approx \frac{h}{g}x + \frac{f^2 - gh}{g}.
\end{equation}
Let
\begin{equation}\label{c7e44}
\alpha^2 = h/g; \beta = (f^2 - gh)/g.
\end{equation}
so that \eqref{c7e43} becomes
\begin{equation}\label{c7e45}
x^\op = \alpha^2x + \beta.
\end{equation}
Since
\[
\td{x^\op}{x} = \alpha^2,
\]
we have from \eqref{c7e41}
\[
\frac{y^\op}{y} = \frac{z^\op}{z} = \pm\alpha.
\]
Using transformations in \eqref{c7e38} we can easily show that
\begin{equation}\label{c7e46}
\td{X^\op}{X} = \alpha^2; \frac{Y^\op}{Y}=\pm\alpha;\frac{Z^\op}{Z}=\pm\alpha;
\end{equation}
The longitudinal and transverse magnifications are constants. Such an image
formation is called \emph{telescopic}.
\end{enumerate}

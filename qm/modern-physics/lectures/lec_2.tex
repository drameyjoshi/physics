\documentclass{beamer}
\usefonttheme[onlymath]{serif}
\usepackage{graphicx,hyperref}
\hypersetup{colorlinks,urlcolor=blue}
\usetheme{Boadilla}

\newcommand{\pvec}[1]{\vec{#1}\mkern2mu\vphantom{#1}}

\title{Special Theory of Relativity - 1}
\author{Amey Joshi}
\date{\today}
\begin{document}
\begin{frame}
\titlepage
\end{frame}

\begin{frame}
\frametitle{Maxwell's theory}
\begin{itemize}
\item In the mid-nineteenth century, Maxwell unified the separate branches of
electricity and magnetism into a single electromagnetic theory. A consequence
of it was the existence of electromagnetic waves which could travel at the
speed of $3 \times 10^8$ ms${}^{-1}$ in vacuum. Hertz demonstrated the 
existence of these waves in 1884.
\item Physicists did not accept the idea of waves travelling in vacuum. After 
all, the undulations of a wave must be in some material. Therefore, they 
proposed the existence of a substance called aether. It was supposed to have
zero density and perfect transparency.
\item Recall from previous lecture that Newton's laws had the same mathematical
form in all inertial frames of reference. However, this quality was found to
be lacking in Maxwell's equation and the equation of electromagnetic waves.
\item This means that one can use electromagnetic phenomena to detect uniform
motion. 
\end{itemize}
\end{frame}

\begin{frame}
\frametitle{How swift is a river? - 1}
Consider a river basin of width $D$ and assume that water flows at a speed $u$.
How will you measure $u$? You have a boat with a speedometer and an odometer. 
You can
\begin{enumerate}
\item Sail downstream for a distance $D$ and return. If $v$ is the boat's speed,
the total time is
\begin{equation}\label{e1}
t_1 = \frac{D}{v + u} + \frac{D}{v - u} = \frac{2Dv}{v^2 - u^2} = 
\frac{2D/v}{1 - u^2/v^2}.
\end{equation}
\item If you want to land on the opposite shore, you must sail with a velocity
$\vec{v} = v^\prime\hat{j} - u\hat{i}$ on onward journey. $v$ is the speed you 
measure on the speedometer. But the time taken to cross the river is determined 
by $v^\prime$. Since $v^2 = {v^\prime}^2 + u^2 \Rightarrow v^\prime = 
\sqrt{v^2 - u^2}$ and time taken for onward journey is $D/v^\prime = 
D/\sqrt{v^2 - u^2}$. The same time is taken by the return journey so that the 
total time is
\begin{equation}\label{e2}
t_2 = \frac{2D}{\sqrt{v^2 - u^2}} = \frac{2D/v}{\sqrt{1 - u^2/v^2}}.
\end{equation}
\end{enumerate}
\end{frame}

\begin{frame}
\frametitle{How swift is a river? - 2}
From equations \eqref{e1} and \eqref{e2}
\begin{equation}\label{e3}
\frac{t_2}{t_1} = \sqrt{1 - \frac{u^2}{v^2}}.
\end{equation}
You can get $u$ if you measure $v, t_1$ and $t_2$. 
\begin{enumerate}
\item Now substitute the river with the mythical aether and the boat with a 
beam of light and you can find the speed of aether.
\item Is the aether flowing? Yes, relatively speaking. The earth moves through
it at a speed of $3 \times 10^4$ ms${}^{-1}$ so that we see the aether going
past at the same speed.
\item This idea is not fanciful. This is exactly what Michelson's interferometer
does.
\end{enumerate}
\end{frame}

\begin{frame}
\frametitle{Michelson interferometer}
A schematic diagram is available on 
\href{https://en.wikipedia.org/wiki/Michelson_interferometer}{the internet}.
Formulae in equations \eqref{e1} and \eqref{e2} hold good except with the 
replacement of $v$ by $c$ so that
\begin{equation}\label{e4}
t_2 - t_1 = \frac{2D}{c}\left(\frac{1}{\sqrt{1 - u^2/c^2}} - 
                              \frac{1}{1-u^2/c^2}\right)
\end{equation}
Since the two light beams take a different time to reach the detector from the
source, they end up having a path difference of $c(t_2 - t_1)$ leading to a 
fringe shift of 
\begin{equation}\label{e5}
n = \frac{c(t_2 - t_1)}{\lambda} = 
  \frac{2D}{\lambda}\left(\frac{1}{\sqrt{1 - u^2/c^2}} - 
                          \frac{1}{1-u^2/c^2}\right)
\end{equation}
If you rotate the interferometer by $90^\circ$ the net fringe shift is $2n$.
\end{frame}

\begin{frame}
\frametitle{Special theory of relativity}
\begin{enumerate}
\item No fringe shift was observed. There is no `river' and the speed of light
is the same in both directions.
\item Michelson got a Nobel prize in 1907 for this discovery.
\item The constancy of the speed of light in spite of the earth's motion 
suggested that the Galilean relativity is not applicable to electromagnetic 
phenomena.
\item Einstein proposed a new theory of relativity in 1905 based on the two 
postulates
\begin{enumerate}
\item The laws of physics remain the same in all \emph{inertial} frames of
reference and
\item The speed of light is a universal constant for all observers.
\end{enumerate}
\item It is called the special theory of relativity because it specialises to
the inertial frame. In 1916 Einstein built a theory of relativity for all
frames of reference. It is now called the general theory of relativity.
\end{enumerate}
\end{frame}

\begin{frame}
\frametitle{Time dilation - 1}
\begin{enumerate}
\item Einstein's postulates have stunning consequences.
\item Construct a clock based on constancy of $c$. Fix a rigid rod of length
$L_0$ with a source and detector of light at one end and a mirror at another 
end.
\item Light travels a distance $2L_0$ before getting detected. The time for 
travel is $t_0 = 2L_0/c$.
\item Align the clock along the $y$ axis and let it move along the $x$ axis
with speed $v$. The path length from source to detector is now
\[
2\sqrt{L_0^2 + \frac{v^2t^2}{4}}
\]
so that the elapsed time is
\begin{equation}\label{e6}
t = \frac{2}{c}\sqrt{L_0^2 + \frac{v^2t^2}{4}} \Rightarrow 
t^2 = 4\frac{L_0^2}{c^2} + \frac{v^2}{c^2}t^2.
\end{equation}
\end{enumerate}
\end{frame}

\begin{frame}
\frametitle{Time dilation - 2}
\begin{enumerate}
\item Introduce the quantity $\beta = v/c$ so that equation \eqref{e6} can be
rewritten as
\[
(1 - \beta^2)t^2 = \left(\frac{2L_0}{c}\right)^2 
\]
or
\begin{equation}\label{e7}
t = \frac{t_0}{\sqrt{1 - \beta^2}}.
\end{equation}
\item If $v < c$, $\beta < 1$ and hence $t > t_0$. A pulse of light that took
 a time $t_0$ when the clock was at rest took a longer time when it was moving.
\item Moving clocks appear to go slow! This was not the case in Galilean
relativity
\item Note the presence of $\beta^2$.
\item If time interval changed how can we be sure that length of the rod didn't
change? Wait for a few slides to get an answer. 
\end{enumerate}
\end{frame}

\begin{frame}
\frametitle{Length contraction - 1}
\begin{enumerate}
\item Measure length in terms of time taken for a light pulse to travel up and
down the length of an object, say a rod. When the rod is at rest, if $t_0$ is
the time for back and forth journey of a light source 
\begin{equation}\label{e8}
L_0 = ct_0.
\end{equation}
\item Set the rod in motion with velocity $\vec{v} = v\hat{i}$ and measure its
length in the `lab frame'. Fire a light pulse from one end in the direction of
motion.
\item If $t_1$ is the time for onward journey then
\begin{equation}\label{e9}
t_1 = \frac{L + vt_1}{c} \Rightarrow t_1 = \frac{L/c}{1 - \beta},
\end{equation}
where $\beta = v/c$
\end{enumerate}
\end{frame}

\begin{frame}
\frametitle{Length contraction - 2}
\begin{enumerate}
\item If $t_2$ is the time for return journey then
\begin{equation}\label{e10}
t_2 = \frac{L - vt_2}{c} \Rightarrow t_2 = \frac{L/c}{1 + \beta}.
\end{equation}
so that the total time is
\begin{equation}\label{e11}
t = t_1 + t_2 = \frac{2L/c}{1 - \beta^2}.
\end{equation}
Using equation \eqref{e7}, $t_0 = t\sqrt{1 - \beta^2}$ so that,
\begin{equation}\label{e12}
t_0 = \frac{2L/c}{\sqrt{1 - \beta^2}} \Rightarrow ct_0 = 2L_0 = \frac{2L}{\sqrt{1 - \beta^2}}
\Rightarrow L = L_0\sqrt{1 - \beta^2}.
\end{equation}

\item If $v < c$, $\beta < 1$ and hence $L < L_0$.
\end{enumerate}
\end{frame}

\begin{frame}
\frametitle{Are these effects real?}
\begin{enumerate}
\item $\mu$ mesons are produced in the upper atmosphere, approximately $10$ km
above the sea level. They are found to reach the sea level in great numbers.
\item They have a half-life of $2$ $\mu$s and a speed of $0.998c$. Half of them
will vanish in $2 \times 10^{-6} \times 0.996c = 600$m. How come so many make it
to the sea level? 
\item For an observer on ground, $\mu$ meson's half life is dilated to 
\[
\frac{2 \times 10^{-6}}{\sqrt{1 - 0.998^2}} = 31.7 \times 10^{-6},
\]
roughly a $16$ fold increase. Therefore, they travel $600m \times 16 = 9.6$ km
before half of them disintegrate.
\item Analyse the situation in the $\mu$ meson's frame of reference.
\end{enumerate}
\end{frame}

\begin{frame}
\frametitle{Length contraction in transverse directions?}
\begin{enumerate}
\item A body appears shrunk in the direction of motion. Are its other dimensions
affected?
\item Consider two rings R$_{1}$ and R$_{2}$ moving at speed $v$ with respect
to the ground in opposite directions along their axis of symmetry. Assume that
the two axes coincide.
\item The ground observer will see them collide.
\item If transverse lengths are also shrunk then R$_{1}$ will see R$_{2}$ shrunk
and therefore will cross it without collision finding it passing through its
inside.
\item Likewise R$_{2}$ will find R$_{1}$ pass through its inside.
\item The same experiment gives three different outcomes in three inertial
frames of reference, violating Einstein's first postulate.
\end{enumerate}
\end{frame}

\begin{frame}[fragile]
\frametitle{Homework}
\begin{itemize}
\item Problem sets 1, 2 and 3 in chapter 1 of the accompanying notes.
\item Submit your assignments within one week of this lecture. Upload your 
files with the name \begin{verbatim}<your-name><hw1>.pdf\end{verbatim}
\end{itemize}
\end{frame}
\end{document}


\documentclass{beamer}
\usefonttheme[onlymath]{serif}
\usepackage{graphicx, hyperref}
\hypersetup{colorlinks, urlcolor=blue}
\usepackage{subcaption}
\usetheme{Boadilla}

\newcommand{\pvec}[1]{\vec{#1}\mkern2mu\vphantom{#1}}

\title{Quantum Mechanics}
\author{Amey Joshi}
\date{\today}

\begin{document}
\begin{frame}
\titlepage
\end{frame}

\begin{frame}
\frametitle{Two formulations of Quantum Mechanics}
\begin{enumerate}
\item Werner Heisenberg gave the first formulation of a new mechanics in 1925 
forelectrons in which dynamical variables were represented as matrices. It is 
called matrix mechanics.
\item Erwin Schr\"{o}dinger gave an alternative formulation in 1926 in which
the dynamical variables were represented as operators acting on a `wave 
function'. The `wave function' itself obeyed a `wave equation'. It is called
wave mechanics. He further showed that matrix mechanics and wave mechanics are
equivalent.
\item 
\end{enumerate}
\end{frame}

\end{document}


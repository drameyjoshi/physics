\documentclass{article}
\usepackage{amsmath}
\newcommand{\un}{\hat{n}}
\newcommand{\uv}[1]{\hat{e}_{#1}}
\newcommand{\grad}[1]{\nabla{#1}}
\newcommand{\dive}[1]{\nabla\cdot\vec{#1}}
\newcommand{\curl}[1]{\nabla\times\vec{#1}}
\newcommand{\op}{\prime}
\newcommand{\pdt}[2]{\frac{\partial{#1}}{\partial{#2}}}
\newcommand{\ke}{\frac{1}{4\pi\epsilon_0}}
\title{The Scalar Potential}
\author{Amey Joshi}
\date{15-Mar-2024}
\begin{document}
\maketitle
\begin{enumerate}
\item For a static electric field, $\curl{E} = 0$. Since curl of a gradient is
always zero, we can write
\begin{equation}\label{e1}
\vec{E} = -\grad\varphi,
\end{equation}
for a scalar function $\varphi$. 

\item The force on a charged particle depends on $\vec{E}$ and not $\varphi$. 
One can add a constant to $\varphi$ to get a different potential that gives 
the same field. Thus, the potential is determined to within a constant.

\item $\curl{E} = 0$ guarantees that
\begin{equation}\label{e2}
\oint\vec{E}\cdot d\vec{l} = 0
\end{equation}
so that the line integral,
\[
\int_A^B \vec{E}\cdot d\vec{l}
\]
is independent of the path along which the integral is taken. Using equation 
\eqref{e1}
\begin{equation}\label{e3}
\int_A^B \vec{E}\cdot d\vec{l} = -\int_A^B\grad{\varphi}\cdot d\vec{l} = 
-(\varphi(B) - \varphi(A)).
\end{equation}

\item It is often easier to compute $\varphi$ and then use \eqref{e1} to compute
the field. 

\item Consider a static electric field produced by a immobile charges. Let a
charge $q$ be moved in the electrostatic field of the immobile charges from a 
point $A$ to a point $B$. Furthermore, assume that the motion is quasi-static
so that the charge $q$ never accelerates. The force $\vec{F}$ that can cause
this motion should just balance the electric force $q\vec{E}$. That is,
\[
\vec{F} = -q\vec{E}.
\]
Work done by $\vec{F}$ is
\begin{equation}\label{e4}
W(B; A) = \int_A^B\vec{F}\cdot d\vec{l} = -q\int_A^B\vec{E}\cdot d\vec{l} = 
q\int_A^B\grad{\varphi}\cdot d\vec{l} = q\varphi(B) - q\varphi(A).
\end{equation}
The terms $q\varphi(B)$ and $q\varphi(A)$ are often interpreted as values of the
potential energy of the charged particle at points $B$ and $A$ respectively.

\item Oftentimes, the point $A$ is at infinity when the charges creating the
electric field are confined to a finite volume in space. In that case, it is 
convenient to let $\varphi(A) = 0$.

\item Since $\vec{E} = -\grad\varphi$, Gauss' law becomes
\begin{equation}\label{e5}
\nabla^2\varphi = -\frac{\rho}{\epsilon_0}.
\end{equation}
This is Poisson's equation. If there is no charge density in a region, it reduces 
to
\begin{equation}\label{e6}
\nabla^2\varphi = 0.
\end{equation}
This is Laplace's equation. Unlike ordinary differential equations, solutions of
partial differential equations show a great variety in their form. For example,
the real and imaginary part of every analytic function is a solution of Laplace's
equation in two dimensions.

\item Complex analytic functions are the smoothest of all. They have derivatives of
all orders. This is a property of the solutions of Laplace's equation in any number
of dimensions.

\item Another property of the solutions of Laplace's equation is that they do not
have a local extremum. For supposing they did at a point $\vec{r}_0$ then 
\[
\nabla^2\varphi(\vec{r}_0) \ne 0
\]
and since $\varphi$ is a smooth function, there is a neighbourhood around 
$\vec{r}_0$ where the function is non zero. If $V$ is such a neighbourhood, then
\[
\nabla^2\varphi \ne 0
\]
for all points in $V$. This is a contradiction if $\varphi$ is a solution to
Laplace's equation. This mathematical property has an important physical
consequence when $U(\vec{r}) = q\varphi(\vec{r})$ is interpreted as the potential
energy of a charged particle in an electric field $\vec{E} = -\grad{\varphi}$. 
If $\varphi$ does not have a local extremum then it means that $U$ too cannot 
have it. In particular, $U$ cannot have a local minimum. Therefore, a charged
particle cannot be in a stable equilibrium in a static electric field. This
results is called \emph{Earnshaw's theorem}.

\item Lines of electric fields are defined analogously to the streamlines in a
fluid. At each point in space, the element of the curve is parallel to the
electric field at the point. Thus, $d\vec{l} \times \vec{E} = 0$ or $E_ydz - 
E_zdy = 0; E_zdx - E_xdz = 0; E_xdy - E_ydx = 0$ or
\begin{equation}\label{e7}
\frac{dx}{E_x} = \frac{dy}{E_y} = \frac{dz}{E_z}.
\end{equation}
This is the equation for electric lines of force. The constant $\lambda$ is
usually taken to be positive so that the line of force is the trajectory of
a positively charged particle in an electric field.

\end{enumerate}
\end{document}
\chapter{Special Theory of Relativity}\label{c1}
\section{Galilean relativity}\label{c1s1}
The idea of relative motion is as old as Newton's mechanics. Consider two 
observers, $O$ and $O^\prime$ with their frames of reference $S$ and $S^\prime$.
If the axes of $S$ and $S^\prime$ are parallel to each other but the origin of
$S^\prime$ is displaced by $\vec{R}$ from the origin of $S$ then a point 
$\vec{r}$ in $S$ appears as 
\begin{equation}\label{c1s1e1}
\vec{r}^\prime = \vec{r} - \vec{R}
\end{equation}
in $S^\prime$. The two observers will disagree on the coordinates of a point. 
However, the disagreement will always be to the extent given in equation \eqref
{c1s1e1}. If the observer $O$ finds $O^\prime$ moving then their measurements 
of velocities of other bodies will not match. $O$ will measure $\vec{v} = 
\dot{\vec{r}}$, $O^\prime$ will measure $\vec{v}^\prime = \dot{\vec{r}}^\prime$
and the two will be related as
\begin{equation}\label{c1s1e2}
\vec{v}^\prime = \vec{v} - \vec{V},
\end{equation}
where $\vec{V} = \dot{\vec{R}}$ is the velocity of $O^\prime$ as measured by
$O$. Likewise, if $O$ finds that $O^\prime$'s velocity changes with time then
their measurements of accelaration will be related as
\begin{equation}\label{c1s1e3}
\vec{a}^\prime = \vec{a} - \vec{A},
\end{equation}
where $\vec{a} = \ddot{\vec{r}}, \vec{a}^\prime = \ddot{\vec{r}}$ and $\vec{A}
= \ddot{\vec{R}}$. If $O$ observes a body moving in a straight line with a
uniform velocity then $O^\prime$ sees it accelarating. This is the origin of
pseudo-forces. Equations \eqref{c1s1e1} are called Galilean transformations
and they described the Galilean relativity.

Galilean transformation does not explicitly mention that the mass of a body
does not change across frames of references. It was assumed to be obvious and
not worth mentioning. Likewise, it was also assumed that the time remains
unaffected by relative motion. Therefore, we also have
\begin{eqnarray}
t^\prime &=& t \label{c1s1e4} \\
m^\prime &=& m. \label{c1s1e5}
\end{eqnarray}
In the next section we will see that Einstein's theory proved that these tacit
assumptions are untenable.

\subsection{Problem set 1}\label{c1s1s1}
\begin{enumerate}
\item Alice is standing on the platform and sees Bob on a train passing at a
uniform speed $V = 40$ kmph. Bob rolls a ball on the floor of the train and
measures its speed to be $v^\prime = 5$ kmph. What will be Alice's measurement
of the ball's speed if the ball moves (a) in the direction of the train, 
(b) against the direction of the train and (c) sideways?
\item Bob is siting on a flat wagon of a train. The floor of the wagon is 
smooth and there is a ball placed on it. The train begins to move. Both Alice
and Bob observe that the ball rolls in a direction opposite to that of the 
train. How will Alice and Bob explain the motion of the ball?
\end{enumerate}

If the observer $O^\prime$ is moving at a uniform velocity with respect to $O$
then $\vec{A} = 0$ and $\vec{a}^\prime = \vec{a}$. Therefore, if $O$ observes
a body at rest then $O^\prime$ observes it to be moving at a \emph{uniform}
speed $\vec{v}^\prime = -\vec{V}$.

Newton's first law states that every body continues to be in a state of rest
or of uniform motion in a straight line unless a force acts on it. A frame of
reference in which Newton's first law holds good is called an \emph{inertial
frame of reference}. 

\subsection{Problem set 2}\label{c1s1s2}
\begin{enumerate}
\item If $S$ is an inertial frame and if $S^\prime$ moves at a uniform velocity
$\vec{V}$ then show that $S^\prime$ is also an inertial frame of reference.
\item If you ignore the earth's motion around its axis and around the sun then
the ground is an inertial frame of reference. An object at rest on the ground
remains so even if the ground is very smooth. Suppose you are standing on the
ground and observing a friend on a merry-go-round. You see the friend in a 
uniform circular motion. Although your friend's speed is constant the direction
of his motion changes at every moment. He is accelarating. (a) What force 
causes the accelaration? (b) Your friend feels that he is being pushed out.
Who is pushing him?
\end{enumerate}

If a body does not feel a force in one inertial frame of reference then it
does not feel it in any other frame of reference moving uniformly with respect
to the first frame. All inertial frames of reference are equivalent. Observers
in inertial frames may disagree on position and velocities of bodies but never
on their accelarations. Newton's laws are valid in all inertial frames of 
reference. As an example, consider two space ships $S$ and $S^\prime$ observing
the planets. Let $\vec{r}_1$ and $\vec{r}_2$ be their positions with respect
to $S$. Then, the force between them is
\begin{equation}\label{c1s1e6}
\vec{F} = -\frac{Gm_1m_2}{|\vec{r}_1 - \vec{r}_2|^3}(\vec{r}_1 - \vec{r}_2).
\end{equation}
Here $m_1$ and $m_2$ are the masses of the planets at positions $\vec{r}_1$
and $\vec{r}_2$ with respect to the origin of $S$ and $G$ is the universal
gravitational constant. The accelaration of mass 
$m_1$ is
\begin{equation}\label{c1s1e7}
\vec{a}_1 = -\frac{Gm_2} {|\vec{r}_1 - \vec{r}_2|^3}(\vec{r}_1 - \vec{r}_2).
\end{equation}
Observer $O^\prime$ will express this force as
\begin{equation}\label{c1s1e8}
\vec{F}^\prime = 
-\frac{Gm_1m_2}{|\vec{r}_1^\prime - \vec{r}_2^\prime|^3}
(\vec{r}_1^\prime - \vec{r}_2^\prime).
\end{equation}
and $m_1$'s acceleration as 
\begin{equation}\label{c1s1e9}
\vec{a}_1^\prime = -\frac{Gm_2} {|\vec{r}_1^\prime - \vec{r}_2^\prime|^3}
(\vec{r}_1^\prime - \vec{r}_2^\prime).
\end{equation}
Equation \eqref{c1s1e1} assures that the right hand sides of equations 
\eqref{c1s1e6} and \eqref{c1s1e8} are the same. The inertial nature of the
frames $S$ and $S^\prime$ assures that the left hand sides are also the same.
The law of motion, Newton's second law, is valid in both frames of reference.
The governing equation are identical, albeit with different variables.

Continuing with the example of the two planets, if $S^\prime$ was accelarating
with respect to $S$ then $\vec{a}_1^\prime$ would have been $\vec{a} - \vec{A}$
and equation \eqref{c1s1e9} would have been
\begin{equation}\label{c1s1e10}
\vec{a}_1^\prime = -\frac{Gm_2} {|\vec{r}_1^\prime - \vec{r}_2^\prime|^3}
(\vec{r}_1^\prime - \vec{r}_2^\prime) + \vec{A}.
\end{equation}
The form of this equation is not the same as \eqref{c1s1e7}. 

Laws of physics are expressed in a mathematical form. When we say that Newton's
laws remain unchanged across inertial frames of reference we mean that their
mathematical form remains unchanged. The variables in one frame are replaced
with the variables in another frame but there are no extra or fewer terms. If
one were to do an experiment in any of these frames one will get identical
results.

When the laws of physics change, one can devise experiments to observe the 
change. For instance, in the accelarating frame in which we got \eqref{c1s1e10}
we can do an experiment whose results will differ from those in the inertial
frame $S$. The reason for the difference is the additional term on the right
hand side of \eqref{c1s1e10}.

We will illustrate how Newton's laws are affected in a non-inertial frame
of reference. We will analyze the circular motion of a person sitting in a 
merry-go-round. From the viewpoint of an observer $S$ on the ground - an 
inertial frame of reference - the person in the merry-go-round, $S^\prime$ is 
accelarating. Therefore, he expects that there is a net force acting on 
$S^\prime$. He lists the forces on $S^\prime$.
\begin{enumerate}
\item $S^\prime$'s weight,
\item The tension in the rod that bears $S^\prime$'s weight,
\item The tension in the rod that connects $S^\prime$ to the merry-go-round's
axis.
\end{enumerate}
$S$ concludes that the first two forces balance each other because $S^\prime$
does not undergo a vertical motion. The third force is responsible for 
$S^\prime$'s accelaration. He writes Newton's second law as
\begin{equation}\label{c1s1e11}
m\frac{v^2}{r} = T,
\end{equation}
where $m$ is $S^\prime$'s mass, $v$ is his speed, $r$ is the radius of the
merry-go-round and $T$ is the tension in the rod that connects $S^\prime$ to
the merry-go-round's axis.

We can also confirm that Newton's third law is valid in $S$'s frame of 
reference. For each of three forces on $S^\prime$ enumerated above, $S^\prime$
also exerts a force of reaction. $S^\prime$ pulls the earth towards him and he
also pulls the two rods, vertical and horizontal, towards him.

Let us now analyze the same experiment from the non-inertial frame of 
$S^\prime$. He includes the same forces acting on his body as $S$ and he
realizes that the net force on him is the horizontal tension $T$. However, he 
is at rest in his frame of reference. So concludes that Newton's first law has
failed. That is why, he further concludes, he is in a non-inertial frame of
reference. Newton's second law also fails because his accelaration is zero
in spite of a horizontal force acting on him. In order to save Newton's second
law he introduces a pseudo-force, the centrifugal force, that exactly balances
the horizontal tension $T$. The horizontal tension pulls him towards the
merry-go-round's axis and the centrifugal force pushes him away by an equal
amount, keeping him at rest.

\subsection{Problem set 3}\label{c1s1s3}
\begin{enumerate}
\item The introduction of centrifugal force allowed $S^\prime$ to extend
Newton's second law to his non-inertial frame. Nevertheless, show that Newton's
third law fails. (Hint: Find all action-reaction pairs.)
\end{enumerate}

For almost two hundred years, Newton's laws and the two types of frames of 
reference sufficed to explain all mechanical phenomena. In the later half of
the nineteenth century, Maxwell unified the fields of electricity and magnetism
in his theory of classical electrodynamics. He predicted the existence of 
electromagnetic waves and proposed that light is an electromagnetic wave. A
wave in three dimensions is the solution of the partial differential equation
of the form
\begin{equation}\label{c1s1e12}
\frac{\partial^2\phi}{\partial x^2} + \frac{\partial^2\phi}{\partial y^2} +
\frac{\partial^2\phi}{\partial z^2} = 
\frac{1}{c^2}\frac{\partial^2\phi}{\partial t^2},
\end{equation}
where the function $\phi = \phi(x, y, z, t)$ represents the oscillating 
quantity and $c$ is the speed of the wave. We will now show that the wave
equation does not retain its form across inertial frames of reference. Consider
two inertial frames of reference $S$ and $S^\prime$. If $S^\prime$ moves at a
speed $v$ along the $x$-axis of $S$ then coordinates in the two frames are
related as
\begin{eqnarray}
x^\prime &=& x - vt \label{c1s1e13} \\
y^\prime &=& y \label{c1s1e14} \\
z^\prime &=& z \label{c1s1e15} \\
t^\prime &=& t \label{c1s1e16}
\end{eqnarray}
We will show in the next problem set that the wave equation \eqref{c1s1e12}
does not retain its form under the transformation of \eqref{c1s1e13} to
\eqref{c1s1e15}. We can, therefore, devise an experiment that detects the
motion of $S^\prime$.

It was believed that there is a certain frame of reference of the fixed stars
that is a true inertial frame and all motion can be measured with respect to it.
The observation that the wave equation does not retain its form across inertial
frames of reference suggested that we can experimentally detect the motion of
the frame of reference in which light travels at a speed $c = 2.998 \times 10^8
$ ms${}^{-1}$.

\subsection{Problem set 4}
\begin{enumerate}
\item If equation \eqref{c1s1e16} is replaced by $t^\prime = t + \alpha$, where
$\alpha$ is a constant, then show that time intervals are remain invariant.
\item Show that distance between two points remains invariant under the 
transformation of equations \eqref{c1s1e13} to \eqref{c1s1e16}. (Hint: Choose
two points $(x_1, y_1, z_1)$ and $(x_2, y_2, z_2)$ in $S$. Find their 
coordinates $(x_1^\prime, y_1^\prime, z_1^\prime)$ and $(x_2^\prime, 
y_2^\prime, z_2^\prime)$ in $S^\prime$. Use the standard formula to compute the
Euclidean distance between the points in the two frames.)
\item We will now show that the wave equation \eqref{c1s1e12} is not invariant
under the transformation defined by equations \eqref{c1s1e13} and 
\eqref{c1s1e16}.
\begin{enumerate}
\item In the frame $S^\prime$, $\phi$ is a function of $x^\prime, y^\prime, 
z^\prime, t^\prime$. Recall the definition of the 
`differential' of $\phi$,
\[
d\phi = \frac{\partial\phi}{\partial x^\prime}dx^\prime + 
\frac{\partial\phi}{\partial y^\prime}dy^\prime + 
\frac{\partial\phi}{\partial z^\prime}dz^\prime + 
\frac{\partial\phi}{\partial t^\prime}dt^\prime 
\]
Find
\[
\frac{\partial\phi}{\partial x} = \frac{\partial\phi}{\partial x^\prime}
\frac{\partial x^\prime}{\partial x} + \frac{\partial\phi}{\partial y^\prime}
\frac{\partial y^\prime}{\partial x} + \frac{\partial\phi}{\partial z^\prime}
\frac{\partial z^\prime}{\partial x} + \frac{\partial\phi}{\partial t^\prime}
\frac{\partial t^\prime}{\partial x}
\]
Hence find
\[
\frac{\partial^2\phi}{\partial {x}^2}.
\]
\item Get the expressions for
\[
\frac{\partial^2\phi}{\partial {y}^2}, \frac{\partial^2\phi}{\partial {z}^2},
\frac{\partial^2\phi}{\partial {t}^2}
\]
using similar steps.
\item Substitute the second partial derivatives obtained in the previous two
steps in the wave equation \eqref{c1s1e12}.
\end{enumerate}
\end{enumerate}

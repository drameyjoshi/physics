\documentclass{beamer}
\usepackage{bm}
\newcommand{\op}{{\;\prime}}
\newcommand{\grad}{\nabla\;}
\newcommand{\dive}{\nabla\cdot}
\newcommand{\curl}{\nabla\times}
\newcommand{\td}[2]{\frac{d{#1}}{d{#2}}}
\newcommand{\pdt}[2]{\frac{\partial{#1}}{\partial{#2}}}
\newcommand{\uv}[1]{\hat{e}_{#1}}
\newcommand{\un}{\hat{n}}

\usefonttheme[onlymath]{serif}
\title{Electricity and Magnetism}
\subtitle{Electrostatics - Charges and Fields \\ Day 1}
\author{Amey Joshi}
\begin{document}
\frame{\titlepage}

\begin{frame}
\frametitle{About the subject}
\begin{enumerate}
\item Classical E\&M deals with charges, currents and their interactions assuming that all quantities can be measured 
simultaneously with $\infty$ precision.
\item Quantum effects to EM fields can be ignored at least up to $10^{-10}$ cm.  There is no known upper limit of distance
over which it is valid.
\item Special relativity is `built in' the theory.
The theory will be used as is in almost all areas of physics.
\item The way mass is a basic physical quantity in gravitation, charge and magnetic dipole are the basic quantities in E\&M.

\end{enumerate}
\end{frame}

\begin{frame}
\frametitle{Electric charge}
\begin{enumerate}
\item Without slipping into the tricky areas of particle physics, we will accept that:
\begin{enumerate}
\item Charges come in two varieties. By convention, electron's charge is negative, proton's is positive.
\item Charge is quantised and conserved.
\item Charge is a scalar, therefore invariant across inertial frames of reference.
\item Electron is roughly $2000$ times lighter than proton. 
\item All positive charge in an atom is concentrated in a nucleus that is of the size $10^{-13}$ cm.
\end{enumerate}
\item Quantisation of charge is mostly irrelvant in classical E\&M. We will mostly consider macroscopic charge and current
densities.
\item We will not agonise over what holds an electron or a proton together.
\end{enumerate}
\end{frame}

\begin{frame}
\frametitle{Coulomb's law}
\begin{enumerate}
\item If a charge $q_1$ is at $\vec{r}_1$, $q_2$ is at $\vec{r}_2$ and both are at rest then the force on $q_2$ due to $q_1$ is
\begin{equation}\label{e1}
\vec{F}_2 = k\frac{q_1q_2}{|\vec{r}_2 - \vec{r}_1|^3}(\vec{r}_2 - \vec{r}_1),
\end{equation}
where $k$ is a constant that depends on the choice of units. Both are assumed to be point charges.

\item In cgs units, $F$ is measured in dynes, $r$ in cm, $k$ is chosen to be $1$ and $q$ is measured in esu. The dimensions of esu
are (dynes)$^{1/2}$ cm. Contrast with SI.

\item From \eqref{e1} it follows that $\vec{F}_1 = -\vec{F}_2$. Newton's third law is taken care of.
\end{enumerate}
\end{frame}

\begin{frame}
\frametitle{Coulomb's law}
\begin{enumerate}
\item Electrons and protons not only have charge but also magnetic dipole moment. However, the magnetic force,
\begin{enumerate}
\item Decays as $r^{-4}$.
\item Is not along $\vec{r}_2 - \vec{r}_1$.
\item Even at a separation of $10^{-8}$ it is $10^{-4}$ times weaker than electric force. At macroscopic distances, it is 
irrelevant.
\end{enumerate}
\item Internal structure of proton is ignored in classical E\&M. Elementary charges are assumed to be so localised that their
positions can be accurately described by $\vec{r}_{1,2}$.
\item Elementary charge is $e \approx 4.8 \times 10^{-10}$ esu ($1.6 \times 10^{-19}$ C). Thus, $1$ C is $\approx 3 \times 
10^9$ esu.
\item Coulomb's law obeys the principle of superposition.
\item Solve problems 1, 2, 3 and 4 from the book. (Review 6).
\end{enumerate}
\end{frame}

\begin{frame}
\frametitle{Work energy theorem}
\begin{enumerate}
\item From Newton's second law 
\begin{equation}\label{e2}
\int_A^B \vec{F}\cdot d\vec{x} = \frac{1}{2}mv_B^2 - \frac{1}{2}mv_A^2,
\end{equation}
where the speed of the particle is $v_A$ at point $A$ and $v_B$ at point $B$. 

\item If $\vec{F}$ is a conservative force then the integral is independent of the path taken by the particle. We can then
define a function $U$ as
\begin{equation}\label{e3}
U(A) = -\int_R^A \vec{F}\cdot d\vec{x},
\end{equation}
where $R$ is a suitable reference point.

\item $U(A) + \frac{1}{2}mv_A^2 = U(B) + \frac{1}{2}mv_B^2$.
\end{enumerate}
\end{frame}

\begin{frame}
\frametitle{Potential energy}
\begin{enumerate}
\item If a charge $q$ is brought from a great distance to the vicinity of a charge $Q$ such that the distance between then
is $r$ then we can easily show that 
\begin{equation}\label{e4}
U = \frac{qQ}{r}.
\end{equation}
\item For an assembly of $N$ charges,
\begin{equation}\label{e5}
U = \frac{1}{2}\sum_{i=1}^N\sum_{j=1, i \ne j}^N \frac{q_iq_j}{r_{ij}}.
\end{equation}
\item Solve problems 7 and 8 from the book.
\end{enumerate}
\end{frame}

\begin{frame}
\frametitle{Electric field and Gauss' law}
\begin{enumerate}
\item Electric field due to a charge density $\rho(\vec{r})$ is
\begin{equation}\label{e6}
\vec{E}(\vec{r})  = \int\frac{\rho(\vec{r}^\op)(\vec{r} - \vec{r}^\op)}{|\vec{r} - \vec{r}^\op|^3}dv^\op.
\end{equation}
\item Gauss' law is
\begin{equation}\label{e7}
\oint_S \vec{E}\cdot d\vec{a} = 4\pi q,
\end{equation}
where $q$ is the charge enclosed by the closed surface $S$. Its differential form is
\begin{equation}\label{e8}
\dive\vec{E} = 4\pi\rho.
\end{equation}

\item Gauss' law is useful only when the problem has a convenient symmetry. It is a happy consequence of
the inverse square dependence of the field.

\item Solve problems 5, 9, 10, 11, 13 to 22.
\end{enumerate}
\end{frame}
\end{document}
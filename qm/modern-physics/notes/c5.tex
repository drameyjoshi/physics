\chapter{Some more one dimensional problems}\label{c5}
In the previous chapter we solved the Schr\"{o}dinger equation for two extreme
situations, the free particle and a particle in an infinite potential well. In
this chapter, we shall restrict ourselves to one dimensional problems so that
Schr\"{o}dinger's time independent equation remains an ordinary differential
equation. We will further consider only those problems where it is possible to
obtain an exact, analytical solution.

\section{Particle in a finite potential well}\label{c5s1}
This problem is a minor, but an interesting, modification of the problem we
considered in section \ref{c4s2}. The only change in the problem is the 
potential,
\begin{equation}\label{c5s1e1}
V(x) = \begin{cases}
0 & \text{ if } -L/2 \le x \le L/2 \\
V & \text{ otherwise.}
\end{cases}
\end{equation}
Schr\"{o}dinger's time independent equation in one dimension is
\begin{equation}\label{c5s1e2}
-\frac{\hslash^2}{2m}\frac{d^2\psi}{dx^2} + V(x)\psi = E\psi.
\end{equation}
In the region $-L/2 \le x \le L/2$, which is in the interior of the well,
the equation becomes
\begin{equation}\label{c5s1e3}
-\frac{\hslash^2}{2m}\frac{d^2\psi}{dx^2} = E\psi
\end{equation}
while outside the well it is
\begin{equation}\label{c5s1e4}
-\frac{\hslash^2}{2m}\frac{d^2\psi}{dx^2} + V\psi = E\psi.
\end{equation}
We can rearrange \eqref{c5s1e3} as
\begin{equation}\label{c5s1e5}
\frac{d^2\psi}{dx^2} + k^2\psi = 0,
\end{equation}
where
\begin{equation}\label{c5s1e6}
k^2 = \frac{2mE}{\hslash^2}.
\end{equation}
The general solution of this equation is, 
\begin{equation}\label{c5s1e7}
\psi(x) = \alpha e^{-ikx} + \beta e^{ikx},
\end{equation}
where $\alpha$ and $\beta$ are constants of integration. Unlike the case of 
an infinitely deep well, we do not have the boundary conditions like 
\eqref{c4s2e24} and \eqref{c4s2e25}. In order to get an appropriate boundary
condition we must solve the Schr\"{o}dinger equation \eqref{c5s1e4} in the 
region outside the well. It can be rearranged as
\begin{equation}\label{c5s1e8}
\frac{d^2\psi}{dx^2} + \frac{2m(E - V)}{\hslash^2}\psi = 0.
\end{equation}
Let us consider the case $E < V$ in which the particle does not have enough
energy to cross the potential well. Define
\begin{equation}\label{c5s1e9}
k_o^2 = \frac{2m(V - E)}{\hslash^2} > 0
\end{equation}
so that we can write \eqref{c5s1e8} as
\begin{equation}\label{c5s1e10}
\frac{d^2\psi}{dx^2} - k_o^2\psi = 0.
\end{equation}
The general solution of this equation is
\begin{equation}\label{c5s1e11}
\psi_o(x) = \alpha_0 e^{-k_ox} + \beta_o e^{k_ox}
\end{equation}
where $\alpha_o$ and $\beta_o$ are constants of integration. This solution
indicates that the wavefunction of the particle is \emph{not zero} outside the
potential well. It means that there is a non-zero probability of finding the
particle outside the well in spite of the fact that the particle lacks the 
energy to cross the potential barrier. This is one example of `quantum
tunnelling', a phenonemon in which a particle appears in a position where no
classical particle would.

We would want the wave function of the particle to be continuous all 
throughtout. Therefore, we insist that
\begin{eqnarray}
\psi_o\left(-\frac{L}{2}\right) &=& \psi\left(-\frac{L}{2}\right) 
  \label{c5s1e12} \\
\psi_o\left(\frac{L}{2}\right) &=& \psi\left(\frac{L}{2}\right) 
  \label{c5s1e13} 
\end{eqnarray}
Further, we also want the solution to converge to zero as $x \rightarrow \pm
\infty$. Therefore, the solution outside the box must be
\begin{equation}\label{c5s1e14}
\psi_o(x) = \begin{cases}
\beta_o e^{k_ox} & \text{ if } x \le -L/2 \\
\alpha_o e^{-k_ox} & \text{ if } x \ge L/2.
\end{cases}
\end{equation}
From equations \eqref{c5s1e12}, \eqref{c5s1e13} and \eqref{c5s1e14} we get
\begin{eqnarray}
\beta_o e^{-k_oL/2} &=& \alpha e^{ikL/2} + \beta e^{-ikL/2} \label{c5s1e15} \\
\alpha_o e^{-k_oL/2} &=& \alpha e^{-ikL/2} + \beta e^{ikL/2} \label{c5s1e16}
\end{eqnarray}
Equation \eqref{c5s1e2} is a second order equation. Therefore, we also insist
on the continuity of the first derivative of $\psi$. We then get the pair of
equations
\begin{eqnarray}
k_o\beta_o e^{-k_oL/2} &=& -ik\alpha e^{ikL/2} + ik\beta e^{-ikL/2} 
   \label{c5s1e17} \\
-k_o\alpha_o e^{-k_oL/2} &=& -ik\alpha e^{-ikL/2} + ik\beta e^{ikL/2}.
   \label{c5s1e18}
\end{eqnarray}
Adding equations \eqref{c5s1e15} and \eqref{c5s1e16} we get
\begin{equation}\label{c5s1e19}
(\alpha_o + \beta_o)e^{-k_oL/2} = 2\cos\left(\frac{kL}{2}\right)(\alpha + \beta)
\end{equation}
while subtracting \eqref{c5s1e18} from \eqref{c5s1e17} gives
\begin{equation}\label{c5s1e20}
k_o(\alpha_o + \beta_o)e^{-k_oL/2} = -2\sin\left(\frac{kL}{2}\right)(\alpha +
\beta).
\end{equation}
From equations \eqref{c5s1e19} and \eqref{c5s1e20} we get
\begin{equation}\label{c5s1e21}
k_o + \tan\left(\frac{kL}{2}\right) = 0.
\end{equation}
From equations \eqref{c5s1e6} and \eqref{c5s1e9} it is clear that the above
equation is really a condition on the energy $E$, since the `depth' $V$ of the
potential is a given quantity. Once again we observe that $E$ cannot take 
arbitrary values but only those permitted by equation \eqref{c5s1e21}.

\subsection{Problem set - 1}
\begin{enumerate}
\item Equation \eqref{c5s1e21} does not have analytical solutions. Can you 
think of a graphical way to solve it?
\item While defining $k_o$ in equation \eqref{c5s1e9} we assumed that $E<V$.
What happens if $E > V$? Without solving the problem mathematically, can you
guess the solution using only physical considerations?
\item Suppose that we define the potential function as
\begin{equation}
V(x) = \begin{cases}
V & \text{ if } -L/2 \le x \le L/2 \\
0 & \text{ otherwise.}
\end{cases}
\end{equation}
so that we instead have the case of a particle incident upon a finite potential
barrier. Consider the case when the particle's energy $E$ is less than $V$.
Will it cross the barrier? If it will then this is a more stark example of
`quantum tunnelling'.
\end{enumerate}

\section{The harmonic oscillator}\label{c5s2}
The harmonic oscillator appears as a toy example in classical mechanics. It is
far more serious in quantum physics. Sidney Coleman, an American theoretical
physicist, once remarked that ``Quantum field theory is harmonic motion taken
to increasing levels of abstraction". We will not have a chance to look at the
simple harmonic oscillator in too many guises in this course. We will use it
to illustrate how to solve a quantum mechanical problem whose classical analogue
is known.

In classical physics, a simple harmonic oscillator is a body of mass $m$ 
subjected to a force proportional to its displacement from a certain point and
in a direction opposite to the displacement. In one dimension, where the
vector notation is superfluous, we can express the force as
\begin{equation}\label{c5s2e1}
F = -kx,
\end{equation}
where the number $k$ is a constant. Newton's second law for a particle subjected
to such a force is
\begin{equation}\label{c5s2e2}
m\ddot{x} = -kx,
\end{equation}
where the a dot overhead signifies a derivative with respect to time. We 
rearrange this equation as
\begin{equation}\label{c5s2e3}
\ddot{x} + \omega^2 x = 0,
\end{equation}
where
\begin{equation}\label{c5s2e4}
\omega = \sqrt{\frac{k}{m}}.
\end{equation}
A general solution of \eqref{c5s2e3} is
\begin{equation}\label{c5s2e5}
x(t) = \alpha e^{-i\omega t} + \beta e^{i\omega t},
\end{equation}
where $\alpha$ and $\beta$ are constants of integration. From this equation it
is evident that the constant $\omega$ is the angular frequency of the harmonic
oscillator. It can be easily show that the particle has a potential energy
\begin{equation}\label{c5s2e6}
V = \frac{1}{2}kx^2
\end{equation}
so that its total energy is
\begin{equation}\label{c5s2e7}
H = T + V = \frac{p^2}{2m} + \frac{1}{2}kx^2.
\end{equation}
The quantity $H$ is called the Hamiltonian and it is the starting point of 
our quantum mechanical analysis.

Given a classical Hamiltonian function we write the corresponding quantum
mechanical Hamiltonian operator by replacing the dynamical variables with their
operator representation. Thus, a harmonic oscillator's Hamiltonian is
\begin{equation}\label{c5s2e8}
\hat{H} = \frac{\hat{p}^2}{2m} + \frac{1}{2}m\omega^2\hat{x}^2.
\end{equation}
From equations \eqref{c4s3e1} and \eqref{c4s3e2} we have
\begin{eqnarray}
\hat{x} &=& x \label{c5s2e9} \\
\hat{p} &=& -i\hslash\frac{d}{dx} \label{c5s2e10}
\end{eqnarray}
so that
\begin{equation}\label{c5s2e11}
\hat{H} = -\frac{\hslash^2}{2m}\frac{d^2}{dx^2} + V(x)
\end{equation}
and hence the Schr\"{o}dinger's time independent equation $\hat{H}\psi = E\psi$
is
\begin{equation}\label{c5s2e12}
-\frac{\hslash^2}{2m}\frac{d^2\psi}{dx^2} + \frac{1}{2}m\omega^2x^2\psi = E\psi.
\end{equation}
We can rearrange it as
\begin{equation}\label{c5s2e13}
\frac{d^2\psi}{dx^2} + \left(\frac{2mE}{\hslash^2} - \frac{m^2\omega^2x^2}{\hslash^2}\right)\psi = 0.
\end{equation}
Introduce a dimensionless variable
\begin{equation}\label{c5s2e14}
\xi = \sqrt{\frac{m\omega}{\hslash}}x
\end{equation}
so that
\[
\frac{d\psi}{dx} = \frac{d\psi}{d\xi}\frac{d\xi}{dx} = \sqrt{\frac{m\omega}{\hslash}}\frac{d\psi}{d\xi}
\]
and
\[
\frac{d^2\psi}{dx^2} = \frac{m\omega}{\hslash}\frac{d^2\psi}{d\xi^2}
\]
so that equation \eqref{c5s2e13} becomes
\begin{equation}\label{c5s2e15}
\frac{d^2\psi}{d\xi^2} + \left(\frac{2E}{\hslash\omega} - \xi^2\right)\psi = 0.
\end{equation}
Let
\begin{equation}\label{c5s2e16}
\alpha = \frac{2E}{\hslash\omega}
\end{equation}
so that equation \eqref{c5s2e15} becomes
\begin{equation}\label{c5s2e17}
\frac{d^2\psi}{d\xi^2} + (\alpha - \xi^2)\psi = 0.
\end{equation}
A standard way to solve equations like \eqref{c5s2e17} is to use Frobenius 
method. It consists in expressing $\psi$ as a power series and finding a 
recurrence relation among the series' coefficients. If we use it for 
\eqref{c5s2e17} we get a three term recurrence relation, which is hard to solve.
Instead we first try to guess the asymptotic form of the solution, that is, the
form of the solution as $\xi \rightarrow \pm\infty$. In this limit, the equation
\eqref{c5s2e17} can be approximated to
\[
\frac{d^2\psi}{d\xi^2} - \xi^2\psi = 0
\]
or, more conveniently,
\[
\frac{d^2\psi}{d\xi^2} - (\xi^2 - 1)\psi = 0.
\]
One can readily verify that the solution of this equation is $\psi(\xi) =
e^{-\xi^2/2}$. Therefore, we express the solution of the original equation as
\begin{equation}\label{c5s2e18}
\psi(\xi) = e^{-\xi^2/2}H(\xi),
\end{equation}
where the function $H$ satisfies the equation
\begin{equation}\label{c5s2e19}
\frac{d^2H}{d\xi^2} - 2\xi\frac{dH}{d\xi} + (\alpha - 1)H(\xi) = 0.
\end{equation}
This equation is called as Hermite's differential equation and its solutions are
called as Hermite polynomials. We will solve \eqref{c5s2e19} using Frobenius 
method. Let
\begin{equation}\label{c5s2e20}
H(\xi) = \sum_{n \ge 0}a_n\xi^n
\end{equation}
so that
\begin{eqnarray}
H^\prime(\xi) &=& \sum_{n \ge 1}na_n \xi^{n-1} \label{c5s2e21} \\
H^{\prime\prime}(\xi) &=& \sum_{n \ge 2}n(n-1)a_n \xi^{n-2} \label{c5s2e22}
\end{eqnarray}
Substituting \eqref{c5s2e20} to \eqref{c5s2e22} in \eqref{c5s2e19} we get
\begin{equation}\label{c5s2e23}
\sum_{n \ge 2}n(n-1)a_n \xi^{n-2} - 2\sum_{n \ge 1}na_n \xi^n + (\alpha - 1)
\sum_{n \ge 0}a_n\xi^n = 0
\end{equation}
Writing all sums so that their indices start from $0$,
\begin{equation}\label{c5s2e24}
\sum_{n \ge 0}(n+2)(n+1)a_{n+2} \xi^n - 2\sum_{n \ge 0}(n+1)a_{n+1} \xi^{n+1} + 
(\alpha - 1)\sum_{n \ge 0}a_n\xi^n = 0
\end{equation}
or
\begin{equation}\label{c5s2e25}
\sum_{n \ge 0}\left[(n+2)(n+1)a_{n+2} + (\alpha - 1)a_n\right]\xi^n - 
2\sum_{n \ge 0}(n+1)a_{n+1} \xi^{n+1} = 0.
\end{equation}
If this equation has to be valid for all values of $\xi$ then the coefficient of
every power of $\xi$ must vanish. In particular,
\begin{equation}\label{c5s2e26}
(n+2)(n+1)a_{n+2} + (\alpha - 1)a_n - 2na_n = 0
\end{equation}
or
\begin{equation}\label{c5s2e27}
a_{n+2} = \frac{2n - \alpha + 1}{(n+2)(n+1)}a_n.
\end{equation}
The recurrence relation terminates when
\begin{equation}\label{c5s2e28}
\alpha = 2n + 1
\end{equation}
or, using equation \eqref{c5s2e16}
\begin{equation}\label{c5s2e29}
E = \left(n + \frac{1}{2}\right)\hslash\omega.
\end{equation}
The only solutions that are physically permissible are the ones for which the
recurrence terminates. For in this case, the series of \eqref{c5s2e20} becomes
a polynomial and the complete solution \eqref{c5s2e18} does indeed assume the
asymptotic form of $\psi(\xi) = e^{-\xi^2/2}$. The energy of the physically
permissible solutions is restricted to the discrete set of values in equation
\eqref{c5s2e29}. For a particular $n$, we distingush the energy eigenvalue by
denoting it as $E_n$ and the corresponding polynomial as $H_n$. Thus, the 
eigenfunctions of the Hamiltonian \eqref{c5s2e11} are
\begin{equation}\label{c5s2e30}
\psi_n(\xi) = e^{-\xi^2/2}H_n(\xi)
\end{equation}
and the corresponding eigenvalues are
\begin{equation}\label{c5s3e31}
E_n = \left(n + \frac{1}{2}\right)\hslash\omega,
\end{equation}
where $n = 0, 1, 2, \ldots$. The lowest energy of a harmonic oscillator is
\begin{equation}\label{c5s3e32}
E_0 = \frac{1}{2}\hslash\omega.
\end{equation}
It is \emph{not} zero. Thus, a quantum harmonic oscillator is never at rest. If 
it were then it would have violated Heisenberg's uncertainty relation. We also 
note that
\begin{equation}\label{c5s2e33}
E_{n+1} - E_n = \hslash\omega.
\end{equation}
Thus, when a harmonic oscillator goes from the ground state of $n = 0$ to the 
first excited state of $n = 1$, it absorbs an energy $\hslash\omega$. When it
goes to the second exsited state, it would have absorbed two quanta, each one 
of energy $\hslash\omega$. In general, when the harmonic oscillator is in the
$m$th excited state, it would have absorbed $m$ quanta of energy $\hslash
\omega$.

It is possible to consider a crystal as an ensemble (the French word for a 
collection) of harmonic oscillators of various frequencies. An individual
harmonic oscillator corresponds to energy quanta of a specific frequency. In
the case of a vibrating crystal, the energy quanta are called \emph{phonons}.
A crystal has $m_1$ phonons of frequency $\omega_1$ if the harmonic oscillator
with frequency $\omega_1$ is in the energy state $E_{m_1} = (m_1 + 1/2)\hslash
\omega_1$.

The harmonic oscillator we considered so far was a mass $m$ that was subject to
a force $-kx$ and we showed that its Hamiltonian is given by equation 
\eqref{c5s2e7}
\[
H = \frac{p^2}{2m} + \frac{1}{2}kx^2.
\]
If we choose our units such that $m = 1$ and $k = 1$ and we start calling the 
coordinate $x$ as $q$ then the Hamiltonian can be written as
\begin{equation}\label{c5s2e34}
H = \frac{p^2}{2} + \frac{q^2}{2}.
\end{equation}
Any system whose Hamiltonian can be written in this form is called a harmonic
oscillator. The quantities $p$ and $q$ are called generalised momentum and 
generalised coordinate. It is possible to write the Hamiltonian of 
electromagnetic radiation in a black body as a sum of terms of the form 
\eqref{c5s2e34}. Analogous to a vibrating crystal, the electromagnetic radiation
in a black body can also be considered to be an ensemble of harmonic oscillators
of various frequencies. The energy quanta of these oscillators are called
\emph{photons}. If an oscillator of frequency $\omega_1$ is in the $m_1$th
excited state then we say that there are $m_1$ photons of frequency $\omega_1$
in the black body. 

When one utters the word  `photon' or a `phonon' we should interpret it in terms
of the excited state of an ensemble of oscillators. We should not visualise them 
as `particles' of light or sound.
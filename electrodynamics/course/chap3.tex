\documentclass{article}
\usepackage{amsmath, graphicx, xcolor, hyperref}
\usepackage{mathabx}
\newcommand{\op}{{\;\prime}}
\newcommand{\grad}{\nabla\;}
\newcommand{\dive}{\nabla\cdot}
\newcommand{\curl}{\nabla\times}
\newcommand{\td}[2]{\frac{d{#1}}{d{#2}}}
\newcommand{\pdt}[2]{\frac{\partial{#1}}{\partial{#2}}}
\newcommand{\uv}[1]{\hat{e}_{#1}}
\newcommand{\un}{\hat{n}}

\begin{document}
%\newcommand{\op}{{\;\prime}}
\newcommand{\grad}{\nabla\;}
\newcommand{\dive}{\nabla\cdot}
\newcommand{\curl}{\nabla\times}
\newcommand{\td}[2]{\frac{d{#1}}{d{#2}}}
\newcommand{\pdt}[2]{\frac{\partial{#1}}{\partial{#2}}}
\newcommand{\uv}[1]{\hat{e}_{#1}}
\newcommand{\un}{\hat{n}}

\title{Electric Fields Around Conductors}
\author{Amey Joshi}
\date{10-Aug-2024}
\maketitle
\begin{enumerate}
\item A charge $q_b$ in one of the cavities results in a lining up of a charge
$-q_b$ on its inner side, which in turn is compensated by a charge $q_b$ on the
surface of the sphere. Likewise, the charge $q_c$ in another cavity will result
in additional charge $q_c$ on the sphere. Thus, the total charge on the surface
is $q_b + q_c$.

$q_b$ will not experience any force from the induced charges because it is at the
centre. Likewise for $q_c$.

The net charge $q_b + q_c$ on the sphere's surface will depend on the location 
and size of the two cavities. In general, it will not be uniformly distributed.
Therefore the force between $q_d$ and the sphere will not have $r^{-2}$ contribution
alone. However, if $r$ is much large than the radius of the sphere, the force
between them is
\[
\frac{q_d(q_b + q_c)}{r^2}.
\]

\item There is no negative mass. Therefore no compensating gravitational fields
can be set up.

\item Let the charge $q$ be at a height $h$ above the plane. Consider the plane
$z = h$. Of all the lines of force emerging from $q$, half of them will never 
cross $z = h$ and the remaining half will initially start their journey upwards
and eventually turn down to meet the plane. All the lines, put together, will
result in a charge $-q$ on the plane. The lines starting below the plane will
result in a charge $-q/2$ on the plane. From equation (8) in the book, the surface
charge density is
\[
\sigma = -\frac{qh}{2\pi(r^2 + h^2)^{3/2}}
\]
If the charge $-q/2$ is induced in a region of radius $R$ from the origin, then
\[
-\frac{q}{2} = \int_0^R\int_0^{2\pi}\sigma rdrd\theta
\]
Therefore,
\[
-\frac{q}{2} = -\int_0^R\int_0^{2\pi}\frac{qh}{2\pi(r^2 + h^2)^{3/2}} rdrd\theta
\Rightarrow
\frac{1}{2h} = \int_0^R \frac{2rdr}{(r^2 + h^2)^{3/2}}
\]
Let $x^2 = r^2 + h^2$ so that $2xdx = 2rdr$. The limits of the integral go from
$h$ to $\sqrt{R^2 + h^2}$. Thus,
\[
\frac{1}{2h} = 2\int_h^{\sqrt{R^2+h^2}} x^{-2}dx = 
-2\left(\frac{1}{\sqrt{R^2+h^2}} - \frac{1}{h}\right)
\]
or
\[
\frac{1}{4h} = \frac{1}{h} - \frac{1}{\sqrt{R^2+h^2}} \Rightarrow
\frac{3}{4h} = \frac{1}{\sqrt{R^2+h^2}} \Rightarrow
9(R^2 + h^2) = 16h^2
\]
so that
\[
R = \frac{h\sqrt{7}}{3}.
\]

\item The solution is quite difficult without the method of images. If $\sigma_+$
and $\sigma_-$ are the surface charge densities on the plate due to $Q$ at height
$10$ and $-Q$ at height $z$ then
\begin{eqnarray*}
\sigma_+ &=& -\frac{10Q}{2\pi(r^2 + 100)^{3/2}} \\
\sigma_- &=& +\frac{zQ}{2\pi(r^2 + z^2)^{3/2}}
\end{eqnarray*}
so that the net density is
\[
\sigma = \frac{Q}{2\pi}\left(\frac{z}{(r^2 + z^2)^{3/2}} - \frac{10}{(r^2 + 100)^{3/2}}\right)
\]
Finding the force due to such a charge distribution is quite challenging. Instead
we note that the electric field likes are exactly due to the charges $Q, -Q, +Q, -Q$
at locations $z = 10, z, -z, -10$. Thus, the force on the charge $-Q$ at $z$ is
\[
F = \frac{Q^2}{(10 - z)^2} - \frac{Q^2}{4z^2} + \frac{Q^2}{(10 + z)^2}.
\]
$F = 0$ when
\[
\frac{1}{(10 - z)^2} + \frac{1}{(10 + z)^2} = \frac{1}{4z^2}
\Rightarrow 7z^4 + 1000z^2 - 10000 = 0.
\]
We can find the roots using
\begin{verbatim}
import numpy as np
p = np.array([7, 0, 1000, 0, -10000])

[np.real(z) for z in np.roots(p) \ 
if np.isreal(z) & (np.real(z) > 0)]
\end{verbatim}
The only positive root is $z = 3.0633$ cm.

\item Align the $z$ axis to be perpendicular to the conducting plane and let the
two charges be on it. When the charge $Q$ is moved, its mirror charge $-Q$ also
retreats by the same amount so that the distance between then is $2z$ and the force
between them is
\[
\vec{F} = -\frac{Q^2}{4z^2}\uv{z}.
\]
and the work done in pulling them apart is
\[
dW = \frac{Q^2}{4z^2}dz
\]
so that
\[
W = \int_h^\infty dW = \frac{Q^2}{4h}.
\]
The image charge is fictitious and it is introduced only to solve a problem. The
force due to it is same as that due to the charge induced on the plane and it is
this latter charge which is the true physical charge. It adjusts itself as the 
charge $Q$ retreats away from the plane.

The answer $Q^2/(2h)$ assumes that work is also done in pulling the image charge
to infinity. However, we do not have to worry about it.

\item This problem too will be hard to solve without using the idea of image
charges. If the wire has a uniform charge density $\lambda$ then its image will
be another wire with charge density $-\lambda$ below the earth's surface. The 
field due to the true/real wire at the earth's surface is
\[
\vec{E}_r = -\frac{\lambda}{2\pi h}\uv{z}.
\]
That due to the `image' wire is
\[
\vec{E}_i = -\frac{\lambda}{2\pi h}\uv{z}
\]
so that the net field is
\[
\vec{E} = -\frac{\lambda}{\pi h}\uv{z}.
\]

\item The charges on $A$ and $B$ are induced by those in $C$ and $D$. When $C$ 
and $D$ are shorted, the charges on them with neutralise each other.

\item Let $\sigma_1$ be the surface density on the upper side and $\sigma_2$ be
it on the lower side of the plate. The compensating densities of opposite side 
are induced on the outer plates. We first have
\[
\sigma_1 + \sigma_2 = 10.
\]
The potential difference between the central plate and the upper plate will be
\[
V_u = 5E_u = 5\sigma_1
\]
while that between the central and the lower plate will be $V_d = 8\sigma_2$.
Since the upper and the lower plates are connected by a conducting wire, they
are at the same potential. Therefore, $5\sigma_1 = 8\sigma_2$. Using this in
$\sigma_1 + \sigma_2 = 10$ gives, $\sigma_2 = 50/13$ and $\sigma_1 = 80/13$
esu per square cm.
\end{enumerate}
\end{document}
\chapter{The Special Theory of Relativity}\label{c1}
\begin{enumerate}
\item A frame of reference is a set of cartesian coordinate axes and a clock.

\item An inertial frame is the one in which Newton's first law is valid. The 
existence of such frames was inferred from the experimental observations of the
seventeenth and the eighteenth centuries.

\item Experiments also showed that a frame fixed to the earth is not inertial
but the one fixed with respect to the distant stars was. If $K$ is an inertial
frame then so is $K^\op$ if it moves at a constant relative velocity with respect
to $K$. Thus, there are an infinite number of inertial frames.

\item Suppose that two events happen in an inertial frame $K$: one at a point
$\vec{a}$, and at time $t$ and another one at a point $\vec{b}$ and at time 
$t + \delta t$. When viewed from another inertial frame $K^\op$ moving at a 
velocity $\vec{v}$ with respect to $K$, the events occured at points $\vec{r}_0 
+ \vec{a} + \vec{v}t$ and $\vec{r}_0 + \vec{b} +\vec{v}(t + \delta t)$. Here 
$\vec{r}_0$ is the point where the origin of $K$ and $K^\op$ coincided. An observer 
in $K$ reports that the two events occured at times $\delta t$ apart and at points 
$\vec{b} - \vec{a}$ apart. An observer in $K^\op$ reports that the two events 
happened at a distance $\vec{b} - \vec{a} + \vec{v}\delta t$ apart. However, 
he does not dispute that the same time $\delta t$ elapsed between the two events. 
If $\delta t$ were zero then the two events will be simultaneous in \emph{all} 
inertial frames of reference although they could have happend at different 
points as seen by different observers. This is a consequence of the implicit 
assumption of an absolute time in classical physics.

\item This idea can be corroborated by a simple example. An person on a platform may
see a passenger lifting a tea cup as the coach entered the platform and take it
to his lips as the coach exited it. The two events thus occurred, from his 
perspective, at different points. However, a co-passenger will report them to
be happening at the same spot. The observers will agree on the duration between
the events.

\item The special theory of relativity has its roots in two facts borne out 
of experiments:
\begin{itemize}
\item The laws of physics are the same in all inertial frames of reference and
\item Changes in the state of a system are propagated with a finite speed.
\end{itemize}

\item It is also confirmed by experiments that the changes in the state of system 
are propagated at a speed not exceeding
\begin{equation}\label{c1e1}
c = 2.998 \times 10^8 \;\text{cm/s}.
\end{equation}
No material body can move at a speed exceeding $c$ because if it could then it
can be used to signal the change of state of another body. Further, $c$ is the
maximum speed in \emph{all} inertial frames. For if it were not then the laws of
physics will not be identical across frames. Thus, $c$ is a universal constant.
It is also the speed of light in vacuum.

\item The idea of an absolute, universal time is in conflict with the experimental 
observation of a finite speed of propagation of light in vacuum. For if the
speed is $c$ in a frame $K$ then its speed will be $3c/2$ in a frame $K^\op$
approaching $K$ with a speed $c/2$ in a direction opposite to that of propagation
of light. That this is not true was confirmed by Michelson and Morley's experiment.
Time elapses differently in different systems and therefore a value of a time
difference must be accompanied by a specification of the frame in which it was 
measured.

\item Consider a frame $K$ with a source of light at its origin and two detectors at
points $(-a, 0)$ and $(a, 0)$. A spherical light front will reach the two 
detectors simultaneously when observed from $K$. Let us consider the experiment
replicated in another frame $K^\op$ that moves with a velocity $v\hat{e}_x$ with
respect to $K$. Suppose that their origins coincided then the light signal was 
emitted in $K$. Since the speed of light is the same in $K$ and $K^\op$ an observer
in $K^\op$ also sees a spherical wavefront propagating isotropically. However, he
will observe the wavefront reaching $(a, 0)$ sooner than it reaches $(-a, 0)$.
Events that are simultaneous in $K$ are not so in $K^\op$.

\item An event is described by the point at which occurred and the time when it
occurred. The four numbers describing an event can be interpreted as points in a
four-dimensional space. They are called the \emph{world points}. They move along
curves in the four-dimensional space called the \emph{world lines}.

\item A material body at a point $\vec{r}$ to which nothing happens travels along
a world line that is parallel to the $t$ axis. In the four-dimensional space, nothing
is still. If a material body moves along the $x$ axis with a uniform speed $v$ then
its world line is a straight line making an angle $\tan^{-1}(v)$ with the $t$ axis.
If it accelerates (decelerates) then the world line will be a curve turning towards 
(away from) the $x$-axis. 
\begin{figure}
\includegraphics[scale=0.8]{ex1}
\caption{World lines}
\label{c1f1}
\end{figure}

\item Consider an experiment in an inertial frame in which light took time 
$\delta t$ to travel between points $(x, y, z)$ and $x + \delta x, y + 
\delta y, z + \delta z)$. The same experiment was observed from another inertial 
frame in which the light pulse travelled from $(x^\op, y^\op, z^\op)$ to $x^\op 
+ \delta x^\op, y^\op + \delta y^\op, z^\op + \delta z^\op)$ in time $\delta t^\op$.
Since the speed of light is the same in two frames,
\begin{eqnarray}
c^2\delta t^2 &=& \delta x^2 + \delta y^2 + \delta z^2 \label{c1e2} \\
c^2\delta {t^\op}^2 &=& \delta {x^\op}^2 + \delta {y^\op}^2 + \delta {z^\op}^2 \label{c1e3}
\end{eqnarray}
From these equations, we conclude that
\begin{equation}\label{c1e4}
c^2\delta t^2 - \delta x^2 - \delta y^2 - \delta z^2 = 
c^2\delta {t^\op}^2 - \delta {x^\op}^2 - \delta {y^\op}^2 - \delta {z^\op}^2
\end{equation}
This suggests that the quantity 
\begin{equation}\label{c1e5}
\delta s^2 = c^2\delta t^2 - \delta x^2 - \delta y^2 - \delta z^2
\end{equation}
is invariant across all inertial frame references. The quantity $\delta s$ is
called an interval between two world points.

\item If $\delta t = 0$ then equation \eqref{c1e4} suggests that $\delta {t^\op}^2$
need not be zero. Events simultaneous in $K$ need not be so in $K^\op$. Further, if
$\delta t = 0$ then $\delta s^2 < 0$. Intervals between world points for which 
$\delta s^2 < 0$ are called \emph{space like}. Two points separated by a space-
like interval will always have different ``space'' coordinates.

\item If $\delta x^2 + \delta y^2 + \delta z^2 = 0$ then $\delta s^2 > 0$. In this
case $\delta t$ can never be zero. Intervals between world points for which 
$\delta s^2 > 0$ are called \emph{time like}. Two points separated by a time-
like interval will always have different ``time'' coordinates.

\item It follows from the definitions of time-like and space-like intervals that
\begin{itemize}
\item If two world points are separated by a time-like interval then there exists
an inertial frame in which the events occur at the same space-point, that is, at
same values of $(x, y, z)$.
\item If two world points are separated by a space-like interval then there exists
an inertial frame in which the events occur at the same time.
\end{itemize}

\item An interval for which $\delta s = 0$ is called ``light-like''.

\item Since the nature of an interval depends on an invariant quantity like
$\delta s^2$, an interval that is space-like (or time-like or null-like) in one
inertial frame is space-like (or time-like or null-like) in all inertial frames.

\item A causal relationship between events can exists only their world-points
are separated by a time-like interval.

\item In the four-dimensional space with coordinates $x, y, z$ and $t$, the equation
\begin{equation}\label{c1e6}
c^2t^2 = x^2 + y^2 + z^2
\end{equation}
defines a double cone with origin as the vertex and the $t$ axis as its principle
axis. It is called the \emph{light cone}. 
World points inside the cone are separated by a time-like interval. World
points outside it are separated by space-like interval. If two world points $A$
and $B$ are inside the cone then in every inertial frame the difference between
their time coordinates will be non-zero. Likewise, if they were outside the cone
then in every inertial frame their space coordinates will not all be the same.
The part of the cone above (below) the origin is called the absolute future 
(past) of the origin.
\end{enumerate}

\chapter{Schr\"{o}dinger Wave Equation}\label{c4}
Bohr's model explained the spectrum of the Hydrogen atom quite well. But not
too well. When a Hydrogen atom is subjected to an electric field, the spectral
lines split. This is called the `Stark effect' after the name of its discoverer.
Similarly, `Zeeman effect' is the splitting of spectral lines when the atom is
subjected to a magnetic field. By splitting we mean that in the place of a 
single line in the absence of a field we now see more than one. A spectral line
indicates the energy level of an electron. A splitting of a spectral line tells
us that electrons which used to have the same energies in the absence of a 
field have different energies when the field is turned on. The Bohr model 
could not explain why this happens. Arnold Sommerfeld tried to extend the Bohr
model by allowing the orbits to be elliptical and by using certain concepts in
advanced classical mechanics. However, it was clear that the theory was getting
\emph{ad hoc} and that a better explanation was needed.

Werner Heisenberg, a German physicist, first came up with a new theory to
explain quantum phenomena \cite{heisenberg1925quantum}. His paper is quite
hard to follow. People of Steven Weinberg's \cite{aitchison2004understanding}
stature have admitted that while they understand what Heisenberg wrote they 
cannot fathom what inspired him to write it. He also found out that the 
equations of quantum mechanics are best written in terms of matrices. As a 
result, his formulation of quantum mechanics is called `matrix mechanics'.

A year later, the Austrian physicist Erwin Schr\"{o}dinger proposed an 
alternative formalism to solve atomic problems. His `equation of motion' was 
conceived using certain optical analogies and is called the wave equation. A
function satisfying the equation is called the wave function and it describes
the matter waves we studied in chapter \ref{c2}. Schr\"{o}dinger's paper
\cite{schrodinger1926undulatory} is remarkably lucid and it offers an insight
into how he came up with his equation. The Schr\"{o}dinger wave equation
is a linear partial differential equation and the physicists of 
Schr\"{o}dinger' time had a expertise and comfort in dealing with them. As a 
result, a first introduction to quantum mechanics is almost always using the
Schr\"{o}dinger equation. Schr\"{o}dinger showed \cite{schrodinger1926relation}
that his approach is equivalent to the one proposed by Heisenberg and developed
by Born and Jordan. The equivalence was put on a firm mathematical footing by
John von Neumann \cite{von2018mathematical}.

In this chapter, we will state Schr\"{o}dinger's equation, interpret its
solution and show how to use it in a few very simple situations. Schr\"{o}dinger
derived \cite{schrodinger1926undulatory} his equation using variational 
principles. However, we shall take it as a starting point of our discussion.
The way we accepted Newton's laws as `correct' because they explained a few
natural phenomena, we will also accept Schr\"{o}dinger's equation because it
explains a lot many phenomena in the atomic world.

\section{Schr\"{o}dinger equation}\label{c4s1}
Before stating and understanding Schr\"{o}dinger's equation that describes
the unfamiliar matter waves, let us consider start in the more familiar world
of sound waves. We know that sound waves travel through the air as a succession
of compressions and rarefactions. We can detect and measure the compression and
rarefaction using a pressure gauge. If $p(x, t)$ is the air pressure at a point
$x$ and at time $t$ then the equation of sound waves is
\begin{equation}\label{c4s1e1}
\frac{\partial^2 p}{\partial x^2} = \frac{1}{c_s^2}
\frac{\partial^2 p}{\partial t^2}
\end{equation}
where $c_s$ is the speed of sound. It is easy to check that any function of 
the form $p(x \pm c_st)$ is a solution of \eqref{c4s1e1}. In particular,
\begin{equation}\label{c4s1e2}
p(x, t) = \cos(x - c_st)
\end{equation}
is a solution. A solution of a wave equation tells us how pressure varys with
position and time. 

Equation \eqref{c4s1e1} described waves in one dimension. If the pressure varies
in all three dimensions then the equation becomes
\begin{equation}\label{c4s1e3}
\frac{\partial^2 p}{\partial x^2} + \frac{\partial^2 p}{\partial y^2} + 
\frac{\partial^2 p}{\partial z^2} = 
\frac{1}{c_s^2} \frac{\partial^2 p}{\partial t^2}
\end{equation}
We can check that
\begin{equation}\label{c4s1e4}
p(x, y, z, t) = \cos(x + y + z - \sqrt{3} c_st)
\end{equation}
is a solution of \eqref{c4s1e4}.

\subsection{Problem set 1}
\begin{enumerate}
\item Verify that \eqref{c4s1e2} is a solution of \eqref{c4s1e1}.
\item Verify that \eqref{c4s1e4} is a solution of \eqref{c4s1e3}.
\item If $p_1(x, t)$ and $p_2(x, t)$ are two solutions of \eqref{c4s1e1} then
show that $\alpha_1 p_1 + \alpha_2 p_2$, where $\alpha_1, \alpha_2$ are real
numbers is also a solution of \eqref{c4s1e1}.
This fact is called the principle of linear superposition. The expression, 
$\alpha_1 p_1 + \alpha_2 p_2$ is called a \emph{linear combination} of $p_1$
and $p_2$.
\end{enumerate}

A sound wave is described by the pressure fluctations it causes. de Broglie
waves are described by the variations of a wave function $\Psi$. It could be a
complex-valued function and therefore it does not have a physical interpretation
by itself. However, its squared modulus $|\Psi|^2 = \Psi\Psi^\ast$ is the 
probability density of finding at a particle at a point $(x, y, z)$ at time $t$.
In particular, $|\Psi|^2 dV$ is the probability of finding a particle in a 
small volume element $dV$ around the point $(x,y,z)$ at time $t$. The way the
air pressure $p$ satisfies the equation \eqref{c4s1e3}, the wave function 
$\Psi$ satisfies 
\begin{equation}\label{c4s1e5}
-\frac{\hslash^2}{2m}\left(\frac{\partial^2\Psi}{\partial x^2} + 
\frac{\partial^2\Psi}{\partial y^2} + \frac{\partial^2\Psi}{\partial z^2}
\right) + V(x,y,z,t)\Psi = i\hslash\frac{\partial\Psi}{\partial t}.
\end{equation}
This is called the Schr\"{o}dinger equation. In this equation,
\begin{enumerate}
\item $\hslash = h/(2\pi)$; we first introduced this quantity when we wrote
\eqref{c3s3e2}.
\item $m$ is the mass of the particle.
\item $V(x, y, z, t)$ is the potential function. It describes the force acting
on the particle. We will describe it more a little later.
\item $i$ is the imaginary number.
\end{enumerate}
A few remarks are in order.
\begin{enumerate}
\item A wave equation always has a second order partial derivative in time.
Equation \eqref{c4s1e5} has just the first order partial derivative in time.
Therefore, strictly speaking, \eqref{c4s1e5} is not a wave equation although
the name `wave function' persists for its solution $\Psi$. It is closer to the
heat equation than it is to a wave equation.
\item The coefficient on the right hand side is an imaginary number. Therefore,
it is not surprising that its solution $\Psi$ is complex valued.
\item The only term in \eqref{c4s1e5} that is specific to a problem is the
potential function $V$. The other terms never change.
\item We remarked that $|\Psi|^2$ has an interpretation of a probability
density function. Therefore, if we want to find the probability of finding
a particle a region of volume $V$ then it is
\begin{equation}\label{c4s1e6}
P(V) = \int_V |\Psi|^2 dV.
\end{equation}
This statement is accurate only when 
\begin{equation}\label{c4s1e7}
\int_{-\infty}^\infty |\Psi|^2dV = 1.
\end{equation}
Such a function is said to be `normalised'. If a wave function is not normalised
, that is, if
\begin{equation}\label{c4s1e8}
\int_{-\infty}^\infty |\Psi|^2dV = N,
\end{equation}
then we can alway consider the function
\begin{equation}\label{c4s1e9}
\Psi_1 = \frac{1}{\sqrt{N}}\Psi.
\end{equation}
We will show in the next problem set that $\Psi_1$ is also a solution of 
\eqref{c4s1e5}. $\Psi_1$ is called the \emph{normalised} wave function and
the number $1/\sqrt{N}$ is called the \emph{normalisation constant}.
\end{enumerate}

\subsection{Problem set 2}
\begin{enumerate}
\item Show that is $\Psi_1$ and $\Psi_2$ are two solutions of \eqref{c4s1e5} 
then $\alpha_1\Psi_1 + \alpha_2\Psi_2$, where $\alpha_1$ and $\alpha_2$ are
complex numbers, is also a solution. Therefore, show that \eqref{c4s1e9} is
also a solution of \eqref{c4s1e5}.
\end{enumerate}

\section{Some solutions of Schr\"{o}dinger equation}\label{c4s2}
Partial differential equations are, in general, much more difficult to solve 
than ordinary differential equations. Schr\"{o}dinger equation is no exception.
We will consider a few situations where an explicit, closed-form solution is
available. In the previous section we remarked that the physics of a problem
is in the choice of the potential function $V$. The situations we alluded to 
are really the different choices of $V$.

\subsection{The free particle}
A free particle is the one not subject to any forces. Therefore, a convenient
choice of $V$ is to set it to zero. Schr\"{o}dinger equation now becomes
\begin{equation}\label{c4s2e1}
-\frac{\hslash^2}{2m}\left(\frac{\partial^2\Psi}{\partial x^2} + 
\frac{\partial^2\Psi}{\partial y^2} + \frac{\partial^2\Psi}{\partial z^2}
\right) = i\hslash\frac{\partial\Psi}{\partial t}.
\end{equation}
Since the particle has no force acting on it, its energy does not change. If
$E$ is its energy then we can write
\begin{equation}\label{c4s2e2}
\Psi(\vec{r}, t) = \psi(\vec{r})e^{-iEt/\hslash}.
\end{equation}
Substituting it in \eqref{c4s2e1} we get
\begin{equation}\label{c4s2e3}
-\frac{\hslash^2}{2m}e^{-iEt/\hslash}\left(\frac{\partial^2\psi}{\partial x^2} 
+ \frac{\partial^2\psi}{\partial y^2} + \frac{\partial^2\psi}{\partial z^2}
\right) = E\psi(\vec{r})e^{-iEt/\hslash}.
\end{equation}
Cancelling the factor $e^{-iEt/\hslash}$ and rearranging a bit, we get
\begin{equation}\label{c4s2e4}
\frac{\partial^2\psi}{\partial x^2} 
+ \frac{\partial^2\psi}{\partial y^2} + \frac{\partial^2\psi}{\partial z^2} =
-\frac{2mE}{\hslash^2}\psi.
\end{equation}
Since we chose $V = 0$, the particle has no potential energy. Its energy is 
entirely kinetic. If $E$ is the kinetic energy then we know that the quantity
$2mE$ is the square of momentum. The coefficient on the right hand side is
\begin{equation}\label{c4s2e5}
-\frac{2mE}{\hslash^2} = -\frac{p^2}{\hslash^2} = -k^2,
\end{equation}
where we have use the de Broglie relation,
\begin{equation}\label{c4s2e6}
p = \frac{h}{\lambda} = \frac{\hslash}{k}.
\end{equation}
Thus, the Schr\"{o}dinger equation simplifies to
\begin{equation}\label{c4s2e7}
\frac{\partial^2\psi}{\partial x^2} 
+ \frac{\partial^2\psi}{\partial y^2} + \frac{\partial^2\psi}{\partial z^2} =
-k^2\psi.
\end{equation}
If we define the vector
\begin{equation}\label{c4s2e8}
\vec{k} = k_x\hat{i} + k_y\hat{j} + k_z\hat{k},
\end{equation}
and let
\begin{equation}\label{c4s2e9}
\psi(\vec{r}) = e^{i\vec{k}\cdot\vec{r}} = e^{i(k_xx+k_yy+k_zz)},
\end{equation}
then we can readily verify that $\psi$ defined in \eqref{c4s2e9} satisfies
\eqref{c4s2e7}. Thus, the complete solution of equation \eqref{c4s2e1} is
\begin{equation}\label{c4s2e10}
\Psi(\vec{r}, t) = e^{i(\vec{k}\cdot\vec{r} - \omega t)},
\end{equation}
where we have used the relation
\begin{equation}\label{c4s2e11}
E = \hslash\omega.
\end{equation}
The solution of \eqref{c4s2e10} is called a plane wave solution. It can have 
any value for its energy $E$.

\subsection{Time-independent potentials}
When $V$ is independent of time we can write
\begin{equation}\label{c4s2e12}
\Psi(\vec{r}, t) = \psi(\vec{r})\tau(t),
\end{equation}
where $\tau$ is a function of time alone. Equation \eqref{c4s1e5} becomes
\begin{equation}\label{c4s2e13}
\tau(t)\left[-\frac{\hslash^2}{2m}\left(\frac{\partial^2\psi}{\partial x^2} + 
\frac{\partial^2\psi}{\partial y^2} + \frac{\partial^2\psi}{\partial z^2}
\right) + V(x,y,z)\psi\right] = i\psi(\vec{r})\hslash\frac{d\tau}{dt}
\end{equation}
Divide both sides by $\Psi = \psi\tau$ to get
\begin{equation}\label{c4s2e14}
\frac{1}{\psi(\vec{r})}\left[-\frac{\hslash^2}{2m}\left(
\frac{\partial^2\psi}{\partial x^2} + 
\frac{\partial^2\psi}{\partial y^2} + \frac{\partial^2\psi}{\partial z^2}
\right) + V(x,y,z)\psi\right] = \frac{i\hslash}{\tau(t)}\frac{d\tau}{dt}
\end{equation}
The left hand side of this equation depends on $\vec{r}$ alone and the 
right hand side on $t$ alone. Therefore, each side must be a constant, say $E$.
Thus, we get two equations
\begin{equation}\label{c4s2e15}
-\frac{\hslash^2}{2m}\left(\frac{\partial^2\psi}{\partial x^2} + 
\frac{\partial^2\psi}{\partial y^2} + \frac{\partial^2\psi}{\partial z^2}
\right) + V(x,y,z)\psi = E\psi
\end{equation}
and
\begin{equation}\label{c4s2e16}
\frac{1}{\tau}\frac{d\tau}{dt} = -i\frac{E}{\hslash}.
\end{equation}
Since $E$ is a constant, equation \eqref{c4s2e16} is particularly easy to
solve. We see that
\begin{equation}\label{c4s2e17}
\tau(t) = \alpha e^{-iEt/\hslash},
\end{equation}
where $\alpha$ is a constant of integration that can be chosen while normalising
the wavefunction.

Equation \eqref{c4s2e15} is called  Schr\"{o}dinger time independent equation. 
In this course, we will deal only with the situation when $V$ is independent of 
$t$ and therefore we will focus on solving \eqref{c4s2e15}.

\subsection{Particle in an infinite potential well}
The free particle is a rather uninteresting. It does not have any features that
are peculiar to a quantum system. We now consider a particles in a box. The box 
prevents the particle from escaping and therefore one can consider it be 
represented by an infinitely high potential wall. That is, it takes an infinte
energy for the particle to escape it. Let us start with a one-dimensional case
where the potential is defined by
\begin{equation}\label{c4s2e18}
V(x) = \begin{cases}
0 \text{ if } -\frac{L}{2} \le x \le \frac{L}{2} \\
\infty \text{ otherwise.}
\end{cases}
\end{equation}
Within the confines of the box, the particle is free. Without, it cannot even 
get. Therefore, we must solve  Schr\"{o}dinger's time independent equation only
in the region
\[
-\frac{L}{2} \le x \le \frac{L}{2}.
\]
Since we are considering a one dimensional box, \eqref{c4s2e15} becomes
\begin{equation}\label{c4s2e19}
-\frac{\hslash^2}{2m}\frac{d^2\psi(x)}{dx^2} = E\psi(x).
\end{equation}
Let
\begin{equation}\label{c4s2e20}
k^2 = \frac{2mE}{\hslash^2}.
\end{equation}
$k$ is called the wave number and it is related to the momentum by the equation
\begin{equation}\label{c4s2e21}
p = \hslash k.
\end{equation}
In terms of $k$, equation \eqref{c4s2e19} becomes
\begin{equation}\label{c4s2e22}
\frac{d^2\psi(x)}{dx^2} + k^2\psi(x) = 0.
\end{equation}
One can check that the general solution of this ordinary differential equation 
is
\begin{equation}\label{c4s2e23}
\psi(x) = \alpha e^{-ikx} + \beta e^{ikx},
\end{equation}
where $\alpha$ and $\beta$ are constants of integration. The infinite potential
barrier of the box prevents the particle from escaping. Therefore, at the walls,
the function $\psi$ must be zero. That is
\begin{eqnarray}
\psi\left(-\frac{L}{2}\right) &=& 0 \label{c4s2e24} \\
\psi\left(\frac{L}{2}\right) &=& 0 \label{c4s2e25}
\end{eqnarray}
From \eqref{c4s2e23} and \eqref{c4s2e24} we have
\[
\alpha e^{ikL/2} + \beta e^{-ikL/2} = 0
\]
or that
\begin{equation}\label{c4s2e26}
\alpha = -\beta e^{-ikL}
\end{equation}
and hence equation \eqref{c4s2e23} becomes
\begin{equation}\label{c4s2e27}
\psi(x) = -\beta e^{-ikL}e^{-ikx} + \beta e^{ikx}.
\end{equation}
From \eqref{c4s2e25} and \eqref{c4s2e27},
\[
\psi\left(\frac{L}{2}\right) = -\beta e^{-3ikL/2} + \beta e^{ikL/2} = 
-\beta e^{-ikL/2}(e^{-ikL} - e^{ikL}) = 0
\]
or
\begin{equation}\label{c4s2e28}
-2i\beta e^{ikL/2}\sin(kL) = 0
\end{equation}
which can be true only if
\begin{equation}\label{c4s2e29}
kL = n\pi, n = 0, \pm 1, \pm 2, \ldots.
\end{equation}
If $n = 0$ then $k = 0$ and from \eqref{c4s2e23} $\psi(x) = \alpha + \beta$. 
Such a solution will not obey the boundary conditions \eqref{c4s2e24} and
\eqref{c4s2e25} unless $\alpha = \beta = 0$. But this is a trivial solution
indicating that there is no particle. Therefore, the only physically interesting
values of $k$ are
\begin{equation}\label{c4s2e30}
k_n = \frac{n\pi}{L}, n = \pm 1, \pm 2, \ldots.
\end{equation}
As a result of this condition, we see that the momentum of the particle is
restricted to
\begin{equation}\label{c4s2e31}
p_n = \frac{n\pi\hslash}{L}, n = \pm 1, \pm 2, \ldots.
\end{equation}
Thus, the particle's momentum is quantised. So is energy, as seen from
\eqref{c4s2e20},
\begin{equation}\label{c4s2e32}
E_n = \frac{n^2\pi^2\hslash^2}{2mL^2}, n = \pm 1, \pm 2, \ldots.
\end{equation}
The different values of $n$ are the energy levels of the particle. If the 
particle has to be pushed from a lower energy then it must be supplied with an
energy equal to the difference between the two energy levels. On the other hand,
when a particle makes a transition from a higher energy level to a lower on it
emits a photon of frequency $\nu$ such that $\Delta E = h\nu$.

Equations \eqref{c4s2e31} and \eqref{c4s2e32} indicate that both $p_n$ and
$E_n$ increase with $n$. The lowest energy of the particle is
\begin{equation}\label{c4s2e33}
E_1 = \frac{\pi^2\hslash^2}{2mL^2}.
\end{equation}
The lowest energy of a classical particle is zero. It remains at rest in the 
box. However, a quantum particle cannot be at rest because that would violate
the uncertainty principle. 

We still have not written the complete solution of the Schr\"{o}dinger equation
\eqref{c4s2e22}. From \eqref{c4s2e27} and \eqref{c4s2e30}, we have
\begin{equation}\label{c4s2e34}
\psi_n(x) = \beta(e^{in\pi x/L} + (-1)^n e^{-in\pi x/L}).
\end{equation} 
We can write it in a simpler form
\begin{equation}\label{c4s2e35}
\psi_n(x) = \begin{cases}
2\beta\cos\left(\frac{n\pi x}{L}\right) & \text{ if } n = \pm 2, \pm 4, \ldots\\
2\beta\sin\left(\frac{n\pi x}{L}\right) & \text{ if } n = \pm 1, \pm 3, \ldots.
\end{cases}
\end{equation}
The constant $\beta$ can be found using the normalisation condition
\begin{equation}\label{c4s2e36}
\int_{-L/2}^{L/2} |\psi_n(x)|^2dx = 1.
\end{equation}
The integration is quite straigtforward and we get
\begin{equation}\label{c4s2e37}
\beta = \frac{1}{\sqrt{2L}}.
\end{equation}
Thus, the complete solution is
\begin{equation}\label{c4s2e38}
\psi_n(x) = \begin{cases}
\sqrt{\frac{2}{L}}\cos\left(\frac{n\pi x}{L}\right) & 
   \text{ if } n = \pm 2, \pm 4, \ldots \\
\sqrt{\frac{2}{L}}\sin\left(\frac{n\pi x}{L}\right) & 
   \text{ if } n = \pm 1, \pm 3, \ldots.
\end{cases}
\end{equation}

\subsection{Problem set 3}
\begin{enumerate}
\item Energy is indendepent of the sign of $n$, but momemtum is not. What is 
the physical significance of the this observation?
\item Choose any $n$, say $n = 4$, for which
\[
\psi_4(x) = \sqrt{\frac{2}{L}}\cos\left(\frac{4\pi x}{L}\right).
\]
At what $x$ is $\psi_4(x) = 0$? Points at which $\psi$ vanishes are called 
nodes.  Thus, there are some points at which the particle just cannot be. This 
is another stark difference between the classical and the quantum particles.
\item The normalisation condition is defined by \eqref{c4s1e7}. Why did we 
change the limits of integration from $\pm\infty$ to $\pm L/2$?
\end{enumerate}

\section{Operators and eigenvalues}\label{c4s3}
The wavefunction $\Psi$ has the entire information about the quantum particle.
If we know $\Psi$, we can find the particle's position, momentum, angular 
momentum and energy, up to a certain extent. We have learnt that 
\begin{itemize}
\item At an atomic scale, particles have a wavelike character and therefore
do not have a precise position or momentum.
\item Therefore, they cannot have precise angular momentum.
\item Under some situations, they do have definite values of energy.
\end{itemize}
How to extract this information from the wavefunction $\Psi$? Dynamical 
variables are represented by operators in quantum mechanics. An operator is
something that maps one function to another. Some examples of operators in
quantum mechanics are
\begin{enumerate}
\item The position operator is just $\hat{r}$. Its action on the wavefunction is
\begin{equation}\label{c4e3e1}
\hat{r}(\Psi) = \vec{r}\Psi.
\end{equation}
\item The momentum operator $\hat{p}$ is
\begin{equation}\label{c4s3e2}
\hat{p}(\Psi) = (i\hslash)\left(\vec{i}\frac{\partial}{\partial x} + 
\vec{j}\frac{\partial}{\partial y} + \vec{k}\frac{\partial}{\partial z}
\right)\Psi
= i\hslash\nabla\Psi.
\end{equation}
\item The energy operator $\hat{E}$ is
\begin{equation}\label{c4s3e3}
\hat{E}(\psi) = i\hslash\frac{\partial\Psi}{\partial t}.
\end{equation}
\end{enumerate}
When the potential energy is a function of $\vec{r}$ alone then the potential
energy operator $\hat{V}(\vec{r})$ is
\begin{equation}\label{c4s3e4}
\hat{V}(\vec{r})(\Psi) = V(\vec{r})\Psi.
\end{equation}
The relation between kinetic energy and momentum is $T = p^2/(2m)$. Therefore,
we can infer that the relationship between the operators is
\begin{equation}\label{c4s3e5}
\hat{T}(\Psi) = \frac{\hat{p}^2}{2m}(\Psi).
\end{equation}
Using \eqref{c4s3e2} we can easily infer that
\begin{equation}\label{c4s3e6}
\hat{T}(\Psi) = -\frac{\hslash^2}{2m}\left(\frac{\partial^2}{\partial x^2} 
+ \frac{\partial^2}{\partial y^2} + \frac{\partial^2}{\partial x^2}\right)
(\Psi)
\end{equation}
The fact that total energy is the sum of kinetic and potential energies can be
expressed in terms of operators are
\begin{equation}\label{c4s3e7}
\hat{E}(\Psi) = \hat{T}(\Psi) + \hat{V}(\Psi).
\end{equation}
Using equations \eqref{c4s3e3}, \eqref{c4s3e4} and \eqref{c4s3e6} we get
\begin{equation}\label{c4s3e8}
i\hslash\frac{\partial\Psi}{\partial t} = -\frac{\hslash^2}{2m}
\left(\frac{\partial^2}{\partial x^2} + \frac{\partial^2}{\partial y^2} +
 \frac{\partial^2}{\partial z^2}\right)\Psi + V(\vec{r})\Psi.
\end{equation} 
This is Schr\"{o}dinger equation. It is just an expression of the fact that the
total energy is the sum of kinetic and potential energies.

If $\hat{O}$ is an operator then an equation of the form
\begin{equation}\label{c4s3e9}
\hat{O}\Psi = \lambda\Psi,
\end{equation}
where $\lambda$ is a number, is called an `eigenvalue' equation. The number 
$\lambda$ is an `eigenvalue` of the operator $\hat{O}$ and the function $\Psi$
is its `eigenfunction'. An example of an eigenvalue equation is
\begin{equation}\label{c4s3e10}
-i\hslash \frac{d}{dx} e^{ikx} = \hslash k e^{ikx}.
\end{equation}

The total energy operator is called the Hamiltonian and is denoted by $\hat{H}$.
It is named after the Irish physicist William Rowan Hamilton who gave an 
alternative formulation of classical mechanics. Thus,
\begin{equation}\label{c4s3e11}
\hat{H} = \hat{T} + \hat{V} = -\frac{\hslash^2}{2m}\nabla^2 + V,
\end{equation}
where the symbol $\nabla^2$ stands for the Laplacian and is defined as
\begin{equation}\label{c4s3e12}
\nabla^2 = \frac{\partial^2}{\partial x^2} + 
\frac{\partial^2}{\partial y^2} +  \frac{\partial^2}{\partial z^2}.
\end{equation}

\section{Postulates of quantum mechanics}\label{c4s5}
Non-relativistic quantum mechanics of a single is based on the following 
postulates.
\begin{enumerate}
\item The state of a quantum mechanical system of one particle is given by a
wavefunction $\Psi(\vec{r}, t)$.
\item A dynamical variable of classical mechanics is represented by a 
(Hermitian) operator in quantum mechanics.
\item The only values of an operator in a measurement are its eigenvalues.
\item The state of a quantum mechanical system can be expressed as a linear
combination of all the eigenfunctions of an operator. If $\hat{O}$ is an 
operator with eigenvalue equations
\begin{equation}\label{c4s3e11}
\hat{O}\Psi_i = \lambda_i\Psi, i = 1, \ldots, n,
\end{equation}
and if $\Phi = c_1\Psi_1 + \cdots + c_n\Psi_n$ is an arbitrary normalised state
then the probability of an experiment giving an eigenvalue $\lambda_i$ is
\begin{equation}\label{c4s3e12}
P(\lambda_i) = |c_i|^2.
\end{equation}
\item The wavefunction evolves according to the Schr\"{o}dinger equation.
\end{enumerate}

Note that we have chosen to restrict our attention to systems with just one
particle. For some time to come, we will restrict ourselves only to these 
systems.

\section{Particle in a three dimensional box}
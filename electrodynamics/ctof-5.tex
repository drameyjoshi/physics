\chapter{Constant electromagnetic fields}\label{c5}
\begin{enumerate}
\item If the fields are independent of time then they obey
\begin{eqnarray}
\dive\vec{E} &=& 4\pi\rho \label{c5e1} \\
\curl\vec{E} &=& 0 \label{c5e2} \\
\dive\vec{H} &=& 0 \label{c5e3} \\
\curl\vec{H} &=& 4\pi\frac{\vec{J}}{c}. \label{c5e4}
\end{eqnarray}
The symmetry of the situation reveals itself if we note the correspondence $\vec{E}
\mapsto \vec{H}$ and $\rho \mapsto \vec{J}/c$ and $\dive \mapsto \curl$. Equations
\eqref{c5e2} and \eqref{c5e3} suggest that
\begin{eqnarray}
\vec{E} &=& -\grad\phi \label{c5e5} \\
\vec{H} &=& \curl\vec{A} \label{c5e6}
\end{eqnarray}
which is understandable if we note that only the time component of $J^\mu$ 
determines $\vec{E}$ and its space component determines $\vec{H}$. Therefore,
only the time component of $\vec{A}^\mu$ suffices to describe the electric
field and its space component suffices to describe the magnetic field.

\item From equations \eqref{c5e1} and \eqref{c5e5} we have
\begin{equation}\label{c5e7}
\nabla^2\phi = -4\pi\rho.
\end{equation}
In region where there are no charges, the potential satisfies Laplace's equation
\begin{equation}\label{c5e8}
\nabla^2\phi = 0.
\end{equation}
This equation immediately implies that the potential function cannot have a local
minimum or a maximum. The nature of extremum of a function of three variables is
determined by its Hessian matrix. It is a minimum (maximum) if the Hessian matrix
is positive (negative) definite. Since the trace of a matrix is the sum of its
eigenvalues, we have an equivalent condition that the trace of the Hessian matrix
is positive (negative) for an extremum to be a minimum (maximum). But the trace
of Hessian is $\nabla^2\phi$ evaluated at the extremum. If $\phi$ is a solution
of Laplace's equation \eqref{c5e8}, then the Hessian can never be positive or
negative definite.

\item The field of a point charge can be derived using Gauss' law. Let the point
charge be at $\vec{r}^\op$. In order to find the field at $\vec{r}$, we consider
a sphere centred at $\vec{r}^\op$ and passing through $\vec{r}$. As the space is
isotropic, we expect the magnitude of $\vec{E}$ to be the same at all points on the
sphere and its direction along $\vec{r} - \vec{r}^\op$. Therefore,
\[
4\pi |\vec{r} - \vec{r}^\op|^2 = q
\]
or
\begin{equation}\label{c5e9}
\vec{E} = \frac{q}{|\vec{r} - \vec{r}^\op|^3}(\vec{r} - \vec{r}^\op).
\end{equation}
If we define
\begin{equation}\label{c5e10}
\vec{R} = \vec{r} - \vec{r}^\op
\end{equation}
then we can write \eqref{c5e9} simply as
\begin{equation}\label{c5e11}
\vec{E} = \frac{q}{R^3}\vec{R} = \frac{q}{R^2}\uv{R}.
\end{equation}
Equation \eqref{c5e11} is \emph{Coulomb's law}. The potential of this field is
\begin{equation}\label{c5e12}
\phi = \frac{q}{R}.
\end{equation}
If there are discrete, point charges at $\vec{r}_a$ then the field due to all of 
them is
\begin{equation}\label{c5e13}
\vec{E} = \sum_a \frac{q_a}{R_a^2}\uv{R_a},
\end{equation}
where $\vec{R}_a = \vec{r}^\op - \vec{r}$. The potential is
\begin{equation}\label{c5e14}
\phi = \sum_a\frac{q_a}{R_a}.
\end{equation}
For a continuum of charges, \eqref{c5e11} and \eqref{c5e12} generalise to
\begin{eqnarray}
\phi(\vec{r}) &=& \int\frac{\rho(\vec{r}^\op)}{R}dV \label{c5e15} \\
\vec{E}(\vec{r}) &=& \int\frac{\rho(\vec{r}^\op)}{R^2}\uv{R}dV  \label{c5e16}
\end{eqnarray}

\item For a point charge, $\rho = q\delta(\vec{R})$. In this case, \eqref{c5e7}
becomes,
\[
\nabla^2\phi = 4\pi q\delta(\vec{R}).
\]
From \eqref{c5e12}, we also have
\begin{equation}\label{c5e17}
\nabla^2\left(\frac{1}{R}\right) = -4\pi\delta(\vec{R}).
\end{equation}

\item In the absence of magnetic field, the energy density is
\begin{equation}\label{c5e18}
W = \frac{1}{8\pi}\int E^2dV.
\end{equation}
From \eqref{c5e5},
\[
W = -\frac{1}{8\pi}\int\vec{E}\cdot\grad\phi dV.
\]
Since $\dive(\phi\vec{E}) = \grad\phi\cdot\vec{E} + \phi\dive\vec{E}$, we have
\begin{eqnarray*}
W &=& \frac{1}{8\pi}\int\phi\dive\vec{E}dV - \frac{1}{8\pi}\int\dive(\phi\vec{E})dV \\
  &=& \frac{1}{8\pi}\int\phi\dive\vec{E}dV - \frac{1}{8\pi}\oint \phi\vec{E}\cdot d\vec{f}.
\end{eqnarray*}
The potential goes as $O(r^{-1})$ and the field as $O(r^{-2})$ so that the surface
integral goes as $O(r^{-1})$. If we choose a large enough surface the second 
integral can be made as small as we desire. Thus, we have
\[
W = \frac{1}{8\pi}\int\phi\dive\vec{E}dV
\]
or using \eqref{c5e1},
\begin{equation}\label{c5e19}
W = \frac{1}{2}\int\rho\phi dV.
\end{equation}
If the charge density consists od point charges, equation \eqref{c5e19} becomes
\begin{equation}\label{c5e20}
W = \sum_a q_a\phi(\vec{r}_a),
\end{equation}
where $\phi(\vec{r}_a)$ is the potential due to all charges at the point 
$\vec{r}_a$ where the charge $q_a$ is located. This formula immediately runs 
into a tricky situation.

\item The potential $\phi$ due to a charge $q$ at $\vec{r}^\op$ at the same point
is infinity according to \eqref{c5e12}. As a result, from \eqref{c5e20}, a point
charge has infinite energy and therefore an infinite mass. To avoid such an absurd
result, we say that the laws of classical electrodynamics are not applicable at
extremely short distances. In fact, we cannot even ask the question if the mass of
a charge has electrodynamic origin, that is, it exists because of electrodynamic
energy.

The energy of a uniformly charged sphere of radius $s$ can be computed as follow.
We first note that its charge density is
\begin{equation}\label{c5e21}
\rho = \frac{Q}{\frac{4\pi}{3}s^3}.
\end{equation}
The field due to a sphere of radius $r$ and a uniform charge density $\rho$ at 
points on and outside it is as if the entire charge was concentrated at its 
centre. If $q(r)$ is the charge in a sphere of radius $r$ then the energy
of a charge $dq$ at a distance $r$ from it is
\[
dU = \frac{qdq}{r},
\]
where 
\[
q = \frac{4\pi}{3}r^3 \rho
\]
and $dq = 4\pi r^2 dr$. Thus,
\[
dU = \frac{16\pi^2}{3}\rho^2 r^4dr
\]
and the energy of the entire sphere is
\begin{equation}\label{c5e22}
U = \int_0^s dU = \frac{3}{5}\frac{Q^2}{s}.
\end{equation}
If we consider an elementary charge $q$ to have an infinitesimal radius $r_0$ and
mass $m$ then if the mass has electrodynamic origin,
\[
mc^2 = \frac{3}{5}\frac{q^2}{s}.
\]
Ignoring the factor of $3/5$, the ``classical radius'' of the charge $q$ is defined
to be
\begin{equation}\label{c5e22}
s = \frac{q}{m} c^2.
\end{equation}

\item Quantum effects become important at distance of the order of $\hslash/mc$. The
ratio of this distance to the classical radius of an electron is
\[
\frac{\hslash}{mc}\frac{mc^2}{e^2} = \frac{\hslash c}{e^2} = \frac{1}{\alpha}
\approx 137,
\]
where $\alpha$ is the fine-structure constant. Thus, quantum effects must be taken
into account at distances much larger than the ``classical radius'' of the electron. 
The lower limit of the applicability of classical electrodynamics is much larger 
than the classical electron radius.

\item To avoid the tricky questions of self-energy of elementary charges, we write
the potential $\phi(\vec{r}_a)$ in equation \eqref{c5e20} as
\begin{equation}\label{c5e23}
\phi(\vec{r}_a) = \sum_{b \ne a} \frac{q_b}{R_{ab}},
\end{equation}
where $R_{ab} = |\vec{r}_a - \vec{r}_b|$.



\end{enumerate}
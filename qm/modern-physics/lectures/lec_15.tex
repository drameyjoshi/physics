\documentclass{beamer}
\usefonttheme[onlymath]{serif}
\usepackage{graphicx, hyperref}
\hypersetup{colorlinks, urlcolor=blue}
\usepackage{subcaption}
\usetheme{Boadilla}

\newcommand{\pvec}[1]{\vec{#1}\mkern2mu\vphantom{#1}}

\title{Solid State Physics - 1}
\author{Amey Joshi}
\date{\today}

\begin{document}
\begin{frame}
\titlepage
\end{frame}

\begin{frame}
\frametitle{States of matter}
\begin{enumerate}
\item At an elementary level, we are told that there are three states of matter
- solid, liquid and gas.
\item The plasma state is introduced at a little later state.
\item The world is complicated even at the elementary level. Sometimes it is not
easy to tell a solid from a liquid. Ketchup, dough, honey, coal-tar, sand, mud 
are all in a state that has properties of both solids and liquids. They are some 
of the most complicated materials to study.
\item We will restrict our attention to materials that are unequivocally solids.
\item Among solids some are crystalline and others are amorphous or glassy. We
will study crystals.
\item There is a great similarity between the structure of liquids and amorphous
materials. Both have a short range order. Gases have no order.
\end{enumerate}
\end{frame}

\begin{frame}
\frametitle{Why are crystals interesting?}
\begin{enumerate}
\item The constituents of a crystal, atoms/ions/molecules, are packed 
sufficiently closely for the wavefunctions of their outermost electrons to 
overlap. Solids are quantum mechanical in nature.
\item They show a vast number of very interesting properties
\begin{enumerate}
\item The ratio of resistivity of quartz to that of silver is $10^{25}$.
\item Some solids are transparent others are not.
\item Some solids are coloured, others are not.
\item Some solids are magnetic, others are not.
\item The resistivity of some solids drop to exactly zero at sufficiently low
temperatures.
\item All semiconductor and photonic devices are crystals with carefully 
controlled properties.
\end{enumerate}
\item Crystals exhibit a vast number of exotic quantum phenomena. Nano physics, 
spintronics, photonics, plasmonics all happen in crystals.
\item Very deep connection with quantum field theory as well.
\end{enumerate}
\end{frame}

\begin{frame}
\frametitle{What is a crystal?}
\begin{enumerate}
\item A lattice is a regular arrangement of points in space. There are 
\href{https://en.wikipedia.org/wiki/Bravais_lattice}{14 types} of lattices in
three dimensions. A lattice extends in three dimensions infinitely.
\item There is an atom, ion or a group of atoms at every point in the lattics. 
It is called a basis.
\item A crystal is a lattice plus the basis.
\item An ideal crystal extends infinitely in all directions. There are no ideal
crystals. Most solids are polycrystalline.
\item Yet, crystals have a long-range order that amorphous materials and liquids
lack.
\item Some materials exist in crystalline as well as amorphous (also called 
glassy or vitreous) states. Example B$_{2}$O$_{3}$.
\end{enumerate}
\end{frame}

\begin{frame}
\frametitle{What binds a crystal? Ionic bonds}
\begin{enumerate}
\item Formed when basis is made up of positive and negative ions. The binding is
through electrical attraction. Collapse is prevented by exclusion principle.
\item Ionic solids tend to be hard, with high melting point owing to the 
strength of the ionic bond.
\item They tend to be brittle because slipping of a crystal plane against 
another is prevented by the nature of the bond.
\item They dissolve in polar solvants like water but not in non-polar solvents
like benzene.
\item Most ionic bonds found in nature are in the ionic crystals. 
\item Examples: NaCl (face centred cubic), CsCl (body centered cubic), Po (simple
cubic).
\end{enumerate}
\end{frame}

\begin{frame}
\frametitle{What binds a crystal? Covalent bonds}
\begin{enumerate}
\item Formed when there is a chemical bond between the basis.
\item Strong bonds make the crystals hard and have high melting points. Many are
insoluble in most common solvents.
\item The pronounced directional nature of covalent bonds prevent these solids
from being ductile.
\item One way to find out if a crystal is ionic or covalent is to check its
dielectric constant, $\kappa$. It is related to refractive index as $n = 
\sqrt{\kappa}$. At low frequencies, both ions and electrons move while at high
frequencies only electrons respond. As a result, ionic crystals have frequency
dependent dielectric constant. That is not so for a covalent crystal.
\item Examples: Diamond, SiC, Si, Ge, semiconductor alloys.
\end{enumerate}
\end{frame}

\begin{frame}
\frametitle{What binds a crystal? Van der Waals forces}
\begin{enumerate}
\item A polar molecule has an electric dipole moment $\vec{p}$ and it creates an
electric field
\begin{equation}\label{e1}
\vec{E} = \frac{1}{4\pi\epsilon_0}
\left(\frac{\vec{p}}{r^3}-\frac{3(\vec{p}\cdot\vec{r})}{r^5}\vec{r}\right)
\end{equation}
\item It polarizes other molecules. They too get a dipole moment
\begin{equation}\label{e2}
\pvec{p}^\prime = \alpha \vec{E},
\end{equation}
where $\alpha$ is the molecular polarizability. Two electric dipoles attract 
each other the way to magnets do. The energy of the dipole is $U = 
-\pvec{p}^\prime \cdot \vec{E}$. It can be shown that it depends on $p^2$.
\item Non polar molecules have $\langle\vec{p}\rangle = 0$ but fluctuations of
electron density makes $\langle p^2 \rangle \ne 0$. As a result, there is Van
der Waals attraction even between non-polar molecules.
\item The bonding is very weak. Crystals have low melting points.
\item Examples: Solid H, Ar, CH$_{4}$.
\end{enumerate}
\end{frame}

\begin{frame}
\frametitle{What binds a crystal? Hydrogen bonds}
\begin{enumerate}
\item Non-polar molecules are attracted to each other by Van der Waals forces.
Polar molecules align with each other via hydrogen bonds.
\item Permanent dipole moment of molecules makes hydrogen bonds stronger than
Van der Waals attraction but they are much weaker than covalent bonds. 
\item Hydrogen bonds remain even in liquid state and give water its peculiar
properties.
\item Ice crystals are formed by hydrogen bonding. Crystal structures are 
usually with lot of open spaces making ice less dense than water. Some of these
bonds are retained even after melting so that water contracts between 273 and 
277 K.
\item Examples: HF, DNA.
\end{enumerate}
\end{frame}

\begin{frame}
\frametitle{What binds a crystal? Metallic bonds}
\begin{enumerate}
\item Metallic bonds are found only in solids.
\item The valence electrons are common to the entire solid. They are considered
to be free if their interactions with ions can be neglected. They are considered
to be independent if their interactions with each other can be neglected.
\item Free and independent electrons form an ideal Fermi gas. Electrons end up
having high kinetic energies.
\item Metallic bond is formed if the increase in kinetic energy of electrons is
more than compensated by a drop in their potential energy due to ion cores.
\item Valence electrons are free to move throughout the crystals. That makes 
them good conductors of electricity.
\item Metallic bonds are not directional. That makes metals malleable and 
ductile.
\end{enumerate}
\end{frame}

\begin{frame}
\frametitle{Is that all?}
\begin{enumerate}
\item Solid state physics would be uninteresting if all crystals fell in these
categories.
\item Tin forms a covalent crystal which is semiconducting below 286.2K (grey
tin). It favours metallic bonds at higher temperatures (white tin) and has
high electrical conductivity.
\item Graphite crystals have sheets of hexagonal arrangements of covalently 
bound carbon atoms. The sheets themselves are bond together with unsaturated 
bonds of a metallic character.
\item Carbon forms many more exotic structures C$_{60}$ bucky balls which are
held together as a solid by Van der Waals forces. They also form nanotubes whose
geometry decides whether they are semiconductors or good conductors.
\item This is just the start. The real fun is in the electronic properties.
\end{enumerate}
\end{frame}
\end{document}
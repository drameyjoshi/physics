\documentclass{beamer}
\usefonttheme[onlymath]{serif}
\usepackage{graphicx,hyperref}
\hypersetup{colorlinks,urlcolor=blue}
\usetheme{Boadilla}

\newcommand{\pvec}[1]{\vec{#1}\mkern2mu\vphantom{#1}}

\title{Particle Properties of Waves}
\author{Amey Joshi}
\date{\today}

\begin{document}
\begin{frame}
\titlepage
\end{frame}

\begin{frame}
\frametitle{Particles and Waves}
\begin{enumerate}
\item A particle is a point mass with a precisely defined position and momentum.
\item The dimensions of a particle are small compared to the other dimensions of
the problem. The earth is a particle when one studies the solar system; not so
when we study ocean currents.
\item Waves are periodic changes in a physical quantity like the pressure, the
strain, the position of fluid particles or electric/magnetic field strengths
at a point.
\item The two were known to be starkly different. Almost ...
\item Light was thought to travel like particles in geometric optics but like
a wave when one studies interference and diffraction.
\end{enumerate}
\end{frame}

\begin{frame}
\frametitle{Black body radiation - 1}
\begin{enumerate}
\item An ideal black body absorbs all radiation incident on it. It emits all
radiation to maintain thermal equilibrium.
\item We consider the spectral energy density of radiation emitted from a black
body. Spectral energy density is the energy of radiation with wavelength between
$\lambda$ and $\lambda + d\lambda$ per unit volume.
\item The peak of the spectral energy density depends on the temperature of the
black body. 
\item The spectral density looks like 
\href{https://www.britannica.com/science/blackbody}{this}.
\item Read the temperatures, spot the maxima. Can you build a thermometer from
 this plot?
\item Classical physics could not explain the shape of the spectrum.
\end{enumerate}
\end{frame}

\begin{frame}
\frametitle{Black body radiation - 2}
Radiation is trapped in the cavity in the form of standing waves. In a cubical
cavity of length $L$, a standing wave will be formed parallel to $x$-axis if
\begin{equation}\label{e1}
L = n_x\frac{\lambda}{2}, n_x = 1, 2, \ldots
\end{equation}
In general, an arbitrary standing wave must obey 
\begin{equation}\label{e2}
n^2 = n_x^2 + n_y^2 + n_z^2 = \left(\frac{2L}{\lambda}\right)^2.
\end{equation}
The number of standing waves with wavelengths between $\lambda$ and $\lambda + 
d\lambda$ in the first octant between radii $n$ and $n + dn$ is proportional to
the volume of the region.
\[
\frac{1}{8} \times 4\pi n^2 dn = \frac{1}{2} \pi n^2 dn.
\]
But there are two polarisation modes. Therefore, the number of standing waves
is twice this quantity,
\begin{equation}\label{e3}
G(n)dn = \pi n^2dn
\end{equation}
\end{frame}

\begin{frame}
\frametitle{Black body radiation - 3}
From \eqref{e2} $n = 2L/\lambda = 2L\nu/c$. Therefore, equation \eqref{e3} 
becomes
\begin{equation}\label{e4}
G(\nu)d\nu = \pi \frac{8L^3}{c^3}\nu^2 d\nu.
\end{equation}
The number of standing waves per unit volume, also called `density of states'
is
\begin{equation}\label{e5}
g(\nu)d\nu = \frac{8\pi}{c^3}\nu^2 d\nu.
\end{equation}
Equivalently, since $g(\lambda)d\lambda = -g(\nu)d\nu$,
\begin{equation}\label{e6}
g(\lambda)d\lambda = \frac{8\pi}{\lambda^4} d\lambda.
\end{equation}
Nothing controversial so far. These calculations were done by Lord Rayleigh
and Sir James Jeans.
\end{frame}

\begin{frame}
\frametitle{Black body radiation - 4}
\begin{enumerate}
\item Lord Rayleigh assumed the classical equipartition theorem to conclude that
the energy per mode is $kT$ and hence the spectral density is
\begin{equation}\label{e7}
u(\nu)d\nu = \frac{8\pi}{c^3} kT \nu^2 d\nu.
\end{equation}
\item $u(\nu) \Rightarrow \infty$ as $\nu \rightarrow \infty$. Not what you 
saw experimentally.
\item Max Planck proposed that energy can be exchanged only in packets of
$h\nu$ and showed that the average energy should be
\begin{equation}\label{e8}
\langle U \rangle = \frac{h\nu}{\exp\left(\frac{h\nu}{kT}\right) - 1}
\end{equation} 
\end{enumerate}
\end{frame}

\begin{frame}
\frametitle{Black body radiation - 5}
The spectral density is,
\begin{equation}\label{e9}
u(\nu)d\nu = 
   \frac{8\pi h\nu^3}{c^3}\frac{d\nu}{\exp\left(\frac{h\nu}{kT}\right) - 1}
\end{equation}
It perfectly explains the black body radiation spectrum. It differs from the
Rayleigh-Jeans formula in just one way, the expression for the average energy 
in \eqref{e8}. Planck derived it using 
\begin{enumerate}
\item the thermodynamic relation between entropy $S$, internal energy $U$ and
temperature $T$, developed in mid 1800s;
\item the relation between entropy and the number of states, developed by
Boltzmann in 1880s;
\item $E = h\nu$.
\end{enumerate}
Question: What does entropy mean in the context of radiation?
\end{frame}

\begin{frame}
\frametitle{What about $E = h\nu$?}
\begin{enumerate}
\item There is nothing in classical electrodynamics to support it. Although
Planck attempted to do so, it was nothing short of revolutionary.
\item Planck restricted the idea only to energy transfer. 
\item Others took it more seriously. Einstein used it to explain the 
photoelectric effect.
\item Yet, this is not a conclusive proof of particle nature of light. It
came from Compton effect.
\end{enumerate}
\end{frame}

\begin{frame}
\frametitle{Compton effect - 1}
\begin{enumerate}
\item In classical physics, waves interfere and particles collide. Can waves
collide?
\item Compton's experiment consisted in shining a monochromatic beam of X-rays
on metals and observing the scattered radiation. He found that the scattered
radiation had a longer wavelength. The difference in wavelengths is called
`Compton shift'.
\item Compton explained it by assuming that a beam of X-rays is really a stream
of particles each with (a) energy $E = h\nu$ and (b) momentum $p = E/c$.
\item He derived an expression for the change in wavelength using the energy 
and momentum conservation laws used in elementary classical mechanics.
\end{enumerate}
\end{frame}

\begin{frame}
\frametitle{Compton effect - 2}
\begin{enumerate}
\item The expression for Compton shift is
\begin{equation}\label{e10}
\lambda^\prime - \lambda = \frac{h}{m_ec}(1 - \cos\phi),
\end{equation}
where $m_e$ is the mass of the electron and $\phi$ is the angle of observation,
with respect to the incident direction, of the scattered direction.
\item Does it look reasonable? Examine the dependence on $\phi$ and $m_e$.
\item Could Compton have used visible light? Look at the order of magnitude
of $h/m_ec$.
\item The chemist G. N. Lewis coined the term `photons' for quanta of light.
\textbf{Do not} visualise photons are corpuscules of light carrying energy
$h\nu$. The idea is subtler than that.
\end{enumerate}
\end{frame}
\end{document}


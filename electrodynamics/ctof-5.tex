\chapter{Constant electromagnetic fields}\label{c5}
\begin{enumerate}
\item If the fields are independent of time then they obey
\begin{eqnarray}
\dive\vec{E} &=& 4\pi\rho \label{c5e1} \\
\curl\vec{E} &=& 0 \label{c5e2} \\
\dive\vec{H} &=& 0 \label{c5e3} \\
\curl\vec{H} &=& 4\pi\frac{\vec{J}}{c}. \label{c5e4}
\end{eqnarray}
The symmetry of the situation reveals itself if we note the correspondence $\vec{E}
\mapsto \vec{H}$ and $\rho \mapsto \vec{J}/c$ and $\dive \mapsto \curl$. Equations
\eqref{c5e2} and \eqref{c5e3} suggest that
\begin{eqnarray}
\vec{E} &=& -\grad\phi \label{c5e5} \\
\vec{H} &=& \curl\vec{A} \label{c5e6}
\end{eqnarray}
which is understandable if we note that only the time component of $J^\mu$ 
determines $\vec{E}$ and its space component determines $\vec{H}$. Therefore,
only the time component of $\vec{A}^\mu$ suffices to describe the electric
field and its space component suffices to describe the magnetic field.

\item From equations \eqref{c5e1} and \eqref{c5e5} we have
\begin{equation}\label{c5e7}
\nabla^2\phi = -4\pi\rho.
\end{equation}
In region where there are no charges, the potential satisfies Laplace's equation
\begin{equation}\label{c5e8}
\nabla^2\phi = 0.
\end{equation}
This equation immediately implies that the potential function cannot have a local
minimum or a maximum. The nature of extremum of a function of three variables is
determined by its Hessian matrix. It is a minimum (maximum) if the Hessian matrix
is positive (negative) definite. Since the trace of a matrix is the sum of its
eigenvalues, we have an equivalent condition that the trace of the Hessian matrix
is positive (negative) for an extremum to be a minimum (maximum). But the trace
of Hessian is $\nabla^2\phi$ evaluated at the extremum. If $\phi$ is a solution
of Laplace's equation \eqref{c5e8}, then the Hessian can never be positive or
negative definite.

\item The field of a point charge can be derived using Gauss' law. Let the point
charge be at $\vec{r}^\op$. In order to find the field at $\vec{r}$, we consider
a sphere centred at $\vec{r}^\op$ and passing through $\vec{r}$. As the space is
isotropic, we expect the magnitude of $\vec{E}$ to be the same at all points on the
sphere and its direction along $\vec{r} - \vec{r}^\op$. Therefore,
\[
4\pi |\vec{r} - \vec{r}^\op|^2 = q
\]
or
\begin{equation}\label{c5e9}
\vec{E} = \frac{q}{|\vec{r} - \vec{r}^\op|^3}(\vec{r} - \vec{r}^\op).
\end{equation}
If we define
\begin{equation}\label{c5e10}
\vec{R} = \vec{r} - \vec{r}^\op
\end{equation}
then we can write \eqref{c5e9} simply as
\begin{equation}\label{c5e11}
\vec{E} = \frac{q}{R^3}\vec{R} = \frac{q}{R^2}\uv{R}.
\end{equation}
Equation \eqref{c5e11} is \emph{Coulomb's law}. The potential of this field is
\begin{equation}\label{c5e12}
\phi = \frac{q}{R}.
\end{equation}
If there are discrete, point charges at $\vec{r}_a$ then the field due to all of 
them is
\begin{equation}\label{c5e13}
\vec{E} = \sum_a \frac{q_a}{R_a^2}\uv{R_a},
\end{equation}
where $\vec{R}_a = \vec{r}^\op - \vec{r}$. The potential is
\begin{equation}\label{c5e14}
\phi = \sum_a\frac{q_a}{R_a}.
\end{equation}
For a continuum of charges, \eqref{c5e11} and \eqref{c5e12} generalise to
\begin{eqnarray}
\phi(\vec{r}) &=& \int\frac{\rho(\vec{r}^\op)}{R}dV^\op \label{c5e15} \\
\vec{E}(\vec{r}) &=& \int\frac{\rho(\vec{r}^\op)}{R^2}\uv{R}dV^\op  \label{c5e16}
\end{eqnarray}

\item For a point charge, $\rho = q\delta(\vec{R})$. In this case, \eqref{c5e7}
becomes,
\[
\nabla^2\phi = 4\pi q\delta(\vec{R}).
\]
From \eqref{c5e12}, we also have
\begin{equation}\label{c5e17}
\nabla^2\left(\frac{1}{R}\right) = -4\pi\delta(\vec{R}).
\end{equation}

\item In the absence of magnetic field, the energy density is
\begin{equation}\label{c5e18}
W = \frac{1}{8\pi}\int E^2dV.
\end{equation}
From \eqref{c5e5},
\[
W = -\frac{1}{8\pi}\int\vec{E}\cdot\grad\phi dV.
\]
Since $\dive(\phi\vec{E}) = \grad\phi\cdot\vec{E} + \phi\dive\vec{E}$, we have
\begin{eqnarray*}
W &=& \frac{1}{8\pi}\int\phi\dive\vec{E}dV - \frac{1}{8\pi}\int\dive(\phi\vec{E})dV \\
  &=& \frac{1}{8\pi}\int\phi\dive\vec{E}dV - \frac{1}{8\pi}\oint \phi\vec{E}\cdot d\vec{f}.
\end{eqnarray*}
The potential goes as $O(r^{-1})$ and the field as $O(r^{-2})$ so that the surface
integral goes as $O(r^{-1})$. If we choose a large enough surface the second 
integral can be made as small as we desire. Thus, we have
\[
W = \frac{1}{8\pi}\int\phi\dive\vec{E}dV
\]
or using \eqref{c5e1},
\begin{equation}\label{c5e19}
W = \frac{1}{2}\int\rho\phi dV.
\end{equation}
If the charge density consists od point charges, equation \eqref{c5e19} becomes
\begin{equation}\label{c5e20}
W = \sum_a q_a\phi(\vec{r}_a),
\end{equation}
where $\phi(\vec{r}_a)$ is the potential due to all charges at the point 
$\vec{r}_a$ where the charge $q_a$ is located. This formula immediately runs 
into a tricky situation.

\item The potential $\phi$ due to a charge $q$ at $\vec{r}^\op$ at the same point
is infinity according to \eqref{c5e12}. As a result, from \eqref{c5e20}, a point
charge has infinite energy and therefore an infinite mass. To avoid such an absurd
result, we say that the laws of classical electrodynamics are not applicable at
extremely short distances. In fact, we cannot even ask the question if the mass of
a charge has electrodynamic origin, that is, it exists because of electrodynamic
energy.

The energy of a uniformly charged sphere of radius $s$ can be computed as follow.
We first note that its charge density is
\begin{equation}\label{c5e21}
\rho = \frac{Q}{\frac{4\pi}{3}s^3}.
\end{equation}
The field due to a sphere of radius $r$ and a uniform charge density $\rho$ at 
points on and outside it is as if the entire charge was concentrated at its 
centre. If $q(r)$ is the charge in a sphere of radius $r$ then the energy
of a charge $dq$ at a distance $r$ from it is
\[
dU = \frac{qdq}{r},
\]
where 
\[
q = \frac{4\pi}{3}r^3 \rho
\]
and $dq = 4\pi r^2 dr$. Thus,
\[
dU = \frac{16\pi^2}{3}\rho^2 r^4dr
\]
and the energy of the entire sphere is
\begin{equation}\label{c5e22}
U = \int_0^s dU = \frac{3}{5}\frac{Q^2}{s}.
\end{equation}
If we consider an elementary charge $q$ to have an infinitesimal radius $r_0$ and
mass $m$ then if the mass has electrodynamic origin,
\[
mc^2 = \frac{3}{5}\frac{q^2}{s}.
\]
Ignoring the factor of $3/5$, the ``classical radius'' of the charge $q$ is defined
to be
\begin{equation}\label{c5e23}
s = \frac{q}{m} c^2.
\end{equation}

\item Quantum effects become important at distance of the order of $\hslash/mc$. The
ratio of this distance to the classical radius of an electron is
\[
\frac{\hslash}{mc}\frac{mc^2}{e^2} = \frac{\hslash c}{e^2} = \frac{1}{\alpha}
\approx 137,
\]
where $\alpha$ is the fine-structure constant. Thus, quantum effects must be taken
into account at distances much larger than the ``classical radius'' of the electron. 
The lower limit of the applicability of classical electrodynamics is much larger 
than the classical electron radius.

\item To avoid the tricky questions of self-energy of elementary charges, we write
the potential $\phi(\vec{r}_a)$ in equation \eqref{c5e20} as
\begin{equation}\label{c5e24}
\phi(\vec{r}_a) = \sum_{b \ne a} \frac{q_b}{R_{ab}},
\end{equation}
where $R_{ab} = |\vec{r}_a - \vec{r}_b|$.

\item We next consider the field of a charge $q$ moving with a uniform velocity
$\vec{v}$. We align the axes such that $\vec{v} = v\uv{x}$ and call the frame
moving with the charge $K^\op$. Let $K$ be the laboratory frame in with axes 
parallel to those of $K^\op$ and let their origins coincide at $t = 0$. Let $P$
be a field point with coordinates $(x^\op, y^\op, z^\op)$ in $K^\op$. Then the
scalar potential at $P$ in $K^\op$ is
\begin{equation}\label{c5e25}
\phi^\op = \frac{q}{R^\op},
\end{equation}
where 
\begin{equation}\label{c5e26}
R^\op = \sqrt{{x^\op}^2 + {y^\op}^2 + {z^\op}^2}
\end{equation}
and the vector potential is
\begin{equation}\label{c5e27}
\vec{A}^\op = 0.
\end{equation}
Thus,
\begin{equation}\label{c5e28}
{A^\op}^\mu = \left(\frac{q}{R^\op}, 0, 0, 0\right).
\end{equation}
From \eqref{c1e46}, the 4-potential in $K$ is
\begin{equation}\label{c5e29}
A^\mu = \left(\gamma\frac{q}{R^\op}, \beta\gamma\frac{q}{R^\op}, 0, 0\right),
\end{equation}
where
\begin{eqnarray*}
\bm{\beta} &=& \frac{\vec{v}}{c} \\
\gamma &=& \frac{1}{\sqrt{1 - \beta^2}}.
\end{eqnarray*}
However, \eqref{c5e29} is not a correct expression because we continue to have
$R^\op$ in it. The relation between $(x, y, z)$ and $(x^\op, y^\op, z^\op)$ is
given the Lorentz transformation
\[
x^\op = \gamma(x - vt) \;;\; y^\op = y \;;\; z^\op = z
\]
so that
\begin{equation}\label{c5e30}
{R^\op}^2 = \gamma^2(x - vt)^2 + (y^2 + z^2).
\end{equation}
We define a function $R^\ast$ of the coordinates $x, y, z$ as
\begin{equation}\label{c5e31}
R^\ast(x, y, z) = \sqrt{(x - vt)^2 + (1 - \beta^2)(y^2 + z^2)}
\end{equation}
so that
\begin{equation}\label{c5e32}
\frac{R^\op(x^\op, y^\op, z^\op)}{\gamma} = R^\ast(x, y, z)
\end{equation}
and the partial Lorentz transformation of \eqref{c5e29} can be completed to
\begin{equation}\label{c5e33}
A^\mu = \left(\frac{q}{R^\ast}, \beta\frac{q}{R^\ast}, 0, 0\right)
\end{equation}
or
\begin{eqnarray}
\phi(x, y, z) &=& \frac{q}{R^\ast} \label{c5e34} \\
\vec{A}(x, y, z) &=& \frac{q\vec{v}}{cR^\ast} \label{c5e35}
\end{eqnarray}

\item The electric and magnetic fields in $K^\op$ frame are
\begin{eqnarray}
\vec{E}^\op &=& \frac{q}{{R^\op}^3}\vec{R}^\op \label{c5e36} \\
\vec{H}^\op &=& 0 \label{c5e37}
\end{eqnarray}
We use equations \eqref{c3e81} to \eqref{c3e83} to get the electric field in the
$K$ frame.
\begin{eqnarray}
E_x &=& E_x^\op \label{c5e38} \\
E_y &=& \gamma E_y^\op \label{c5e39} \\
E_z &=& \gamma E_z^\op \label{c5e40}
\end{eqnarray}
Using \eqref{c5e36} we get
\begin{eqnarray}
E_x &=& \frac{q}{{R^\op}^3}x^\op \label{c5e41} \\
E_y &=& \gamma\frac{q}{{R^\op}^3}y^\op \label{c5e42} \\
E_z &=& \gamma\frac{q}{{R^\op}^3}z^\op \label{c5e43}
\end{eqnarray}
These equations are not satisfactory because their rhs are written in terms of
the primed coordinates. Using $R^\op = \gamma R^\ast$ from \eqref{c5e32} and the
Lorentz transformation formulae
\begin{eqnarray}
E_x &=& \frac{q}{{R^\ast}^3}\frac{x - vt}{\gamma^2} \label{c5e44} \\
E_y &=& \frac{q}{{R^\ast}^3}\frac{y}{\gamma^2} \label{c5e45} \\
E_z &=& \frac{q}{{R^\ast}^3}\frac{z}{\gamma^2} \label{c5e46}
\end{eqnarray}
so that
\begin{equation}\label{c5e47}
\vec{E} = (1 - \beta^2)\frac{q}{{R^\ast}^3}\vec{R},
\end{equation}
where
\begin{equation}\label{c5e48}
\vec{R} = (x - vt)\uv{x} + y\uv{y} + z\uv{z}.
\end{equation}
If $\theta$ is the angle between $\vec{R}$ and $\vec{v}$ then
\begin{equation}\label{c5e49}
\vec{v}\cdot\vec{R} = v(x - vt) = vR\cos\theta \Rightarrow x - vt = R\cos\theta
\end{equation}
so that,
\begin{equation}\label{c5e50}
y^2 + z^2 = R^2 - (x - vt)^2 = R^2\sin^2\theta.
\end{equation}
From \eqref{c5e31}, \eqref{c5e49} and \eqref{c5e50} we get
\begin{equation}\label{c5e51}
{R^\ast}^2 = R^2\cos^2\theta + (1 - \beta^2)R^2\sin^2\theta = 
R^2(1 - \beta^2\sin^2\theta).
\end{equation}
We can now write \eqref{c5e47} solely in terms of $\vec{R}$ as
\begin{equation}\label{c5e52}
\vec{E} = \frac{q\vec{R}}{R^3} \frac{1 - \beta^2}{(1 - \beta^2\sin^2\theta)^{3/2}}.
\end{equation}
Note that $\vec{R}$ is the position of the (moving) charge in $K$ frame.

\item Lorentz transformation of $\vec{E}$ and $\vec{H}$ are a consequence of the
transformation properties of the field tensor. One should use \eqref{c3e81} to
\eqref{c3e84} instead of evaluating $-\grad\phi$ with $\phi$ given by \eqref{c5e34}.
That is, although
\[
\phi^\op = \frac{q}{R^\op} \mapsto \phi = \frac{q}{R^\ast},
\]
the mapping
\[
-\grad^\op\phi^\op \mapsto -\grad\phi
\]
does not hold.

\item In \eqref{c5e52}, $\theta$ is the angle between the radius vector of the
field point $\vec{R}$ and the direction of motion. The magnitude of the field
is
\begin{equation}\label{c5e53}
E = \frac{q}{R^2}\frac{1 - \beta^2}{(1 - \beta^2\sin^2\theta)^{3/2}}
\end{equation}
Figure \ref{c5f1} shows how the magnitude of the electric field varies with 
$\theta$ for different values of $\beta$. As the speed of the charge approaches
$c$, the electric field gets increasingly concentrated along the directions
perpendicular to the motion of the particle.
\begin{figure}[!ht]
\includegraphics[scale=0.8]{ex2}
\caption{$E(\theta)$}
\label{c5f1}
\end{figure}
For an observer in the $K$ frame, the charge in motion creates an electric current
and therefore expects to observe magnetic field. In fact, equations \eqref{c3e84}
to \eqref{c3e86} give
\begin{eqnarray}
H_x &=& 0 \label{c5e54} \\
H_y &=& -\gamma\beta E^\op_z \label{c5e55} \\
H_z &=& \gamma\beta E^\op_y \label{c5e56}
\end{eqnarray}
Since $\beta = v/c = v_x/c$, using \eqref{c5e39} and \eqref{c5e40}, we get
\begin{eqnarray}
H_x &=& 0 \label{c5e57} \\
H_y &=& -\frac{1}{c}v_xE_z \label{c5e58} \\
H_z &=& \frac{1}{c}v_xE_y. \label{c5e59}
\end{eqnarray}
These three equations can be combined as
\begin{equation}\label{c5e60}
\vec{H} = \frac{1}{c}\vec{v} \times \vec{E} = \bm{\beta} \times \vec{E}.
\end{equation}

\item We now consider the motion of a charge $q$ in the electric field produced
by another one $Q$. Let their masses by $m$ and $M$ respectively. Assume that
$m \ll M$ so that the charge $Q$ is almost stationary. The problem now reduces to
studying the motion of $q$ in a potential $\phi = Q/r$.

The energy of the charged particle is
\begin{equation}\label{c5e61}
\mathcal{E} = \sqrt{p^2c^2 + m^2c^4} + \frac{\alpha}{r},
\end{equation}
where
\begin{equation}\label{c5e62}
\alpha = qQ.
\end{equation}
The motion in a central field is confined to a plane. Align the coordinate axes
such that $Q$ is at the origin and the motion happens in $xy$ plane. Furthermore,
since the torque on the particle is zero, its angular momentum is a constant of
motion. The particle's velocity in polar coordinates is
\begin{equation}\label{c5e63}
\vec{v} = \dot{r}\uv{r} + r\dot{\theta}\uv{\theta}
\end{equation}
so that $v^2 = \dot{r}^2 + r^2\dot{\theta}^2$. Since $\vec{L} = \vec{r} \times
\vec{p} = m\gamma\vec{r} \times \vec{p}$, (using \eqref{c2e8}) we have $L = 
\gamma mr^2\dot{\theta}$. We can express $v^2$ in terms of $L$ as
\[
v^2 = \dot{r}^2 + \frac{L^2}{m^2\gamma^2 r^2}
\]
so that
\begin{equation}\label{c5e64}
p^2 = m^2\gamma^2\dot{r}^2 + \frac{L^2}{r^2} = p_r^2 + \frac{L^2}{r^2}.
\end{equation}
Therefore, \eqref{c5e61} becomes
\begin{equation}\label{c5e65}
\mathcal{E} = c\sqrt{p_r^2 + \frac{L^2}{r^2} + m^2c^2} + \frac{\alpha}{r}.
\end{equation}

\item If $\alpha > 0$ then as $r$ increases, the rhs of \eqref{c5e64} also
increases even if $p_r \rightarrow 0$. Since $\mathcal{E}$ is a constant of
motion, a drop in $r$ can be compensated by a drop in $p_r$ only to a limited
extent. This prevents the two charges from coming arbitrarily close to each 
other.

To analyse the condition $\alpha < 0$, write \eqref{c5e65} as
\[
\mathcal{E} = \frac{Lc}{r}\sqrt{\frac{r^2(p_r^2 + m^2c^2)}{L^2} + 1} - \frac{|\alpha|}{r}.
\]
so that we can approximate
\[
\mathcal{E} \approx \frac{Lc}{r} + \frac{cr}{2L}(p_r^2 + m^2c^2) - \frac{|\alpha|}{r}.
\]
for small $r$. If $Lc > |\alpha|$ then the rhs blows up as $r \rightarrow 0$. 
Therefore, to enforce constancy of $\mathcal{E}$, $r$ cannot be permitted to
decrease indefinitely. If $Lc < |\alpha|$, if $p_r = m\gamma\dot{r}$ can become
indefinitely large, $r$ may be allowed to become arbitrarily small while still
keeping $\mathcal{E}$ constant.

In the non-relativistic case, we would have had
\[
\mathcal{E} = \frac{1}{2}mv_r^2 + \frac{L^2}{2mr^2} - \frac{|\alpha|}{r}.
\]
If $L \ne 0$ then the first two terms, which are always positive, rise much faster
than the third term falls. Therefore, energy conservation does not allow the two
charges to come arbitrarily close to each other. On the other hand, if $L = 0$, 
then $v_r$ can rise enough to compensate the drop in the third term and still
keep  $\mathcal{E}$ constant. Thus, the two charges can come arbitrarily close only
if they approach to each other head-on.

\item The Hamilton-Jacobi equation \eqref{c3e15} for this problem is
\[
(\grad S)^2 - \frac{1}{c^2}\left(\pdt{S}{t} + \frac{\alpha}{r}\right)^2 + m^2c^2 = 0.
\]
Since $Q$ is stationary, $q$ experiences only the electric field and therefore 
$\vec{A} = 0$. The gradient in plane-polar coordinates is
\begin{equation}\label{c5e66}
\grad S = \pdt{S}{r}\uv{r} + \frac{1}{r}\pdt{S}{\phi}
\end{equation}
so that the Hamilton-Jacobi equation becomes
\begin{equation}\label{c5e67}
\left(\pdt{S}{r}\right)^2 + \frac{1}{r^2}\left(\pdt{S}{\phi}\right)^2 - \frac{1}{c^2}
\left(\pdt{S}{t} + \frac{\alpha}{r}\right)^2 + m^2c^2 = 0.
\end{equation}
or
\[
\left(\pdt{S}{r}\right)^2 + \frac{1}{r^2}\left(\pdt{S}{\phi}\right)^2 - \frac{1}{c^2}
\left(\pdt{S}{t}\right)^2 + 2\frac{\alpha}{rc^2}\pdt{S}{t} + \frac{\alpha^2}{c^2r^2}
+ m^2c^2 = 0.
\]
This is a non-linear, first-order pde. We try a solution of the form
\begin{equation}\label{c5e68}
S(r, \phi, t) = -\mathcal{E}t + L\phi + f(r)
\end{equation}
to get
\[
{f^\op(r)}^2 + \frac{L^2}{r^2} - \frac{\mathcal{E}^2}{c^2} - 2\frac{\alpha\mathcal{E}}{rc^2}
+ \frac{\alpha^2}{c^2r^2} + m^2c^2 = 0.
\]
or
\[
f^\op = \pm\sqrt{\frac{1}{c^2}\left(\mathcal{E} - \frac{\alpha}{r}\right)^2 - \frac{L^2}{r^2} 
-m^2c^2}
\]
The solution, thus, is
\begin{equation}\label{c5e69}
S(r, \phi, t) = -\mathcal{E}t + L\phi \pm 
\frac{1}{c}\int\sqrt{\left(\mathcal{E}-\frac{\alpha}{r}\right)^2-\frac{L^2c^2}{r^2} -m^2c^4}dr
\end{equation}
The trajectories are given by (find out why)
\begin{equation}\label{c5e70}
\pdt{S}{L} = \text{constant.}
\end{equation}
Differentiating \eqref{c5e69},
\begin{equation}\label{c5e71}
\pdt{S}{L} = \phi \mp Lc\int\frac{dr}{r\sqrt{A^2r^2 - 2Br + C^2}},
\end{equation}
where
\begin{eqnarray}
A^2 &=& \mathcal{E}^2 - m^2c^4 \label{c5e72} \\
B   &=& \alpha\mathcal{E} \label{c5e73} \\
C^2 &=& \alpha^2 - L^2c^2 \label{c5e74}
\end{eqnarray}
We now use the result
\begin{equation}\label{c5e75}
\int\frac{dx}{x\sqrt{a^2x^2 - 2bx + c^2}} = 
-\frac{1}{c}\tanh^{-1}\left(\frac{c^2 - bx}{c\sqrt{a^2x^2 - 2bx + c^2}}\right) +
\text{const}.
\end{equation}
in equation \eqref{c5e71} to get
\begin{equation}\label{c5e76}
\phi_0 = \phi \pm \frac{Lc}{C}\tanh^{-1}\left(\frac{C^2 - Br}{C\sqrt{A^2r^2 - 2Br + C^2}}\right),
\end{equation}
where $\phi_0$ is a constant of integration.

We consider the following cases:
\begin{enumerate}
\item $\alpha^2 > L^2c^2$, that is $C$ is real. Then \eqref{c5e71} can be written as
\[
\tanh\left(\frac{C}{Lc}(\phi_0 - \phi)\right) = \frac{C^2 - Br}{C\sqrt{A^2r^2 - 2Br + C^2}}.
\]
Let
\begin{equation}\label{c5e77}
X = \frac{C}{Lc}(\phi_0 - \phi) = \sqrt{\frac{\alpha^2}{L^2c^2} - 1}(\phi_0 - \phi)
\end{equation}
so that
\begin{eqnarray*}
\tanh^2 X &=& \frac{(C^2 - Br)^2}{C^2(A^2r^2 - 2Br + C^2)} \\
C^2(A^2r^2 + C^2 - 2Br)\tanh^2X &=& C^4 - 2BC^2r + B^2r^2 \\
C^2A^2r^2\tanh^2X &=& (C^4 - 2BC^2r)\sech^2X + B^2r^2 \\
C^2A^2r^2\sinh^2X &=& C^2(C^2 - 2B^2r) + B^2r^2\cosh^2X \\
C^2A^2r^2(\cosh^2X - 1) &=& C^2(C^2 - 2B^2r) + B^2r^2\cosh^2X \\
(C^2A^2 - B^2)r^2\cosh^2X &=& C^2(A^2r^2 - 2B^2r + C^2)
\end{eqnarray*}
Now,
\[
A^2C^2-B^2 = (L^2c^2 -\alpha^2)m^2c^4 -\mathcal{E}^2L^2c^2
\]
Since we assumes $\alpha^2 > L^2c^2$, the rhs of the above equation is negative.
Therefore, before taking the square-root, we flip the signs to get
\begin{eqnarray*}
(B^2 - C^2A^2)r^2\cosh^2X &=& C^2(2B^2r - A^2r^2 - C^2) \\
\sqrt{B^2 - C^2A^2}\cosh X &=& \frac{C^2}{r}\sqrt{2\frac{B^2}{C^2}r - \frac{A^2}{C^2}r^2 - 1}
\end{eqnarray*}
Thus,
\[
\pm c\sqrt{(L\mathcal{E})^2 + m^2c^2(\alpha^2 - L^2c^2)}\cosh X = 
\frac{C^2}{r}\sqrt{2\frac{B^2}{C^2}r - \frac{A^2}{C^2}r^2 - 1}
\]
Since $\cosh$ is an even function,
\begin{eqnarray}
\pm c\sqrt{(L\mathcal{E})^2 + m^2c^2(\alpha^2 - L^2c^2)}
\cosh\left((\phi - \phi_0)\sqrt{\frac{\alpha^2}{L^2c^2} - 1}\right) &=& \nonumber\\
\frac{C^2}{r}\sqrt{2\frac{B^2}{C^2}r - \frac{A^2}{C^2}r^2 - 1} \label{c5e78}
\end{eqnarray}
We can choose $\phi_0$ such that this equation reduces to
\begin{equation}\label{c5e79}
\pm c\sqrt{(L\mathcal{E})^2 + m^2c^2(\alpha^2 - L^2c^2)}
\cosh\left((\phi - \phi_0)\sqrt{\frac{\alpha^2}{L^2c^2} - 1}\right) = \frac{C^2}{r}.
\end{equation}
This can always be done because \eqref{c5e78} is of the form
\begin{equation}\label{c5e80}
A\cosh(B(x_0-x)) = C^2f(x).
\end{equation}
We can write it as
\begin{equation}\label{c5e81}
x_0 = x + \frac{1}{B}\cosh^{-1}\left(\frac{C^2f(x)}{A}\right).
\end{equation}
Choose $x_0$ such that if $f(x^\op) = 1$ then
\begin{equation}\label{c5e82}
x_0 = x^\op + \frac{1}{B}\cosh^{-1}\left(\frac{C^2}{A}\right).
\end{equation}

\item $\alpha^2 < L^2c^2$. In this case,
\begin{eqnarray*}
\cosh\left((\phi - \phi_0)\sqrt{\frac{\alpha^2}{L^2c^2} - 1}\right) 
 &=& \cosh\left(i(\phi - \phi_0)\sqrt{1 - \frac{\alpha^2}{L^2c^2}}\right) \\
 &=& \cos\left((\phi - \phi_0)\sqrt{1 - \frac{\alpha^2}{L^2c^2}}\right)
\end{eqnarray*}
and \eqref{c5e79} becomes
\begin{equation}\label{c5e83}
\pm c\sqrt{(L\mathcal{E})^2 - m^2c^2(L^2c^2 - \alpha^2)}
\cos\left((\phi - \phi_0)\sqrt{1 - \frac{\alpha^2}{L^2c^2}}\right) = \frac{C^2}{r}.
\end{equation}

Note that, when $\phi \mapsto \phi + 2\pi$ increases by $2\pi$, 
\begin{eqnarray*}
\cos\left((\phi - \phi_0)\sqrt{1 - \frac{\alpha^2}{L^2c^2}}\right)  
 &\mapsto& \cos\left((\phi - \phi_0)\sqrt{1 - \frac{\alpha^2}{L^2c^2}} + 2\pi\sqrt{1 - \frac{\alpha^2}{L^2c^2}}\right) \\
 &\ne& \cos\left((\phi - \phi_0)\sqrt{1 - \frac{\alpha^2}{L^2c^2}}\right)
\end{eqnarray*}
so that the orbits described by \eqref{c5e83} are not closed curves like 
ellipses but are rosettes.

\item $\alpha^2 = L^2c^2$. In this case, \eqref{c5e71} becomes
\begin{equation}\label{c5e84}
\phi_0 = \phi \mp Lc\int\frac{dr}{r\sqrt{A^2r^2 - 2Br}} =  
\phi \mp \frac{Lc}{B}\frac{\sqrt{A^2r^2 - 2Br}}{r}.
\end{equation}
Squaring both sides,
\[
(\phi - \phi_0)^2 = \frac{Lc}{B}\frac{A^2r^2 - 2Br}{r^2}
\]
Using the expressions for $A$ and $B$ in \eqref{c5e72} and \eqref{c5e73}, we get
\[
(\phi - \phi_0)^2 = \frac{L^2c^2}{\alpha^2\mathcal{E}^2}
\frac{(\mathcal{E}^2 - m^2c^4)r^2 - 2\alpha\mathcal{E}r}{r^2}
\]
or
\[
\left(\frac{\mathcal{E}\alpha}{Lc}\right)^2(\phi - \phi_0)^2 = 
\mathcal{E}^2 - m^2c^4 - 2\frac{\alpha\mathcal{E}}{r}
\]
Choosing initial conditions such that $\phi_0 = 0$, we finally get

\begin{equation}\label{c5e85}
2\frac{\alpha\mathcal{E}}{r} = \mathcal{E}^2 - m^2c^4 -
\left(\frac{\mathcal{E}\alpha}{Lc}\right)^2\phi^2.
\end{equation}
\end{enumerate}

\item Consider a localised collection of discrete charges $q_a$ at locations 
$\vec{R}_a$. The potential due to all of them at a point $\vec{R}_0$ is
\begin{equation}\label{c5e86}
\varphi(\vec{R}_0) = \sum_a\frac{q_a}{|\vec{R}_0 - \vec{R}_a|}.
\end{equation}
The rhs is a sum over $\vec{R}_a$ of functions of the form $q_af(\vec{R}_0-\vec{R}_a)$.
We can expand them as
\[
f(\vec{R}_0-\vec{R}_a) = f(\vec{R}_0) - \vec{R}_a\cdot\grad f|_{\vec{R}_0}.
\]
Here we have ignored the higher order terms. In this case, $f(\vec{R}) = 1/\vec{R}$
so that
\[
\grad f = -\frac{\vec{R}}{R^3}.
\]
Thus, we can write \eqref{c5e86} as
\begin{equation}\label{c5e87}
\varphi(\vec{R}_0) = 
\sum_a\left(\frac{q_a}{R_0} + q_a\vec{R}_a\cdot\frac{\vec{R_0}}{R_0^3}\right).
\end{equation}
If
\begin{eqnarray}
Q &=& \sum_a q_a \label{c5e88} \\
\vec{p} &=& \sum_a q_a\vec{R}_a \label{c5e89}
\end{eqnarray}
then we can write \eqref{c5e87} as
\begin{equation}\label{c5e90}
\varphi(\vec{R}_0) = \frac{Q}{R_0} + \frac{\vec{p}\cdot\vec{R_0}}{R_0^3}.
\end{equation}
$Q$ is the total charge in the collection and $\vec{p}$ is called its \emph{dipole
moment}.

\item The dipole moment depends on $\vec{R}_a$ and therefore on the choice of the
origin. If we shift the origin to $\vec{b}$ then the new position vectors are
$\vec{R}_a - \vec{b}$. The new dipole moment is
\begin{equation}\label{c5e91}
\vec{p}^\op = \sum_a q_a (\vec{R}_a - \vec{b}) = \vec{p} - Q\vec{b}.
\end{equation}
If $Q = 0$ then $\vec{p}^\op = \vec{p}$. 

\item Let us denote the positive charges by $q_a^+$ and the negative ones by
$-q_a^-$. Let their positions be denoted by $\vec{R}_a^+$ and $\vec{R}_a^-$
respectively. Then,
\begin{equation}\label{c5e92}
\vec{p} =  \sum_a q_a^+\vec{R}_a^+ - \sum_a q_a^-\vec{R}_a^-.
\end{equation}
Let,
\begin{eqnarray*}
\vec{R}^+ &=& \frac{\sum_a q_a^+\vec{R}_a^+}{\sum_a q_a^+} \\
\vec{R}^- &=& \frac{\sum_a q_a^-\vec{R}_a^-}{\sum_a q_a^-}
\end{eqnarray*}
so that \eqref{c5e92} becomes
\begin{equation}\label{c5e93}
\vec{p} = \left(\sum_a q_a^+\right)\vec{R}^+ - \left(\sum_a q_a^-\right)\vec{R}^-.
\end{equation}
If
\begin{equation}\label{c5e94}
\sum_a q_a^+ = \sum_a q_a^- = Q_0,
\end{equation}
say, then equation \eqref{c5e93} becomes
\begin{equation}\label{c5e95}
\vec{p} = Q_0\vec{R}^{+-},
\end{equation}
where
\begin{equation}\label{c5e96}
\vec{R}^{+-} = \vec{R}^+ - \vec{R}^-.
\end{equation}
This shows that if the total charge of the collection is zero then we can write 
its dipole moment as a product of $Q_0$ the sum of its positive charges and 
$\vec{R}^{+-}$, which can be viewed as the separation of the ``centres of masses
`' of the charges of each kind. The term ``centre of mass'' is not inappropriate
given the similarity of the definitions of $R^+$ and $R^-$ with that of the centre
of mass of a collection of discrete particles.

\item We have been considering the case $Q = 0$ in the previous two points. It is
interesting and important because most macroscopic bodies are electrically neutral
although they are composed of positive and negative charges. Continuing to analyse
this case further, if $Q = 0$, \eqref{c5e90} simplifies to
\begin{equation}\label{c5e97}
\varphi(\vec{R}_0) = \frac{\vec{p}\cdot\vec{R}_0}{R_0^3}.
\end{equation}
The electric field at $\vec{R}_0$ is
\begin{equation}\label{c5e98}
\vec{E} = -\grad\varphi(\vec{R}_0) = -\grad\frac{\vec{p}\cdot\vec{R}_0}{R_0^3}.
\end{equation}
Now,
\[
\grad\frac{\vec{p}\cdot\vec{R}_0}{R_0^3} = \frac{\vec{p}}{R_0^3} - 
\vec{p}\cdot\vec{R}_0\grad\frac{1}{R_0^3} = \frac{\vec{p}}{R_0^3} -
\frac{3(\vec{p}\cdot\vec{R}_0)\vec{R}_0}{R_0^5}.
\]
If 
\begin{equation}\label{c5e99}
\un = \frac{\vec{R}_0}{R_0}
\end{equation}
then
\begin{equation}\label{c5e100}
\grad\frac{\vec{p}\cdot\vec{R}_0}{R_0^3} = 
\frac{\vec{p}}{R_0^3} - \frac{3(\un\cdot\vec{p})\un}{R_0^3}
\end{equation}
and \eqref{c5e98} becomes
\begin{equation}\label{c5e101}
\vec{E} = \frac{3(\un\cdot\vec{p})\un - \vec{p}}{R_0^3}.
\end{equation}
The potential of a dipole varies as $R_0^{-2}$ and the field as $R_0^{-3}$.

\item The Taylor expansion of $f(\vec{R}_0-\vec{R}_a)$ up to second order term is
\begin{equation}\label{c5e102}
f(\vec{R}_0-\vec{R}_a) = f(\vec{R}_0) - \vec{R}_a\cdot\grad f\Big|_{\vec{R}_0} +
\frac{1}{2}\sum_{i, j}R_{ai}R_{aj}
\frac{\partial^2 f}{\partial x_i\partial x_j}\Big|_{\vec{R}_0}.
\end{equation}
Here $R_{ai}, R_{aj}$ are components of of $\vec{R}_a$.
If $f = 1/|\vec{R}_0-\vec{R}_a|$ then
\[
\pdt{f}{x_i} = -\frac{x_i}{R_0^3}
\]
and
\[
\frac{\partial^2 f}{\partial x_j\partial x_i} = -\frac{\delta_{ij}}{R_0^3} + 3\frac{x_ix_j}{R_0^5}
= \frac{3x_ix_j - R_0^2\delta_{ij}}{R_0^5}.
\]
The second order term in Taylor expansion of the potential of \eqref{c5e86} is
\begin{eqnarray*}
T_2 &=& \frac{1}{2}\sum_a\sum_{ij} q_aR_{ai}R_{aj}\frac{3x_ix_j - R_0^2\delta_{ij}}{R_0^5}\Big|_{\vec{R}_0} \\
 &=& \frac{1}{2}\sum_a\sum_{ij} q_aR_{ai}R_{aj}\frac{3R_{0i}R_{0j} - R_0^2\delta_{ij}}{R_0^5} \\
 &=& \frac{1}{2}\sum_{ij}\sum_a q_a\frac{3R_{ai}R_{aj}R_{0i}R_{0j} - R_{ai}R_{aj}R_0^2\delta_{ij}}{R_0^5}
\end{eqnarray*}
Now,
\[
\sum_{ij}R_{ai}R_{aj}R_0^2\delta_{ij} = R_a^2R_0^2 = \sum_{ij}R_a^2\delta_{ij}R_{ai}R_{aj}
\]
so that
\[
T_2 = \frac{1}{2}\sum_{ij}R_{0i}\sum_a q_a(3R_{ai}R_{aj} - R_a^2\delta_{ij}R_{0j})\frac{1}{R_0^5}.
\]
We can now extend the expression in \eqref{c5e87} to
\begin{equation}\label{c5e103}
\varphi(R_0) = \frac{Q}{R_0} + \frac{\vec{p}\cdot\vec{R}_0}{R_0^3} + 
\frac{1}{2}\frac{R_{0i}D_{ij}R_{0j}}{R_0^5}.
\end{equation}
where we have used the summation convention and
\begin{equation}\label{c5e104}
D_{ij} = \sum_a q_a(3R_{ai}R_{aj} - R_a^2\delta_{ij})
\end{equation}
is the quadrupole moment of the collection of charges. $D_{ij}$ is a symmetric,
second-order 3-tensor. Further, it is also traceless, for
\[
D_{ii} = \sum_a q_a(3R_{ai}R_{ai} - R_a^2\delta_{ii}) = \sum_a q_a(3R_a^2 - 3R_a^2) = 0.
\]
Therefore, only five of its components are independent. Using \eqref{c5e99}, we
can further simplify \eqref{c5e103} to
\begin{equation}\label{c5e105}
\varphi(R_0) = \frac{Q}{R_0} + \frac{d_in_i}{R_0^2} + 
\frac{1}{2}\frac{n_iD_{ij}n_j}{R_0^3}.
\end{equation}

\item A symmetric axis can be diagonalised. If the charge distribution has a 
symmetry axis then align the $z$-axis with it. The symmetry about the $z$ axis
makes $D_{11} = D_{22}$. Further, since the trace of $D_{ij}$ is zero, we have 
$D_{11} + D_{22} + D_{33} = 0$ or, taking since $D_{11} = D_{22}$,
\begin{equation}\label{c5e106}
D_{11} = D_{22} = -\frac{D_{33}}{2} = -\frac{D}{2},
\end{equation}
where defined $D = D_{33}$. Therefore,
\[
n_i D_{ij} n_j = n_1^2D_{11} + n_2^2D_{22} + n_3^2D_{33} = 
D\left(n_3^2 - \frac{n_1^2 + n_2^2}{2}\right).
\]
If $\un$ has polar coordinates $(1, \phi, \theta)$ where the $z$-axis is the 
symmetry axis of the charge distribution then $n_3 = \cos\theta, n_2 = \sin\theta
\sin\phi, n_1 = \sin\theta\cos\phi$ and hence,
\begin{equation}\label{c5e107}
n_iD_{ij}n_j = \frac{D}{2}(3\cos^2\theta - 1).
\end{equation}
We can then write \eqref{c5e105} as
\begin{equation}\label{c5e108}
\varphi(R_0) = \frac{Q}{R_0} + \frac{d\cos\theta}{R_0^2} + 
\frac{D}{4R_0^3}(3\cos^2\theta - 1)
\end{equation}
Since $P_1(\cos\theta) = 1, P_2(\cos\theta) = \cos\theta$ and $P_3(\cos\theta)
= (3\cos^2\theta - 1)/2$, $P_n$ being the Legendre polynomial of order $n$, we 
can write \eqref{c5e109} as
\begin{equation}\label{c5e109}
\varphi(R_0) = \frac{Q}{R_0}P_1(\cos\theta) + \frac{d}{R_0^2}P_2(\cos\theta)
+ \frac{D}{2R_0^3}P_3(\cos\theta).
\end{equation}

\item We can continue to expand the potential to get higher order terms expressed
as multiples of Legendra polynomials using
\begin{equation}\label{c5e110}
\frac{1}{|\vec{R}_0 - \vec{r}|} = \frac{1}{\sqrt{R_0^2 + r^2 - 2rR_0\cos\chi}}
= \sum_{l=0}^\infty \frac{r^l}{R_0^{l+1}}P_l(\cos\chi).
\end{equation}
If the spherical angles of $\vec{R}_0$ are $\Theta$ and $\Phi$ and those of
$\vec{r}_0$ are $\theta$ and $\phi$ then
\begin{equation}\label{c5e111}
P_l(\cos\chi) = \sum_{m=-l}^l\frac{(l -|m|)!}{(l + |m|)!}
P_l^{|m|}(\cos\Theta)P_l^{|m|}(\cos\theta)e^{-im(\Phi-\phi)},
\end{equation}
where $P_l^{|m|}$ are called the associated Legendre polynomials. We next introduce
the spherical harmonics as
\begin{equation}\label{c5e112}
Y_{lm}(\theta,\phi)=(-1)^m\sqrt{\frac{2l+1}{4\pi}\frac{(l-m)!}{(l+m)!}}P_l^m(\cos\theta)e^{im\phi},
m \ge 0
\end{equation}
so that we can write \eqref{c5e111} as
\begin{equation}\label{c5e113}
P_l(\cos\chi)=\sum_{m=-l}^l\frac{4\pi}{2l+1}Y_{lm}^\ast(\Theta,\Phi)Y_{lm}(\theta,\phi)
\end{equation}
and \eqref{c5e110} as
\begin{equation}\label{c5e114}
\frac{1}{|\vec{R}_0 - \vec{r}_a|} = \sum_{l=0}^\infty\sum_{m=-l}^l\frac{r^l}{R_0^{l+1}}
\frac{4\pi}{2l+1}Y_{lm}^\ast(\Theta,\Phi)Y_{lm}(\theta_a,\phi_a).
\end{equation}
This allows us to write \eqref{c5e86} as
\begin{eqnarray*}
\varphi(\vec{R}_0) &=& \sum_a\frac{q_a}{|\vec{R}_0 - \vec{r}|} \\
 &=& \sum_a q_a\sum_{l=0}^\infty\sum_{m=-l}^l\frac{r^l}{R_0^{l+1}}\frac{4\pi}{2l+1}Y_{lm}^\ast(\Theta,\Phi)
     Y_{lm}(\theta_a,\phi_a) \\
 &=& \sum_{l=0}^\infty\frac{1}{R_0^{l+1}}\sum_{m=-l}^l\sum_a q_ar^l\frac{4\pi}{2l+1}Y_{lm}^\ast(\Theta,\Phi)
     Y_{lm}(\theta_a,\phi_a) \\
 &=& \sum_{l=0}^\infty\frac{1}{R_0^{l+1}}\sum_{m=-l}^l\sqrt{\frac{4\pi}{2l+1}}Y_{lm}^\ast(\Theta,\Phi)
     \sum_a q_ar^l\sqrt{\frac{4\pi}{2l+1}}Y_{lm}(\theta_a,\phi_a)
\end{eqnarray*}
Define
\begin{equation}\label{c5e115}
Q_m^l = \sum_a q_ar^l\sqrt{\frac{4\pi}{2l+1}}Y_{lm}(\theta_a,\phi_a)
\end{equation}
so that
\begin{equation}\label{c5e116}
\varphi(\vec{R}_0) = \sum_{l=0}^\infty\frac{1}{R_0^{l+1}}\sum_{m=-l}^l\sqrt{\frac{4\pi}{2l+1}}
Y_{lm}^\ast(\Theta,\Phi)Q_m^l.
\end{equation}
For each $l$, $m$ can take $2l + 1$ values from $-l$ to $l$. They form the 
components of the $2^l$-moment of the collection of charges.

\item Now consider a collection of charges $q_a$ at positions $\vec{r}_a$ with 
respect to an origin within the collection. Let it be exposed to an external 
electric field which varies slowly over the extent of the collection. If $\varphi$
is the potential of the field then the energy of the collection is
\begin{equation}\label{c5e117}
U = \sum_aq_a\varphi(\vec{r}_a).
\end{equation}
Expand $U$ about the origin so that
\begin{equation}\label{c5e118}
U = \sum_{n=0}^\infty U^{(n)}.
\end{equation}
Then 
\begin{equation}\label{c5e119}
U^{(0)} = \sum_a q_a\varphi(0) = Q\varphi(0).
\end{equation}
The first order term is
\[
U^{(1)} = \sum_a q_a\vec{r}_a\cdot\grad\varphi|_0 = 
-\left(\sum_a q_a\vec{r}_a\right)\cdot\vec{E}_0,
\]
where $\vec{E}_0 = -\grad\varphi|_0$ is the electric field at the origin. From 
\eqref{c5e89} we know that the term in the bracket is the dipole moment of the
distribution so that
\begin{equation}\label{c5e120}
U^{(1)} = -\vec{p}\cdot\vec{E}_0,
\end{equation}
The force on the distribution is
\begin{equation}\label{c5e121}
\vec{F} = -\grad U
\end{equation}
so that up to the first order, the force on the distribution is
\[
\vec{F} = -Q\grad\varphi|_0 + \grad(\vec{p}\cdot\vec{E})|_0
\]
or, since $\vec{p}$ is a constant,
\begin{equation}\label{c5e122}
\vec{F} = Q\vec{E}_0 + (\vec{p}\cdot\grad)\vec{E}|_0.
\end{equation}
The torque on the collection with respect to the origin, to the lowest order term,
is
\begin{equation}\label{c5e123}
\vec{N} = \sum_aq_a\vec{r}_a\times\vec{E}_0  = \vec{p}\times\vec{E}_0.
\end{equation}

\item Now consider two systems with net charge $0$ so that up to the lowest order
the energy is of the first system is
\begin{equation}\label{c5e124}
U_1 = -\vec{p}_1\cdot\vec{E},
\end{equation}
$\vec{E}$ being the external electric field \emph{at the origin within its extent}.
If $\vec{E}$ is due to the other system whose origin is at $\vec{R}$, then using
\eqref{c5e101},
\[
\vec{E} = \frac{(3\vec{R}\cdot\vec{p}_2)\vec{R} - \vec{p}_2R^2}{R^5}
\]
so that
\begin{equation}\label{c5e125}
U_1 = \frac{(\vec{p}_1\cdot\vec{p}_2)R^2 - 3(\vec{p}_1\cdot\vec{R})(\vec{p}_2\cdot\vec{R})}{R^5}.
\end{equation}

If the second system had a net charge $Q_2$ then
\[
\vec{E} = -\frac{Q}{R^3}\vec{R}
\]
and
\begin{equation}\label{c5e126}
U_1 = Q\frac{\vec{p}\cdot\vec{R}}{R^3}.
\end{equation}

\item The second order term in the expansion of \eqref{c5e118} is
\[
U^{(2)} = \frac{1}{2}\sum_a q_a r_{ai}r_{aj}\frac{\partial^2\varphi}{\partial x_i\partial x_j}\Big|_0.
\]
Since $\varphi$ is a solution of the Laplace equation,
\[
\frac{\partial^2\varphi}{\partial x_i\partial x_j} = 
(1 - \delta_{ij})\frac{\partial^2\varphi}{\partial x_i\partial x_j}
\]
so that
\[
\frac{\partial^2\varphi}{\partial x_i\partial x_j}\Big|_0 = 
(1 - \delta_{ij})\frac{\partial^2\varphi}{\partial x_i\partial x_j}\Big|_0
\]
and hence
\begin{eqnarray*}
U^{(2)} &=& \frac{1}{2}\sum_a q_a r_{ai}r_{aj}(1 - \delta_{ij}) \\
 &=& \frac{1}{2}\frac{\partial^2\varphi}{\partial x_i\partial x_j}\Big|_0\sum_a q_a r_{ai}r_{aj}(1 - \delta_{ij}) \\
 &=& \frac{1}{2}\frac{\partial^2\varphi}{\partial x_i\partial x_j}\Big|_0\sum_a q_a (r_{ai}r_{aj} - r_a^2\frac{\delta_{ij}}{3}) \\
 &=& \frac{1}{6}\frac{\partial^2\varphi}{\partial x_i\partial x_j}\Big|_0\sum_a q_a (3r_{ai}r_{aj} - r_a^2\delta_{ij})
\end{eqnarray*}
Using the definition of the quadrupole moment tensor of \eqref{c5e104},
\begin{equation}\label{c5e127}
U^{(2)} = \frac{1}{6}\frac{\partial^2\varphi}{\partial x_i\partial x_j}\Big|_0D_{ij}
\end{equation}

\item The spherical harmonics form a complete set of basis functions. Therefore,
one can write
\begin{equation}\label{c5e128}
\phi(\vec{r}) = \sum_{l=0}^\infty r^l\sum_{m=-l}^l\sqrt{\frac{4\pi}{2l+1}}a_{lm}Y_{lm}(\theta,\phi),
\end{equation}
where $a_{lm}$ are constants. Substituting this in equation \eqref{c5e117},
\begin{eqnarray*}
U &=& \sum_a q_a\sum_{l=0}^\infty r_a^l\sum_{m=-l}^l\sqrt{\frac{4\pi}{2l+1}}a_{lm}Y_{lm}(\theta,\phi) \\
  &=& \sum_{l=0}^\infty\sum_{m=-l}^l a_{lm}\left(\sum_a q_ar_a^l\sqrt{\frac{4\pi}{2l+1}}Y_{lm}(\theta,\phi)\right)
\end{eqnarray*}
Using \eqref{c5e115},
\[
U = \sum_{l=0}^\infty\sum_{m=-l}^l a_{lm}Q_m^l.
\]
from which we infer that
\begin{equation}\label{c5e129}
U^{(l)} = \sum_{m=-l}^l a_{lm}Q_m^l.
\end{equation}

\item Consider the magnetic field produced by charges in motion over a finite
portion of space and with momenta that also remain finite. This requirement is 
same under which the virial theorem was proved and under which the time averages
of time derivatives of functions is zero. Refer to definition \eqref{c4e112} 
for the time average of a function and the proof of \eqref{c4e114} for the 
vanishing of the time average of its derivative.

For such a system, we will perform the time averaging of the Maxwell equations
for the magnetic field to get
\begin{eqnarray}
\dive\ta{\vec{H}} &=& 0 \label{c5e130} \\
\curl\ta{\vec{H}} &=& \frac{4\pi}{c}\ta{\vec{J}} \label{c5e131}
\end{eqnarray}

\item One could have got similar equations had one assumed that the fields are
independent of time. We wrote \eqref{c5e3} and \eqref{c5e4} under the same 
assumption. However, one cannot consider the electric field to be time independent
of there are currents in the system. Therefore, the correct way to introduce
constant fields in this case is to assume that the $\langle\vec{E}\rangle = 0$.

\item We introduce $\langle\vec{A}\rangle$ using
\begin{equation}\label{c5e132}
\ta{\vec{H}} = \curl\ta{\vec{A}}.
\end{equation}
Substituting it in \eqref{c5e131} we get
\begin{equation}\label{c5e133}
\grad\dive\ta{\vec{A}} - \nabla^2\ta{\vec{A}} = \frac{4\pi}{c}\ta{\vec{J}}.
\end{equation}
Since the curl of a gradient is always zero, a vector potential
\[
\tav{A}^\op = \tav{A} + \grad\psi
\]
also gives the same $\tav{H}$. Now,
\[
\dive\tav{A}^\op = \dive\tav{A} + \nabla^2\psi.
\]
If we insist on choosing a $\psi$ that is a solution of the Laplace equation then
we get $\dive\tav{A}^\op = \dive\tav{A} = 0$. Choosing
\begin{equation}\label{c5e134}
\dive\tav{A} = 0
\end{equation} is equivalent of choosing $\psi$ to be a harmonic function. The 
choice of \eqref{c5e134} is called the \emph{Coulomb gauge}.

\item Under the Coulomb gauge, \eqref{c5e133} becomes
\begin{equation}\label{c5e135}
\nabla^2\ta{\vec{A}} = -\frac{4\pi}{c}\ta{\vec{J}}.
\end{equation}
Thus, $\tav{A}$ obeys the vector Poisson equation the way $\varphi$ obeys the
scalar Poisson equation of \eqref{c5e7}. The solution of \eqref{c5e135} is similar
to the solution \eqref{c5e15} of \eqref{c5e7}. Thus,
\begin{equation}\label{c5e136}
\tav{A}(\vec{r}) = \frac{1}{c}\int\frac{\tav{J}(\vec{r}^\op)}{|\vec{r} - \vec{r}^\op|}dV^\op.
\end{equation}
We can write the rhs of \eqref{c5e136} for discrete charges as
\begin{equation}\label{c5e137}
\tav{A} = \frac{1}{c}\sum_a \ta{\frac{q_a\vec{v}_a}{R_a}},
\end{equation}
where 
\begin{equation}\label{c5e138}
\vec{R}_a = \vec{r} - \vec{r}_a.
\end{equation}

\item From \eqref{c5e136}, we can immediately get
\[
\tav{H} = \frac{1}{c}\curl\int\frac{\tav{J}(\vec{r}^\op)}{|\vec{r} - \vec{r}^\op|}dV^\op.
\]
The curl is with respect to the unprimed variables. Therefore,
\[
\tav{H} = \frac{1}{c}\int\curl\frac{\tav{J}(\vec{r}^\op)}{|\vec{r} - \vec{r}^\op|}dV^\op
 = \frac{1}{c}\int\grad\frac{1}{|\vec{r} - \vec{r}^\op|} \times \tav{J}(\vec{r}^\op)dV^\op,
\]
where we used the identity
\begin{equation}\label{c5e139}
\curl(f\vec{F}) = f\curl\vec{F} + \grad f \times \vec{F}.
\end{equation}
From this we get the \emph{Biot-Savart} law,
\begin{equation}\label{c5e140}
\tav{H} = \frac{1}{c}\int\frac{\tav{J}(\vec{r}^\op) \times (\vec{r} - \vec{r}^\op)}{|\vec{r} - \vec{r}^\op|^3}dV^\op.
\end{equation}

\item We go back to equation \eqref{c5e137} which described the vector potential
due to a collection of charges. Choose the origin to be somewhere within the
positions of the charges and let the field point be $\vec{R}_0$. Expanding the
function $1/R_a$ up to the first order terms, we get
\[
\frac{1}{R_a} = \frac{1}{|\vec{R}_0 - \vec{r}_a|} = 
\frac{1}{R_0} - \vec{r}_a\cdot\grad\frac{1}{R}\Big|_{R_0}
\]
Therefore,
\begin{equation}\label{c5e141}
\tav{A} = \frac{1}{c}\sum_a\frac{q_a\tav{v}_a}{R_0} - 
\frac{1}{c}\sum_aq_a\ta{\vec{v}_a\vec{r}_a\cdot\grad\frac{1}{R}\Big|_{R_0}}.
\end{equation}
For a system of finite extent, $\tav{v}_a = 0$ so that
\begin{equation}\label{c5e142}
\tav{A} = +\frac{1}{cR_0^3}\sum_aq_a\ta{\vec{v}_a\vec{r}_a\cdot\vec{R}_0}.
\end{equation}
Now,
\[
\frac{d}{dt}\sum_a q_a\vec{r}_a(\vec{r}_a\cdot\vec{R}_0) = 
\sum_a q_a\left(\vec{v}_a(\vec{r}_a\cdot\vec{R}_0) + \vec{r}_a(\vec{v}_a\cdot\vec{R}_0)\right)
\]
so that
\[
\frac{1}{2}\sum_a q_a \vec{v}_a(\vec{r}_a\cdot\vec{R}_0) = 
\frac{1}{2}\frac{d}{dt}\sum_a q_a\vec{r}_a(\vec{r}_a\cdot\vec{R}_0) - 
\frac{1}{2}\sum_a q_a \vec{r}_a(\vec{v}_a\cdot\vec{R}_0).
\]
and hence
\begin{equation}\label{c5e143}
\ta{\frac{1}{2}\sum_a q_a \vec{v}_a(\vec{r}_a\cdot\vec{R}_0)} = 
-\ta{\frac{1}{2}\sum_a q_a \vec{r}_a(\vec{v}_a\cdot\vec{R}_0)},
\end{equation}
as the average of the time derivative vanishes. If we write
\[
\tav{A} = \frac{1}{cR_0^3}\sum_a\frac{q_a}{2}\ta{\vec{v}_a\vec{r}_a\cdot\vec{R}_0}
 + \frac{1}{cR_0^3}\sum_a\frac{q_a}{2}\ta{\vec{v}_a\vec{r}_a\cdot\vec{R}_0}.
\]
and use \eqref{c5e143} for one of the factors, we get
\begin{equation}\label{c5e144}
\tav{A} = \frac{1}{2cR_0^3}\sum_aq_a\ta{\vec{v}_a(\vec{r}_a\cdot\vec{R}_0) - 
\vec{r}_a(\vec{v}_a\cdot\vec{R}_0)}.
\end{equation}
We now introduce the magnetic dipole moment
\begin{equation}\label{c5e145}
\vec{m} = \frac{1}{2c}\sum_a q_a\vec{r}_a \times \vec{v}_a
\end{equation}
so that
\[
\vec{m} \times \vec{R}_0 = \frac{1}{2c}
\sum_a q_a(\vec{v}_a(\vec{r}_a\cdot\vec{R}_0) - \vec{r}_a(\vec{v}_a\cdot\vec{R}_0))
\]
(Note that this is $(\vec{A} \times \vec{B})\times\vec{C}$ and not $\vec{A}\times
(\vec{B}\times\vec{C})$.) and \eqref{c5e144} becomes
\begin{equation}\label{c5e146}
\tav{A} = \frac{\tav{m} \times \vec{R}_0}{R_0^3}.
\end{equation}

\item Using \eqref{c5e146}, we can calculate
\begin{eqnarray*}
\tav{H} &=& \curl\left(\tav{m} \times \frac{\vec{R_0}}{R_0^3}\right) \\
 &=& \frac{\vec{R_0}}{R_0^3}\cdot\grad\tav{m} - \tav{m}\cdot\grad\frac{\vec{R_0}}{R_0^3} + \\
 & & \tav{m}\dive\left(\frac{\vec{R_0}}{R_0^3}\right) - \frac{\vec{R_0}}{R_0^3}\dive\tav{m}
\end{eqnarray*}
$\tav{m}$ is a function of $\vec{r}_a$ alone. Therefore, its derivatives are all
zero and
\begin{equation}\label{c5e147}
\tav{H} = \tav{m}\dive\left(\frac{\vec{R_0}}{R_0^3}\right) -
\tav{m}\cdot\grad\frac{\vec{R_0}}{R_0^3}.
\end{equation}
Now,
\begin{eqnarray*}
\dive\left(\frac{\vec{R_0}}{R_0^3}\right) &=& \frac{3}{R_0^3} - \vec{R}_0\cdot\grad R_0^{-3} \\
 &=& \frac{3}{R_0^3} - 3\frac{R_0^2}{R_0^5} \\
 &=& 0
\end{eqnarray*}
so that
\begin{eqnarray*}
\tav{H} &=& -\tav{m}\cdot\grad\frac{\vec{R_0}}{R_0^3} \\
 &=& -\left(\ta{m}_x\frac{\partial}{\partial X} + \ta{m}_y\frac{\partial}{\partial Y} + 
     \ta{m}_z\frac{\partial}{\partial Z}\right)\frac{\vec{R_0}}{R_0^3} \\
 &=& \frac{\tav{m}}{R_0^3} - \frac{3\vec{R}_0(\tav{m}\cdot\vec{R}_0)}{R_0^5}.
\end{eqnarray*}
If $\un$ is a unit vector along $\vec{R}_0$ then
\begin{equation}\label{c5e148}
\tav{H} = \frac{3\un(\tav{m}\cdot\un) - \tav{m}}{R_0^3}.
\end{equation}

\item The definition of the magnetic dipole moment vector is very similar to the
angular momentum of a collection of masses $m_a$ at positions $\vec{r}_a$ and 
moving with velocity $\vec{v}_a$. The role of mass $m_a$ is played by $q_a/(2c)$.
If $q_a/m_a = \alpha$, that is, if the charge to mass ratio of all charges is the
same then,
\begin{equation}\label{c5e149}
\tav{m} = \frac{\alpha}{2c}\sum_a m_a\vec{r}_a \times \vec{v}_a = 
\frac{\alpha}{2c}\vec{L}.
\end{equation}
The ratio of magnetic moment and angular momentum is thus a constant. It is called
the \emph{gyromagnetic ratio}.

\item If the collection of charges is in a constant, uniform magnetic field then
the force on them is
\[
\vec{F} = \sum_a q_a\frac{\vec{v}_a}{c}\times\vec{H} = 
\frac{d}{dt}\sum_a q_a\frac{\vec{r}_a}{c}\times\vec{H}.
\]
As the force is expressed as a time derivative, its average under the assumptions
we have made so far is $0$. The torque on the collection is
\begin{equation}\label{c5e150}
\vec{N} = \sum_a q_a\vec{r}_a \times \left(\frac{\vec{v}_a}{c}\times\vec{H}\right).
\end{equation}
It cannot be expressed as a time derivative and therefore its average may not
vanish. We can write it as
\begin{eqnarray*}
\vec{N} &=& \sum_a\frac{q_a}{c}(\vec{v}_a(\vec{r}_a\cdot\vec{H}) - \vec{H}(\vec{r}_a\cdot\vec{v}_a)) \\
 &=& \sum_a\frac{q_a}{c}\left(\vec{v}_a(\vec{r}_a\cdot\vec{H}) - \vec{H}\left(\vec{r}_a\cdot\td{\vec{r}_a}{t}\right)\right) \\
  &=& \sum_a\frac{q_a}{c}\left(\vec{v}_a(\vec{r}_a\cdot\vec{H}) - \frac{\vec{H}}{2}\td{r_a^2}{t}\right)
\end{eqnarray*}
Taking the time average gives us
\begin{equation}\label{c5e151}
\tav{N} = \sum_a\frac{q_a}{c}\ta{\vec{v}_a(\vec{r}_a\cdot\vec{H})}.
\end{equation}
We repeat the manipulations we did to get \eqref{c5e144} from \eqref{c5e142} by 
starting with
\[
\frac{d}{dt}(\vec{r}_a(\vec{r}_a\cdot\vec{H}) = \vec{v}_a(\vec{r}_a\cdot\vec{H}) + \vec{r}_a(\vec{v}_a\cdot\vec{H})
\]
so that, upon taking the time derivative, we get
\begin{equation}\label{c5e152}
\ta{\vec{v}_a(\vec{r}_a\cdot\vec{H})} = -\ta{\vec{r}_a(\vec{v}_a\cdot\vec{H})}
\end{equation}
so that
\begin{eqnarray}
\tav{N} &=& \sum_a\frac{q_a}{c}\ta{\vec{v}_a(\vec{r}_a\cdot\vec{H})} \nonumber \\
 &=& \sum_a\frac{q_a}{2c}\ta{\vec{v}_a(\vec{r}_a\cdot\vec{H})} + 
 \sum_a\frac{q_a}{2c}\ta{\vec{v}_a(\vec{r}_a\cdot\vec{H})} \nonumber \\
 &=& \sum_a\frac{q_a}{2c}\ta{\vec{v}_a(\vec{r}_a\cdot\vec{H})} - 
 \sum_a\frac{q_a}{2c}\ta{\vec{r}_a(\vec{v}_a\cdot\vec{H})} \nonumber \\
 &=& \frac{1}{2c}\sum_a q_a(\ta{\vec{v}_a(\vec{r}_a\cdot\vec{H})} - \ta{\vec{r}_a(\vec{v}_a\cdot\vec{H})}) 
 \nonumber \\
 &=& \tav{m} \times \tav{H}. \label{c5e153}
\end{eqnarray}

\item The canonical momentum of a charged particle in an electromagnetic field 
has a term $q\vec{A}\cdot\vec{v}/c$ in addition to the $m\vec{v}$. For a constant 
magnetic field, using \eqref{c3e37}, we observe that
\begin{equation}\label{c5e154}
\frac{q}{c}\vec{A}\cdot\vec{v} = \frac{q}{2c}(\vec{H} \times \vec{r})\cdot\vec{v} = 
\frac{q}{2c}(\vec{r}\times\vec{v})\cdot\vec{H} = \vec{m}\cdot\vec{H}.
\end{equation}
Thus, the additional term is indeed the potential energy of the magnetic dipole
in a constant magnetic field. The analogous term for an electric dipole is 
$\vec{p}\cdot\vec{E}$.

\item Consider a collection of charges all having the same magnitude $q$ and mass
$m$. If $\vec{r}_a$ and $\vec{v}_a$ describe their positions and velocities then
its aaverage ngular momentum is
\[
\tav{L} = m\sum_a\ta{\vec{r}_a \times \vec{v}_a}
\]
and its magnetic dipole moment is
\[
\tav{m} = \frac{q}{2c}\sum_a\ta{\vec{r}_a \times \vec{v}_a}.
\]
From equation \eqref{c5e153}
\[
\td{\tav{L}}{t} = \frac{q}{2c}\sum_a\ta{\vec{r}_a \times \vec{v}_a} \times \vec{H}
\]
Let
\begin{equation}\label{c5e155}
\bm{\Omega} = \frac{q}{2mc}\vec{H}
\end{equation}
then 
\begin{equation}\label{c5e156}
\td{\tav{L}}{t} = \tav{L} \times \bm{\Omega} = -\bm{\Omega} \times \tav{L}.
\end{equation}
Thus the $\tav{L}$ rotates in a magnetic field as if the system itself rotates
with an angular velocity $\bm{\Omega}$. Its magnitude is called the \emph{Larmor
frequency}.
\end{enumerate}

\section{Problems}
%To do: change the equation labels.
\begin{enumerate}
\item Determine the quadrupole moment of a uniformly charged ellipsoid with 
respect to its centre.
\item[Solution:] We align the cartesian coordinate axes to coincide with the 
principal axes of the ellipsoid. Therefore, the tensor of \eqref{c5e104} can be
written as
\[
D_{xx} = \sum_a q_a(3x^2 - x^2 - y^2 - z^2) = \sum_a q_a(2x^2-y^2-z^2).
\]
If we generalise it to a continuous distribution, 
\begin{equation}\label{c5e157}
D_{xx} = \int_V\rho(2x^2-y^2-z^2)dxdydz.
\end{equation}
where $V$ is the volume of the ellipsoid. If the equation of the ellipsoid is
\begin{equation}\label{c5e158}
\frac{x^2}{a^2} + \frac{y^2}{b^2} + \frac{z^2}{c^2} = 1
\end{equation}
then introduce the transformation $x_1 = x/a, y_1 = y/b, z_1 = z/c$ so that the 
Jacobian of transformation is $abc$ and
\[
D_{xx} = \rho abc\int_{V_1}\left(2a^2x_1^2 -b^2y_1^2 - c^2z_1^2\right)dx_1dy_1dz_1
\]
and the volume of integration $V_1$ is over $x_1^2 + y_1^2 + z_1=1$, a unit sphere. 
Introduce spherical coordinates so that $x_1 = \sin\theta\cos\phi, y_1 = 
\sin\theta\sin\phi, z_1 = \cos\theta$ so that
\begin{eqnarray*}
\frac{D_{xx}}{\rho abc}
 &=& \int_0^\pi\int_0^{2\pi}(2a^2\sin^2\theta\cos^2\phi-b^2\sin^2\theta
 \sin^2\phi-c^2\cos^2\theta)\sin\theta d\theta d\phi \\
 &=& 2a^2\int_0^\pi\sin^3\theta d\theta\int_0^{2\pi}\cos^2\phi d\phi - 
      b^2\int_0^\pi\sin^3\theta d\theta\int_0^{2\pi}\sin^2\phi d\phi - \\
 & & c^2\int_0^\pi\sin\theta\cos^2\theta d\theta \int_0^{2\pi}d\phi \\
 &=& 2a^2\left(\frac{4}{3}\right)\pi - b^2\left(\frac{4}{3}\right)\pi
     - c^2\left(\frac{2}{3}\right)2\pi \\
 &=& \frac{4\pi}{3}(2a^2-b^2-c^2)
\end{eqnarray*}
so that
\[
D_{xx} = \rho\frac{4\pi}{3}abc(2a^2-b^2-c^2)
\]
Since the volume of the ellipsoids is $(4\pi/3)abc$, if $Q$ is the total charge
on the ellipsoid then
\begin{equation}\label{c5e159}
D_{xx} = Q(2a^2-b^2-c^2).
\end{equation}
Analogous to \eqref{c5e157} we have
\begin{equation}\label{c5e160}
D_{yy} = \int_V(2y^2 - x^2 - z^2)dxdydz
\end{equation}
which, upon the same transformation as before, becomes
\[
D_{yy} = \rho abs\int_{V_1}(2b^2y_1^2 - a^2x_a^2 - c^2z_1^2)dx_1dy_1dz_1.
\]
This integral can be evaluated in the same way so that
\begin{equation}\label{c5e151}
D_{yy} = Q(2b^2 - a^2 - c^2).
\end{equation}
Since the quadrupole moment tensor is traceless,
\begin{equation}\label{c5e162}
D_{zz} = -D_{xx} - D_{yy} = Q(2c^2 - a^2 - b^2).
\end{equation}

\item Find the gyromagnetic ratio of a system of two charges moving non-
relativistically.
\item[Solution:] The magnetic moment, like the angular momentum, depends on the
frame of reference. Let us work in frame of reference of the centre of mass of 
the two charges. If $\vec{r}_1$ and $\vec{r}_2$ are the positions of the two
charges then,
\begin{equation}\label{c5e163}
m_1\vec{r}_1 = -m_2\vec{r}_2.
\end{equation}
In this frame, the two momenta are equal and opposite. That is,
\begin{equation}\label{c5e164}
\vec{p}_1 = -\vec{p}_2.
\end{equation}
Therefore, the angular momentum of the system is
\[
\vec{L} = m_1\vec{r}_1\times\vec{p}_1 + m_2\vec{r}_2\times\vec{p}_2 
\]
which, using equations \eqref{c5e163} and \eqref{c5e164} becomes
\begin{equation}\label{c5e165}
\vec{L} = \frac{m_1 + m_2}{m_2}\vec{r}_1 \times \vec{p}_1.
\end{equation}
The magnetic moment is
\[
\vec{m} = \frac{1}{2c}(q_1\vec{r}_1\times\vec{v}_1 + q_2\vec{r}_2\times\vec{v}_2).
\]
Since the motion is non-relativistic, $\vec{p}_1 = m_1\vec{v}_1$ and $\vec{p}_2
= m_2\vec{v}_2$. Therefore,
\[
\vec{m} = \frac{1}{2c}\left(q_1\vec{r}_1\times\frac{\vec{p}_1}{m_1} + 
q_2\vec{r}_2\times\frac{\vec{p}_1}{m_1}\right).
\]
Using \eqref{c5e163} and \eqref{c5e164} we get
\[
\vec{m} = \frac{1}{2c}\left(\frac{q_1}{m_1} + q_2\frac{m_1}{m_2^2}\right)\vec{r}_1\times\vec{p}_1
= \frac{1}{2c}\left(\frac{q_1}{m_1^2} + \frac{q_2}{m_2^2}\right)m_1\vec{r}_1\times\vec{p}_1
\]
Substitute for $\vec{r}_1 \times \vec{p}_1$ from \eqref{c5e165} to get
\begin{equation}\label{c5e166}
\vec{m} = \frac{1}{2c}\left(\frac{q_1}{m_1^2} + \frac{q_2}{m_2^2}\right)\frac{m_1m_2}{m_1+m_2}\vec{L}.
\end{equation}
The scalar on the right hand side is the gyromagnetic ratio.
\end{enumerate}
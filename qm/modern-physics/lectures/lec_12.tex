\documentclass{beamer}
\usefonttheme[onlymath]{serif}
\usepackage{graphicx, hyperref}
\hypersetup{colorlinks, urlcolor=blue}
\usepackage{subcaption}
\usetheme{Boadilla}

\newcommand{\pvec}[1]{\vec{#1}\mkern2mu\vphantom{#1}}

\title{The Molecules}
\author{Amey Joshi}
\date{\today}

\begin{document}
\begin{frame}
\titlepage
\end{frame}

\begin{frame}
\begin{enumerate}
\frametitle{Many-electron systems}
\item The spin is a separate degree of freedom. The wave function $\psi$ depends
on $\vec{r}$ alone. We need another function $s$ to denote spin. It is a 2-row
column matrix in non-relativistic quantum mechanics. The complete, time 
independent wave function is, therefore, $\psi(\vec{r})s$.
\item When there are two electrons, the wave function of the system is 
\begin{equation}\label{e1}
\psi = \psi_a(\vec{r}_1)s_a\psi_b(\vec{r}_2)s_b,
\end{equation}
which expresses the fact that one electron is in the state `a' and the other in 
the state `b'.
\item A shorthand to the express the same thing is
\begin{equation}\label{e2}
\psi = \psi_a(1)\psi_b(2),
\end{equation}
where the arguments `1' and `2' include both space and spin degrees of freedom 
of an electron.
\end{enumerate}
\end{frame}

\begin{frame}
\frametitle{Indistinguishable particles}
\begin{enumerate}
\item When the wave functions of two particles overlap, it is impossible to tell
one particle from the other. Therefore, the state $\psi_a(1)\psi_b(2)$ cannot be
told from $\psi_a(2)\psi_b(1)$. Such a state is therefore expressed as a linear
combination of the two,
\begin{equation}\label{e3}
\psi = \alpha\psi_a(1)\psi_b(2) + \beta\psi_a(2)\psi_b(1).
\end{equation}
\item It is a \emph{law of nature} that particles with integral spins have a 
symmetric wave function while those with half-integral spins have anti-symmetric
wave function.
\item A symmetric combination is
\begin{equation}\label{e4}
\psi = \psi_a(1)\psi_b(2) + \psi_a(2)\psi_b(1)
\end{equation}
while an anti-symmetric combination is
\begin{equation}\label{e5}
\psi = \psi_a(1)\psi_b(2) - \psi_a(2)\psi_b(1)
\end{equation}
\end{enumerate}
\end{frame}

\begin{frame}
\frametitle{Indistinguishable particles}
\begin{enumerate}
\item Now let the two states coincide. Then $\psi = 2\psi_a(1)\psi_a(2)$ in the
symmetric case but $\psi = 0$ in the anti-symmetric case.
\item It is impossible for two particles with half-integral spins to be in the 
same state.
\item Recall that, in the context of the hydrogen atom, a stationary state of
and electron is completely satisfied by four quantum numbers. The previous point
therefore means that no two electron can have the same set of quantum numbers.
This is Pauli exclusion principle and it is an immediate consequence of the fact
that electrons have a spin $1/2$.
\item Indistinguishable particles means that all permutations of the electrons 
are equivalent.
\end{enumerate}
\end{frame}

\begin{frame}
\frametitle{Odd and even permutations}
\begin{enumerate}
\item Every permutation can be achieved by a finite number of `exchanges' of the
objects.
\item The sign of a permutation is `$+$' if it needs and even number of `exchanges'
starting from the arrangement $\{1, \ldots, n\}$. Otherwise, the sign is `$-$'. 
\item If $p$ denotes a permutation of $\{1, \ldots, n\}$ then $p(1)$ is the 
position of $1$ in the rearrangement, $p(2)$ is the position of $2$ in the 
rearrangement, ...
\item If there are $n$ electrons then the anti-symmetric wave function of a system
of all of them is
\begin{equation}\label{e6}
\psi(1, \ldots, n) = \frac{1}{\sqrt{n!}}
 \sum_{p}\text{sgn}(p)\psi_{p(1)}(1)\ldots\psi_{p(n)}(n)
\end{equation}
Here the sum is over all partitions.
\end{enumerate}
\end{frame}

\begin{frame}
\frametitle{Bosons and Fermions}
\begin{enumerate}
\item The symmetry or anti-symmetry of wave functions has a profound implication.
\item Particles with spins $0, 1, \ldots$ are called Bosons. Examples are - a
photon, a phonon, Deuterium nucleus, $\alpha$-particle.
\item Particles with spins $1/2, 3/2, \ldots$ are called Fermions. Examples are
- an electron, a proton, a neutron, a neutrino.
\item Bosons and Fermions have drastically different statistical properties.
\item Bulk matter is largely stable because of Pauli exclusion principle which
the electrons have to obey. Likewise, a lot of fascinating properties of the
crystalline matter is because of the fermionic character of the electrons and
holes.
\end{enumerate}
\end{frame}

\begin{frame}
\frametitle{H$_{2}$ molecule}
\begin{enumerate}
\item The wave function of the electrons in a Hydrogen molecule is $\psi(1, 2) =
\psi(\vec{r}_1, \vec{r}_2)s(1, 2)$.
\item $\psi(1, 2)$ must be anti-symmetric, which is possible only if
\begin{enumerate}
\item $\psi(\vec{r}_1, \vec{r}_2)$ is symmetric and $s(1, 2)$ is anti-symmetric,
that is, the spins are anti-parallel or
\item $\psi(\vec{r}_1, \vec{r}_2)$ is anti-symmetric and $s(1, 2)$ is symmetric,
that is, the spins are parallel.
\end{enumerate}
\item It turns out that the first possibility has lower energy and is therefore
favoured. It is called the `bonding orbital'. The other one is called the `anti-
bonding' orbital.
\item Molecular orbitals are a generalisation of the atomic orbitals. It is a
one-electron wave function in a molecule.
\item Molecular orbitals are expressed as linear combinations of atomic 
orbitals. Simplest examples are hybridised orbitals.
\end{enumerate}
\end{frame}

\begin{frame}
\frametitle{The chemical bond}
\begin{enumerate}
\item Covalent and coordinate-covalent bonds
\item Ionic bond
\item Hydrogen bond
\item Metallic bond
\item Polar and non-polar bonds
\end{enumerate}
\end{frame}

\begin{frame}
\frametitle{Molecular spectroscopy}
\begin{enumerate}
\item A large number energy states possible because of
\begin{enumerate}
\item electronic states,
\item vibrational states,
\item rotational states,
\item vibration-rotation states.
\end{enumerate}
\item Molecular spectra therefore appear as bands instead of lines.
\item Rotations of polar molecules are like oscillations of an electric dipole.
They can absorb and emit electromagnetic radiation. These states give rise to 
the rotational spectra of molecules. Non-polar molecules do not have absorption
spectra arising out of rotational states. Energy levels are in meV.
\item The separation in the vibrational energy levels is of the order of $0.1$
eV. Their fine structure consists of the rotational spectra.
\item The separation in the electronic states is of the order of a few eV.
\end{enumerate}
\end{frame}

\begin{frame}
\frametitle{Fluorescence and Phosphorescence}
\begin{enumerate}
\item A molecule absorbs radiation of a frequency $\nu_1$ to get excited into a
higher electronic state. It them loses energy through a series of vibrational
transitions. After reaching the lowest vibrational state, it emits a photon of
frequency $\nu_2 < \nu_1$.
\item Fluorescent lamps contain mercury vapour which emits ultraviolet radiation
when electric current is passed through it. It tube is coated with a material 
that fluoresces by emitting the lower energy visible radiation.
\item Sometimes the lowest vibrational level is such that a transition to the
ground state is `forbidden' by selection rules of the electric dipole type. 
However, other classes of transitions may be permitted, although they happen
at a much lower rate.
\item Such states are called metastable states. The molecule stays in these 
states for a long time, sometimes even for hours. The resulting radiation is
emitted much after the incident radiation is turned off. This is called
phosphorescence.
\end{enumerate}
\end{frame}
\end{document}

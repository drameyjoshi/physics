\documentclass{beamer}
\usefonttheme[onlymath]{serif}
\usepackage{graphicx,hyperref}
\hypersetup{colorlinks,urlcolor=blue}
\usetheme{Boadilla}

\newcommand{\pvec}[1]{\vec{#1}\mkern2mu\vphantom{#1}}

\title{Particle Properties of Waves}
\author{Amey Joshi}
\date{\today}

\begin{document}
\begin{frame}
\titlepage
\end{frame}

\begin{frame}
\frametitle{Particles and Waves}
\begin{enumerate}
\item A particle is a point mass with a precisely defined position and momentum.
\item The dimensions of a particle are small compared to the other dimensions of
the problem. The earth is a particle when one studies the solar system; not so
when we study ocean currents.
\item Waves are periodic changes in a physical quantity like the pressure, the
strain, the position of fluid particles or electric/magnetic field strengths
at a point.
\item The two were known to be starkly different. Almost ...
\item Light was thought to travel like particles in geometric optics but like
a wave when one studies interference and diffraction.
\end{enumerate}
\end{frame}

\begin{frame}
\frametitle{Black body radiation - 1}
\begin{enumerate}
\item An ideal black body absorbs all radiation incident on it. It emits all
radiation to maintain thermal equilibrium.
\item We consider the spectral energy density of radiation emitted from a black
body. Spectral energy density is the energy of radiation with wavelength between
$\lambda$ and $\lambda + d\lambda$ per unit volume.
\item The peak of the spectral energy density depends on the temperature of the
black body. 
\item The spectral density looks like 
\href{https://www.britannica.com/science/blackbody}{this}.
\item Read the temperatures, spot the maxima. Can you build a thermometer from
 this plot?
\item Classical physics could not explain the shape of the spectrum.
\end{enumerate}
\end{frame}

\begin{frame}
\frametitle{Black body radiation - 2}
Radiation is trapped in the cavity in the form of standing waves. In a cubical
cavity of length $L$, a standing wave will be formed parallel to $x$-axis if
\begin{equation}\label{e1}
L = n_x\frac{\lambda}{2}, n_x = 1, 2, \ldots
\end{equation}
In general, an arbitrary standing wave must obey 
\begin{equation}\label{e2}
n^2 = n_x^2 + n_y^2 + n_z^2 = \left(\frac{2L}{\lambda}\right)^2.
\end{equation}
The number of standing waves with wavelengths between $\lambda$ and $\lambda + 
d\lambda$ in the first octant between radii $n$ and $n + dn$ is proportional to
the volume of the region.
\[
\frac{1}{8} \times 4\pi n^2 dn = \frac{1}{2} \pi n^2 dn.
\]
But there are two polarisation modes. Therefore, the number of standing waves
is twice this quantity,
\begin{equation}\label{e3}
G(n)dn = \pi n^2dn
\end{equation}
\end{frame}

\begin{frame}
\frametitle{Black body radiation - 3}
From \eqref{e2} $n = 2L/\lambda = 2L\nu/c$. Therefore, equation \eqref{e3} 
becomes
\begin{equation}\label{e4}
G(\nu)d\nu = \pi \frac{8L^3}{c^3}\nu^2 d\nu.
\end{equation}
The number of standing waves per unit volume, also called `density of states'
is
\begin{equation}\label{e5}
g(\nu)d\nu = \frac{8\pi}{c^3}\nu^2 d\nu.
\end{equation}
Equivalently, since $g(\lambda)d\lambda = -g(\nu)d\nu$,
\begin{equation}\label{e6}
g(\lambda)d\lambda = \frac{8\pi}{\lambda^4} d\lambda.
\end{equation}
Nothing controversial so far. These calculations were done by Lord Rayleigh
and Sir James Jeans.
\end{frame}

\begin{frame}
\frametitle{Black body radiation - 4}
\begin{enumerate}
\item Lord Rayleigh assumed the classical equipartition theorem to conclude that
the energy per mode is $kT$ and hence the spectral density is
\begin{equation}\label{e7}
u(\nu)d\nu = \frac{8\pi}{c^3} kT \nu^2 d\nu.
\end{equation}
\item $u(\nu) \Rightarrow \infty$ as $\nu \rightarrow \infty$. Not what you 
saw experimentally.
\item Max Planck proposed that energy can be exchanged only in packets of
$h\nu$ and showed that the average energy should be
\[
\frac{h\nu}{\exp\left(\frac{h\nu}{kT}\right) - 1}
\] 
and hence the spectral density should be
\begin{equation}\label{e8}
u(\nu)d\nu = 
   \frac{8\pi h\nu^3}{c^3}\frac{d\nu}{\exp\left(\frac{h\nu}{kT}\right) - 1}
\end{equation}
\end{enumerate}
\end{frame}

\end{document}


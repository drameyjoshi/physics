\documentclass{article}
\begin{document}
\begin{enumerate}
\item Mankind, for a large part of the history, have viewed the sky to be 
and eternal heaven, in which luminous bodies move along predictable paths.
The sky was viewed as the inside of a large, spherical surface. The 
luminous bodies moved, or appeared to move, on that two-dimensional surface.

\item Some stars that did not confirm to orderly motion of the rest of the
vast majority were called wanderers. Later on, they came to be called as
the planets.

\item No one knew the physical properties of stars and planets. Why they 
shone, why some were brighter than others or had a different colour.

\item Friedrich Willhelm Bessel was a skilled astronomer and mathematician.
He built telescopes for accurate measurements of stellar movements. He 
reported that Sirius, a very bright star, moved irregularly about a smooth
trajectory. He suspected that it was part of a binary system. But where
was its companion?

\item The period of irregularity was measured to be close to $50$ years by
Bessl's successor at the K\"{o}nisberg observatory.

\item Alvan Clark Jr first spotted the companion of Sirius when he was 
testing a new lens. Following the conventions in astronomy, the newly 
spotted very faint star was called \emph{Sirius B}. Eddington referred to
it as \emph{Sirius coms}. It is $10000$ times less brighter than Sirius
$A$.
\end{enumerate}
\end{document}

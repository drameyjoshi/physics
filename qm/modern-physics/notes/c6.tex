\chapter{The hydrogen atom}\label{c6}
\section{The Hamiltonian}\label{c6s1}
The quantum mechanical treatment of the hydrogen atom resolved the mystery of 
Bohr's second postulate. It also demonstrated the convenience of the 
Schr\"{o}dinger's approach. His equation can be solved exactly for the hydrogen
atom. The hydrogen atom consists of a single electron in the electric field od a
single proton. Since the electron is almost $1836$ times lighter than the proton
we can, as a first approximation, assume that the proton is at rest. Therefore, 
the energy of the system is
\begin{equation}\label{c6s1e1}
H = \frac{p^2}{2m} - \frac{1}{4\pi\epsilon_0}\frac{e^2}{r},
\end{equation}
where $e$ is the electronic charge and $r$ is the distance between the two. The
operator representation of this Hamiltonian is
\begin{equation}\label{c6s1e2}
\hat{H} = -\frac{\hslash^2}{2m}\nabla^2 - \frac{1}{4\pi\epsilon_0}\frac{e^2}{r}
\end{equation}
and the Schr\"{o}dinger's time independent equation is
\begin{equation}\label{c6s1e3}
-\frac{\hslash^2}{2m}\nabla^2\psi(\vec{r}) - 
\frac{1}{4\pi\epsilon_0}\frac{e^2}{r}\psi(\vec{r}) = E\psi(\vec{r}).
\end{equation}
The spherical symmetry of the problem suggests that we should work in the 
spherical polar coordinate system. The Laplacian in these coordinates is
\begin{equation}\label{c6s1e4}
\nabla^2 = \frac{1}{r^2}\frac{\partial}{\partial r}
\left(r\frac{\partial}{\partial r}\right) + 
\frac{1}{r^2\sin\theta}\frac{\partial}{\partial\theta}
\left(\sin\theta\frac{\partial}{\partial\theta}\right) + 
\frac{1}{r^2\sin^2\theta}\frac{\partial^2}{\partial\phi^2}
\end{equation}
The equation \eqref{c6s1e3} is solved using the method of `separation of 
variables'. We write
\begin{equation}\label{c6s1e5}
\psi(\vec{r}) = \psi(r, \theta, \phi) = R(r)\Theta(\theta)\Phi(\phi)
\end{equation}
so that equation \eqref{c6s1e3} becomes
\begin{eqnarray}
-\frac{\hslash^2}{2m}\left[
\frac{\Theta\Phi}{r^2}\frac{d}{dr}\left(r\frac{dR}{dr}\right) +
\frac{R\Phi}{r^2\sin\theta}\frac{d}{d\theta}
\left(\sin\theta\frac{d\Theta}{d\theta}\right) + 
\frac{R\Theta}{r^2\sin^2\theta}\frac{d^2\Phi}{d\phi^2}\right] &=& \nonumber \\ 
\left(E + \frac{1}{4\pi\epsilon_0}\frac{e^2}{r}\right)R\Theta\Phi & & 
\label{c6s1e6}
\end{eqnarray}
Multiplying both sides by
\[
\frac{r^2\sin^2\theta}{R\Theta\Phi}
\]
and rearranging a bit, we get
\begin{equation}\label{c6s1e7}
\frac{\sin^2\theta}{R}\frac{d}{dr}\left(r\frac{dR}{dr}\right) +
\frac{\sin\theta}{\Theta}\left(\sin\theta\frac{d\Theta}{d\theta}\right) +
\frac{2m}{\hslash^2}\left(E + \frac{1}{4\pi\epsilon_0}\frac{e^2}{r}\right)
r^2\sin^2\theta =
-\frac{1}{\Phi}\frac{d^2\Phi}{d\phi^2}. 
\end{equation}
The left hand side of this equation is a function of $r$ and $\theta$ while
the right hand side depends on $\phi$ alone. Therefore, each side must be equal
to a constant, say $\alpha$. In particular, we get
\begin{equation}\label{c6s1e8}
\frac{1}{\Phi}\frac{d^2\Phi}{d\phi^2} = -\alpha
\end{equation}
or
\begin{equation}\label{c6s1e9}
\frac{d^2\Phi}{d\phi^2} + \alpha\Phi = 0.
\end{equation}
This equation will have a periodic solution if $\alpha > 0$ and an exponential
one if $\alpha < 0$. The hydrogen atom is clearly periodic in $\phi$. Therefore,
we can write
\begin{equation}\label{c6s1e10}
\alpha = m_l^2
\end{equation} 
so that we get
\begin{equation}\label{c6s1e11}
\frac{d^2\Phi}{d\phi^2} + m_l^2\Phi = 0
\end{equation}
and
\begin{equation}\label{c6s1e12}
\frac{\sin^2\theta}{R}\frac{d}{dr}\left(r\frac{dR}{dr}\right) +
\frac{\sin\theta}{\Theta}\left(\sin\theta\frac{d\Theta}{d\theta}\right) +
\frac{2m}{\hslash^2}\left(E + \frac{1}{4\pi\epsilon_0}\frac{e^2}{r}\right)
r^2\sin^2\theta = m_l^2.
\end{equation}
The solution of the equation \eqref{c6s1e11} is
\begin{equation}\label{c6s1e13}
\Phi(\phi) = \alpha_1 e^{im_l\phi} + \alpha_2 e^{-im_l\phi},
\end{equation}
where $\alpha_1$ and $\alpha_2$ are constants of integration. Since $\Phi(\phi)
= \Phi(\phi + 2\pi)$ $m_l$ must be an integer. We can therefore write the 
solution \eqref{c6s1e13} as just
\begin{equation}\label{c6s1e13}
\Phi(\phi) = \alpha_1 e^{im_l\phi}
\end{equation}
as the other factor is automatically included in the first one. In order to 
separate equation \eqref{c6s1e12}, divide both sides by $\sin^2\theta$ and
rearrange a bit to get
\begin{equation}\label{c6s1e14}
\frac{1}{R}\frac{d}{dr}\left(r\frac{dR}{dr}\right) +
\frac{2m}{\hslash^2}\left(E + \frac{1}{4\pi\epsilon_0}\frac{e^2}{r}\right)r^2 =
-\frac{1}{\Theta\sin\theta}\left(\sin\theta\frac{d\Theta}{d\theta}\right) +
\frac{m_l^2}{\sin^2\theta}.
\end{equation}
Once again we have a situation in which the left hand side of the equation 
depends on one variable, $r$, and the right hand side on another, $\theta$
so that each side is a constant, say $\beta$. Thus, we have
\begin{equation}\label{c6s1e16}
-\frac{1}{\Theta\sin\theta}\left(\sin\theta\frac{d\Theta}{d\theta}\right) +
\frac{m_l^2}{\sin^2\theta} = \beta
\end{equation}
and
\begin{equation}\label{c6s1e17}
\frac{1}{R}\frac{d}{dr}\left(r\frac{dR}{dr}\right) +
\frac{2m}{\hslash^2}\left(E + \frac{1}{4\pi\epsilon_0}\frac{e^2}{r}\right)r^2 =
\beta.
\end{equation}
We will solve these equations in the following sections. And yet, before 
proceeding we notice that the energy $E$ is part of the $r$-equation. This 
indicates that the energy of an electron in the Hydrogen atom depends on $r$
alone, in agreement with the Bohr model.

\section{The $\theta$-equation}\label{c6s2}
\section{The $r$-equation}\label{c6s3}


\chapter{The Electromagnetic Field Equations}\label{c4}
\begin{enumerate}
\item From equations \eqref{c3e19} and \eqref{c3e20} it follows that
\begin{eqnarray}
\curl\vec{E} &=& -\frac{1}{c}\pdt{\vec{H}}{t} \label{c4e1} \\
\dive\vec{H} &=& 0 \label{c4e2}
\end{eqnarray}
These equations can be easily cast in integral form as 
\begin{eqnarray}
\oint\vec{E}\cdot d\vec{l} &=& -\frac{1}{c}\frac{\partial}{\partial t}\int\vec{H}\cdot d\vec{f} \label{c4e3} \\
\oint\vec{H}\cdot d\vec{f} &=& 0 \label{c4e4}
\end{eqnarray}
Note that, in equation \eqref{c4e3}, the surface integral on the rhs is over the
surface bounded by the countour along which the line integral of lhs is calculated.
The line integral is over a closed curve but the surface integral is not over a 
closed surface. The equation states that the circulation of the electric field
is equal to $-1/c$ times the flux of $\vec{H}$ through the surface bound by the
countour. On the other hand, \eqref{c4e4} tells that flux of $\vec{H}$ through
any closed surface is zero.

\item Equations \eqref{c4e1} and \eqref{c4e2} form the first pair of Maxwell
equations. They do not describe the electromagnetic field completely because
they involve the time derivative of $\vec{H}$ alone and not $\vec{E}$.

\item It is possible to write equations \eqref{c4e1} and \eqref{c4e2} in terms
of the electromagnetic field tensor $F_{\mu\nu}$. Since
\[
F_{\mu\nu} = \pdt{A_\nu}{x^\mu} - \pdt{A_\mu}{x^\nu},
\]
we have
\begin{eqnarray*}
\pdt{F_{\mu\nu}}{x^\rho} &=& \pdt{A^2_\nu}{x^\mu x^\rho} - \pdt{A^2_\mu}{x^\nu x^\rho} \\
\pdt{F_{\nu\rho}}{x^\mu} &=& \pdt{A^2_\rho}{x^\nu x^\mu} - \pdt{A^2_\nu}{x^\rho x^\mu} \\
\pdt{F_{\rho\mu}}{x^\nu} &=& \pdt{A^2_\mu}{x^\rho x^\nu} - \pdt{A^2_\rho}{x^\mu x^\nu}
\end{eqnarray*}
from which we get
\begin{equation}\label{c4e5}
\pdt{F_{\mu\nu}}{x^\rho} + \pdt{F_{\nu\rho}}{x^\mu} + \pdt{F_{\rho\mu}}{x^\nu} = 0.
\end{equation}

\item We will examine \eqref{c4e5} in greater details.
\begin{enumerate}
\item $\mu = 0, \nu = 1, \rho = 2$:
\[
\pdt{F_{01}}{x^2} + \pdt{F_{12}}{x^0} + \pdt{F_{20}}{x^1} = 0 \Rightarrow
\pdt{E_y}{x} - \pdt{E_x}{y} = - \frac{1}{c}\pdt{H_z}{t}.
\]

\item $\mu = 0, \nu = 1, \rho = 3$:
\[
\pdt{F_{01}}{x^3} + \pdt{F_{13}}{x^0} + \pdt{F_{30}}{x^1} = 0 \Rightarrow
\pdt{E_x}{z} - \pdt{E_z}{x} = - \frac{1}{c}\pdt{H_y}{t}.
\]

\item $\mu = 0, \nu = 2, \rho = 3$:
\[
\pdt{F_{02}}{x^3} + \pdt{F_{23}}{x^0} + \pdt{F_{30}}{x^2} = 0 \Rightarrow
\pdt{E_z}{y} - \pdt{E_y}{z} = - \frac{1}{c}\pdt{H_x}{t}.
\]

\item $\mu = 1, \nu = 2, \rho = 3$:
\[
\pdt{F_{12}}{x^3} + \pdt{F_{23}}{x^1} + \pdt{F_{31}}{x^2} = 0 \Rightarrow
-\pdt{H_z}{z} - \pdt{H_x}{x} - \pdt{H_y}{y} = 0.
\]
\end{enumerate}
Thus, \eqref{c4e5} encodes the two Maxwell equations \eqref{c4e1} and \eqref{c4e2}.
We also note that
\begin{enumerate}
\item If any two indices in \eqref{c4e5} are equal then we get the identity $0=0$.
\item If all three indices are equal then each term on lhs of \eqref{c4e5} is
zero.
\item Let us examine what happens when we consider other combinations of indices.
If we swap the values of $\mu$ and $\nu$ in case (a), that is, if $\mu = 1, \nu = 0,
\rho = 2$ then we have
\[
\pdt{F_{10}}{x^2} + \pdt{F_{02}}{x^1} + \pdt{F_{21}}{x^0} = 0 \Rightarrow
-\pdt{E_x}{y} + \pdt{E_y}{x} - \frac{1}{c}\pdt{H_z}{t} = 0,
\]
which is same as the conclusion of case (a). We can similarly show that all swaps
give 
\[
\curl\vec{E} = -\frac{1}{c}\pdt{\vec{H}}{t}.
\]
\end{enumerate}

\item Now consider the expression,
\begin{equation}\label{c4e6}
C^\mu = \epsilon^{\mu\nu\rho\sigma}\pdt{F_{\rho\sigma}}{x^\nu}.
\end{equation}
Then
\[
C^0 = \pdt{F_{23}}{x^1} - \pdt{F_{32}}{x^1} - \pdt{F_{13}}{x^2} + \pdt{F_{31}}{x^2}
 + \pdt{F_{12}}{x^3} - \pdt{F_{21}}{x^3}
\]
which, owing to anti-symmetric nature of $F_{\mu\nu}$ is
\[
C^0 = 2\left(\pdt{F_{23}}{x^1} + \pdt{F^{31}}{x^2} + \pdt{F_{12}}{x^3}\right)
\]
From equation \eqref{c4e5}, it is $C^0 = 0$. Similarly we can show that the other
components too vanish. Therefore, equation \eqref{c4e5} is equivalent to
\begin{equation}\label{c4e7}
\epsilon^{\mu\nu\rho\sigma}\pdt{F_{\rho\sigma}}{x^\nu} = 0.
\end{equation}

\item We next develop the action function for a system consisting of particles
in an electromagnetic field. From equations \eqref{c2e3} and \eqref{c2e6}, the
action for a free particle is
\begin{equation}\label{c4e8}
S_m = -mc\int ds.
\end{equation}
If there are many particles, we can write it as
\begin{equation}\label{c4e9}
S_m = -\sum_i m_ic\int ds.
\end{equation}
The subscript `m' indicates that the action depends only of the mass of the 
particles and therefore pertains to free particles alone. When the particles
interact with the field, we need an additional term,
\begin{equation}\label{c4e10}
S_{mf} = -\sum_i \frac{e_i}{c}\int A_\mu dx^\mu.
\end{equation}
It is a mild generalisation of the second term on the rhs of \eqref{c3e2}.
We need a term $S_f$ that determines the behaviour of the fields in absence of
matter so that the action of the system is
\begin{equation}\label{c4e11}
S = S_m + S_{mf} + S_f.
\end{equation}

It is an experimental fact that the electromagnetic fields obey the principle of
superposition. Therefore, the differential equations describing them must be linear.
Since the equations are derived from the variational principle, we require that the
integrand of the action integral must be quadratic in the field tensor. Furthermore,
we require the action integral to be a scalar. Therefore, we also require the 
integrand to be a scalar. The only scalar, quadratic quantity that depends on the
field tensor alone is $F_{\mu\nu}F^{\mu\nu}$. We propose that
\begin{equation}\label{c4e12}
S_f = a\int F_{\mu\nu}F^{\mu\nu} d\Omega.
\end{equation}

The quantity $\epsilon^{\mu\rho\nu\sigma}F^{\mu\rho}F^{\nu\sigma}$ is not 
considered to be part of $S_f$ because it is a pseudo-scalar. We showed in 
\eqref{c3e98} that it is $\vec{E}\cdot\vec{H}$. Since $\vec{H}$ is an axial vector,
the dot product is a pseudo-scalar.

Equation \eqref{c3e97} gave us
\begin{equation}\label{c4e13}
F_{\mu\nu}F^{\mu\nu} = 2(H^2 - E^2).
\end{equation}
In terms of the potentials,
\begin{equation}\label{c4e14}
F_{\mu\nu}F^{\mu\nu} = |\curl\vec{A}|^2 - |\grad\varphi|^2 + 
2\grad\varphi\pdt{\vec{A}}{t} - \left|\pdt{\vec{A}}{t}\right|^2.
\end{equation}
This shows that $F_{\mu\nu}F^{\mu\nu}$ can be made arbitrarily negative if $\vec{A}$
can be made to vary arbitrarily quickly with $t$. In that case, we cannot minimise
the action integral of \eqref{c4e12}. To prevent that from happening, we require
the constant $a$ to be negative. Its value depends on our choice of units. In
gaussian units, it is 
\[
a = -\frac{1}{16\pi c}
\]
so that the action for the field is
\begin{equation}\label{c4e15}
S_f = -\frac{1}{16\pi c}\int F_{\mu\nu}F^{\mu\nu} d\Omega.
\end{equation}
Using \eqref{c4e13},
\begin{equation}\label{c4e16}
S_f = \frac{1}{8\pi}\int (E^2 - H^2) dt dV,
\end{equation}
where we used the fact that $d\Omega = cdtdV$. We can also write it as
\begin{equation}\label{c4e17}
S_f = \int dt \left(\frac{1}{8\pi}\int dV (E^2 - H^2)\right) = \int dt L,
\end{equation}
where
\begin{equation}\label{c4e18}
L_f = \frac{1}{8\pi}\int dV (E^2 - H^2) = \int dV \mathcal{L}_f.
\end{equation}
The quantity,
\begin{equation}\label{c4e19}
\mathcal{L}_f = \frac{1}{8\pi}(E^2 - H^2)
\end{equation}
is called the Lagrangian density of the field and $L_f$ the Lagrangian of the 
field.
\end{enumerate}
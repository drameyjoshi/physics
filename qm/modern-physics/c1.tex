\chapter{Special Theory of Relativity}\label{c1}
\section{Galilean relativity}\label{c1s1}
The idea of relative motion is as old as Newton's mechanics. Consider two 
observers, $O$ and $O^\prime$ with their frames of reference $S$ and $S^\prime$.
If the axes of $S$ and $S^\prime$ are parallel to each other but the origin of
$S^\prime$ is displaced by $\vec{R}$ from the origin of $S$ then a point 
$\vec{r}$ in $S$ appears as 
\begin{equation}\label{c1s1e1}
\vec{r}^\prime = \vec{r} - \vec{R}
\end{equation}
in $S^\prime$. The two observers will disagree on the coordinates of a point. 
However, the disagreement will always be to the extent given in equation \eqref
{c1s1e1}. If the observer $O$ finds $O^\prime$ moving then their measurements 
of velocities of other bodies will not match. $O$ will measure $\vec{v} = 
\dot{\vec{r}}$, $O^\prime$ will measure $\vec{v}^\prime = \dot{\vec{r}}^\prime$
and the two will be related as
\begin{equation}\label{c1s1e2}
\vec{v}^\prime = \vec{v} - \vec{V},
\end{equation}
where $\vec{V} = \dot{\vec{R}}$ is the velocity of $O^\prime$ as measured by
$O$. Likewise, if $O$ finds that $O^\prime$'s velocity changes with time then
their measurements of accelaration will be related as
\begin{equation}\label{c1s1e3}
\vec{a}^\prime = \vec{a} - \vec{A},
\end{equation}
where $\vec{a} = \ddot{\vec{r}}, \vec{a}^\prime = \ddot{\vec{r}}$ and $\vec{A}
= \ddot{\vec{R}}$. If $O$ observes a body moving in a straight line with a
uniform velocity then $O^\prime$ sees it accelarating. This is the origin of
pseudo-forces.

\subsection{Problem set 1}\label{c1s1s1}
\begin{enumerate}
\item Alice is standing on the platform and sees Bob on a train passing at a
uniform speed $V = 40$ kmph. Bob rolls a ball on the floor of the train and
measures its speed to be $v^\prime = 5$ kmph. What will be Alice's measurement
of the ball's speed if the ball moves (a) in the direction of the train, 
(b) against the direction of the train and (c) sideways?
\item Bob is siting on a flat wagon of a train. The floor of the wagon is 
smooth and there is a ball placed on it. The train begins to move. Both Alice
and Bob observe that the ball rolls in a direction opposite to that of the 
train. How will Alice and Bob explain the motion of the ball?
\end{enumerate}

If the observer $O^\prime$ is moving at a uniform velocity with respect to $O$
then $\vec{A} = 0$ and $\vec{a}^\prime = \vec{a}$. Therefore, if $O$ observes
a body at rest then $O^\prime$ observes it to be moving at a \emph{uniform}
speed $\vec{v}^\prime = -\vec{V}$.

Newton's first law states that every body continues to be in a state of rest
or of uniform motion in a straight line unless a force acts on it. A frame of
reference in which Newton's first law holds good is called an \emph{inertial
frame of reference}. 

\subsection{Problem set 2}\label{c1s1s2}
\begin{enumerate}
\item If $S$ is an inertial frame and if $S^\prime$ moves at a uniform velocity
$\vec{V}$ then show that $S^\prime$ is also an inertial frame of reference.
\item If you ignore the earth's motion around its axis and around the sun then
the ground is an inertial frame of reference. An object at rest on the ground
remains so even if the ground is very smooth. Suppose you are standing on the
ground and observing a friend on a merry-go-round. You see the friend in a 
uniform circular motion. Although your friend's speed is constant the direction
of his motion changes at every moment. He is accelarating. (a) What force 
causes the accelaration? (b) Your friend feels that he is being pushed out.
Who is pushing him?
\end{enumerate}

If a body does not feel a force in one inertial frame of reference then it
does not feel it in any other frame of reference moving uniformly with respect
to the first frame. All inertial frames of reference are equivalent. Observers
in inertial frames may disagree on position and velocities of bodies but never
on their accelarations. Newton's laws are valid in all inertial frames of 
reference. As an example, consider two space ships $S$ and $S^\prime$ observing
the planets. Let $\vec{r}_1$ and $\vec{r}_2$ be their positions with respect
to $S$. Then, the force between them is
\begin{equation}\label{c1s1e4}
\vec{F} = -\frac{Gm_1m_2}{|\vec{r}_1 - \vec{r}_2|^3}(\vec{r}_1 - \vec{r}_2).
\end{equation}
Here $m_1$ and $m_2$ are the masses of the planets at positions $\vec{r}_1$
and $\vec{r}_2$ with respect to the origin of $S$. The accelaration of mass 
$m_1$ is
\begin{equation}\label{c1s1e5}
\vec{a}_1 = -\frac{Gm_2} {|\vec{r}_1 - \vec{r}_2|^3}(\vec{r}_1 - \vec{r}_2).
\end{equation}
Observer $O^\prime$ will express this force as
\begin{equation}\label{c1s1e6}
\vec{F}^\prime = 
-\frac{Gm_1m_2}{|\vec{r}_1^\prime - \vec{r}_2^\prime|^3}
(\vec{r}_1^\prime - \vec{r}_2^\prime).
\end{equation}
and $m_1$'s acceleration as 
\begin{equation}\label{c1s1e7}
\vec{a}_1^\prime = -\frac{Gm_2} {|\vec{r}_1^\prime - \vec{r}_2^\prime|^3}
(\vec{r}_1^\prime - \vec{r}_2^\prime).
\end{equation}
Equation \eqref{c1s1e1} assures that the right hand sides of equations 
\eqref{c1s1e4} and \eqref{c1s1e6} are the same. The inertial nature of the
frames $S$ and $S^\prime$ assures that the left hand sides are also the same.
The law of motion, Newton's second law, is valid in both frames of reference.
The governing equation are identical, albeit with different variables.







\chapter{Charges in Electromagnetic Fields}\label{c3}
\begin{enumerate}
\item Unless they are in touch with each other, particles interact with each
other via fields.

\item Disturbances in fields propagate information. It is a postulate of the 
special theory of relativity that the changes in the state of a system are
progagated at a finite speed not exceeding $c$, the speed of light in vacuum.

\item This restriction rejects the hypothesis of an ideal rigid body in relativity.
For suppose a force is applied to one end of a rigid body. Then in order that the
inter-particle distance is unchanged, the force has to be propagated instantaneously
to the other end. Therefore, in elementary analysis of the theory of relativity
all particles are assumed to be points.

\item The principle of action for a free particle is given by equation \eqref{c2e4}
and \eqref{c2e7},
\begin{equation}\label{c3e1}
S = -\int_{t_1}^{t_2}mcds.
\end{equation}
For a particle in an electromagnetic field, the right hand side of this equation
should be augmented with another terms that represents the interaction of the
particle with the field. It should comprise of quantities describing the particle
as well as the field.

The particle's interaction with the field is described by a single quantity, its
charge $e$ and the field by its 4-potential $A_\mu$. The complete action is
\begin{equation}\label{c3e2}
S = -\int_{t_1}^{t_2} \left(mcds + \frac{e}{c}A_\mu dx^\mu\right).
\end{equation}

The 4-potential can be written as 
\begin{equation}\label{c3e3}
A_\mu = (\varphi, \vec{A})
\end{equation}
so that we can write \eqref{c3e2} as
\begin{equation}\label{c3e4}
S = \int_{t_1}^{t_2}\left(-mcds - e\varphi dt + 
\frac{e}{c}\vec{A}\cdot d\vec{x}\right).
\end{equation}
Since $ds = c\sqrt{1 - \beta^2}dt$ and $d\vec{x} = \vec{v}dt$,
\begin{equation}\label{c3e5}
S = \int_{t_1}^{t_2}\left(-mc^2\sqrt{1 - \beta^2} - e\varphi + 
\frac{e}{c}\vec{A}\cdot\vec{v}\right)dt.
\end{equation}
Thus, the Lagrangian for a particle in an electromagnetic field is
\begin{equation}\label{c3e6}
L = -mc^2\sqrt{1 - \beta^2} - e\varphi + \frac{e}{c}\vec{A}\cdot\vec{v}
= -mc^2\sqrt{1 - \beta^2} - e\varphi + \frac{e}{c}A^i v_i.
\end{equation}
From the Lagrangian, we get the generalised momentum
\begin{equation}\label{c3e7}
p^i = \pdt{L}{v^i} = m\gamma v^i + \frac{e}{c}A^i.
\end{equation}
From equation \eqref{c2e8}, the first term is just the generalised momentum for
the Lagrangian of \eqref{c2e7}. The Hamiltonian is given be the Legendre
transform
\begin{eqnarray*}
H &=& p^iv_i - L \\
  &=& m\gamma v^iv_i + \frac{e}{c}A^iv_i + mc^2\sqrt{1 - \beta^2} + e\varphi - 
  \frac{e}{c}A^iv_i \\
  &=& \frac{mv^2}{\sqrt{1 - \beta^2}} + mc^2\sqrt{1 - \beta^2} + e\varphi \\
  &=& \frac{mc^2}{\sqrt{1 - \beta^2}} + e\varphi
\end{eqnarray*}
or that
\begin{equation}\label{c3e8}
H = mc^2\gamma + e\varphi.
\end{equation}
Stricly speaking \eqref{c3e9} is not the correct form of the Hamiltonian because 
it is expressed in terms of the generalised velocity. To remedy this defect, we 
write
\eqref{c3e9} as
\[
\frac{(H - e\varphi)^2}{c^2} = m^2c^2 \gamma^2.
\]
Now,
\begin{equation}\label{c3e9}
\gamma^2 = \frac{1}{1 - \beta^2} = 1 + \frac{\beta^2}{1 - \beta^2} = 
1 + \beta^2\gamma^2
\end{equation}
so that
\[
\frac{(H - e\varphi)^2}{c^2} = m^2c^2(1 + \beta^2\gamma^2) 
\]
or that
\[
\frac{(H - e\varphi)^2}{c^2} = m^2c^2 + m^2\gamma^2 v^2,
\]
From \eqref{c3e7} the last term on the rhs can be written as
\begin{equation}\label{c3e10}
\frac{(H - e\varphi)^2}{c^2} = m^2c^2 + \left(p^i - \frac{e}{c}A^i\right)
\left(p_i - \frac{e}{c}A_i\right)
\end{equation}
Thus,
\begin{equation}\label{c3e11}
H = e\varphi + \sqrt{m^2c^4 + c^2 
\left(p^i - \frac{e}{c}A^i\right)\left(p_i - \frac{e}{c}A_i\right)}
\end{equation}
is the correct form of the Hamiltonian. For speeds small as compared to $c$,
\[
\sqrt{1 - \beta^2} = 1 - \frac{1}{2}\beta^2
\]
so that the Lagrangian of \eqref{c3e6} can be written as
\begin{equation}\label{c3e12}
L = \frac{1}{2}mv^2 - e\varphi + \frac{e}{c}\vec{A}\cdot\vec{v},
\end{equation}
where we ignored the constant terms $-mc^2$. At non-relativistic speeds, $\gamma$
can be well-approximated by $1$ so that the generalised momentum is
\begin{equation}\label{c3e13}
\vec{p} = m\vec{v} + \frac{e}{c}\vec{A}
\end{equation}
and the Hamiltonian is
\begin{equation}\label{c3e14}
H = \frac{1}{2m}\left(\vec{p} - \frac{e}{c}\vec{A}\right)^2 + e\varphi.
\end{equation}

From \eqref{c2e27},
\[
p^\nu = g^{\mu\nu}p_\mu = -g^{\mu\nu}\pdt{S}{x^\mu}
\]
so that
\[
p^0 = -\frac{1}{c}\pdt{S}{t}, \vec{p} = \grad S
\]
and hence the Hamilton-Jacobi equation is, from \eqref{c3e10},
\[
\frac{1}{c^2}\left(\pdt{S}{t} + e\varphi\right)^2 = m^2c^2 + 
\left(\grad S - \frac{e}{c}\vec{A}\right)^2
\]
where we replaced $H$ in \eqref{c3e10} with $-(1/c)\partial_t S$ because it is
the energy of the particle. We can reexpress the relativistic Hamilton-Jacobi
equation as 
\begin{equation}\label{c3e15}
\left(\grad S - \frac{e}{c}\vec{A}\right)^2 - 
\frac{1}{c^2}\left(\pdt{S}{t} + e\varphi\right)^2 + m^2c^2 = 0.
\end{equation}
\end{enumerate}
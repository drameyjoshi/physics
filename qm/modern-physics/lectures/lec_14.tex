\documentclass{beamer}
\usefonttheme[onlymath]{serif}
\usepackage{graphicx, hyperref}
\hypersetup{colorlinks, urlcolor=blue}
\usepackage{subcaption}
\usetheme{Boadilla}

\newcommand{\pvec}[1]{\vec{#1}\mkern2mu\vphantom{#1}}

\title{Statistical Mechanics - Some Applications}
\author{Amey Joshi}
\date{\today}

\begin{document}
\begin{frame}
\titlepage
\end{frame}

\begin{frame}
\frametitle{A brief recap}
\begin{enumerate}
\item Statistical methods are needed when the system consists of a very large
number of entities. We are not interested in position and momentum of every one
of them but only certain gross, experimentally verifiable quantities.
\item We confine our attention to ``ideal gases''.
\item Suppose that the entities in the gas can have energy $\epsilon_1, 
\epsilon_2, \ldots$ and that it is known that the system has an energy $E$ then
if we have $n_i$ particles with energy $\epsilon_i$ then we must have
\begin{equation}\label{e1}
n_1\epsilon_1 + n_2\epsilon_2 + \cdots = E.
\end{equation}
\item The goal is to find the most probable distribution of $n_1, n_2, \ldots$
in the form
\begin{equation}\label{e2}
n(\epsilon) = g(\epsilon)f(\epsilon),
\end{equation}
where $g$ is the number of states and $f$ is the probability of occupancy of the
state with energy $\epsilon$.
\end{enumerate}
\end{frame}

\begin{frame}
\frametitle{Maxwell-Boltzmann statistics - 1}
\begin{enumerate}
\item Applicable to a gas of identical but distingushable particles.
\item The probability of occupancy is
\begin{equation}\label{e3}
f(\epsilon) = A\exp\left(\frac{-\epsilon}{kT}\right),
\end{equation}
where $A$ is a constant that depends on the number of particles.
\item Therefore, the most probable distribution of gas atoms among the energy 
states is
\begin{equation}\label{e4}
n(\epsilon) = Ag(\epsilon)\exp\left(\frac{-\epsilon}{kT}\right).
\end{equation}
\item An immediate consequence is
\begin{equation}\label{e5}
\frac{n(\epsilon_2)}{n(\epsilon_1)} = \frac{g(\epsilon_2)}{g(\epsilon_1)}
\exp\left(-\frac{(\epsilon_2 - \epsilon_1)}{kT}\right).
\end{equation}
\end{enumerate}
\end{frame}

\begin{frame}
\frametitle{Maxwell-Boltzmann statistics - 2}
\begin{enumerate}
\item The ratio $g(\epsilon_2)/g(\epsilon_1)$ depends only on the nature of the
gas. It does not depend on the temperature.
\item If $\epsilon_2 > \epsilon_1$ then the exponential is less than 1; since 
$k$ is very small, we almost always have $n(\epsilon_2) < n(\epsilon_1)$.
\item Generally, most gas atoms prefer to be in lower energy states.
\item If, by some way, $\epsilon_2>\epsilon_1$ and $n(\epsilon_2)>n(\epsilon_1)$
then we say that the system has gone in the state of `population inversion'. The
temperature of the system then has to be negative on the absolute scale.
\item Systems with negative temperatures are ``hotter'' than those with positive
temperatures in the sense that when they are in contact, heat flows from the
former to the latter.
\end{enumerate}
\end{frame}

\begin{frame}
\frametitle{Bose-Einstein statistics - 1}
\begin{enumerate}
\item Applicable to identical, indistinguishable particles with integral spins.
\item The wave function of the system is symmetric under the exchange of 
particles.
\item The probability of occupancy is
\begin{equation}\label{e6}
f(\epsilon) = \frac{1}{e^\alpha e^{\epsilon/kT} - 1},
\end{equation}
where the constant $\alpha$ depends on the number of particles and temperature.
\item If the number of particles is not conserved, $\alpha = 0$ and the we have
\begin{equation}\label{e7}
f(\epsilon) = \frac{1}{e^{\epsilon/kT} - 1}.
\end{equation}
\item More particles per state than Maxwell-Boltzmann statistics at low enough
energies.
\end{enumerate}
\end{frame}

\begin{frame}
\frametitle{Bose-Einstein statistics - 2}
\begin{enumerate}
\item Photons are bosons. The energy distribution of photons in a cavity at a
temperature $T$ is given by the Bose-Einstein statistics. 
\item It immediately gives the Planck's radiation formula
\item Phonons in a crystal are also bosons. Einstein applied his statistics to 
the problem of specific heat of solids and explained why it goes to zero as
$T \rightarrow 0$ and why it is $3R$ at high temperatures (more than 273 K).
Here, $R$ is the universal gas constant.
\item Einstein assumed that each atom in a crystal vibrated with a frequency
$\nu$ with a probability
\begin{equation}\label{e8}
f(\nu) = \frac{1}{e^{h\nu/kT} - 1}.
\end{equation}
\item Einstein's theory was later on improved by Debye who assumed that 
vibrations in solids are best understood as an assembly of phonons.
\end{enumerate}
\end{frame}

\begin{frame}
\frametitle{Fermi-Dirac statistics - 1}
\begin{enumerate}
\item Applicable to identical, indistinguishable particles with half-integral 
spins.
\item The wave function of the system is anti-symmetric under the exchange of 
particles.
\item The probability of occupancy is
\begin{equation}\label{e9}
f(\epsilon) = \frac{1}{e^\alpha e^{\epsilon/kT} + 1},
\end{equation}
where the constant $\alpha$ depends on the number of particles and temperature.
\item Define the Fermi energy $\epsilon_F$ by the equation
\begin{equation}\label{e10}
f(\epsilon_F) = \frac{1}{2}.
\end{equation}
\item From equations \eqref{e9} and \eqref{e10} one readily gets $\alpha = 
-\epsilon_F$.
\end{enumerate}
\end{frame}

\begin{frame}
\frametitle{Fermi-Dirac statistics - 2}
\begin{enumerate}
\item In terms of Fermi energy, the probability of occupancy is
\begin{equation}\label{e11}
f(\epsilon) = \frac{1}{e^{(\epsilon - \epsilon_F)/kT} + 1},
\end{equation}
\item At $T = 0$, $f(\epsilon) = 0$ if $\epsilon > \epsilon_F$ and $f(\epsilon)
= 1$ if $\epsilon < \epsilon_F$. All energy levels up to $\epsilon_F$ are 
filled.
\item Fewer particles per state than Maxwell-Boltzmann statistics at low enough
energies - an immediate consequence of Pauli exclusion principle.
\item At room temperature $kT \approx 0.025$ eV. Fermi energy of electrons in
metals is of the order of a few eV, at least $40$ times greater!
\item When a solid metal is heated at room temperature, very few electrons with
energy levels close to $\epsilon_F$ get excited to higher states. Therefore,
the electrons make only a tiny contribution to the specific heat of solids.
\end{enumerate}
\end{frame}

\begin{frame}
\frametitle{Fermi-Dirac statistics - 3}
\begin{enumerate}
\item Gives rise to electron degeneracy pressure because of which ordinary 
matter remains stable.
\item Electron degeneracy pressure also prevents gravitational collapse in 
white dwarfs.
\item In heavier stars, electron degeneracy pressure can be overcome by 
gravity leading to a neutron star. The neutron star remains stable because 
of the neutron degeneracy pressure.
\item In the heaviest stars, the gravitational force can overcome neutron
degeneracy pressure as well leading to a black hole.
\item Fermi-Dirac statistics of electrons in solids give rise to their band 
structure, explaining most their physical properties.
\end{enumerate}
\end{frame}
\end{document}
\chapter{Electromagnetic Waves}\label{c6}
\begin{enumerate}
\item Maxwell equations in vacuum are:
\begin{eqnarray}
\dive\vec{E} &=& 0 \label{c6e1} \\
\curl\vec{E} &=& -\frac{1}{c}\pdt{\vec{H}}{t} \label{c6e2} \\
\dive\vec{H} &=& 0 \label{c6e3} \\
\curl\vec{H} &=& \frac{1}{c}\pdt{\vec{E}}{t} \label{c6e4}
\end{eqnarray}
If the fields are also time independent, then we will have $\dive\vec{E} = 0,
\curl\vec{E} = 0, \dive\vec{H} = 0$ and $\curl\vec{H} = 0$. From equations
\eqref{c5e16} (Coulomb's law) and \eqref{c5e140} (Biot-Savart's law), we get
$\vec{E} = 0$ and $\tav{H} = 0$. Therefore, we assume that the fields depend
on time.

\item We will soon show that equations \eqref{c6e1} to \eqref{c6e4} have a
non-zero solution in vacuum. That is, electromagnetic fields can exist in vacuum.
We will also show that they satisfy the wave equation. Solutions of Maxwell
equations in vacuum are called electromagnetic waves.

\item As usual, it is easier to analyse the potentials. Let us choose them such
that $\varphi = 0$ and $\dive\vec{A} = 0$. Equation \eqref{c6e2} becomes
\[
-\frac{1}{c}\curl\pdt{\vec{A}}{t} = 
-\frac{1}{c}\frac{\partial}{\partial t}\curl\vec{A},
\]
an identity. While, equation \eqref{c6e4} becomes
\[
\curl\curl\vec{A} = -\frac{1}{c^2}\spdt{\vec{A}}{t}.
\]
The lhs of this equation is $\grad\dive\vec{A} - \nabla^2\vec{A}$. Since we chose
$\dive{\vec{A}} = 0$, we get
\begin{equation}\label{c6e5}
\nabla^2\vec{A} - \frac{1}{c^2}\spdt{\vec{A}}{t} = 0.
\end{equation}
The choice $\dive{\vec{A}} = 0$ is called the Coulomb gauge and that it can be
exercised without violating physics is explained in the point leading to 
\eqref{c5e134}. Taking the curl of this equation gives
\begin{equation}\label{c6e6}
\nabla^2\vec{H} - \frac{1}{c^2}\spdt{\vec{H}}{t} = 0.
\end{equation}
Taking a partial derivative of \eqref{c6e5} and recalling that we chose $\varphi
= 0$, gives
\begin{equation}\label{c6e7}
\nabla^2\vec{E} - \frac{1}{c^2}\spdt{\vec{E}}{t} = 0.
\end{equation}

\end{enumerate}
\documentclass{beamer}
\usefonttheme[onlymath]{serif}
\usepackage{graphicx, hyperref}
\hypersetup{colorlinks, urlcolor=blue}
\usepackage{subcaption}
\usetheme{Boadilla}

\newcommand{\pvec}[1]{\vec{#1}\mkern2mu\vphantom{#1}}

\title{Solid State Physics - 2}
\author{Amey Joshi}
\date{\today}

\begin{document}
\begin{frame}
\titlepage
\end{frame}

\begin{frame}
\frametitle{Band theory of solids}
\begin{enumerate}
\item When two H atoms come together, their 1s orbitals combine to form two 
molecular orbitals. One of them is symmetric in space variables and the other
anti-symmetric. The former is favoured because of its lower energy. Two atomic
orbitals combine to form two molecular orbitals.
\item If $N \sim O(10^{23})$ atoms come together, their atomic orbitals combine
to create those many `solid state' orbitals. Since the number is so huge, the 
orbitals are so closely spaced that they can be considered to form a continuous
band of allowed energies.
\item The valence electrons form the valence band. The higher oribitals form the
higher bands. Of these the one right above the valence band is called the 
conduction band.
\item Sometimes the bands overlap, sometimes they do not.
\end{enumerate}
\end{frame}

\begin{frame}
\frametitle{Consequences of band structure}
\begin{enumerate}
\item When the valence band and the conduction band overlap the electrons close
to the Fermi level have enough states available to get into when they acquire
energy from the applied electric field. Such solids are good conductors of 
electricity.
\item When the valence and the conduction band are separated, we say that there
is a band gap.
\item If the band gap is more than 1.2 eV, the solid is an insulator at room
temparature and below.
\item If the band gap is less than 1.2 eV, the solid is a semiconductor at room
temparature and above. It is an insulator at lower temperatures.
\item The band structure is usually very complicated. Bands may overlap in one
direction but not in another.
\end{enumerate}
\end{frame}

\begin{frame}
\frametitle{Optical properties of solids}
\end{frame}

\begin{frame}
\frametitle{Electrons and holes}
\end{frame}

\begin{frame}
\frametitle{Band structure of semiconductors}
\end{frame}

\begin{frame}
\frametitle{Ohm's law}
\end{frame}

\begin{frame}
\frametitle{Hall effect}
\end{frame}

\begin{frame}
\frametitle{Superconductors}
\end{frame}

\begin{frame}
\frametitle{Josephson junctions and SQUID}
\end{frame}
\end{document}
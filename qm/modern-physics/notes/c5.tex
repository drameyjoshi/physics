\chapter{Some more one dimensional problems}\label{c5}
In the previous chapter we solved the Schr\"{o}dinger equation for two extreme
situations, the free particle and a particle in an infinite potential well. In
this chapter, we shall restrict ourselves to one dimensional problems so that
Schr\"{o}dinger's time independent equation remains an ordinary differential
equation. We will further consider only those problems where it is possible to
obtain an exact, analytical solution.

\section{Particle in a finite potential well}\label{c5s1}
This problem is a minor, but an interesting, modification of the problem we
considered in section \ref{c4s2}. The only change in the problem is the 
potential,
\begin{equation}\label{c5s1e1}
V(x) = \begin{cases}
0 & \text{ if } -L/2 \le x \le L/2 \\
V & \text{ otherwise.}
\end{cases}
\end{equation}
Schr\"{o}dinger's time independent equation in one dimension is
\begin{equation}\label{c5s1e2}
-\frac{\hslash^2}{2m}\frac{d^2\psi}{dx^2} + V(x)\psi = E\psi.
\end{equation}
In the region $-L/2 \le x \le L/2$, which is in the interior of the well,
the equation becomes
\begin{equation}\label{c5s1e3}
-\frac{\hslash^2}{2m}\frac{d^2\psi}{dx^2} = E\psi
\end{equation}
while outside the well it is
\begin{equation}\label{c5s1e4}
-\frac{\hslash^2}{2m}\frac{d^2\psi}{dx^2} + V\psi = E\psi.
\end{equation}
We can rearrange \eqref{c5s1e3} as
\begin{equation}\label{c5s1e5}
\frac{d^2\psi}{dx^2} + k^2\psi = 0,
\end{equation}
where
\begin{equation}\label{c5s1e6}
k^2 = \frac{2mE}{\hslash^2}.
\end{equation}
The general solution of this equation is, 
\begin{equation}\label{c5s1e7}
\psi(x) = \alpha e^{-ikx} + \beta e^{ikx},
\end{equation}
where $\alpha$ and $\beta$ are constants of integration. Unlike the case of 
an infinitely deep well, we do not have the boundary conditions like 
\eqref{c4s2e24} and \eqref{c4s2e25}. In order to get an appropriate boundary
condition we must solve the Schr\"{o}dinger equation \eqref{c5s1e4} in the 
region outside the well. It can be rearranged as
\begin{equation}\label{c5s1e8}
\frac{d^2\psi}{dx^2} + \frac{2m(E - V)}{\hslash^2}\psi = 0.
\end{equation}
Let us consider the case $E < V$ in which the particle does not have enough
energy to cross the potential well. Define
\begin{equation}\label{c5s1e9}
k_o^2 = \frac{2m(V - E)}{\hslash^2} > 0
\end{equation}
so that we can write \eqref{c5s1e8} as
\begin{equation}\label{c5s1e10}
\frac{d^2\psi}{dx^2} - k_o^2\psi = 0.
\end{equation}
The general solution of this equation is
\begin{equation}\label{c5s1e11}
\psi_o(x) = \alpha_0 e^{-k_ox} + \beta_o e^{k_ox}
\end{equation}
where $\alpha_o$ and $\beta_o$ are constants of integration. This solution
indicates that the wavefunction of the particle is \emph{not zero} outside the
potential well. It means that there is a non-zero probability of finding the
particle outside the well in spite of the fact that the particle lacks the 
energy to cross the potential barrier. This is one example of `quantum
tunnelling', a phenonemon in which a particle appears in a position where no
classical particle would.

We would want the wave function of the particle to be continuous all 
throughtout. Therefore, we insist that
\begin{eqnarray}
\psi_o\left(-\frac{L}{2}\right) &=& \psi\left(-\frac{L}{2}\right) 
  \label{c5s1e12} \\
\psi_o\left(\frac{L}{2}\right) &=& \psi\left(\frac{L}{2}\right) 
  \label{c5s1e13} 
\end{eqnarray}
Further, we also want the solution to converge to zero as $x \rightarrow \pm
\infty$. Therefore, the solution outside the box must be
\begin{equation}\label{c5s1e14}
\psi_o(x) = \begin{cases}
\beta_o e^{k_ox} & \text{ if } x \le -L/2 \\
\alpha_o e^{-k_ox} & \text{ if } x \ge L/2.
\end{cases}
\end{equation}
From equations \eqref{c5s1e12}, \eqref{c5s1e13} and \eqref{c5s1e14} we get
\begin{eqnarray}
\beta_o e^{-k_oL/2} &=& \alpha e^{ikL/2} + \beta e^{-ikL/2} \label{c5s1e15} \\
\alpha_o e^{-k_oL/2} &=& \alpha e^{-ikL/2} + \beta e^{ikL/2} \label{c5s1e16}
\end{eqnarray}
Equation \eqref{c5s1e2} is a second order equation. Therefore, we also insist
on the continuity of the first derivative of $\psi$. We then get the pair of
equations
\begin{eqnarray}
k_o\beta_o e^{-k_oL/2} &=& -ik\alpha e^{ikL/2} + ik\beta e^{-ikL/2} 
   \label{c5s1e17} \\
-k_o\alpha_o e^{-k_oL/2} &=& -ik\alpha e^{-ikL/2} + ik\beta e^{ikL/2}.
   \label{c5s1e18}
\end{eqnarray}
Adding equations \eqref{c5s1e15} and \eqref{c5s1e16} we get
\begin{equation}\label{c5s1e19}
(\alpha_o + \beta_o)e^{-k_oL/2} = 2\cos\left(\frac{kL}{2}\right)(\alpha + \beta)
\end{equation}
while subtracting \eqref{c5s1e18} from \eqref{c5s1e17} gives
\begin{equation}\label{c5s1e20}
k_o(\alpha_o + \beta_o)e^{-k_oL/2} = -2\sin\left(\frac{kL}{2}\right)(\alpha +
\beta).
\end{equation}
From equations \eqref{c5s1e19} and \eqref{c5s1e20} we get
\begin{equation}\label{c5s1e21}
k_o + \tan\left(\frac{kL}{2}\right) = 0.
\end{equation}
From equations \eqref{c5s1e6} and \eqref{c5s1e9} it is clear that the above
equation is really a condition on the energy $E$, since the `depth' $V$ of the
potential is a given quantity. Once again we observe that $E$ cannot take 
arbitrary values but only those permitted by equation \eqref{c5s1e21}.

\section{The harmonic oscillator}\label{c5s2}


\documentclass{beamer}
\usefonttheme[onlymath]{serif}
\usepackage{graphicx, hyperref}
\hypersetup{colorlinks, urlcolor=blue}
\usepackage{subcaption}
\usetheme{Boadilla}

\newcommand{\pvec}[1]{\vec{#1}\mkern2mu\vphantom{#1}}

\title{The Atomic Spectra}
\author{Amey Joshi}
\date{\today}

\begin{document}
\begin{frame}
\titlepage
\end{frame}

\begin{frame}
\frametitle{The spin}
\begin{enumerate}
\item Consider a problem in classical physics. A point charge $q$ is moving in a
circle of radius $r$ with a frequency $f$. It makes a current $I = qf$.
\item A current carrying loop has a magnetic moment $\mu = IA$ where $A$ is the
area of the loop. In this case, $\mu = \pi qfr^2$ or
\begin{equation}\label{e1}
\mu = \frac{q}{2}\omega r^2.
\end{equation}
\item If the mass of the charged particle is $m$ then its angular momentum is
\begin{equation}\label{e2}
L = m\omega r^2.
\end{equation}
\item From equations \eqref{e1} and \eqref{e2} it is clear that $\mu=\gamma L$,
where
\begin{equation}\label{e3}
\gamma = \frac{q}{2m}
\end{equation}
is called the gyromagnetic ratio of the charged particle.
\end{enumerate}
\end{frame}

\begin{frame}
\frametitle{The spin}
\begin{enumerate}
\item There is a close relationship between magnetic moment and angular momentum
in classical physics. It is much deeper in quantum physics.
\item Some particles have an intrinsic magnetic moment. It is their fundamental
property the way their mass and electric charge are.
\item Particles with an intrinsic magnetic moment can be considered to also have
an intrinsic angular momentum. Intrinsic angular momentum is called the spin.
\item The quantum mechanical generalisation of the classical relation $\mu = 
\gamma L$ is
\begin{equation}\label{e1}
\mu = g\gamma S,
\end{equation}
where $\gamma$ is given by \eqref{e3} and the constant $g$ is called the $g$-
factor. It is (very close to) $2$ for an electron. $S$ is the spin angular 
momentum.
\item Uncharged particles like the photon and the neutrino also have a spin that 
is unrelated to magnetic moment.
\end{enumerate}
\end{frame}
\end{document}
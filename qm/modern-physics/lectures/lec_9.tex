\documentclass{beamer}
\usefonttheme[onlymath]{serif}
\usepackage{graphicx, hyperref}
\hypersetup{colorlinks, urlcolor=blue}
\usepackage{subcaption}
\usetheme{Boadilla}

\newcommand{\pvec}[1]{\vec{#1}\mkern2mu\vphantom{#1}}

\title{The Harmonic Oscillator}
\author{Amey Joshi}
\date{\today}

\begin{document}
\begin{frame}
\titlepage
\end{frame}

\begin{frame}
\frametitle{The Classical Harmonic Oscillator}
\begin{enumerate}
\item A mass $m$ is attached to a spring whose one end is fixed. It is pulled
slightly from its equilibrium position. The restoring force is proportional to
the extension and is directed against the extension.
\item Thus, $F = -kx$, $k$ is a constant of proportionality, called the spring
constant.
\item Newton's second law is $m\ddot{x} = -kx$ or $\ddot{x} + \omega^2 x = 0$
where
\begin{equation}\label{e1}
\omega^2 = \frac{k}{m}.
\end{equation}
\item The solution of the equation of motion is
\begin{equation}\label{e2}
x(t) = \alpha e^{i\omega t} + \beta e^{-i\omega t}
\end{equation}
where $\alpha, \beta$ are constants of integration, determined from the initial
conditions. $\omega$ is the angular frequency.
\end{enumerate}
\end{frame}

\begin{frame}
\frametitle{The Classical Harmonic Oscillator}
\begin{enumerate}
\item One can as well write \eqref{e2} as
\begin{equation}\label{e3}
x(t) = (\alpha + \beta)\cos\omega t + i(\alpha - \beta)\sin\omega t.
\end{equation}
The displacement $x$ thus varies sinusoidally.
\item The potential energy of the oscillator is $kx^2/2$ and the kinetic energy
is, as usual, $p^2/(2m)$. It can be shown that
\begin{equation}\label{e4}
\frac{1}{2}kx^2 + \frac{p^2}{2m} = E,
\end{equation}
is a constant. $E$ is called the total energy of the system.
\item The energy $E$ can take any non-negative value, that is, $E \ge 0$.
\item If $E = 0$ then $x = 0$ and $p = 0$. That is, the particle is at rest.
\end{enumerate}
\end{frame}

\begin{frame}
\frametitle{The Quantum Harmonic Oscillator}
\begin{enumerate}
\item Equation \eqref{e4} suggests that the Hamiltonian is
\begin{equation}\label{e5}
\hat{H} = \frac{1}{2}k\hat{x}^2 + \frac{\hat{p}^2}{2m}.
\end{equation}
so that the Schr\"{o}dinger equation $\hat{H}\psi = E\psi$ is
\begin{equation}\label{e6}
-\frac{\hslash^2}{2m}\frac{d^2\psi}{dx^2} + \frac{1}{2}kx^2\psi = E\psi.
\end{equation}
\item We rearrange this equation to get
\begin{equation}\label{e7}
\frac{d^2\psi}{dx^2} + \left(\frac{2mE}{\hslash^2} - 
\frac{m^2\omega^2 x^2}{\hslash^2}\right)\psi = 0.
\end{equation}
\item Solving this equation is not a trivial affair, but neither is it very 
hard. 
\end{enumerate}
\end{frame}

\begin{frame}
\frametitle{The Quantum Harmonic Oscillator}
\begin{enumerate}
\item In the Lagrangian and the Hamiltonian formulation of classical mechanics
it is customary to denote the coordinate as $q$. Therefore, the expression for
the total energy, \eqref{e4}, is written as
\begin{equation}\label{e8}
\frac{1}{2}kq^2 + \frac{p^2}{2m} = E.
\end{equation}
\item The total energy of many systems can be written as 
\begin{equation}\label{e9}
aq^2 + bp^2 = E,
\end{equation}
where $a$ and $b$ are constants specific to the system. Any system whose energy
can be written in the form of \eqref{e9} is called a harmonic oscillator. If
the energy is written as a sum of terms, each of the form \eqref{e9}, the system
can be considered as an \emph{ensemble} of harmonic oscillators.
\end{enumerate}
\end{frame}

\begin{frame}
\frametitle{The Quantum Harmonic Oscillator}
\begin{enumerate}
\item A harmonic oscillator is not a `toy' example in quantum mechanics. 
\item Sidney Coleman, an American theoretical physicist, one remarked that 
``Quantum field theory is harmonic motion taken to increasing levels of 
abstraction''. We are not even scratching the surface in this course.
\item Standing electromagnetic waves in a cavity and a lattice of nuclei in a
crystal can both be considered as an ensemble of quantum harmonic oscillators.
\item There are two ways of solving this very important problem in quantum 
mechanics. The traditional way is to solve the differential equation \eqref{e7}
using the Frobenius series solution method. We will follow that one.
\item The modern method expresses the Schr\"{o}dinger equation as an expression
of `creation' and `annihilation' operators. These quantum mechanical operators
do not have classical analogues.
\item The modern method is algebraically easy but conceptually harder.
\end{enumerate}
\end{frame}

\begin{frame}
\frametitle{The solution}
\begin{enumerate}
\item The solution of \eqref{e7} is
\begin{equation}\label{e10}
\psi_n(\xi) = N_ne^{-\xi^2/2}H_n(\xi),
\end{equation}
where $N_n$ is the normalisation constant,
\begin{equation}\label{e11}
\xi = \sqrt{\frac{m\omega}{\hslash}}x.
\end{equation}
and $n = 0, 1, 2, \ldots.$.
\item Unsurprisingly, the energy cannot take any positive value but is 
restricted to
\begin{equation}\label{e12}
E_n = \left(n + \frac{1}{2}\right)\hslash\omega.
\end{equation}
\item The lowest energy state is $n = 0$ with energy $E_0 = \hslash\omega/2$.
\end{enumerate}
\end{frame}

\begin{frame}
\frametitle{Photons}
\begin{enumerate}
\item $E_{n+1} - E_n = \hslash\omega/2$ for all $n$. The energy levels are 
equally spaced.
\item Recall our discussion on blackbody radiation. The cavity of the blackbody
had standing waves of all possible wavelengths. A standing wave of a particular
wavelength corresponds to a single harmonic oscillator. If there are $N$ photons
of this wavelength then the corresponding oscillator has energy $(N + 1/2)
\hslash\omega$. It is said to have $N$ photons.
\item Electromagnetic radiation can always be written as a Fourier series of
pure sine waves. Each sine wave can be looked upon as an oscillator of that
frequency. 
\item An oscillator corresponds to the state of a photon; the state is described
in terms of the frequency. It \emph{does not} correspond to the photon. Its
energy state tells how many photons are there.
\end{enumerate}
\end{frame}

\begin{frame}
\frametitle{Phonons}
\begin{enumerate}
\item A crystal lattice is a set of ions/radicals bound together by chemical
bonds. The ions/radicals vibrate about their mean positions. The lattice is
thus a large number of coupled oscillators.
\item It is possible to consider them as an ensemble of uncoupled oscillators
of various frequencies. We once again have a large number of harmonic 
oscillators. 
\item If the harmonic oscillator of a frequency $\omega$ is in the energy state
$E_N = (N + 1/2)\hslash\omega$ then we say that there are $N$ phonons.
\item Phonons are quanta of acoustic radiation the way photons are quanta of 
electromagnetic radiation.
\item In either cases, the energy is exchanged only in quanta of $h\nu = 
\hslash\omega$.
\end{enumerate}
\end{frame}
\end{document}

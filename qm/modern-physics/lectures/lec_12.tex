\documentclass{beamer}
\usefonttheme[onlymath]{serif}
\usepackage{graphicx, hyperref}
\hypersetup{colorlinks, urlcolor=blue}
\usepackage{subcaption}
\usetheme{Boadilla}

\newcommand{\pvec}[1]{\vec{#1}\mkern2mu\vphantom{#1}}

\title{The Molecules}
\author{Amey Joshi}
\date{\today}

\begin{document}
\begin{frame}
\titlepage
\end{frame}

\begin{frame}
\begin{enumerate}
\frametitle{Many-electron systems}
\item The spin is a separate degree of freedom. The wavefunction $\psi$ depends
on $\vec{r}$ alone. We need another function $s$ to denote spin. It is a 2-row
column matrix in non-relativistic quantum mechanics. The complete, time 
independent wave function is, therefore, $\psi(\vec{r})s$.
\item When there are two electrons, the wavefunction of the system is 
\begin{equation}\label{e1}
\psi = \psi_a(\vec{r}_1)s_a\psi_b(\vec{r}_2)s_b,
\end{equation}
which expresses the fact that one electron is in the state `a' and the other in 
the state `b'.
\item A shorthand to the express the same thing is
\begin{equation}\label{e2}
\psi = \psi_a(1)\psi_b(2),
\end{equation}
where the arguments `1' and `2' include both space and spin degrees of freedom 
of an electron.
\end{enumerate}
\end{frame}

\begin{frame}
\frametitle{Indistingushable particles}
\begin{enumerate}
\item When the wavefunctions of two particles overlap, it is impossible to tell
one particle from the other. Therefore, the state $\psi_a(1)\psi_b(2)$ cannot be
told from $\psi_a(2)\psi_b(1)$. Such a state is therefore expressed as a linear
combination of the two,
\begin{equation}\label{e3}
\psi = \alpha\psi_a(1)\psi_b(2) + \beta\psi_a(2)\psi_b(1).
\end{equation}
\item It is a \emph{law of nature} that particles with integral spins have a 
symmetric wavefunction while those with half-integral spins have anti-symmetric
wavefunction.
\item A symmetric combination is
\begin{equation}\label{e4}
\psi = \psi_a(1)\psi_b(2) + \psi_a(2)\psi_b(1)
\end{equation}
while an anti-symmetric combination is
\begin{equation}\label{e5}
\psi = \psi_a(1)\psi_b(2) - \psi_a(2)\psi_b(1)
\end{equation}
\end{enumerate}
\end{frame}

\begin{frame}
\frametitle{Indistingushable particles}
\begin{enumerate}
\item Now let the two states coincide. Then $\psi = 2\psi_a(1)\psi_a(2)$ in the
symmetric case but $\psi = 0$ in the anti-symmetric case.
\item It is impossible for two particles with half-integral spins to be in the 
same state.
\item Recall that, in the context of the hydrogen atom, a stationary state of
and electron is completely satisfied by four quantum numbers. The previous point
therefore means that no two electron can have the same set of quantum numbers.
This is Pauli exclusion principle and it is an immediate consequence of the fact
that electrons have a spin $1/2$.
\item Indistingushable particles means that all permutations of the electrons 
are equivalent.
\end{enumerate}
\end{frame}

\begin{frame}
\frametitle{Odd and even permutations}
\begin{enumerate}
\item Every permutation can be achieved by a finite number of `exchanges' of the
objects.
\item The sign of a permutation is `$+$' if it needs and even number of `exchanges'
starting from the arrangement $\{1, \ldots, n\}$. Otherwise, the sign is `$-$'. 
\item If $p$ denotes a permutation of $\{1, \ldots, n\}$ then $p(1)$ is the 
position of $1$ in the rearrangement, $p(2)$ is the position of $2$ in the 
rearrangement, ...
\item If there are $n$ electrons then the anti-symmetric wavefunction of a system
of all of them is
\begin{equation}\label{e6}
\psi(1, \ldots, n) = \frac{1}{\sqrt{n!}}
 \sum_{p}\text{sgn}(p)\psi_{p(1)}(1)\ldots\psi_{p(n)}(n)
\end{equation}
Here the sum is over all partitions.
\end{enumerate}
\end{frame}

\begin{frame}
\frametitle{Bosons and Fermions}
\begin{enumerate}
\item The symmetry or anti-symmetry of wavefunctions has a profound implication.
\item Particles with spins $0, 1, \ldots$ are called Bosons. Examples are - a
photon, a phonon, Deuterium nuclues, $\alpha$-particle.
\item Particles with spins $1/2. 3/2, \ldots$ are called Fermions. Examples are
- an electron, a proton, a neutron, a neutrino.
\item Bosons and Fermions have drastically different statistical properties.
\end{enumerate}
\end{frame}
\end{document}

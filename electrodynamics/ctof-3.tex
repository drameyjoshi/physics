\chapter{Charges in Electromagnetic Fields}\label{c3}
\begin{enumerate}
\item Unless they are in touch with each other, particles interact with each
other via fields.

\item Disturbances in fields propagate information. It is a postulate of the 
special theory of relativity that the changes in the state of a system are
progagated at a finite speed not exceeding $c$, the speed of light in vacuum.

\item This restriction rejects the hypothesis of an ideal rigid body in relativity.
For suppose a force is applied to one end of a rigid body. Then in order that the
inter-particle distance is unchanged, the force has to be propagated instantaneously
to the other end. Therefore, in elementary analysis of the theory of relativity
all particles are assumed to be points.

\item The principle of action for a free particle is given by equation \eqref{c2e4}
and \eqref{c2e7},
\begin{equation}\label{c3e1}
S = -\int_{A}^{B}mcds,
\end{equation}
where $A$ and $B$ are world points.
For a particle in an electromagnetic field, the right hand side of this equation
should be augmented with another terms that represents the interaction of the
particle with the field. It should comprise of quantities describing the particle
as well as the field.

The particle's interaction with the field is described by a single quantity, its
charge $e$ and the field by its 4-potential $A_\mu$. The complete action is
\begin{equation}\label{c3e2}
S = -\int_A^B \left(mcds + \frac{e}{c}A_\mu dx^\mu\right).
\end{equation}

The 4-potential can be written as 
\begin{equation}\label{c3e3}
A_\mu = (\varphi, \vec{A})
\end{equation}
so that we can write \eqref{c3e2} as
\begin{equation}\label{c3e4}
S = \int_A^B\left(-mcds - e\varphi dt + 
\frac{e}{c}\vec{A}\cdot d\vec{x}\right).
\end{equation}
Since $ds = c\sqrt{1 - \beta^2}dt$ and $d\vec{x} = \vec{v}dt$,
\begin{equation}\label{c3e5}
S = \int_{t_1}^{t_2}\left(-mc^2\sqrt{1 - \beta^2} - e\varphi + 
\frac{e}{c}\vec{A}\cdot\vec{v}\right)dt.
\end{equation}
Thus, the Lagrangian for a particle in an electromagnetic field is
\begin{equation}\label{c3e6}
L = -mc^2\sqrt{1 - \beta^2} - e\varphi + \frac{e}{c}\vec{A}\cdot\vec{v}
= -mc^2\sqrt{1 - \beta^2} - e\varphi + \frac{e}{c}A^i v_i.
\end{equation}
From the Lagrangian, we get the generalised momentum
\begin{equation}\label{c3e7}
p^i = \pdt{L}{v^i} = m\gamma v^i + \frac{e}{c}A^i.
\end{equation}
From equation \eqref{c2e8}, the first term is just the generalised momentum for
the Lagrangian of \eqref{c2e7}. The Hamiltonian is given be the Legendre
transform
\begin{eqnarray*}
H &=& p^iv_i - L \\
  &=& m\gamma v^iv_i + \frac{e}{c}A^iv_i + mc^2\sqrt{1 - \beta^2} + e\varphi - 
  \frac{e}{c}A^iv_i \\
  &=& \frac{mv^2}{\sqrt{1 - \beta^2}} + mc^2\sqrt{1 - \beta^2} + e\varphi \\
  &=& \frac{mc^2}{\sqrt{1 - \beta^2}} + e\varphi
\end{eqnarray*}
or that
\begin{equation}\label{c3e8}
H = mc^2\gamma + e\varphi.
\end{equation}
Stricly speaking \eqref{c3e9} is not the correct form of the Hamiltonian because 
it is expressed in terms of the generalised velocity. To remedy this defect, we 
write
\eqref{c3e9} as
\[
\frac{(H - e\varphi)^2}{c^2} = m^2c^2 \gamma^2.
\]
Now,
\begin{equation}\label{c3e9}
\gamma^2 = \frac{1}{1 - \beta^2} = 1 + \frac{\beta^2}{1 - \beta^2} = 
1 + \beta^2\gamma^2
\end{equation}
so that
\[
\frac{(H - e\varphi)^2}{c^2} = m^2c^2(1 + \beta^2\gamma^2) 
\]
or that
\[
\frac{(H - e\varphi)^2}{c^2} = m^2c^2 + m^2\gamma^2 v^2,
\]
From \eqref{c3e7} the last term on the rhs can be written as
\begin{equation}\label{c3e10}
\frac{(H - e\varphi)^2}{c^2} = m^2c^2 + \left(p^i - \frac{e}{c}A^i\right)
\left(p_i - \frac{e}{c}A_i\right)
\end{equation}
Thus,
\begin{equation}\label{c3e11}
H = e\varphi + \sqrt{m^2c^4 + c^2 
\left(p^i - \frac{e}{c}A^i\right)\left(p_i - \frac{e}{c}A_i\right)}
\end{equation}
is the correct form of the Hamiltonian. For speeds small as compared to $c$,
\[
\sqrt{1 - \beta^2} = 1 - \frac{1}{2}\beta^2
\]
so that the Lagrangian of \eqref{c3e6} can be written as
\begin{equation}\label{c3e12}
L = \frac{1}{2}mv^2 - e\varphi + \frac{e}{c}\vec{A}\cdot\vec{v},
\end{equation}
where we ignored the constant terms $-mc^2$. At non-relativistic speeds, $\gamma$
can be well-approximated by $1$ so that the generalised momentum is
\begin{equation}\label{c3e13}
\vec{p} = m\vec{v} + \frac{e}{c}\vec{A}
\end{equation}
and the Hamiltonian is
\begin{equation}\label{c3e14}
H = \frac{1}{2m}\left(\vec{p} - \frac{e}{c}\vec{A}\right)^2 + e\varphi.
\end{equation}

From \eqref{c2e27},
\[
p^\nu = g^{\mu\nu}p_\mu = -g^{\mu\nu}\pdt{S}{x^\mu}
\]
so that
\[
p^0 = -\frac{1}{c}\pdt{S}{t}, \vec{p} = \grad S
\]
and hence the Hamilton-Jacobi equation is, from \eqref{c3e10},
\[
\frac{1}{c^2}\left(\pdt{S}{t} + e\varphi\right)^2 = m^2c^2 + 
\left(\grad S - \frac{e}{c}\vec{A}\right)^2
\]
where we replaced $H$ in \eqref{c3e10} with $-(1/c)\partial_t S$ because it is
the energy of the particle. We can reexpress the relativistic Hamilton-Jacobi
equation as 
\begin{equation}\label{c3e15}
\left(\grad S - \frac{e}{c}\vec{A}\right)^2 - 
\frac{1}{c^2}\left(\pdt{S}{t} + e\varphi\right)^2 + m^2c^2 = 0.
\end{equation}

\item The Lagrangian of \eqref{c3e12} allows us to derive the equation of motion.
\begin{eqnarray*}
\grad L &=& -e\grad\varphi + \frac{e}{c}(\vec{A}\cdot\grad\vec{v} + 
 \vec{v}\cdot\grad\vec{A} + \vec{A}\times\curl\vec{v} + \vec{v}\times\curl\vec{A})\\ 
 &=& -e\grad\varphi + \frac{e}{c}(\vec{v}\cdot\grad\vec{A} + \vec{v}\times\curl\vec{A})\\ 
\grad_{\vec{v}}L &=& m\vec{v} + \frac{e}{c}\vec{A}.
\end{eqnarray*}
The equation for $\grad L$ simplifies because $\vec{v}$ is independent of $\vec{x}$.
\begin{equation}\label{c3e16}
\frac{d}{dt}\left(m\vec{v} + \frac{e}{c}\vec{A}\right) = 
-e\grad\varphi + \frac{e}{c}(\vec{v}\cdot\grad\vec{A} + \vec{v}\times\curl\vec{A})
\end{equation}
is the equation of motion. Now.
\begin{equation}\label{c3e17}
\td{\vec{A}}{t} = \pdt{\vec{A}}{t} + \vec{v}\cdot\vec{A}
\end{equation}
so that \eqref{c3e16} is simplified to
\begin{equation}\label{c3e18}
\td{\vec{p}}{t} = e\left(-\grad\varphi - \frac{1}{c}\pdt{\vec{A}}{t}\right)
+ \frac{e}{c}\vec{v}\times\curl\vec{A}
\end{equation}
We define
\begin{eqnarray}
\vec{E} &=& -\grad\varphi - \frac{1}{c}\pdt{\vec{A}}{t} \label{c3e19} \\
\vec{B} &=& \curl\vec{A} \label{c3e20}
\end{eqnarray}
and call $\vec{E}$ as the electric field and $\vec{B}$ as the magnetic field. The
quantity
\begin{equation}\label{c3e21}
\vec{F} = e\vec{E} + \frac{\vec{v}}{c}\times\vec{B}
\end{equation}
is called the Lorentz force.

\item Recall that 
\begin{equation}\label{c3e22}
p^\mu = mcu^\mu = (mc\gamma, mv^i\gamma) = \left(\frac{\mathcal{E}}{c}, \vec{p}\right)
\end{equation}
so that the energy of the particle is
\begin{equation}\label{c3e23}
\mathcal{E} = mc^2\gamma
\end{equation}
so that
\begin{equation}\label{c3e24}
\td{\mathcal{E}}{t} = mc^2\td{\gamma}{t} = 
-\frac{mc^2}{2}\gamma^3\left(\frac{-2\vec{v}}{c^2}\cdot\td{\vec{v}}{t}\right)
= m\gamma^3 \vec{v}\cdot\td{\vec{v}}{t}
\end{equation}
and
\[
\td{\vec{p}}{t} = m\gamma\td{\vec{v}}{t} + m\gamma^3\frac{\vec{v}}{c^2}\vec{v}\cdot\td{\vec{v}}{t} = 
m\gamma^3\left(m\gamma^{-2}\td{\vec{v}}{t} + \frac{\vec{v}}{c^2}\vec{v}\cdot\td{\vec{v}}{t}\right)
\]
or, since
\[
\gamma = \frac{1}{\sqrt{1 - v^2/c^2}},
\]
we have,
\[
\td{\vec{p}}{t} = m\gamma^3\left(\left(1 - \frac{v^2}{c^2}\right)\td{\vec{v}}{t} + 
\frac{\vec{v}}{c^2}\vec{v}\cdot\td{\vec{v}}{t}\right)
\]
Therefore,
\begin{equation}\label{c3e25}
\vec{v}\cdot\td{\vec{p}}{t} = m\gamma^3\left(\left(1 - \frac{v^2}{c^2}\right)\vec{v}\cdot\td{\vec{v}}{t}
+ \frac{v^2}{c^2}\vec{v}\cdot\td{\vec{v}}{t}\right) = m\gamma^3\vec{v}\cdot\td{\vec{v}}{t}
\end{equation}
From equations \eqref{c3e24} and \eqref{c2e25},
\begin{equation}\label{c3e26}
\td{\mathcal{E}}{t} = \vec{v}\cdot\td{\vec{p}}{t}.
\end{equation}
If we apply it to the equation of motion \eqref{c3e18} and use the definitions 
of the fields, we get
\begin{equation}\label{c3e27}
\td{\mathcal{E}}{t} = e\vec{v}\cdot\vec{E}.
\end{equation}

Here we have used the fact that
\begin{equation}\label{c3e28}
\td{v^2}{t} = \td{(\vec{v}\cdot\vec{v})}{t^2} = 2\vec{v}\cdot\td{\vec{v}}{t}.
\end{equation}

\item The equation of motion
\begin{equation}\label{c3e29}
\td{\vec{p}}{t} = e\left(\vec{E} + \frac{\vec{v}}{c}\times\vec{B}\right)
\end{equation}
is invariant under time reversal if we also reverse the direction of the magnetic
field.

\item A 4-potential $A_\mu$ gives a unique combination of electric and magnetic
fields. But the relationship is many-to-one. This is because, the potential
\begin{equation}\label{c3e30}
A_\mu^\op = A_\mu - \pdt{f}{x^\mu}
\end{equation}
results in
\begin{eqnarray}
\varphi^\op &=& \varphi - \frac{1}{c}\pdt{f}{t} \label{c3e31} \\
\vec{A}^\op &=& \vec{A} + \grad f \label{c3e32}
\end{eqnarray}
and the fields
\begin{eqnarray}
\vec{E}^\op &=& -\grad{\varphi^\op} - \frac{1}{c}\pdt{\vec{A}^\op}{t} \nonumber \\
 &=& -\grad\varphi + \frac{1}{c}\grad\pdt{f}{t} - 
      \frac{1}{c}\pdt{\vec{A}}{t} - \frac{1}{c}\grad{\pdt{f}{t}} \nonumber \\
 &=& -\grad\varphi - \frac{1}{c}\pdt{\vec{A}}{t} \nonumber \\
 &=& \vec{E} \label{c3e33} \\
\vec{B}^\op &=& \curl\vec{A}^\op \nonumber \\
 &=& \curl\vec{A} \nonumber \\
 &=& \vec{B} \label{c3e34}
\end{eqnarray}
Equation \eqref{c3e30} is called the gauge transformation and an invariance of
functions under it is called gauge invariance. All physically meaningful quantities
must be gauge invariant.

\item If $A_\mu$ is independent of time then so are the fields. The electric field
depends only on $\varphi$. Further, since the Lagrangian is independent of time, the
energy of the system is a constant and it coincides with the Hamiltonian. From
\eqref{c3e8}, it is
\begin{equation}\label{c3e35}
\mathcal{E} (= H) = mc^2\gamma + e\varphi.
\end{equation} 
The effect of the fields is just to add a term $e\varphi$ to the energy. This term
is called the \emph{potential energy}. It is independent of $\vec{A}$ which means
that the magnetic field has no influence on the particle's energy.

If the fields are constant then one can easily verify that
\begin{eqnarray}
\varphi &=& -\vec{E}\cdot\vec{x} \label{c3e36} \\
\vec{A} &=& \frac{1}{2}\vec{B} \times \vec{x} \label{c3e37}
\end{eqnarray}

\item We will now consider the motion of a charge particle in a constant electric
field. From the equation of motion \eqref{c3e29},
\begin{equation}\label{c3e38}
\td{\vec{p}}{t} = e\vec{E}.
\end{equation}
If $\vec{E} = E\uv{x}$ then we have
\begin{eqnarray}
p_x(t) &=& p_x(0) + eEt \label{c3e39} \\
p_y(t) &=& p_y(0) \label{c3e40}.
\end{eqnarray}
Let us choose the zero of time at an instant when $p_x(0) = 0$ and $p_y(0) = p_0$,
the initial momentum of the particle. Since $\vec{p} = m\gamma\vec{v}$ and 
$\mathcal{E} = mc^2$, we have 
\begin{equation}\label{c3e41}
\vec{v} = \frac{\vec{p}c^2}{\mathcal{E}}
\end{equation}
so that
\begin{eqnarray}
\dot{x}(t) &=& \frac{ec^2Et}{\mathcal{E}} \label{c3e42} \\
\dot{y}(t) &=& p_0 \label{c3e43}
\end{eqnarray}
A particle in an electric field gains energy. Therefore, $\mathcal{E}$ is not a
constant and hence \eqref{c3e42} cannot be integrated as is. However, we know 
that $\mathcal{E}^2 = m^2c^4 + p^2c^2 = m^2c^4 + p_0^2c^2 + e^2E^2t^2c^2$ so that
equation \eqref{c3e42} becomes
\begin{equation}\label{c3e44}
\dot{x}(t) = \frac{ec^2Et}{\sqrt{m^2c^4 + p_0^2c^2 + c^2e^2E^2t^2}}.
\end{equation}
and hence
\[
x(t) = \int \frac{ec^2Et dt}{\sqrt{m^2c^4 + p_0^2c^2 + c^2e^2E^2t^2}} + x(0).
\]
If $ceEt = u, dt = du/(ceE)$ and
\[
x(t) = \frac{1}{eE}\int\frac{u du}{\sqrt{\mathcal{E}_0^2 + u^2}} + x(0)
\]
If $u = \mathcal{E}_0\sinh v$, $du = \mathcal{E}_0\cosh  vdv$ and
\begin{eqnarray*}
x(t) &=& \frac{1}{eE}\int\frac{\mathcal{E}_0^2\sinh v\cosh v dv}{\mathcal{E}_0\cosh v} + x(0) \\
 &=& \frac{\mathcal{E}_0}{eE}\int\sinh v dv + x(0) \\
 &=& \frac{\mathcal{E}_0}{eE}\cosh v + x(0) \\
 &=& \frac{1}{eE}\sqrt{\mathcal{E}_0^2 + u^2} + x(0)
\end{eqnarray*}
or
\begin{equation}\label{c3e45}
x(t) = \frac{1}{eE}\sqrt{\mathcal{E}_0^2 + e^2c^2E^2t^2} + x(0)
\end{equation}
Analogous to \eqref{c3e44}
\[
\dot{y}(t) = \frac{p_0c^2}{\sqrt{\mathcal{E}_0^2 + e^2E^2c^2t^2}}
\]
so that
\[
y(t) = \int \frac{p_0c^2}{\sqrt{\mathcal{E}_0^2 + e^2E^2c^2t^2}} dt + y(0).
\]
Put $ecEt = \mathcal{E}_0\sinh u$ so that $ecEdt = \mathcal{E}_0\cosh u du$
and
\[
y(t) = \frac{p_0c}{eE}\int \frac{\mathcal{E}_0\cosh u du}{\mathcal{E}_0\cosh u}
 + y(0) = \frac{p_0c}{eE} u + y(0) 
\]
that is
\begin{equation}\label{c3e46}
y(t) = \frac{p_0c}{eE} \sinh^{-1}\left(\frac{ecEt}{\mathcal{E}_0}\right) + y(0).
\end{equation}
We choose the origin such that $x(0) = 0, y(0) = 0$ so that from \eqref{c3e46}
we have
\[
ecEt = \mathcal{E}_0\sinh\left(\frac{eEy}{p_0c}\right)
\]
Substituting this in \eqref{c3e45} we get the equation of the orbit
\begin{equation}\label{c3e47}
x(t) = \frac{\mathcal{E}_0}{eE}\cosh\left(\frac{eEy}{p_0c}\right).
\end{equation}
It describes a catenary symmetric about the $x$-axis. When $eEy \ll p_0c$, we
can approximate \eqref{c3e47} as
\begin{equation}\label{c3e48}
x(t) = \frac{\mathcal{E}_0}{eE}\left(1 + \frac{1}{2}\frac{e^2E^2y}{p_0^2c^2}\right),
\end{equation}
which is an equation of a parabola.

\item When a charged particle is subjected to a constant magnetic field, the
equation of motion is
\begin{equation}\label{c2e49}
\td{\vec{p}}{t} = e\frac{\vec{v}}{c} \times \vec{B}.
\end{equation}
Once again, we use the fact that $\mathcal{E} = m\gamma c^2$ and $\vec{p} = 
m\gamma\vec{v}$ to get
\[
\vec{p} = \frac{\mathcal{E}}{c^2}\vec{v}
\]
so that \eqref{c2e48} becomes
\[
\td{\vec{v}}{t} = \frac{ec}{\mathcal{E}}\vec{v} \times \vec{B}.
\]
In a magnetic field, the energy of a particle remains unchanges. Therefore, we 
could pull $\mathcal{E}$ out of the derivative. Let us align the $z$ axis along
the magnetic field so that $\vec{B} = B\uv{z}$ and $\vec{v} \times \vec{B} = 
\uv{x}(v_yB - 0) + \uv{y}(-v_xB) + \uv{z}(0)$. Therefore,
\begin{eqnarray*}
\dot{v}_x(t) &=& \frac{ecB}{\mathcal{E}}v_y \\
\dot{v}_y(t) &=& -\frac{ecB}{\mathcal{E}}v_x \\
\dot{v}_z(t) &=& 0
\end{eqnarray*}
From these equations, we get
\begin{eqnarray*}
\ddot{v}_x(t) &=& -\omega^2 v_x \\
\ddot{v}_y(t) &=& -\omega^2 v_y \\
\ddot{v}_z(t) &=& 0
\end{eqnarray*}
where
\begin{equation}\label{c3e50}
\omega = \frac{ecB}{\mathcal{E}},
\end{equation}
is called the cyclotron frequency. At low speeds, $\gamma \approx 1$, $\mathcal{E}
= mc^2$ and the cyclotron frequency is $(eB)/(mc)$. Note that $\omega$ is a 
constant. We can write solutions to the differential equations of velocity
components as
\begin{eqnarray*}
v_x(t) &=& A_1\omega\cos\omega t + A_2\omega\sin\omega t \\
v_y(t) &=& A_3\omega\cos\omega t + A_4\omega\sin\omega t \\
v_z(t) &=& v_z(0)
\end{eqnarray*}
here $A_1, A_2, A_3, A_4$ are the constants of integration. If the initial
conditions are $v_x(0) = v_{\perp}, v_y(0) = 0$ then $A_1 = v_{\perp}$ and 
$A_3 = 0$ so that
\begin{eqnarray*}
v_x(t) &=& v_\perp\cos\omega t + A_2\omega\sin\omega t \\
v_y(t) &=& A_4\omega\sin\omega t \\
v_z(t) &=& v_z(0)
\end{eqnarray*}
from which we get
\begin{eqnarray*}
x(t) &=& \frac{v_\perp}{\omega}\sin\omega t + A_2\cos\omega t \\
y(t) &=& -A_4\cos\omega t \\
z(t) &=& z(0) + v_z(0)t
\end{eqnarray*}
If $x(0) = 0, y(0) = r$ then $A_2 = 0$ and $A_4 = -r$ so that
\begin{eqnarray}
x(t) &=& \frac{v_\perp}{\omega}\sin\omega t \label{c3e51} \\
y(t) &=& r\cos\omega t \label{c3e52} \\
z(t) &=& z(0) + v_z(0)(t) \label{c3e53}
\end{eqnarray}
The constants $v_\perp$ and $r$ are not unrelated to each other. Using them in
the equations of velocity components we get
\begin{eqnarray*}
v_x(t) &=& v_\perp\cos\omega t \\
v_y(t) &=& -r\omega\sin\omega t
\end{eqnarray*}
so that $\dot{v}_x(t) = v_\perp\omega\sin\omega t$. From the equation of motion
the right hand side is $\omega v_y = -r\omega^2\sin\omega t$ so that we get
\begin{equation}\label{c3e54}
v_\perp = -r\omega.
\end{equation}
The equations of the trajectory therefore simplify to
\begin{eqnarray}
x(t) &=& -r\sin\omega t \label{c3e55} \\
y(t) &=& r\cos\omega t \label{c3e56} \\
z(t) &=& z(0) + v_z(0)(t) \label{c3e57}
\end{eqnarray}
This is an equation of an helix.

\item Now consider a charged particle in the presence of a constant magnetic field
$\vec{B} = B\uv{z}$ and a constant electric field $\vec{E} = E_y\uv{y} + E_z\uv{z}$.
We restrict ourselves to non-relativistic regime so that $\vec{p} = m\vec{v}$ and
the equation of motion \eqref{c3e29} becomes
\begin{eqnarray}
m\dot{v}_x &=& \frac{e}{c}v_yB \label{c3e58} \\
m\dot{v}_y &=& eE_y - \frac{e}{c}v_xB \label{c3e59} \\
m\dot{v}_z &=& eE_z \label{c3e60}
\end{eqnarray}
The last equation immediately gives
\[
v_z = \frac{e}{m}E_zt + v_z(0)
\]
and
\begin{equation}\label{c3e61}
z(t) = \frac{e}{2m}E_zt^2 + v_z(0)t + z(0).
\end{equation}
In order to solve the coupled equations \eqref{c3e57} and \eqref{c3e58}, we can 
either differentiate them once more and decouple them or use a different trick.
We demonstrated the former in the previous point. Here, we will use the latter.
Let $\tilde{v} = v_x + iv_y$ so that
\[
m\td{\tilde{v}}{t} = \frac{e}{c}v_yB + ieE_y - \frac{e}{c}{iv_xB}
= ieE_y -i\frac{e}{c}(B(v_x + iv_y)) = ieE_y - i\frac{eB}{c}\tilde{v}
\]
or
\begin{equation}\label{c3e62}
\td{\tilde{v}}{t} = -i\frac{eB}{mc}\tilde{v} + i\frac{e}{m}E_y,
\end{equation}
It can be readily integrated to
\[
\ln\left(\tilde{v} - \frac{eE_y}{m\omega}\right) = -i\omega t + \ln\tilde{v}(0)
\]
or
\begin{equation}\label{c3e63}
\tilde{v} = \tilde{v}(0)e^{-i\omega t} + \frac{eE_y}{m\omega},
\end{equation}
where we approximated \eqref{c3e50} for non-relativistic speeds. Equating the
real and imaginary parts,
\begin{eqnarray}
v_x(t) &=& v_x(0)\cos\omega t  + \frac{e}{m\omega}E_y\label{c3e64} \\
v_y(t) &=& -v_y(0)\sin\omega t \label{c3e65}
\end{eqnarray}
From these equations, we readily get
\begin{eqnarray}
\langle v_x(t) \rangle &=& \frac{e}{m\omega}E_y \label{c3e66} \\
\langle v_y(t) \rangle &=& 0 \label{c3e67}
\end{eqnarray}
Thus, the motion remains bounded on the $y$ axis but not so on the other axes. 
Equations \eqref{c3e65} and \eqref{c3e66} can be readily integrated to
\begin{eqnarray}
x(t) &=& \frac{v_x(0)}{\omega}\sin\omega t + \frac{e}{m\omega}E_yt + x(0) \label{c3e68} \\
y(t) &=& \frac{v_y(0)}{\omega}\cos\omega t + y(0) \label{c3e69}
\end{eqnarray}
Equations \eqref{c3e61}, \eqref{c3e68} and \eqref{c3e69} are the equations of the
trajectory.

\item We derived the equations of motion in point (4) from the Lagrangian which
was obtained after writing the principle of least action as an integral over time.
We can, instead, apply the variational method to equation \eqref{c3e2}. Thus,
\begin{eqnarray*}
\delta S &=& -\delta\int_A^B\left(mcds + \frac{e}{c}A_\mu dx^\mu\right) \\
 &=& -\int_A^B\left(mc\delta ds + \frac{e}{c}\delta(A_\mu dx^\mu)\right) \\
 &=& -\int_A^B\left(mc\delta\sqrt{dx_\mu dx^\mu} + \frac{e}{c}(\delta(A_\mu) dx^\mu + A_\mu d\delta x^\mu)\right) \\
 &=& -\int_A^B\left(mc\frac{1}{2}\frac{dx_\mu d\delta x^\mu + d(\delta x_\mu)dx^\mu}{ds}
     + \frac{e}{c}(\delta(A_\mu) dx^\mu + A_\mu d\delta x^\mu)\right)
\end{eqnarray*}
Now $d(\delta x_\mu)dx^\mu = d(\delta x^\mu)dx_\mu = dx_\mu d(\delta x^\mu)$ so that
\begin{equation}\label{c3e70}
\delta S = -\int_A^B\left(mc\frac{dx_\mu d(\delta x^\mu)}{ds}
     + \frac{e}{c}(\delta(A_\mu) dx^\mu + A_\mu d\delta x^\mu)\right)
\end{equation}    
Using \eqref{c1e78},
\begin{equation}\label{c3e71}
\delta S = -\int_A^B\left((mcu_\mu + \frac{e}{c}A_\mu)d\delta x^\mu + \frac{e}{c}\delta(A_\mu) dx^\mu\right).
\end{equation}
We can integrate first term by parts,
\[
\delta S = -\left(mcu_\mu + \frac{e}{c}A_\mu\right)\delta x^\mu\big|_A^B + 
 \int_A^B \left(\delta x^\mu d\left(mcu_\mu + \frac{e}{c}A_\mu\right) - 
 \frac{e}{c}\delta(A_\mu) dx^\mu\right)
\]
Since the variations $\delta x^\mu$ vanish at the end points,
\[
\delta S = \int_A^B\left(\delta x^\mu\left(mcdu_\mu + \frac{e}{c}dA_\mu\right) - 
 \frac{e}{c}\delta(A_\mu) dx^\mu\right)
\]
We now write
\[
dA_\mu = \pdt{A_\mu}{x^\nu}dx^\nu;\; \delta A_\mu = \pdt{A_\mu}{x^\nu}\delta x^\nu dx^\mu
\]
so that
\begin{eqnarray*}
\delta S &=& \int_A^B\left(mcdu_\mu\delta x^\mu + \frac{e}{c}\pdt{A_\mu}{x^\nu}dx^\nu\delta x^\mu
 - \frac{e}{c}\pdt{A_\mu}{x^\nu}\delta x^\nu dx^\mu\right) \\
 &=& \int_A^B\left(mcdu_\mu\delta x^\mu + \frac{e}{c}\pdt{A_\mu}{x^\nu}dx^\nu\delta x^\mu
 - \frac{e}{c}\pdt{A_\nu}{x^\mu}\delta x^\mu dx^\nu\right) \\
 &=& \int_A^B\left(mcdu_\mu - \frac{e}{c}\left(\pdt{A_\nu}{x^\mu}
 - \pdt{A_\mu}{x^\nu}\right)dx^\nu\right)\delta x^\mu \\
 &=& \int_A^B\left(mc\td{u_\mu}{s}ds - \frac{e}{c}\left(\pdt{A_\nu}{x^\mu}
 - \pdt{A_\mu}{x^\nu}\right)u^\nu ds\right)\delta x^\mu \\
 &=& \int_A^B\left(mc\td{u_\mu}{s} - \frac{e}{c}\left(\pdt{A_\nu}{x^\mu}
 - \pdt{A_\mu}{x^\nu}\right)u^\nu \right)ds\delta x^\mu
\end{eqnarray*}
$\delta S = 0$ implies that
\[
\int_A^B\left(mc\td{u_\mu}{s} - \frac{e}{c}\left(\pdt{A_\nu}{x^\mu}
 - \pdt{A_\mu}{x^\nu}\right)u^\nu \right)ds\delta x^\mu = 0.
\]
Since this is true for all $\delta x^\mu = 0$,
\[
\int_A^B\left(mc\td{u_\mu}{s} - \frac{e}{c}\left(\pdt{A_\nu}{x^\mu}
 - \pdt{A_\mu}{x^\nu}\right)u^\nu \right)ds = 0.
\]
This integral is always zero, we must have
\begin{equation}\label{c3e72}
mc\td{u_\mu}{s} = \frac{e}{c}\left(\pdt{A_\nu}{x^\mu} - \pdt{A_\mu}{x^\nu}\right)u^\nu
\end{equation}
Define the electromagnetic field tensor
\begin{equation}\label{c3e73}
F_{\mu\nu} = \pdt{A_\nu}{x^\mu} - \pdt{A_\mu}{x^\nu}
\end{equation}
so that the equation of motion becomes
\[
mc\td{u_\mu}{s} = \frac{e}{c}F_{\mu\nu}u^\nu.
\]
equivalently,
\begin{equation}\label{c3e74}
mc\td{u^\mu}{s} = \frac{e}{c}F^{\mu\nu}u_\nu.
\end{equation}
Recall that $dx^\mu = (cdt, dx^1, dx^2, dx^3)$ so that $dx_\mu = (cdt, -dx^1, 
-dx^2, -dx^3)$. Further, $A_\mu = (\varphi, -A_1, -A_2, -A_3)$ so that the 
components of $F_{\mu\nu}$ are
\[
\begin{bmatrix}
0 & -c^{-1}\partial_t A_1 - \partial_{x^1}\varphi & -c^{-1}\partial_t A_2 - \partial_{x^2}\varphi & -c^{-1}\partial_t A_3 - \partial_{x^3}\varphi \\
c^{-1}\partial_t A_1 - \partial_{x^1}\varphi & 0 & -\partial_{x^1}A_2+\partial_{x^2}A^1 & -\partial_{x^1}A_3+\partial_{x^3}A^3\\
c^{-1}\partial_t A_2 - \partial_{x^2}\varphi & \partial_{x^1}A_2-\partial_{x^2}A_1 & 0 & -\partial_{x^2}A_3+\partial_{x^3}A_2\\
c^{-1}\partial_t A_3 - \partial_{x^3}\varphi & \partial_{x^1}A_3-\partial_{x^3}A_1 & \partial_{x^3}A_2-\partial_{x^3}A_2 & 0
\end{bmatrix}
\]
From equations \eqref{c3e19} and \eqref{c3e20},
\begin{equation}\label{c3e75}
A_{\mu\nu} = 
\begin{bmatrix}
0 & E_1 & E_2 & E_3 \\
-E_1 & 0 & -B_3 & B_2\\
-E_2 & B_3 & 0 & -B_1\\
-E_3 & -B_2 & B_1 & 0
\end{bmatrix}
\end{equation}
In order to transform this into a contravariant tensor, we use the equation
\begin{equation}\label{c3e76}
A^{\mu\nu} = g^{\mu\sigma}A_{\sigma\rho}g^{\rho\nu}.
\end{equation}
The factors are written in a manner that mimics the matrix multiplication. The matrix
representing the metric is $\text{diag}(1, -1, -1, -1)$ so that the result is
\begin{equation}\label{c3e77}
A^{\mu\nu} = 
\begin{bmatrix}
0 & -E_1 & -E_2 & -E_3 \\
E_1 & 0 & -B_3 & B_2\\
E_2 & B_3 & 0 & -B_1\\
E_3 & -B_2 & B_1 & 0
\end{bmatrix}
\end{equation}
Note that $E_i, B_i$ are not space components of a 4-vector. They should be treated
as scalars. That is $E_i = E^i$ and $B_i = B^i$.


\item Let us write \eqref{c3e74} in components form. Since
\[
u^\mu = \left(\gamma, -\frac{\gamma}{c}v^1, -\frac{\gamma}{c}v^2, -\frac{\gamma}{c}v^3\right),
\]
we have
\[
c\td{u^\mu}{s} = \left(c\td{\gamma}{s}, -\td{(\gamma v^1)}{s}, 
-\td{(\gamma v^2)}{s}, -\td{(\gamma v^3)}{s}\right)
\]
and
\[
u^\mu = \left(\gamma, \frac{\gamma}{c}v_1, \frac{\gamma}{c}v_2, \frac{\gamma}{c}v_3\right),
\]
and hence
\begin{eqnarray*}
mc\td{\gamma}{s} &=& \frac{e}{c}(F^{00}\gamma - F^{01}\gamma \beta_1 - F^{02}\gamma \beta_2 -F^{03}\gamma \beta_3) \\
-m\frac{d}{ds}(\gamma v^1)&=&\frac{e}{c}(F^{10}\gamma + F^{11}\gamma \beta_1 + F^{12}\gamma \beta_2 + F^{13}\gamma \beta_3) \\
-m\frac{d}{ds}(\gamma v^2)&=&\frac{e}{c}(F^{20}\gamma + F^{21}\gamma \beta_1 + F^{22}\gamma \beta_2 + F^{23}\gamma \beta_3) \\
-m\frac{d}{ds}(\gamma v^3)&=&\frac{e}{c}(F^{30}\gamma + F^{31}\gamma \beta_1 + F^{32}\gamma \beta_2 + F^{33}\gamma \beta_3) 
\end{eqnarray*}
Since $v^i = -v_i$ and $ds = cdt/\gamma$,
\begin{eqnarray*}
mc\td{\gamma}{t} &=& e(E_1 \beta_1 + E_2 \beta_2 + E_3 \beta_3) \\
m\frac{d}{dt}(\gamma v^1) &=& e(E_1 + B_3 \beta_2 - B_2 \beta_3) \\
m\frac{d}{dt}(\gamma v^2) &=& e(E_2 - B_3 \beta_1 + B_1 \beta_3) \\
m\frac{d}{dt}(\gamma v^3) &=& e(E_3 + B_2 \beta_1 - B_1 \beta_2).
\end{eqnarray*}
The last three equations can be written in vector form as
\[
\frac{d}{dt}(m\gamma\vec{v}) = \td{\vec{p}}{t} = e(\vec{E} + \vec{\beta} \times \vec{B})
\]
which is indeed the same as \eqref{c3e29}. The first equation can be written as
\[
mc^2\td{\gamma}{t} = \frac{d}{dt}(mc^2\gamma) = \td{\mathcal{E}}{t} = e\vec{E}\cdot\vec{v}
\]
This is the ``relativistic'' form of the work-energy theorem of \eqref{c3e27}.

\item From the equation of motion \eqref{c3e74} we also have
\[
mcu_\mu\td{u^\mu}{s} = \frac{e}{c}u_\mu F^{\mu\nu}u_\nu.
\]
The left hand side is zero because of \eqref{c1e86} while the right hand side 
vanishes because of anti-symmetry of the electromagnetic field tensor. This means
that the four equations of motion in the previous point are not independent.

\item We go back to equation \eqref{c3e71}
\[
\delta S = -\int_A^B\left((mcu_\mu + \frac{e}{c}A_\mu)d\delta x^\mu + \frac{e}{c}\delta(A_\mu) dx^\mu\right).
\]
and its immediate consequence,
\[
\delta S = -\left(mcu_\mu + \frac{e}{c}A_\mu\right)\delta x^\mu\big|_A^B + 
 \int_A^B \left(\delta x^\mu d\left(mcu_\mu + \frac{e}{c}A_\mu\right) - 
 \frac{e}{c}\delta(A_\mu) dx^\mu\right)
\]
The first term is a difference of a quantity at two fixed points while the second
one is a variation over all paths between those points. Since the variations 
vanish at the fixed points the first term is zero while the second term, after some
manipulations, gives the equations of motion.

If we consider only the actual paths starting from the world-point $A$ and let $B$
be any other point on it, then the second term will necessarily be zero while the 
first one will not be zero. The equation then gives,
\[
\frac{\delta S}{\delta x^\mu} = -\left(mcu_\mu + \frac{e}{c}A_\mu\right).
\]
In the limit $\delta x^\mu \rightarrow 0$ this equation becomes
\begin{equation}\label{c3e78}
-\pdt{S}{x^\mu} = mcu_\mu + \frac{e}{c}A_\mu.
\end{equation}
From equation \eqref{c2e26}, the left hand side is the generalised momentum
\begin{equation}\label{c3e79}
p_\mu = mcu_\mu + \frac{e}{c}A_\mu.
\end{equation}
This agrees with \eqref{c3e7}.

\item From equation \eqref{c1e46}, the Lorentz transformation of the 4-potential
gives
\begin{equation}\label{c3e80}
\varphi = \gamma(\bar{\varphi} + \beta\bar{A}_1);
\; A_1 = \gamma(\bar{A}_1 + \beta\bar{\varphi}); A_2 = \bar{A}_2;\; A_3 = \bar{A}_3.
\end{equation}
Using the solution to problem 2 in chapter \ref{c1}, we have $F^{01} = 
\bar{F}^{01}$ (from \eqref{c1e98}) so that 
\begin{equation*}
E_1 = \bar{E}_1.
\end{equation*}
From \eqref{c1e90}
\[
F^{02} = \frac{\bar{F}^{02} + \beta\bar{F}^{12}}{1 - \beta^2}
\]
so that
\begin{equation*}
E_2 = \frac{\bar{E}_2 + \beta\bar{B}_3}{1 - \beta^2}.
\end{equation*}
From \eqref{c1e91}
\[
F^{03} = \frac{\bar{F}^{03} + \beta\bar{F}^{13}}{1 - \beta^2}
\]
so that
\begin{equation*}
E_3 = \frac{\bar{E}_3 - \beta\bar{B}_2}{1 - \beta^2}.
\end{equation*}
From \eqref{c1e93}
\[
F^{12} = \frac{\bar{F}^{12} + \beta\bar{F}^{02}}{1 - \beta^2}
\]
so that
\begin{equation*}
-B_3 = \frac{-\bar{B}_3 - \beta\bar{E}_2}{1 - \beta^2} \Rightarrow 
B_3 = \frac{\bar{B}_3 + \beta\bar{E}_2}{1 - \beta^2}
\end{equation*}
From \eqref{c1e94}
\[
F^{13} = \frac{\bar{F}^{13} + \beta\bar{F}^{03}}{1 - \beta^2}
\]
so that
\begin{equation*}
B_2 = \frac{\bar{B}_2 - \beta\bar{E}_3}{1 - \beta^2}.
\end{equation*}
Finally, $F^{23} = \bar{F}^{23}$ implies
\begin{equation*}
B_1 = \bar{B}_1.
\end{equation*}
We collect these formulae together as
\begin{eqnarray}
E_1 &=& \bar{E}_1 \\ \label{c3e81}
E_2 &=& \frac{\bar{E}_2 + \beta\bar{B}_3}{1 - \beta^2} \\ \label{c3e82}
E_3 &=& \frac{\bar{E}_3 - \beta\bar{B}_2}{1 - \beta^2} \\ \label{c3e83}
B_1 &=& \bar{B}_1 \\ \label{c3e84}
B_2 &=& \frac{\bar{B}_2 - \beta\bar{E}_3}{1 - \beta^2} \\ \label{c3e85}
B_3 &=& \frac{\bar{B}_3 + \beta\bar{E}_2}{1 - \beta^2} \label{c3e86}
\end{eqnarray}

\item At low speeds, $1 - \beta^2 \approx 1$ and the transformation equations
become
\begin{eqnarray}
E_1 &=& \bar{E}_1 \\ \label{c3e87}
E_2 &=& \bar{E}_2 + \beta\bar{B}_3 \\ \label{c3e88}
E_3 &=& \bar{E}_3 - \beta\bar{B}_2 \\ \label{c3e89}
B_1 &=& \bar{B}_1 \\ \label{c3e90}
B_2 &=& \bar{B}_2 - \beta\bar{E}_3 \\ \label{c3e92}
B_3 &=& \bar{B}_3 + \beta\bar{E}_2 \label{c3e92}
\end{eqnarray}
Since $\beta = v/c$ or really $v_1/c$, we can write these equations in vector form as
\begin{eqnarray}
\vec{E} = \vec{E}^\op + \frac{1}{c}\vec{B}^\op \times \vec{v} \label{c3e93} \\
\vec{B} = \vec{B}^\op - \frac{1}{c}\vec{E}^\op \times \vec{v}, \label{c3e94}
\end{eqnarray}
where $\vec{E}^\op = (\bar{E}_1, \bar{E}_2, \bar{E}_3)$ and $\vec{B}^\op = 
(\bar{B}_1, \bar{B}_2, \bar{B}_3)$, These equations tell that even if $\vec{B}^\op
= 0$ ($\vec{E}^\op = 0$), $\vec{B}$ ($\vec{E}$) is not. Further, it is clear that
$\vec{E}$ is perpendicular to $\vec{B}$ in both cases.

\item From equations \eqref{c3e93} and \eqref{c3e94}, we have
\begin{equation}\label{c3e95}
\vec{E}\cdot\vec{B} = (1 + \beta^2)\vec{E}^\op\cdot\vec{B}^\op - 
(\vec{\beta}\cdot\vec{E}^\op)(\vec{\beta}\cdot\vec{B}^\op)
\end{equation}
\end{enumerate}


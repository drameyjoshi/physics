\documentclass{beamer}
\usefonttheme[onlymath]{serif}
\usepackage{graphicx, hyperref}
\hypersetup{colorlinks, urlcolor=blue}
\usepackage{subcaption}
\usetheme{Boadilla}

\newcommand{\pvec}[1]{\vec{#1}\mkern2mu\vphantom{#1}}

\title{Quantum Mechanics}
\author{Amey Joshi}
\date{\today}

\begin{document}
\begin{frame}
\titlepage
\end{frame}

\begin{frame}
\frametitle{Two formulations of Quantum Mechanics}
\begin{enumerate}
\item Werner Heisenberg gave the first formulation of a new mechanics in 1925 
for electrons in which dynamical variables were represented as matrices. It is 
called matrix mechanics.
\item Erwin Schr\"{o}dinger gave an alternative formulation in 1926 in which
the dynamical variables were represented as operators acting on a `wave 
function'. The `wave function' itself obeyed a `wave equation'. It is called
wave mechanics. He further showed that matrix mechanics and wave mechanics are
equivalent.
\item We will study wave mechanics, that is, the wave equation, the wave 
function and its consequences. For most practical calculations, wave mechanics
is preferred over matrix mechanics.
\item We will restrict ourselves to non-relativitistic quantum mechanics.
\end{enumerate}
\end{frame}

\begin{frame}
\frametitle{Schr\"{o}dinger equation - 1}
\begin{enumerate}
\item Matrix mechanics is largely a flash of insight; it is hard to imagine how
Heisenberg must have thought about it. Schr\"{o}dinger, on the other hand, gave
a plausible argument for his equation. Yet, we will take his equation as our
postulate.
\item The `unknown' in his equation is the wave function $\Psi(\vec{r}, t)$.
\item The equation is
\begin{equation}\label{e1}
-\frac{\hslash^2}{2m}\nabla^2\Psi + V(\vec{t},t)\Psi(\vec{r}, t) = 
i\hslash\frac{\partial\Psi}{\partial t},
\end{equation}
where
\begin{equation}\label{e2}
\nabla^2 = \frac{\partial^2}{\partial x^2} + \frac{\partial^2}{\partial y^2} + 
\frac{\partial^2\Psi}{\partial z^2} 
\end{equation}
is called the Laplacian operator.
\end{enumerate}
\end{frame}

\begin{frame}
\frametitle{Schr\"{o}dinger equation - 2}
\begin{enumerate}
\item It is a linear partial differential equation. Linear means that if 
$\Psi_1$ and $\Psi_2$ are any two solutions then so is $\alpha_1\Psi_1 + 
\alpha_2\Psi_2$, where $\alpha_1, \alpha_2$ are complex constants.
\item Note the presence if $i$ on the right hand side. In general, a solution
of the Schr\"{o}dinger equation will be a complex-valued function.
\item The equation in the form \eqref{e1} describes the time evolution of a 
single particle system. When there a more particles, it is a bit different.
\item Neither does the equation fit in the framework of the special theory of
relativity. Yet, it suffices to a very large extent in molecular and solid
state physics.
\item The equation is usually solved using numerical techniques although exact, 
analytical solutions can be found for a few simple situations.
\end{enumerate}
\end{frame}

\begin{frame}
\frametitle{The wavefunction}
\begin{enumerate}
\item In equation \eqref{e1}, $V$ is the potential energy. This is the only term
that changes from problem to problem.
\item The solution $\Psi$ is expected to have a few properties like continuity,
continuity of the first derivative and going to zero at infinity.
\item The entire physics of the system is in the wave function. If you know it,
you can calculate anything that can be measured about the system.
\item Recall that $|\Psi(\vec{r}, t)|^2dV$ is the probability of finding the
particle in a small volume element $dV$ around the point $\vec{r}$. In order
to make this happen
\begin{equation}\label{e3}
\int_{-\infty}^\infty |\Psi|^2dV = 1
\end{equation}
must always be true. Supposing the right hand side was $N$ then use the wave
function $\Psi/\sqrt{N}$. This is called a `normalised' wavefunction.
\item We will always assume a normalised wavefunction.
\end{enumerate}
\end{frame}

\begin{frame}
\frametitle{Operators - 1}
\begin{enumerate}
\item An operator is something that takes a function and produces another one.
If $\hat{O}$ is an operator then $\hat{O}f = g$, where $f$ and $g$ are 
functions.
\item If $f = \Psi(\vec{r},t)$, a few operators are
\begin{eqnarray}
\hat{O} &=& \alpha \in \mathbb{C} \label{e4} \\
\hat{O} &=& x \label{e5} \\
\hat{O} &=& \vec{r} \label{e6} \\
\hat{O} &=& \frac{\partial}{\partial x} \label{e7} \\
\hat{O} &=& -i\hslash\nabla \label{e8} \\
\hat{O} &=& i\hslash\frac{\partial}{\partial t} \label{e9} \\
\hat{O} &=& -\frac{\hslash^2}{2m}\nabla^2 \label{e10}
\end{eqnarray}
\end{enumerate}
\end{frame}

\begin{frame}
\frametitle{Operators - 2}
\begin{enumerate}
\item Dynamical variables are represented as operators in quantum mechanics.
Many operators in equations \eqref{e4} to \eqref{e10} represent important 
dynamical variables.
\item Do you understand the term dynamical variables?
\item An equation of the form
\begin{equation}\label{e11}
\hat{O} f = \lambda f, \lambda \in \mathbb{C}
\end{equation}
is called an eigenvalue equation. $\lambda$ is the eigenvalue and $f$ is the
eigenfunction. In general, there is a set (finite, countable or uncountable) of
$\lambda$s.
\item If $\hat{O}$ is a dynamical variable and if it satisfies \eqref{e11} then
an experiment measuring the dynamical variable will record one of the 
eigenvalues and nothing else.
\item Operators and their eigenvalues have a great amount of physics in them.
\end{enumerate}
\end{frame}

\begin{frame}
\frametitle{Operators - 3}
\begin{enumerate}
\item Position is represented by $\vec{r}$, momentum by $-i\hslash\nabla$,
energy by $i\hslash\partial/\partial t$.
\item The operator representation of potential energy is same as itself.
\item Kinetic energy is $T = p^2/2m$. Therefore, the corresponding operator is
\begin{equation}\label{e12}
\hat{T} = -\frac{\hslash^2}{2m}\nabla^2.
\end{equation}
\item The fact that total energy is the sum of kinetic and potential energies,
that is $E = T + V$ is represented as
\begin{equation}\label{e13}
i\hslash\frac{\partial}{\partial t} = -\frac{\hslash^2}{2m}\nabla^2 + V.
\end{equation} 
\item The operator $\hat{T} + \hat{V}$ is denoted by $\hat{H}$, the Hamiltonian
operator. Equation \eqref{e14} thus is
\begin{equation}\label{e14}
i\hslash\frac{\partial\Psi}{\partial t} = \hat{H}\Psi = 
-\frac{\hslash^2}{2m}\nabla^2\Psi + V\Psi.
\end{equation}
Does this look familiar? 
\end{enumerate}
\end{frame}

\begin{frame}
\frametitle{Operators - 4}
\begin{enumerate}
\item Operators do not always commute. $\hat{A}\hat{B} f \ne \hat{B}\hat{A} f$.
\item The commutator is defined as $[\hat{A}, \hat{B}] = \hat{A}\hat{B} -
\hat{B}\hat{A}$.
\item Does the remind you of another mathematical object? Does it give you a
hint of why Schr\"{o}dinger's and Heisenberg's formulations could be equivalent?
\item The lack of commutativity has a very important relation with the 
simultaneous measurement of the dynamical variables.
\item This is another way to understand uncertainty relations.
\end{enumerate}
\end{frame}
\end{document}


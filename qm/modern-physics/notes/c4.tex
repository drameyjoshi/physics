\chapter{Schr\"{o}dinger Wave Equation}\label{c4}
Bohr's model explained the spectrum of the Hydrogen atom quite well. But not
too well. When a Hydrogen atom is subjected to an electric field, the spectral
lines split. This is called the `Stark effect' after the name of its discoverer.
Similarly, `Zeeman effect' is the splitting of spectral lines when the atom is
subjected to a magnetic field. By splitting we mean that in the place of a 
single line in the absence of a field we now see more than one. A spectral line
indicates the energy level of an electron. A splitting of a spectral line tells
us that electrons which used to have the same energies in the absence of a 
field have different energies when the field is turned on. The Bohr model 
could not explain why this happens. Arnold Sommerfeld tried to extend the Bohr
model by allowing the orbits to be elliptical and by using certain concepts in
advanced classical mechanics. However, it was clear that the theory was getting
\emph{ad hoc} and that a better explanation was needed.

Werner Heisenberg, a German physicist, first came up with a new theory to
explain quantum phenomena \cite{heisenberg1925quantum}. His paper is quite
hard to follow. People of Steven Weinberg's \cite{aitchison2004understanding}
stature have admitted that while they understand what Heisenberg wrote they 
cannot fathom what inspired him to write it. He also found out that the 
equations of quantum mechanics are best written in terms of matrices. As a 
result, his formulation of quantum mechanics is called `matrix mechanics'.

A year later, the Austrian physicist Erwin Schr\"{o}dinger proposed an 
alternative formalism to solve atomic problems. His `equation of motion' was 
conceived using certain optical analogies and is called the wave equation. A
function satisfying the equation is called the wave function and it describes
the matter waves we studied in chapter \ref{c2}. Schr\"{o}dinger's paper
\cite{schrodinger1926undulatory} is remarkably lucid and it offers an insight
into how he came up with his equation. The Schr\"{o}dinger wave equation
is a linear partial differential equation and the physicists of 
Schr\"{o}dinger' time had a expertise and comfort in dealing with them. As a 
result, a first introduction to quantum mechanics is almost always using the
Schr\"{o}dinger equation. Schr\"{o}dinger showed \cite{schrodinger1926relation}
that his approach is equivalent to the one proposed by Heisenberg and developed
by Born and Jordan. The equivalence was put on a firm mathematical footing by
John von Neumann \cite{von2018mathematical}.

In this chapter, we will state Schr\"{o}dinger's equation, interpret its
solution and show how to use it in a few very simple situations. Schr\"{o}dinger
derived \cite{schrodinger1926undulatory} his equation using variational 
principles. However, we shall take it as a starting point of our discussion.
The way we accepted Newton's laws as `correct' because they explained a few
natural phenomena, we will also accept Schr\"{o}dinger's equation because it
explains a lot many phenomena in the atomic world.

\section{Schr\"{o}dinger equation}\label{c4s1}
Before stating and understanding Schr\"{o}dinger's equation that describes
the unfamiliar matter waves, let us consider start in the more familiar world
of sound waves. We know that sound waves travel through the air as a succession
of compressions and rarefactions. We can detect and measure the compression and
rarefaction using a pressure gauge. If $p(x, t)$ is the air pressure at a point
$x$ and at time $t$ then the equation of sound waves is
\begin{equation}\label{c4s1e1}
\frac{\partial^2 p}{\partial x^2} = \frac{1}{c_s^2}
\frac{\partial^2 p}{\partial t^2}
\end{equation}
where $c_s$ is the speed of sound. It is easy to check that any function of 
the form $p(x \pm c_st)$ is a solution of \eqref{c4s1e1}. In particular,
\begin{equation}\label{c4s1e2}
p(x, t) = \cos(x - c_st)
\end{equation}
is a solution. A solution of a wave equation tells us how pressure varys with
position and time. 

Equation \eqref{c4s1e1} described waves in one dimension. If the pressure varies
in all three dimensions then the equation becomes
\begin{equation}\label{c4s1e3}
\frac{\partial^2 p}{\partial x^2} + \frac{\partial^2 p}{\partial y^2} + 
\frac{\partial^2 p}{\partial z^2} = 
\frac{1}{c_s^2} \frac{\partial^2 p}{\partial t^2}
\end{equation}
We can check that
\begin{equation}\label{c4s1e4}
p(x, y, z, t) = \cos(x + y + z - \sqrt{3} c_st)
\end{equation}
is a solution of \eqref{c4s1e4}.

\subsection{Problem set 1}
\begin{enumerate}
\item Verify that \eqref{c4s1e2} is a solution of \eqref{c4s1e1}.
\item Verify that \eqref{c4s1e4} is a solution of \eqref{c4s1e3}.
\item If $p_1(x, t)$ and $p_2(x, t)$ are two solutions of \eqref{c4s1e1} then
show that $\alpha_1 p_1 + \alpha_2 p_2$, where $\alpha_1, \alpha_2$ are real
numbers is also a solution of \eqref{c4s1e1}.
This fact is called the principle of linear superposition. The expression, 
$\alpha_1 p_1 + \alpha_2 p_2$ is called a \emph{linear combination} of $p_1$
and $p_2$.
\end{enumerate}

A sound wave is described by the pressure fluctations it causes. de Broglie
waves are described by the variations of a wave function $\Psi$. It could be a
complex-valued function and therefore it does not have a physical interpretation
by itself. However, its squared modulus $|\Psi|^2 = \Psi\Psi^\ast$ is the 
probability density of finding at a particle at a point $(x, y, z)$ at time $t$.In particular, $|\Psi|^2 dV$ is the probability of finding a particle in a 
small volume element $dV$ around the point $(x,y,z)$ at time $t$. The way the
air pressure $p$ satisfies the equation \eqref{c4s1e3}, the wave function 
$\Psi$ satisfies 
\begin{equation}\label{c4s1e5}
-\frac{\hslash^2}{2m}\left(\frac{\partial^2\Psi}{\partial x^2} + 
\frac{\partial^2\Psi}{\partial y^2} + \frac{\partial^2\Psi}{\partial z^2}
\right) + V(x,y,z,t)\Psi = -i\hslash\frac{\partial\Psi}{\partial t}.
\end{equation}
This is called the Schr\"{o}dinger equation. In this equation,
\begin{enumerate}
\item $\hslash = h/(2\pi)$; we first introduced this quantity when we wrote
\eqref{c3s3e2}.
\item $m$ is the mass of the particle.
\item $V(x, y, z, t)$ is the potential function. It describes the force acting
on the particle. We will describe it more a little later.
\item $i$ is the imaginary number.
\end{enumerate}
A few remarks are in order.
\begin{enumerate}
\item A wave equation always has a second order partial derivative in time.
Equation \eqref{c4s1e5} has just the first order partial derivative in time.
Therefore, strictly speaking, \eqref{c4s1e5} is not a wave equation although
the name `wave function' persists for its solution $\Psi$. It is closer to the
heat equation than it is to a wave equation.
\item The coefficient on the right hand side is an imaginary number. Therefore,
it is not surprising that its solution $\Psi$ is complex valued.
\item The only term in \eqref{c4s1e5} that is specific to a problem is the
potential function $V$. The other terms never change.
\item We remarked that $|\Psi|^2$ has an interpretation of a probability
density function. Therefore, if we want to find the probability of finding
a particle a region of volume $V$ then it is
\begin{equation}\label{c4s1e6}
P(V) = \int_V |\Psi|^2 dV.
\end{equation}
This statement is accurate only when 
\begin{equation}\label{c4s1e7}
\int_{-\infty}^\infty |\Psi|^2dV = 1.
\end{equation}
Such a function is said to be `normalised'. If a wave function is not normalised
, that is, if
\begin{equation}\label{c4s1e8}
\int_{-\infty}^\infty |\Psi|^2dV = N,
\end{equation}
then we can alway consider the function
\begin{equation}\label{c4s1e9}
\Psi_1 = \frac{1}{\sqrt{N}}\Psi.
\end{equation}
We will show in the next problem set that $\Psi_1$ is also a solution of 
\eqref{c4s1e5}. $\Psi_1$ is called the \emph{normalised} wave function and
the number $1/\sqrt{N}$ is called the \emph{normalisation constant}.
\end{enumerate}

\subsection{Problem set 2}
\begin{enumerate}
\item Show that is $\Psi_1$ and $\Psi_2$ are two solutions of \eqref{c4s1e5} 
then $\alpha_1\Psi_1 + \alpha_2\Psi_2$, where $\alpha_1$ and $\alpha_2$ are
complex numbers, is also a solution. Therefore, show that \eqref{c4s1e9} is
also a solution of \eqref{c4s1e5}.
\end{enumerate}

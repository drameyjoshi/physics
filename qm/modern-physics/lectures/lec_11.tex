\documentclass{beamer}
\usefonttheme[onlymath]{serif}
\usepackage{graphicx, hyperref}
\hypersetup{colorlinks, urlcolor=blue}
\usepackage{subcaption}
\usetheme{Boadilla}

\newcommand{\pvec}[1]{\vec{#1}\mkern2mu\vphantom{#1}}

\title{The Atomic Spectra}
\author{Amey Joshi}
\date{\today}

\begin{document}
\begin{frame}
\titlepage
\end{frame}

\begin{frame}
\frametitle{Radiative transitions}
\begin{enumerate}
\item The eigenfunctions of the hydrogen's Hamiltonian are stationary states.
They do not radiate.
\item Consider a transition between state $\psi_1$ with energy $E_1$ to another
state $\psi_2$ with energy $E_2$. During the transition, the electron is in the
superposed state $\psi = \alpha\psi_1 + \beta\psi_2$.
\item The time dependent wavefunction is $\Psi = \alpha\psi_1 e^{-iE_1t/\hslash}
+ \beta\psi_2 e^{-iE_2t/\hslash}$
\item The expectation value of the position in the superposed state is
\begin{equation}\label{e1}
\langle\vec{r}\rangle = \int_{-\infty}^\infty \Psi\vec{r}\Psi^\ast dV.
\end{equation}
\item As $\psi_1$ and $\psi_2$ are normalised,
\begin{eqnarray}
\langle\vec{r}\rangle &=& |\alpha|^2\int_{-\infty}^{\infty}\vec{r}|\psi_1|^2dV +
|\beta|^2\int_{-\infty}^{\infty}\vec{r}|\psi_2|^2dV + \nonumber \\
 & & 2\Re\left[\alpha\beta^\ast
\int_{-\infty}^\infty\vec{r}\psi_1^\ast\psi_2 dV\right]e^{-i(E_1 - E_2)t/\hslash} 
\label{e2}.
\end{eqnarray}
\end{enumerate}
\end{frame}

\begin{frame}
\frametitle{Radiative transitions}
\begin{enumerate}
\item This is an oscillatory motion with frequency $\omega = (E_1-E_2)/\hslash$.
\item The energy is radiated as a photon of energy $E_1 - E_2$.
\item This argument is semi-classical. The assumption that a charged particle
oscillating at frequency $\omega$ emits radiation of the same frequency follows
from classical electrodynamics.
\item Quantum electrodynamics provides the full explanation of radiative 
transitions.
\item The amplitude of the emitted radiation depends on the factor
\begin{equation}\label{e3}
\int_{-\infty}^\infty\psi_1^\ast\psi_2 dV.
\end{equation}
\item It is zero unless $\Delta l = \pm 1$ and $\Delta m_l = 0, \pm 1$. These
equations are called \emph{selection rules}.
\end{enumerate}
\end{frame}

\begin{frame}
\frametitle{Radiative transitions}
\begin{enumerate}
\item Transitions between states that do not follow the selection rules are 
forbidden.
\item The rule $\delta l = \pm 1$ follows from the conservation of angular
momentum. The change in electron's angular momentum is accounted by the
emitted photon's angular momentum.
\item Electronic transition that are analysed by considering the integral in
equation \eqref{e2} are called electric dipole transitions. The selection rules
$\Delta l = \pm 1$ and $\Delta m_l = 0, \pm 1$ are applicable only to these 
transitions.
\item There are magnetic dipole, electric and magnetic quadrupole, electric
and magnetic octupole transitions as well. They have their own selection rules.
\item We have had just a glimpse of the fascinating subject of spectroscopy - an
extremely valuable experimental tool to characterize materials.
\end{enumerate}
\end{frame}

\begin{frame}
\frametitle{The spin}
\begin{enumerate}
\item Consider a problem in classical physics. A point charge $q$ is moving in a
circle of radius $r$ with a frequency $f$. It makes a current $I = qf$.
\item A current carrying loop has a magnetic moment $\mu = IA$ where $A$ is the
area of the loop. In this case, $\mu = \pi qfr^2$ or
\begin{equation}\label{e4}
\mu = \frac{q}{2}\omega r^2.
\end{equation}
\item If the mass of the charged particle is $m$ then its angular momentum is
\begin{equation}\label{e5}
L = m\omega r^2.
\end{equation}
\item From equations \eqref{e4} and \eqref{e5} it is clear that $\mu=\gamma L$,
where
\begin{equation}\label{e6}
\gamma = \frac{q}{2m}
\end{equation}
is called the gyromagnetic ratio of the charged particle.
\end{enumerate}
\end{frame}

\begin{frame}
\frametitle{The spin}
\begin{enumerate}
\item There is a close relationship between magnetic moment and angular momentum
in classical physics. It is much deeper in quantum physics.
\item Some particles have an intrinsic magnetic moment. It is their fundamental
property the way their mass and electric charge are.
\item Particles with an intrinsic magnetic moment can be considered to also have
an intrinsic angular momentum. Intrinsic angular momentum is called the spin.
\item The quantum mechanical generalisation of the classical relation $\mu = 
\gamma L$ is
\begin{equation}\label{e7}
\mu = g\gamma S,
\end{equation}
where $\gamma$ is given by \eqref{e3} and the constant $g$ is called the $g$-
factor. It is (very close to) $2$ for an electron. $S$ is the spin angular 
momentum.
\end{enumerate}
\end{frame}

\begin{frame}
\frametitle{The spin}
\begin{enumerate}
\item Uncharged particles like the photon and the neutrino also have a spin 
without having a magnetic moment.
\item The classical relation $\mu = \gamma L$ is valid for orbital angular 
momentum. Note that magnetic moment and angular momentum are both vectors.
Therefore, we should be using
\begin{eqnarray}
\vec{\mu} &=& \gamma\vec{L} \label{e8} \\
\vec{\mu} &=& g\gamma\vec{S} \label{e9}
\end{eqnarray}
\item An electron's spin is always quantised. The angular momentum of an 
electron in a hydrogen atom is also quantised. Therefore, $\mu$ of an electron
in a hydrogen atom is also quantised.
\end{enumerate}
\end{frame}

\begin{frame}
\frametitle{The magnetic quantum number}
\begin{enumerate}
\item The orbital quantum number gives the magnitude of the electron's angular
momentum. Thus, $L = \sqrt{l(l + 1)}\hslash$.
\item The magnetic quantum number gives the z-component of the angular momentum.
That is, $L_z = m_l\hslash$. 
\item Since $L_z = L\cos\theta$, we see that
\begin{equation}\label{e10}
\cos\theta = \frac{m_l}{\sqrt{l(l + 1)}}.
\end{equation}
Thus, the angle $\theta$ between $L_z$ and $L$ can take only a certain values. 
This is sometimes called `space quantisation'.
\item The same is true for $\mu$ and $\mu_z$.
\end{enumerate}
\end{frame}

\begin{frame}
\frametitle{Magnetic dipole in a magnetic field}
\begin{enumerate}
\item A magnetic dipole of moment $\vec{\mu}$ in a magnetic field $\vec{B}$ 
experiences a torque $\vec{\mu} \times \vec{B}$ and has a potential energy
$U_m = -\vec{\mu}\cdot\vec{B}$.
\item The z-axis is aligned parallel to $\vec{B}$ so that $U_m=-\mu\cos\theta B
= -\mu_z B$.
\item For an electron $\gamma = -e/(2m)$ so that $\mu_z = \gamma L_z$ or
\begin{equation}\label{e11}
U_m = \frac{e}{2m}m_l\hslash B.
\end{equation}
\item In (an electric dipole) radiative transition $\Delta m_l = 0, \pm 1$.
Therefore, the change in energy only because of change in $m_l$ is
\begin{equation}\label{e12}
\Delta U_m = \frac{e}{2m}\Delta m_l\hslash B.
\end{equation}
\end{enumerate}
\end{frame}

\begin{frame}
\frametitle{Zeeman effect}
\begin{enumerate}
\item A single line corresponding to an energy difference $E$ now splits into 
three, corresponding to energies 
\[
E - \frac{e}{2m}m_l\hslash, E, E + \frac{e}{2m}m_l\hslash.
\]
\item This is the `normal' Zeeman effect. While it is observed for some 
transitions, more often than not, it a single line is observed to split into
more than three lines.
\item Often times, a single line is observed to consist of two closely spaced
lines. This was observed by Michelson in the late 1800s and is called the 
`fine structure' of the hydrogen spectrum.
\item It can be partially explained by the electron's spin. The electron is in
the magnetic field (?) of the nucleus. Having a magnetic moment of its own, it 
itself can exist in two closely separated energy levels depending on whether it
aligns with or against the magnetic field.
\end{enumerate}
\end{frame}
\end{document}
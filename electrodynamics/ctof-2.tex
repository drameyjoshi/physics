\chapter{Relativistic Mechanics}\label{c2}
\begin{enumerate}
\item In the previous chapter, we argued that the expression
\begin{equation}\label{c2e1}
\delta t = \frac{1}{c}\int_A^B ds
\end{equation}
has a maximum for a particle if it is stationary. In this expression, $A$ and 
$B$ are world points and 
\[
ds = \sqrt{dx^\nu dx_\nu}.
\]
It follows immediately that, among all particles that start their journey at 
$A$ and end it at $B$, the expression
\begin{equation}\label{c2e2}
-\frac{1}{c}\int_A^B ds,
\end{equation}
has a minimum for the particle that is stationary.

\item We also know that for a classical system, there exists an action integral
$S$ whose value is an extremum for a path followed by the system in the 
configuration space. It is a minimum over an infinitesimal length of the path.

\item If we have to extend this idea to relativistic mechanics then the integral
must be invariant under Lorentz transformation. Therefore, it must be a true 
scalar. One example of such an integral is given by \eqref{c2e2}. We can mildly 
generalise it to
\begin{equation}\label{c2e3}
S = -\alpha \int_A^B ds,
\end{equation}
where $\alpha$ is a constant. From the discussion around equations \eqref{c2e1}
and \eqref{c2e3}, we infer that $\alpha > 0$.

\item We would like to write the action integral for relativistic systems as
\begin{equation}\label{c2e4}
S = \int_{t_1}^{t_2}Ldt,
\end{equation}
in analogy for the classical systems. From equations \eqref{c2e3} and 
\eqref{c2e4}, we get
\begin{equation}\label{c2e5}
L = -\alpha c\sqrt{1 - \frac{v^2}{c^2}},
\end{equation}
where
\[
v^i = \td{x^i}{t}
\]
is the 3-velocity of the particle. Can we guess $\alpha$?

\item As $v/c \rightarrow 0$, $L$ of \eqref{c2e5} should go over to the 
classical Lagrangian for a free particle, which is just $mv^2/2$. For small 
$v/c$, we can write \eqref{c2e5} as
\[
L = -\alpha c\left(1 - \frac{1}{2}\frac{v^2}{c^2}\right) = 
-\alpha c + \frac{\alpha}{2}\frac{v^2}{c}.
\]
A constant term in a Lagrangian can always be ignored and we infer that
\begin{equation}\label{c2e6}
\alpha = mc.
\end{equation}
Therefore, we guess the correct form of the Lagrangian for a free relativistic
particle to be
\begin{equation}\label{c2e7}
L = -mc^2\sqrt{1 - \frac{v^2}{c^2}}.
\end{equation}

\item The generalised momentum corresponding to this Lagrangian is
\begin{equation}\label{c2e8}
p_i = \pdt{L}{v^i} = (-mc^2)\frac{1}{2}\left(1 - \frac{v^2}{c^2}\right)^{-1/2}
\frac{-2v_i}{c^2} = \frac{mv_i}{\sqrt{1 - v^2/c^2}}.
\end{equation}
Note that $v^2 = v_iv^i = \eta_{ij}v^jv^i$ so that
\[
\frac{\partial}{\partial v^i}v^2 = \eta_{ij}(\delta_{ij}v^i + v^j) = \eta_{ij}
(v^j + v^j) = 2v_i.
\]
If
\begin{eqnarray}
\vec\beta &=& \frac{\vec{v}}{c} \\ \label{c2e9}
\gamma &=& \frac{1}{\sqrt{1 - \beta^2}} \label{c2e10}
\end{eqnarray}
then
\begin{equation}\label{c2e11}
\td{p^i}{t} = m\gamma \td{v^i}{t} + m\td{\gamma}{t}v^i.
\end{equation}
This equation is quite different from the one used in classical mechanics.

\item The energy of the particle is
\[
\mathcal{E} = p^i v_i - L = mv^2\gamma + \frac{mc^2}{\gamma} = m\gamma
\left(v^2 + \frac{c^2}{\gamma^2}\right)
\]
so that
\begin{equation}\label{c2e12}
\mathcal{E} = m\gamma\left(v^2 + c^2\left(1 - \frac{v^2}{c^2}\right)\right) 
= mc^2\gamma
\end{equation}

\item From \eqref{c2e12} we see that the energy of the particle is not zero in
the relativistic framework even if the particle is resting. When $v = 0$, 
$\gamma = 1$ and the particle's energy is
\begin{equation}\label{c2e13}
\mathcal{E} = mc^2.
\end{equation}
For small velocities, we can approximate \eqref{c2e12} to
\begin{equation}\label{c2e14}
\mathcal{E} = mc^2\left(1 + \frac{1}{2}\frac{v^2}{c^2} + O(\beta^4)\right) = 
mc^2 + \frac{1}{2}mv^2 + O(\beta^4).
\end{equation}

\item Since $\beta < 1$ for material bodies, $\gamma > 1$ and hence, from 
\eqref{c2e12} $\mathcal{E} \ge mc^2 > 0$. Thus energy of a free particle is 
always positive.  This is also true in classical mechanics for an elementary 
particle, that is a particle without internal degrees of freedom. For a 
composite body, in classical mechanics, energy can be negative and is 
determined to within a constant. However, in the relativistic regime this is 
not so, $\mathcal{E} > 0$ is always true.

\item For a composite body at rest, the energy $\mathcal{E} = Mc^2$ is not the 
same as 
\[
\sum_{i=1}^N m_ic^2,
\]
where $m_i, i = 1, \ldots, N$ are the rest masses of the constituent particles.
The rest energy will also include the energy of interaction between the 
particles.
Since 
\[
M \ne \sum_{i=1}^N m_i
\]
conservation of mass is not true in relativistic mechanics.

\item From \eqref{c2e8},
\[
p^2 = p_ip^i = m^2v^2\gamma^2
\]
and from \eqref{c2e12}
\[
\mathcal{E}^2 = m^2c^4\gamma^2 \Rightarrow \mathcal{E}^2 - \mathcal{E}^2\beta^2 
= m^2c^4 \Rightarrow \mathcal{E}^2 - p^2c^2  = m^2c^4,
\]
so that
\begin{equation}\label{c2e15}
\mathcal{E}^2 = p^2c^2 + m^2c^4.
\end{equation}
Since $\mathcal{E}$ is expressed in terms of $p$, we can as well write the 
Hamiltonian as
\begin{equation}\label{c2e16}
\mathcal{H} = c\sqrt{p^2 + m^2c^2}
\end{equation}

\item Since $p^i = mv^i\gamma$ and $\mathcal{E} = mc^2\gamma$, we also have
\begin{equation}\label{c2e17}
p^i = \frac{\mathcal{E}}{c^2}v^i.
\end{equation}
If $v \rightarrow c$, $\gamma \rightarrow \infty$. In this limit, both 
$\mathcal{E}$ and $p^i$ blow up unless $m = 0$. In that case, the two are 
related by
\begin{equation}\label{c2e18}
\mathcal{E} = pc.
\end{equation}

\item From equations \eqref{c2e4}, \eqref{c2e5} and \eqref{c2e6}, we can write
the principle of least action as
\begin{equation}\label{c2e19}
\delta S = -mc\delta\int_{t_1}^{t_2} ds = 0.
\end{equation}
Since $ds = \sqrt{dx^\mu dx_\mu}$, 
\begin{eqnarray*}
\delta ds 
 &=& \frac{1}{2}\frac{(\delta dx^\mu)dx_\mu + dx^\mu (\delta dx_\mu)}{ds} \\
 &=& \frac{1}{2}\frac{(\delta dx^\mu)dx_\mu + dx_\mu (\delta dx^\mu)}{ds} \\
 &=& \frac{dx_\mu(\delta dx^\mu)}{ds} \\
 &=& u_\mu \delta(dx^\mu) \\
 &=& u_\mu d(\delta x^\mu),
\end{eqnarray*}
where we used \eqref{c1e78}. Thus, equation \eqref{c2e19} becomes
\begin{equation}\label{c2e20}
\delta S = -mc\int_{t_1}^{t_2}u_\mu d(\delta x^\mu)
\end{equation}
Integrating by parts,
\begin{equation}\label{c2e21}
\delta S = -mc u_\mu \delta x^\mu\Big|_{t_1}^{t_2} + 
mc\int_{t_1}^{t_2}\delta x^\mu \td{u_\mu}{s}ds.
\end{equation}
Since variations in trajectories vanish at the end points, the first term on the
right hand side is zero and hence
\begin{equation}\label{c2e22}
\delta S = mc\int_{t_1}^{t_2}\delta x^\mu \td{u_\mu}{s}ds.
\end{equation}
$\delta S = 0$ for all possible variations $\delta x^\mu$ therefore implies 
that for a free particle,
\begin{equation}\label{c2e23}
\td{u_\mu}{s} = 0.
\end{equation}
The particle travels with a uniform velocity, unsurprisingly.

\item Now fix $t_1$ and let $t_2$ be varied for true trajectories, that is the
ones which occur in reality and for which \eqref{c2e23} is valid. Then we have 
from
\eqref{c2e21},
\begin{equation}\label{c2e24}
\delta S = -mcu_\mu \delta (x^\mu)_{t_2},
\end{equation}
so that
\begin{equation}\label{c2e25}
mcu_\mu = -\pdt{S}{x^\mu}.
\end{equation}
The 4-vector $mcu^\mu$, from equations \eqref{c1e79} to \eqref{c1e82} is
\[
u^\mu = \left(\gamma, \frac{v^i}{c}\gamma\right)
\]
so that
\[
mcu^\mu = \left(mc\gamma, mv^i\gamma\right) = 
\left(\frac{\mathcal{E}}{c}, p^i\right)
\]
We define the 4-momemtum $p^\mu$ as
\begin{equation}\label{c2e26}
p^\mu = \left(\frac{\mathcal{E}}{c}, p^i\right)
\end{equation}
so that \eqref{c2e25} becomes
\begin{equation}\label{c2e27}
p_\mu = -\pdt{S}{x^\mu}.
\end{equation}
Note that if $p^\mu$ is given by \eqref{c2e26} then
\begin{equation}\label{c2e28}
p_\mu = \left(\frac{\mathcal{E}}{c}, -p^i\right).
\end{equation}
The transformation equations for $p^\mu$ are given by
\begin{equation}\label{c2e29}
\frac{\mathcal{E}}{c} = 
\gamma\left(\frac{\bar{\mathcal{E}}}{c} + \beta\bar{p}^1\right);
p^1 = \gamma\left(\bar{p}^1 + \beta\frac{\bar{\mathcal{E}}}{c}\right); p^2 = 
\bar{p}^2; p^3 = \bar{p}^3.
\end{equation}
We also have
\begin{equation}\label{c2e30}
p^\mu p_\mu = \frac{\mathcal{E}^2}{c^2} - p^2 = m^2c^2.
\end{equation}
We now define the 4-force as 
\begin{equation}\label{c2e31}
g^\mu = \td{p^\mu}{s}.
\end{equation}
Since $ds = \sqrt{c^2(dt)^2 - dx_i dx^i}$,
\begin{equation}\label{c2e32}
ds = cdt\sqrt{1 - \frac{v^2}{c^2}} = \frac{c}{\gamma}dt.
\end{equation}
Therefore,
\[
g^\mu = \frac{\gamma}{c}
\left(\frac{1}{c}\td{\mathcal{E}}{t}, \td{\vec{p}}{t}\right).
\]
Since $\dot{\mathcal{E}} = \vec{f}\cdot\vec{v}$, we can write this as
\begin{equation}\label{c2e33}
g^\mu = \left(\frac{\gamma}{c^2}\vec{f}\cdot\vec{v}, 
\frac{\gamma}{c}\td{\vec{p}}{t}\right).
\end{equation}

\item From equations \eqref{c2e27} and \eqref{c2e30},
\[
\left(-\pdt{S}{x^\mu}\right)\left(-\pdt{S}{x_\mu}\right) = m^2c^2
\]
or
\begin{equation}\label{c2e34}
\pdt{S}{x^\mu}\pdt{S}{x_\mu} = g^{\mu\nu}\pdt{S}{x^\mu}\pdt{S}{x^\nu} = m^2c^2
\end{equation}
is the relativistic Hamilton-Jacobi equation. Expanding the sum
\begin{equation}\label{c2e35}
\frac{1}{c^2}\left(\pdt{S}{t}\right)^2 - \left(\pdt{S}{x}\right)^2 
- \left(\pdt{S}{y}\right)^2 - \left(\pdt{S}{z}\right)^2 = m^2c^2.
\end{equation}

\item We now consider the transformation of a distribution function in momentum
space. If $f$ is such a function then $f(\vec{p})dp_xdp_ydp_z$ is the number of
particles with momenta in the range $(p_x, p_y, p_z)$ to $(p_x + dp_x, p_y + 
dp_y, p_z + dp_z)$. To understand the properties of the ``volume'' element 
$dp_xdp_ydp_z$, we consider the 4-dimensional energy-momentum space whose axes 
determine the coordinates of 4-momentum $p^\mu$. From equation \eqref{c2e30}, 
the 4-momentum is a constant. Therefore, the motion of the system happens on 
the hypersurface with equation \eqref{c2e30}, that is, $p_\mu p^\mu = m^2c^2$. 
An element on the hypersurface is a 4-vector normal to it, the way an element 
on the surface of a 3-sphere is normal to it. From equations \eqref{c1e71} 
and \eqref{c1e72}, this vector is
\begin{equation}\label{c2e36}
dS^\mu = -\frac{1}{6}\epsilon^{\mu\nu\rho\sigma}dS_{\nu\rho\sigma}.
\end{equation}
so that $dS^0 = dp_xdp_ydp_z$. (Note that $S^\mu$ is a vector and has no
relationship with the action defined in the previous point) Further, the way 
the element on the 2-surface of a 3-sphere at $\vec{r}$ is parallel to 
$\vec{r}$, $dS^\mu$ too is parallel to $p^\mu$. Therefore, there exists a 
constant, say $k$, such that $dS^\mu = kp^\mu$. In particular, for $\mu = 0$, 
the ratio
\begin{equation}\label{c2e37}
\frac{dS^0}{p^0} = \frac{dp_xdp_ydp_z}{\mathcal{E}} = k,
\end{equation}
is an invariant under a Lorentz transformation. The number of particles 
$f(\vec{p}) dp_xdp_ydp_z$ is also an invariant. If we write
\[
f(\vec{p})dp_xdp_ydp_z = f(\vec{p})\mathcal{E}\frac{dp_xdp_ydp_z}{\mathcal{E}}
\]
then we see that
\[
f(\vec{p})dp_xdp_ydp_z = f^\op(\vec{p}^\op)dp_x^\op dp_y^\op dp_z^\op
\]
implies, from the invariance of $dS^0/p^0$ that
\begin{equation}\label{c2e38}
f^\op(\vec{p}^\op) = \frac{f(\vec{p})\mathcal{E}}{\mathcal{E}^\op}.
\end{equation}
In ``spherical'' momentum coordinates $dp_xdp_ydp_z = p^2dpdo$, where $do$ is 
the solid angle in momentum space about $p^i$. Putting this in equation 
\eqref{c2e37} we get the invariance of
\[
\frac{dp_xdp_ydp_z}{\mathcal{E}} = \frac{p^2dpdo}{\mathcal{E}} 
\]
Since $\mathcal{E}^2 = p^2c^2 + m^2c^4$, $\mathcal{E}d\mathcal{E} = c^2 pdp$, 
therefore,
\[
\frac{p^2dpdo}{\mathcal{E}} = \frac{pd\mathcal{E}do}{c^2}
\]
is also an invariant. In particular $pd\mathcal{E}do$ is also invariant.

\item We will now look at the distribution function over phase space. Thus,
$f(\vec{r}, \vec{p})dxdydzdp_xdp_ydp_z$ is the number of particles in the phase
space volume element $dxdydzdp_xdp_ydp_z$ around a point $(\vec{r}, \vec{p})$ in
a reference frame $K$. We want to find $f^\op(\vec{r}^\op, \vec{p}^\op)$ to 
which $f$ transforms in another frame $K^\op$.

Let $K_0$ be the frame in which the particles in volume element are at rest. It
is possible to find such a frame if we choose a small enough volume element. If
$dV_0$ is the volume of (position) space in $K_0$ then
\begin{equation}\label{c2e39}
dV = \frac{dV_0}{\gamma(v)}; dV^\op = \frac{dV_0}{\gamma(v^\op)}
\end{equation}
are the same volumes as measured in $K$ and $K^\op$ respectively. From these 
equations we have
\begin{equation}\label{c2e40}
\frac{dV}{dV^\op} = \frac{\gamma(v^\op)}{\gamma(v)}.
\end{equation}
If $\mathcal{E}_0, \mathcal{E}, \mathcal{E}^\op$ are the values of energy in 
the three frames then we also have
\begin{equation}\label{c2e41}
\mathcal{E} = \mathcal{E}_0\gamma(v); \mathcal{E}^\op = 
\mathcal{E}_0\gamma(v^\op)
\end{equation}
because $p_0^i = 0$. Using \eqref{c2e41} in \eqref{c2e40} we get
\begin{equation}\label{c2e42}
\frac{dV}{dV^\op}  = \frac{\mathcal{E}^\op}{\mathcal{E}}.
\end{equation}
From \eqref{c2e37} we also have
\begin{equation}\label{c2e43}
\frac{dp_xdp_ydp_z}{dp_x^\op dp_y^\op dp_z^\op} = 
\frac{\mathcal{E}}{\mathcal{E}^\op}.
\end{equation}
The previous two equations give
\begin{equation}\label{c2e44}
dVdp_xdp_ydp_z = dV^\op dp_x^\op dp_y^\op dp_z^\op
\end{equation}
that is, the phase space volume is invariant under a Lorentz transformation. 
Therefore, we also have
\begin{equation}\label{c2e45}
f(\vec{r}, \vec{p}) = f(\vec{r}^\op, \vec{p}^\op).
\end{equation}

\item Before proceeding further, we note that for any vector $A^\mu$, the 
differential $d^4A = dA^0dA^1dA^2dA^3$ is Lorentz invariant. This is because, 
$d^4A^\op = \det J  d^4A$, where $J$ is the Jacobian for the transformation. 
Since a Lorentz transformation is a rotation in 4-space, $\det J = 1$. We can 
quickly confirm it by writing the transformation in matrix form as
\[
\begin{bmatrix}x^0 \\ x^1 \\ x^2 \\ x^3\end{bmatrix} = 
\begin{bmatrix} \gamma & \beta\gamma & 0 & 0 \\
\beta\gamma & \gamma & 0 & 0 \\
0 & 0 & 1 & 1 \\
0 & 0 & 0 & 1\end{bmatrix}
\begin{bmatrix}\bar{x}^0 \\ \bar{x}^1 \\ \bar{x}^2 \\ \bar{x}^3\end{bmatrix}.
\]
Clearly,
\[
\det J = \gamma^2(1 - \beta^2) = 1.
\]

\item We now consider the problem of writing an expression for the cross-section
of collisions. Let a beam of particles with number density $n_1$ collide another
beam with number density $n_2$. Then the number of collisions $d\nu$ between 
them in a time $dt$ and a volume $dV$ depends on $n_1, n_2, dV, dt$ and $v_r$, 
the relative speed of one beam with respect to the other. Thus,
\[
d\nu \propto n_1n_2v_r dVdt
\]
or
\begin{equation}\label{c2e46}
d\nu = \sigma n_1n_2v_r dVdt,
\end{equation}
where $\sigma$ is called the collision cross section. Its dimensions are that of
an area. Since $dVdt = dx^\mu/c$, it is invariant under a Lorentz 
transformation.  $d\nu$ is invariant because it is a scalar. Therefore, the 
quantity $\sigma v_r n_1n_2$ is also an invariant.

However, this expression is developed only in the frame of the reference in 
which one of the beams is stationary. In order to generalise it to an arbitrary 
frame, we surmise that equation \eqref{c2e46} is generalised to
\begin{equation}\label{c2e47}
d\nu = An_1n_2dVdt,
\end{equation}
where $A = \sigma v_r$ in the frame of reference in which one of the beams is
stationary. 

The number of particles in a volume $dV$ is $ndV$ and is invariant. In another
frame of reference, it would be $n^\op dV^\op$. If $dV$ is the volume in 
measured in a frame in which it is at rest, $dV^\op = dV/\gamma$. Thus,
\begin{equation}\label{c2e48}
n^\op dV^\op = n dV \Rightarrow n^\op\frac{dV}{\gamma} = ndV \Rightarrow n^\op
= n\gamma.
\end{equation}
From equation \eqref{c2e12}, 
\begin{equation}\label{c2e49}
n^\op = \frac{n\mathcal{E}}{mc^2}.
\end{equation}
Invariance of $An_1n_2$ thus follows from that of $A\mathcal{E}_1\mathcal{E}_2$.
The dot product of $p_1^\mu$ and $p_2^\mu$ is a scalar and therefore invariant.
Therefore, the quantity
\[
\frac{A\mathcal{E}_1\mathcal{E}_2}{p_1^\mu {p_2}_\mu}
\]
is also an invariant. In the rest frame of beam 2, it becomes $\sigma v_r$, so 
that
\begin{equation}\label{c2e50}
A = \sigma v_r\frac{p_1^\mu {p_2}_\mu}{\mathcal{E}_1\mathcal{E}_2}.
\end{equation}
In the rest frame of particle 2, $p_2^\mu = (m_2c, 0, 0, 0)$, so that 
\[
p_1^\mu {p_2}_\mu = m_1\gamma(v_r) m_2 c^2 = 
\frac{m_1m_2c^2}{\sqrt{1 - v_r^2/c^2}} \Rightarrow 1 - \frac{v_r^2}{c^2} = 
\frac{m_1^2m_2^2c^4}{(p_1^\mu {p_2}_\mu)^2},
\]
or that
\begin{equation}\label{c2e51}
v_r = c\sqrt{1 - \frac{m_1^2m_2^2c^4}{(p_1^\mu {p_2}_\mu)^2}}.
\end{equation}
Now, 
\[
p_1^\mu {p_2}_\mu = m_1m_2\gamma(v_1)\gamma(v_2)c^2
\left(1 - \frac{\vec{v}_1\cdot\vec{v}_2}{c^2}\right).
\]
so that
\begin{eqnarray*}
v_r &=& c\sqrt{1 - \frac{1}{\gamma^2(v_1)\gamma^2(v_2)
    \left(1 - \frac{\vec{v}_1\cdot\vec{v}_2}{c^2}\right)^2}} \\
 &=& \frac{1}{\gamma(v_1)\gamma(v_2)
     \left(1 - \frac{\vec{v}_1\cdot\vec{v}_2}{c^2}\right)}
 	\sqrt{\gamma^2(v_1)\gamma^2(v_2)
 	\left(1 - \frac{\vec{v}_1\cdot\vec{v}_2}{c^2}\right)^2 - 1} \\
 &=& \frac{1}{\left(1 - \frac{\vec{v}_1\cdot\vec{v}_2}{c^2}\right)}
 \sqrt{\left(1 - \frac{\vec{v}_1\cdot\vec{v}_2}{c^2}\right)^2 - 
  \gamma^{-2}(v_1)\gamma^{-2}(v_2)} \\
 &=& \frac{1}{\left(1 - \frac{\vec{v}_1\cdot\vec{v}_2}{c^2}\right)}
 \sqrt{\left(1 - \frac{\vec{v}_1\cdot\vec{v}_2}{c^2}\right)^2 - 
 \left(1 - \frac{v_1^2}{c^2}\right)
 \left(1 - \frac{v_2^2}{c^2}\right)}
\end{eqnarray*}
We now simplify the notation a bit, and let
\begin{eqnarray}
\vec{\beta}_1 &=& \frac{\vec{v}_1}{c} \label{c2e52} \\
\vec{\beta}_2 &=& \frac{\vec{v}_2}{c} \label{c2e53}
\end{eqnarray}
so that
\begin{eqnarray*}
v_r &=& \frac{\sqrt{(1 - \vec{\beta}_1\cdot\vec{\beta}_2)^2 - 
	(1 - \beta_1^2 - \beta_2^2 + \beta_1^2\beta_2^2)}}
	{1 - \vec{\beta}_1\cdot\vec{\beta}_2} \\
 &=& \frac{\sqrt{1 - 2\vec{\beta}_1\cdot\vec{\beta}_2 + 
 			\beta_1^2\beta_2^2\cos^2\theta - 
			 (1 - \beta_1^2 - \beta_2^2 + \beta_1^2\beta_2^2)}}
			 {1 - \vec{\beta}_1\cdot\vec{\beta}_2} \\
 &=& \frac{\sqrt{\beta_1^2 - 2\vec{\beta}_1\cdot\vec{\beta}_2 + 
 				 \beta_2^2 - \beta_1^2\beta_2^2\sin^2\theta }}
		 {1 - \vec{\beta}_1\cdot\vec{\beta}_2}
\end{eqnarray*}
Finally, we get
\begin{equation}\label{c2e54}
v_r = 
\frac{\sqrt{(\vec{\beta}_1 - \vec{\beta}_2)^2 - 
	  (\vec{\beta}_1 \times \vec{\beta}_2)^2}}
  {1 - \vec{\beta}_1\cdot\vec{\beta}_2}
\end{equation}
In terms of $\vec{\beta}_i$,
\begin{equation}\label{c2e55}
p_1^\mu {p_2}_\mu = 
m_1m_2\gamma(v_1)\gamma(v_2)c^2\left(1 - \vec{\beta}_1\cdot\vec{\beta}_2\right).
\end{equation}
Substituting \eqref{c2e54} and \eqref{c2e55} in \eqref{c2e50}, we get
\begin{equation}\label{c2e56}
A = \sigma 
\sqrt{(\vec{\beta}_1-\vec{\beta}_2)^2-(\vec{\beta}_1 \times \vec{\beta}_2)^2},
\end{equation}
where we also used the fact that $\mathcal{E}_i = m_ic^2\gamma$. Therefore, 
equation \eqref{c2e47} gives
\begin{equation}\label{c2e57}
d\nu = \sigma 
\sqrt{(\vec{\beta}_1-\vec{\beta}_2)^2-(\vec{\beta}_1 \times \vec{\beta}_2)^2}
	n_1n_2dVdt.
\end{equation}

\item We now consider elastic collisions between particles. Let a particle of
mass $m_1$, energy $\mathcal{E}_1$ and momentum $\vec{p}_1$ collide with a 
particle of mass $m_2$, energy $\mathcal{E}_2$ and momentum $\vec{p}_2$. Let 
$\mathcal{E}_1^\op, \vec{p}_1^\op$ ($\mathcal{E}_2^\op, \vec{p}_2^\op$) be the 
energy and momenta after collision. By the conservation of 4-momentum,
\begin{equation}\label{c2e58}
p_1^\mu + p_2^\mu = p_1^{\op\mu} + p_2^{\op\mu}.
\end{equation}
Then $p_1^\mu + p_2^\mu - p_1^{\op\mu} = p_2^{\op\mu}$. Squaring each side, that
is, taking a product with the covariant components of each side, we get
\begin{equation}\label{c2e59}
m_1^2c^2 + p_{1\mu}p_2^\mu - p_{1\mu}p_1^{\op\mu} - p_{2\mu}p_1^{\op\mu} = 0,
\end{equation}
where we used the fact that $p_\mu p^\mu = \mathcal{E}^2/c^2 - p^2 = 
(\mathcal{E}^2 - p^2c^2)/c^2 = m^2c^2$.
Likewise, squaring $p_1^\mu + p_2^\mu - p_2^{\op\mu} = p_1^{\op\mu}$ gives
\begin{equation}\label{c2e60}
m_2^2c^2 + p_{1\mu}p_2^\mu - p_{1\mu}p_2^{\op\mu} - p_{2\mu}p_2^{\op\mu} = 0.
\end{equation}
If $m_2$ was at rest in the laboratory frame of reference then $p_2^\mu = 
(\mathcal{E}_2/c, 0)$. Using it in \eqref{c2e59} gives
\begin{equation}\label{c2e61}
m_1^2c^2 + \frac{\mathcal{E}_1\mathcal{E}_2}{c^2} - p_{1\mu}p_1^{\op\mu} 
- \frac{\mathcal{E}_1^\op \mathcal{E}_2}{c^2} = 0.
\end{equation}
Since $\mathcal{E}_2 = m_2c^2$ and $p_{1\mu}p_1^{\op\mu} = 
\mathcal{E}_1\mathcal{E}_1^\op/c^2 - \vec{p}_1\cdot\vec{p}_1^\op$, \eqref{c2e61}
becomes
\begin{equation}\label{c2e62}
m_1^2c^2 + m_2\mathcal{E}_1 - \mathcal{E}_1\mathcal{E}_1^\op/c^2 + 
\vec{p}_1\cdot\vec{p}_1^\op - m_2E_1^\op = 0,
\end{equation}
or
\[
p_1p_1^\op\cos\theta_1 = \mathcal{E}_1^\op(\mathcal{E}_1/c^2 + m_2) - 
E\mathcal{E}_1m_2 - m_1^2c^2
\]
so that
\begin{equation}\label{c2e63}
\cos\theta_1 = \frac{\mathcal{E}_1^\op(\mathcal{E}_1/c^2 + m_2) - 
\mathcal{E}_1m_2 - m_1^2c^2}{p_1p_1^\op}
\end{equation}
Using $p_2^\mu = (\mathcal{E}_2/c, 0)$ in \eqref{c2e60} we get
\[
m_2^2c^2 + \frac{\mathcal{E}_1\mathcal{E}_2}{c^2} - \frac{\mathcal{E}_1
\mathcal{E}_2^\op}{c^2} + 
\vec{p}_1\cdot\vec{p}_2^\op - \frac{\mathcal{E}_2\mathcal{E}_2^\op}{c^2} = 0.
\]
Since $\mathcal{E}_2 = m_2c^2$, this equation can be written as
\[
m_2^2c^2 + m_2\mathcal{E}_1 - \frac{\mathcal{E}_1\mathcal{E}_2^\op}{c^2} + 
\vec{p}_1\cdot\vec{p}_2^\op - m_2\mathcal{E}_2^\op = 0.
\]
Reorganising it, we get
\begin{equation}\label{c2e64}
\cos\theta_2 = 
\frac{(\mathcal{E}_2^\op - m_2c^2)(\mathcal{E}_1/c^2 + m_2)}{p_1p_2^\op}.
\end{equation}

An experiment records $\theta_1$ and $\theta_2$ for controlled values of $E_1$
and fixed masses $m_1$ and $m_2$. Equations \eqref{c2e63} and \eqref{c2e64}
can be inverted, in principle, to get $\mathcal{E}_1^\op$ and 
$\mathcal{E}_2^\op$ in terms of the scattering angles $\theta_1$ and $\theta_2$.

\item We will examine an elastic collision considered in the previous point in 
the centre of mass reference frame. Here, the momenta of the two particles 
retain their magnitudes but change their directions. They rotate by an angle, 
say $\chi$.
We will denote the quantities in this frame with a bar. Thus,
\begin{equation}\label{c2e65}
\bar{p}_{1\mu}\bar{p}_1^{\op\mu} = 
\frac{\bar{\mathcal{E}}_1\bar{\mathcal{E}}_1^\op}{c^2} - 
\bar{\vec{p}}_1\cdot\bar{\vec{p}}+1^\op = 
\frac{\bar{\mathcal{E}}_1\bar{\mathcal{E}}_1^\op}{c^2} - 
\bar{p}^2_1\cos\chi,
\end{equation}
because only the direction of momentum changes in this frame. Further, in an
elastic collision, $\bar{\mathcal{E}}_1 = \bar{\mathcal{E}}_1^\op$ so that
\begin{equation}\label{c2e66}
\bar{p}_{1\mu}\bar{p}_1^{\op\mu} = \frac{\bar{\mathcal{E}}_1^2}{c^2} - 
\bar{p}_1^2\cos\chi.
\end{equation}
Since $\bar{\mathcal{E}}_1^2 = m_1^2c^4 + \bar{p}^2c^2$, 
\begin{equation}\label{c2e67}
\bar{p}_{1\mu}\bar{p}_1^{\op\mu} = m_1^2c^2 + \bar{p}_1^2(1 - \cos\chi).
\end{equation}
If the particle with mass $m_2$ was at rest in the laboratory frame,
\begin{eqnarray}
p_{1\mu}p_2^\mu &=& \mathcal{E}_1m_2c^2 \label{c2e68} \\
p_{2\mu}p_1^{\op\mu} &=& m_2c^2 \mathcal{E}_1^\op \label{c2e69}
\end{eqnarray}
so that
\begin{equation}\label{c2e70}
m_2c^2(\mathcal{E}_1^\op - \mathcal{E}_1) = p_{2\mu}p_1^{\op\mu} - 
p_{1\mu}p_2^\mu.
\end{equation}
Since scalars are invariant under a transformation to centre of frame reference
we can as well use \eqref{c2e66} to write
\begin{equation}\label{c2e71}
p_{1\mu}p_1^{\op\mu} = m_1^2c^2 + \bar{p}_1^2(1 - \cos\chi).
\end{equation}
Substituting \eqref{c2e70} and \eqref{c2e71} into \eqref{c2e59}, we get
\begin{equation}\label{c2e72}
\mathcal{E}_1^\op - \mathcal{E}_1 = -\frac{\bar{p}_1^2}{m_2c^2}(1 - \cos\chi).
\end{equation}
We next use the fact that $p_{1\mu}p_2^\mu = \bar{p}_{1\mu}\bar{p}_2^\mu$ so 
that, using \eqref{c2e68},
\begin{equation}\label{c2e73}
m_2c^2\mathcal{E}_1 = \bar{\mathcal{E}}_1\bar{\mathcal{E}}_2 - 
\bar{\vec{p}}_1\cdot\bar{\vec{p}}_2 = 
\bar{\mathcal{E}}_1\bar{\mathcal{E}}_2 + \bar{p}_1^2,
\end{equation}
as the two momenta are equal and opposite in the centre of mass system. Further,
$\bar{\mathcal{E}}_1 = \sqrt{\bar{p}_1^2 + m_1^2c^4}, \bar{\mathcal{E}}_2 = 
\sqrt{\bar{p}_2^2 + m_2^2c^4}
= \sqrt{\bar{p}_1^2 + m_2^2c^4}$ so that equation \eqref{c2e72} becomes
\begin{equation}\label{c2e74}
m_2c^2\mathcal{E}_1 - \bar{p}_1^2 = \sqrt{\bar{p}_1^2 + m_1^2c^4}
\sqrt{\bar{p}_1^2 + m_2^2c^4},
\end{equation}
whose solution is
\begin{equation}\label{c2e75}
\bar{p}_1^2 = 
\frac{m_2c^2(\mathcal{E}_1^2 - m_1^2c^4)}{m_1^2 + m_2^2 + 2m_2\mathcal{E}_1}.
\end{equation}
Substituting \eqref{c2e75} in \eqref{c2e72} we get
\begin{equation}\label{c2e76}
\mathcal{E}_1^\op = \mathcal{E}_1 - 
\frac{m_2(\mathcal{E}_1^2 - m_1^2c^4)}
{m_1^2 + m_2^2 + 2m_2\mathcal{E}_1}(1 - \cos\chi).
\end{equation}
This equation gives the energy of the scatterd particle in terms of its initial
energy and angle of scattering in the centre of mass frame.

Energy conservation gives us $\mathcal{E}_1 + m_2c^2 = \mathcal{E}_1^\op + 
\mathcal{E}_2^\op$ so that
\[
\mathcal{E}_2^\op = \mathcal{E}_1 - \mathcal{E}_1^\op - m_2c^2
\]
so that
\begin{equation}\label{c2e77}
\mathcal{E}_2^\op = m_2c^2 + 
\frac{m_2(\mathcal{E}_1^2 - m_1^2c^4)}{m_1^2 + m_2^2 + 2m_2\mathcal{E}_1}
(1 - \cos\chi).
\end{equation}

\item Let $x^\mu$ be the world-coordinates of particle. Under an infinitesimal 
rotation of the reference frame, the coordinates now become $x^{\op\mu}$. The 
two are so close that one can express the relation between them as
\begin{equation}\label{c2e78}
x^{\op\mu} = x^\mu + x_\nu\delta\Omega^{\nu\mu}.
\end{equation}
Since the norm of this vector remains unchanged, $x^\op_\mu x^{\op\mu} = x_\mu 
x^\mu$. Now,
\[
x^\op_\mu = 
g_{\mu\nu}x^{\op\nu} = g_{\mu\nu}(x^\nu + x_\rho \delta\Omega^{\rho\nu})
= x_\mu + x_\rho g_{\mu\nu}\delta\Omega^{\rho\nu},
\]
so that
\begin{eqnarray*}
x^\op_\mu x^{\op\mu} &=& 
 (x_\mu + x_\rho g_{\mu\nu}\delta\Omega^{\rho\nu})(x^\mu + 
x_\nu\delta\Omega^{\nu\mu}) \\
 &=& x_\mu x^\mu + x_\mu x_\nu \delta\Omega^{\nu\mu} + x_\rho g_{\mu\nu}x^\mu 
 \delta\Omega^{\rho\nu} \\
 &=& x_\mu x^\mu + x_\mu x_\nu \delta\Omega^{\nu\mu} + x_\rho x_\nu\delta
 \Omega^{\rho\nu} \\
 &=& x_\mu x^\mu + x_\mu x_\nu \delta\Omega^{\nu\mu} + x_\mu x_\nu\delta
 \Omega^{\mu\nu} \\
 &=& x_\mu x^\mu + x_\mu x_\nu (\delta\Omega^{\nu\mu} + \delta \Omega^{\mu\nu}).
\end{eqnarray*}
Here we have ignored terms quadratic in $\delta\Omega^{\mu\nu}$.
If $x^\op_\mu x^{\op\mu} = x_\mu x^\mu$ then we must have
\begin{equation}\label{c2e79}
\delta\Omega^{\nu\mu} = -\delta \Omega^{\mu\nu},
\end{equation}
that is, the tensor $\delta\Omega^{\mu\nu}$ us anti-symmetric.

\item For an infinitesimal change of coordinates, the change in action is
\begin{equation}\label{c2e80}
\delta S = \sum_{a} p_{\mu a}\delta x^{\mu a},
\end{equation}
where we have used equations \eqref{c2e24} and \eqref{c2e26} and the summation
is over all particles of the system. From equation \eqref{c2e78},
\begin{equation}\label{c2e81}
\delta S = \sum_{a} p_{\mu a}x_{\nu a}\delta\Omega^{\mu\nu}.
\end{equation}
The infinitesimal transformation $\delta\Omega^{\mu\nu}$ is the same for all
particles in the system. Therefore,
\begin{equation}\label{c2e82}
\delta S = \delta\Omega^{\mu\nu}\sum_{a} p_{\mu a}x_{\nu a}.
\end{equation}
We write the tensor $p_{\mu a}x_{\nu a}$ as a sum of symmmetric and 
anti-symmetric tensors
\begin{equation}\label{c2e83}
p_{\mu a}x_{\nu a} = \frac{p_{\mu a}x_{\nu a} + p_{\nu a}x_{\mu a}}{2} +
\frac{p_{\mu a}x_{\nu a} - p_{\nu a}x_{\mu a}}{2}
\end{equation}
Since $\delta\Omega^{\mu\nu}$ is anti-symmetric,
\begin{eqnarray*}
\delta\Omega^{\mu\nu}\frac{p_{\mu a}x_{\nu a} + p_{\nu a}x_{\mu a}}{2} &=&
\frac{\delta\Omega^{\mu\nu}p_{\mu a}x_{\nu a}}{2} + 
\frac{\delta\Omega^{\mu\nu}p_{\nu a}x_{\mu a}}{2} \\
 &=& \frac{\delta\Omega^{\mu\nu}p_{\mu a}x_{\nu a}}{2} + 
\frac{\delta\Omega^{\nu\mu}p_{\mu a}x_{\nu a}}{2} \\
 &=& \frac{1}{2}(\delta\Omega^{\mu\nu} + \delta\Omega^{\nu\mu})
 p_{\mu a}x_{\nu a}.
\end{eqnarray*}
Because of the anti-symmetric nature of $\delta\Omega^{\mu\nu}$, the right hand 
side is zero. Thus, only the anti-symmetric part of \eqref{c2e83} survives in
the summation over $\mu\nu$ in \eqref{c2e82}. We, therefore, write \eqref{c2e82}
as
\begin{equation}\label{c2e84}
\delta S = 
\frac{1}{2}\delta\Omega^{\mu\nu}\sum_{a}p_{\mu a}x_{\nu a} - p_{\nu a}x_{\mu a}.
\end{equation}
Equivalently,
\begin{equation}\label{c2e85}
\delta S = 
\frac{1}{2}\delta\Omega_{\mu\nu}\sum_{a}p^{\mu a}x^{\nu a} - p^{\nu a}x^{\mu a}.
\end{equation}
The tensor
\begin{equation}\label{c2e86}
L^{\mu\nu} = x^{\mu}p^{\nu} - x^{\nu}p^{\mu}
\end{equation}
is called the angular momentum 4-tensor. Since $x^\mu = (ct, x^1, x^2, x^3), 
p^\mu = (\mathcal{E}/c, p^1, p^2, p^3)$,
\[
x^\mu p^\nu = \begin{bmatrix}t\mathcal{E} & p^1ct & p^2ct & p^3ct \\
x^1\mathcal{E}/c & x^1p^1 & x^1p^2 & x^1p^3 \\
x^2\mathcal{E}/c & x^2p^1 & x^2p^2 & x^2p^3 \\
x^3\mathcal{E}/c & x^3p^1 & x^3p^2 & x^3p^3
\end{bmatrix}
\]
and hence
\[
L^{\mu\nu} = \begin{bmatrix}t\mathcal{E} & p^1ct & p^2ct & p^3ct \\
x^1\mathcal{E}/c & x^1p^1 & x^1p^2 & x^1p^3 \\
x^2\mathcal{E}/c & x^2p^1 & x^2p^2 & x^2p^3 \\
x^3\mathcal{E}/c & x^3p^1 & x^3p^2 & x^3p^3
\end{bmatrix} - 
\begin{bmatrix}t\mathcal{E} & x^1\mathcal{E}/c & x^2\mathcal{E}/c & 
x^3\mathcal{E}/c \\
p^1ct & x^1p^1 & x^2p^1 & x^3p^1 \\
p^2ct & x^1p^2 & x^2p^2 & x^3p^2 \\
p^3ct & x^1p^3 & x^2p^3 & x^3p^3
\end{bmatrix}
\]
or
\begin{equation}\label{c2e87}
L^{\mu\nu} = \begin{bmatrix}
0 & p^1ct - x^1\mathcal{E}/c & p^2ct - x^2\mathcal{E}/c & p^3ct - 
x^3\mathcal{E}/c\\
p^1ct - x^1\mathcal{E}/c & 0 & x^1p^2 - x^2p^1 & x^1p^3 - x^3p^1 \\
p^2ct - x^2\mathcal{E}/c & x^2p^1 - x^1p^2 & 0 & x^2p^3 - x^3p^2 \\
p^2ct - x^3\mathcal{E}/c & x^3p^1 - x^3p^1 & x^3p^2 - x^2p^3 & 0
\end{bmatrix}
\end{equation}
This tensor is very similar to the electromagnetic field tensor with the 
electric part given by the vector
\begin{equation}\label{c2e88}
\vec{L}_e = \vec{p}ct - \vec{x}\frac{\mathcal{E}}{c}
\end{equation}
and the magnetic part by the vector
\begin{equation}\label{c2e89}
\vec{L}_m = -(\vec{x} \times \vec{p}).
\end{equation}
If $L^{\mu\nu}$ is conserved then so are $\vec{L}_e$ and $\vec{L}_m$ 
individually.  For a system of multiple particles, it means that the quantity
\begin{equation}\label{c2e90}
\sum_a \left(\vec{p}_act - \vec{x}_a\frac{\mathcal{E}_a}{c}\right) = 
\text{constant.}
\end{equation}
Equivalently,
\begin{equation}\label{c2e91}
\frac{\sum_a\vec{p}_ac^2t}{\sum_a \mathcal{E}_a} - \frac{\sum_a\vec{x}_a 
\mathcal{E}_a}{\sum \mathcal{E}_a} = \text{constant.}
\end{equation}
Thus a point
\begin{equation}\label{c2e92}
\vec{R} = \frac{\sum_a\vec{x}_a \mathcal{E}_a}{\sum \mathcal{E}_a}
\end{equation}
travels with uniform velocity
\begin{equation}\label{c2e93}
\vec{V} = c^2\frac{\sum_a\vec{p}_a}{\sum_a \mathcal{E}_a}.
\end{equation}
Equation \eqref{c2e92} is analogous to the centre of mass vector of classical 
mechanics. The formula merely replaces mass in classical mechanics with energy.
\end{enumerate}

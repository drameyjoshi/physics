\documentclass{beamer}
\title{Key Ideas in Thermodynamics}
\author{Amey Joshi}
\newcommand{\dbar}{{\mathchar'26\mkern-13mu d}}
\begin{document}
\frame{\titlepage}

\begin{frame}
\frametitle{A thermodynamic system}
\begin{enumerate}
\item A portion of space which we want to study is the \emph{system} everything else is its \emph{surrounding}.
\item The system is described in terms of a few quantities called \emph{thermodynamic variables}.
\item Thermodynamic variables are macroscopic, directly measured and make no assumptions about the matter in the system.
\item Two systems described by variables $(X_1, Y_1)$ and $(X_2, Y_2)$ are said to be in thermal equilibrium if the variables do
not change when the two systems are in contact with each other.
\item Zeroth law of thermodynamics.
\item This leads to the idea that systems in equilibrium can be characterised by a single number called their \emph{temperature}.
\end{enumerate}
\end{frame}

\begin{frame}
\frametitle{Equilibrium}
\begin{enumerate}
\item A system is in \emph{thermodynamic equilibrium} if it is in:
\begin{enumerate}
\item Mechanical equilibrium, with no unbalances forces on the interior of the system.
\item Chemical equilibrium, when there is an absence of diffusion, phase change, change in composition.
\item Thermal equilibrium, when there is no change in thermodynamic variables when the system is separated from its surrounding
by a diathermic wall.
\end{enumerate}
\item A system in thermodynamic equilibrium can be described in terms of thermodynamic variables that do not change with time.
\item Systems not in equilibrium cannot be described by variables common to all parts of the systems. We focus on systems in 
equilibrium.
\end{enumerate}
\end{frame}

\begin{frame}
\frametitle{Equation of state}
\begin{enumerate}
\item Usually the thermodynamics variables describing a system are not all independent. They are constrained by an equation of
state.
\item An example of a thermodynamic variable is the pressure $P$. A differential change in it is $dP$. In thermodynamics, $dP$
is small compared to $P$ but very large compared to the molecular fluctuations in stress. Same is true for differential changes
to all thermodynamic variables.
\item An extensive thermodynamic variable's value is halved if we consider only half the system. An intensive variable's value
remains unchanged. Volume is extensive whilr pressure is intensive.
\end{enumerate}
\end{frame}

\begin{frame}
\frametitle{Work}
\begin{enumerate}
\item When the external force acting on a system is in the same (opposite) direction as the displacement of the system, work is 
done on (by) the system and is considered to be positive (negative). 
\item The displacement of the system happens \emph{quasi-statically}. 
\item For a hydrostatic system, $\dbar W = -PdV$. When $dV < 0$, work is done on the system and is positive. For a volume change
from $V_1$ to $V_2$, the work done is
\[
W = -\int_{V_1}^{V_2}PdV,
\]
and its value depends on the path taken by the quasi-static process in the $PV$ plane. That is why, the differential work
is an inexact differential $\dbar W$.

\end{enumerate}
\end{frame}

\begin{frame}
\frametitle{Work}
\begin{enumerate}
\item In general, work done by/on a system is $\dbar W = YdX$, where $Y$ is an intensive property called the generalised
force and $X$ is an extensive property called the generalised displacement. Their product has the dimensions of work.
\item For a composite system, $\dbar W = Y_1dX_1 + Y_2dX_2$.
\item It is experimentally found that adiabatic work is independent of path. We can then write
\[
W_{\text{adiabatic}} = -\int_{V_1}^{V_2}PdV = U_2 - U_1,
\]
where $U_1. U_2$ are numbers characteristic of the inital and final states.
\item $U$ is analogous to the potential in a mechanical system and is called its \emph{internal energy}.
\end{enumerate}
\end{frame}

\begin{frame}
\frametitle{Heat}
\begin{enumerate}
\item We can change the internal energy of the system from $U_1$ to $U_2$ even without adiabatic work by bringing it in contact
with another system at a different temperature.
\item The energy that is transferred to the system solely because of a temperature difference and enough to compensate for the 
change $U_2 - U_1$ in its internal energy is called \emph{heat}. This is the thermodynamic definition of heat.
\item A calorimetric definition of heat is that which flows between two systems when they are at different temperatures.
\end{enumerate}
\end{frame}

\begin{frame}
\frametitle{First law of thermodynamics}
\begin{enumerate}
\item $U_2 - U_1 = Q + W$. Its differential form is $dU = \dbar Q + \dbar W$. 
\item The first law of thermodynamics introduces three important ideas:
\begin{enumerate}
\item The principle of conservation of energy.
\item The existence of the internal energy function.
\item The definition of heat at energy in transit as a result of temperature difference.
\end{enumerate}
\item For a hydrostatic system, $dU = \dbar Q - PdV$.
\item The quantity $\dbar Q/dT$ is called the \emph{heat capacity}. It can be measured accurately and it plays an important
role in checking if the internal energy derived in statistical mechanics is correct.
\end{enumerate}
\end{frame}

\begin{frame}
\frametitle{Second law of thermodynamics - 1}
\begin{enumerate}
\item If the system is a resistor immersed in flowing water, the heat caused by passage of current is transferred to the water
and the system remains in the same state. Work done by the battery is completely converted to heat and the process can happen
indefinitely.
\item Can heat be completely converted to work? First law of thermodynamics will not object to this possibility. The second law 
forbids it.
\item  First law assures that energy is conserved. The second law dictates the direction in which it flows and the proportion
by which it transforms between heat and work.
\item Work is a result of orderly motion of the system. Heat manifests as disorderly motion. That is why conversion of heat
to work is never $100\%$.
\end{enumerate}
\end{frame}

\begin{frame}
\frametitle{Second law of thermodynamics - 2}
\begin{enumerate}
\item These two formulations of the second law are equivalent:
\begin{itemize}
\item  Kelvin-Planck: No process is possible whose \emph{sole} result is absorption of hear from a reservoir and its conversion
into work.
\item Clausius: No process is possible whose \emph{sole} result is the transfer of heat from a cooler to a hotter body.
\end{itemize}
\item A reversible process is quasi-static and non-dissipative.
\item All natural, spontaneous processes are irreversible. They are either not quasi-static, that is not in thermodynamic
equilibrium, or dissipative or both.
\end{enumerate}
\end{frame}

\begin{frame}
\frametitle{Pfaffian differential forms}
\begin{enumerate}
\item One can write $\dbar Q = dU - \dbar W = dU - \sum_{i}Y_idX_i$. Such an expression is called a \emph{Pfaffian differential
form}. In general, it is not integrable. That is, there is no function $f$ such that $df = dU - \sum_{i}Y_idX_i$.
\item An example is $-ydx + xdy$. When integrated from $(0, 0)$ to $(1, 1)$ along the line $y = x$ we get zero. Along $y = x^2$,  
we get $1/3$.
\item A Pfaffian differential form in two variables can always be multiplied by a function $\phi(x, y)$ such that the result is
integrable. $\phi(x, y)$ is called an integrating factor. (Recall the theory of ODEs.) A Pffafian differential form in more
than two variables is, in general, not integrable.

\end{enumerate}
\end{frame}

\begin{frame}
\frametitle{Second law of thermodynamics - 3}
\begin{enumerate}
\item It is a law of Nature that $\dbar Q$ for \emph{all} systems is integrable and have the same integrating factor $1/T$,
where $T$ is the absolute temperature, provided that $\dbar Q$ is a reversible change. 
\item This is another, more abstract, statement of the second law of thermodynamics.
\item Thus, $\dbar Q/T$ is an exact differential and can be written as 
\[
dS = \frac{\dbar Q_R}{T}.
\]
The quantity $S$ is called the entropy. The subscript $R$ in $\dbar Q_R$ emphasises that the change is reversible.
\end{enumerate}
\end{frame}

\begin{frame}
\frametitle{Second law of thermodynamics - 4}
\begin{enumerate}
\item For a reversible change, a transfer $\dbar Q_R$ to the system from its surroundings is accompanied with a change 
$-\dbar Q_R$ in the surroundings. Therefore, the net change in entropy of the system plus the surrounding is zero.
\item If the heat transfer happens irreversibly, then the net change in entropy is positive.
\item Since more natural processes are dissipative, they lead to an increase in entropy.
\item Entropy is also a measure of energy unavailable for work. An increase in entropy is a degradation of energy.
\end{enumerate}
\end{frame}

\begin{frame}
\frametitle{What I want to learn}
\begin{enumerate}
\item The connection between the definitions $dS = \dbar Q_R/T$ and $S = k\ln\Omega$.
\item The statistical interpretation of inexact differentials like $\dbar W$.
\item Equilibrium statistical mechanics is a mature discipline and these questions are satisfactorily answered. My understanding
is faded in these areas.
\end{enumerate}
\end{frame}
\end{document}
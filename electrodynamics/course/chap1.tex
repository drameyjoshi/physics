\documentclass{article}
\usepackage{amsmath, graphicx, xcolor}
\newcommand{\op}{{\;\prime}}
\newcommand{\grad}{\nabla\;}
\newcommand{\dive}{\nabla\cdot}
\newcommand{\curl}{\nabla\times}
\newcommand{\td}[2]{\frac{d{#1}}{d{#2}}}
\newcommand{\pdt}[2]{\frac{\partial{#1}}{\partial{#2}}}
\newcommand{\uv}[1]{\hat{e}_{#1}}
\newcommand{\un}{\hat{n}}

\title{Electrostatics - Charges and Fields}
\author{Amey Joshi}
\begin{document}
\begin{enumerate}
\item The magnitude of electric and gravitational forces are
\begin{eqnarray}
F_e &=& \frac{q^2}{r^2} \label{e1} \\
F_g &=& \frac{Gm^2}{r^2} \label{e2}
\end{eqnarray}
where $m$ is the mass of the proton and $q$ is the charge. Therefore,
\[
\frac{F_e}{F_g} = \frac{q^2}{Gm^2}.
\]
The ratio depends only on the fundamental constants. In the case of a pair of
protons, it is $1.235 \times 10^{36}$. An electron is roughly $2000$ times lighter
than a proton. Therefore, the same ratio for a pair of electrons will be roughly
a million times more, that is, the electric repulsion will be $10^{42}$ times
stronger than the gravitational attraction.

\begin{itemize}
\item Will the ratio depend on units?
\end{itemize}

\item From the free body diagram of an electron, at equilibrium 
\begin{equation}\label{e3}
\frac{e^2}{r^2} = m_eg
\end{equation}
so that
\begin{equation}\label{e4}
r = \frac{e}{\sqrt{m_e g}} \approx 500 \text{ cm}.
\end{equation}
If a proton was to be balanced by an electron, then we would have had to place it
closer to the electron because a heavier object is to be balanced.

\item Using the free body diagram of one of the balls,
\begin{eqnarray}
mg &=& T\cos\theta \label{e5} \\
\frac{q^2}{r^2} &=& T\sin\theta \label{e6}
\end{eqnarray}
so that
\[
\tan\theta = \frac{q^2}{mgr^2}.
\]
From the geometry of the figure, we also have $\tan\theta = 0.1$. Therefore,
\begin{equation}\label{e7}
q = \sqrt{mg\tan\theta}\; r = 85.73 \text{ esu}. 
\end{equation}

\item Due to symmetry of the arrangement, the force on each $q$ is the same. The
force on any one of them is
\[
F = \frac{q^2}{a^2} + \frac{q^2}{a^2} + \frac{q^2}{2a^2} - \frac{2qQ}{a^2},
\]
so that equilibrium requires
\begin{equation}\label{e8}
Q = \frac{5}{4}q.
\end{equation}
The ratio $Q/q$ is independent of the size of the square.

\item Align the axes such that the origin coincides with the centre of the semi-
cirle and the semi-circle lies above the $x$ axis. The field due to an elementary
charge $dq$ at $\vec{r}$ on the ring is
\[
d\vec{E} = \frac{dq}{r^2}\uv{r} = \frac{dq}{r^3}\vec{r}.
\]

Before solving the problem analytically, from the geometry alone, can you
\begin{itemize}
\item Guess the $x$-component? The $z$-component?
\item Infer the field at the origin if the ring was full?
\end{itemize}
Now, $dq = \lambda Rd\theta$, $\vec{r} = R\cos(\pi + \theta)\uv{x} + R\sin(\pi + \theta)\uv{y}
= -R\cos\theta\uv{x} - R\sin\theta\uv{y}$ so
that
\[
d\vec{E} = -(R\cos\theta\uv{x} + R\sin\theta\uv{y})\frac{\lambda R d\theta}{R^3}
\]
and
\begin{equation}\label{e9}
\vec{E} = -\frac{2Q}{\pi R^2}\uv{y}.
\end{equation}

\item A numerical problem.

\item Align the axes such that one electron is at $(-a, 0)$ and the other at 
$(a, 0)$. Let the proton be at $(x, 0)$. The potential energy of the arrangement
is
\begin{equation}\label{e10}
U = -\frac{e^2}{x - a} - \frac{e^2}{x + a} + \frac{e^2}{2a},
\end{equation}
where $e$ is the electronic charge. The roots of the equation $U = 0$ are
\begin{equation}\label{e11}
x = (2 \pm \sqrt{5})a.
\end{equation}
Since distances must be positive, we drop the root $(2 - \sqrt{5})a$. Therefore,
the proton should be placed at $x = (2 + \sqrt{5})a$.

If the proton were at $(x, y)$, the locus of curve of zero potential energy is
\begin{equation}\label{e12}
\frac{1}{2a} = \frac{1}{\sqrt{(x - a)^2 + y^2}} + \frac{1}{\sqrt{(x + a)^2 + y^2}}.
\end{equation}

If
\begin{equation}\label{e13}
F(a, x, y) = \frac{1}{2a} - \frac{1}{\sqrt{(x - a)^2 + y^2}} + \frac{1}{\sqrt{(x + a)^2 + y^2}}
\end{equation}
then a the locus of points of zero potential energy looks like can be generated using
\begin{verbatim}
import numpy as np
import matplotlib.pyplot as plt

from functools import partial
from scipy.optimize import fsolve

def F(a, x, y):
    return 1/(2 * a) - 1/np.sqrt((x - a)**2 + y**2) \
     - 1/np.sqrt((x + a)**2 + y**2)

a_curve = partial(F, 1)

a = 1
X = np.linspace(-4, 4, 1000)
Y = []

for x in X:
    roots = fsolve(a_curve, a, x)
    Y.append(roots[0])

Xs = X.tolist()
Xs.extend(Xs)
Ys = Y + [-y for y in Y]

plt.scatter(Xs, Ys, s = 1)
plt.xlabel(r'$x$')
plt.ylabel(r'$y$')
plt.title(r'$F(a=1, x, y) = 0$')
plt.savefig('purcell_c1p7.png')
plt.show()
\end{verbatim}

For $a = 1$, it looks like the one shown in figure \ref{c1f1}.
\begin{figure}[!ht]
\center
\includegraphics[scale=0.6]{purcell_c1p7}
\caption{Locus of points for which $U=0$.}
\label{c1f1}
\end{figure}

\item Let $a$ be the distance between the ions. Starting from any ion, the potential
energy is
\[
U = \left(-\frac{e^2}{a} - \frac{e^2}{a}\right) + \left(\frac{e^2}{2a} + \frac{e^2}{2a}\right) + \cdots
\]
or,
\begin{equation}\label{e14}
U = -2\frac{e^2}{a}\left(1 - \frac{1}{2} + \frac{1}{3} - \frac{1}{4} + \cdots\right) 
= -2\ln 2\frac{e^2}{a}
\end{equation}

\item By Gauss' law the electric field on a closed surface is only due to the 
charges enclosed by the surface. Consider a portion of the sphere of radius $r$
and let $q$ be the charge in it. Let $dq$ be the change on a thin shell with 
radii $r$ and $r + dr$. It has a charge $dq = \rho 4\pi r^2dr$. This combination
of charges gives rise to an energy
\[
dU = \frac{qdq}{r}.
\]
Since $q = \rho(4\pi r^3)/3$,
\[
dU = \frac{(4\pi\rho)^2}{3}r^4dr
\]
so that, the energy of a uniformly charged sphere of radius $R$ is
\begin{equation}\label{e15}
U = \int_0^a dU = \frac{16\pi^2\rho^2}{3}\frac{a^5}{5}.
\end{equation}
Since
\[
\rho = \frac{3Q}{4\pi a^3},
\]
equation \eqref{e13} becomes
\begin{equation}\label{e16}
U = \frac{3}{5}\frac{Q^2}{a}.
\end{equation}

\item If $m$ is the mass of an electron and if an electron is considered to be a
sphere of radius $a$, then
\[
mc^2 =  \frac{3}{5}\frac{Q^2}{a}
\]
so that
\begin{equation}\label{e17}
a = \frac{3}{5}\frac{Q^2}{mc^2}
\end{equation}
is called the `classical electron radius'. This picture of an electron does not 
tell what holds so the negative charge together. This question is left unanswered
in classical physics.

\item A charge of $-1$ esu is at the origin and another charge of $2$ esu is at
$(1, 0)$. At a point $(x, 0)$, the electric field is
\[
\vec{E} = -\frac{1}{x^2}\uv{x} + \frac{2}{(x - 1)^2}\uv{x}
\]
so that $E = 0$ when
\[
\frac{1}{x^2} = \frac{2}{(x - 1)^2} \Rightarrow x = -1 \pm \sqrt{2}.
\]
For a point on the $y$ axis,
\[
\vec{E}(y) = -\frac{1}{y^2}\uv{y} + 2\frac{-\uv{x} + y\uv{y}}{(1 + y^2)^{3/2}} =
\frac{2}{(1 + y^2)^{3/2}}\uv{x} + \left(\frac{2y}{(1 + y^2)^{3/2}} - \frac{1}{y^2}\right)\uv{y}.
\]
$\vec{E}$ will be parallel to $x$ axis if the $y$ component vanishes. That will
happen if $2y^3 = (1 + y^2)^{3/2}$ or $4y^6 = (1 + y^2)^3 = 1 + 3y^2 + 3y^4 + y^6$,
or
\[
3y^6 - 3y^4 - 3y^2 - 1 = 0.
\]
The only real roots of this equation are $y = \pm 1.30476603$. The roots can be
found using
\begin{verbatim}
import numpy as np

coeffs = [3, 0, -3, 0, -3, 0, -1]
[r for r in np.roots(coeffs) if np.isreal(r)]
\end{verbatim}

\item Let the equilateral triangle have vertices at $(\pm a, 0)$ with positive
ions and $(0, a\sqrt{3})$ with a negative ion. The electric field at a point
$(x, y)$ is
\begin{eqnarray*}
\vec{E} &=& \uv{x}\left[\frac{x-a}{((x-a)^2+y^2)^{3/2}}+
            \frac{x+a}{((x+a)^2+y^2)^{3/2}}-\frac{x}{(x^2+(y-a\sqrt{3})^2)^{3/2}}\right]+\\
 & &\uv{y}\left[\frac{y}{((x-a)^2+y^2)^{3/2}}+\frac{y}{((x+a)^2+y^2)^{3/2}}-
    \frac{y - a\sqrt{3}}{(x^2+(y-a\sqrt{3})^2)^{3/2}}\right]
\end{eqnarray*}
Symmetry suggests that the equilibrium point is likely to be on the $y$-axis. The
$x$ component of the field is zero at all points on the $y$-axis. The $y$-component
is
\[
E_y = \frac{2y}{(a^2+y^2)^{3/2}} - \frac{1}{(y-a\sqrt{3})^2}
\]    
$E_y = 0$ if $4y^2(y - a\sqrt{3})^4 = (a^2 + y^2)^3$, that is,
\begin{equation}\label{e18}
3y^6 - 24a\sqrt{3}y^5 + 117 a^2y^4 - 72\sqrt{3}a^3y^3 - 3a^4y^2 + (36a^4 - a^6) = 0.
\end{equation}
If we choose $a = 1$ then the roots of this equation are found using
\begin{verbatim}
import numpy as np

coeffs = [3, -24*np.sqrt(3), 117, -72*np.sqrt(3), -3, 0, 35]
[np.round(r, 4) for r in np.roots(coeffs) if np.isreal(r)]
\end{verbatim}
They are $y = 10.5277, 0.8501$.

\item Assume that the cloud is like an infinite plane. Then $E = 2\pi\sigma$ so
that the surface charge density on it is
\[
\sigma = \frac{0.1}{2\pi}\text{ esu/cm}^{2}.
\]
If $A$ is the area of the cloud then the volume of water in $0.25$ cm rain is
$V = 0.25A$. If $r$ is the radius of a rain drop, then the number of rain drops
in this volume is
\begin{equation}\label{e19}
N\times \frac{4\pi}{3}r^3 = 0.25A
\end{equation}
that is,
\begin{equation}\label{e20}
N = \frac{0.75A}{4\pi r^3}
\end{equation}
Let $q$ be the charge on each of them. Then, $Nq = \sigma A$ so that
\begin{equation}\label{e21}
\frac{0.75q}{4\pi r^3} = \frac{0.1}{2\pi}
\end{equation}
that is,
\begin{equation}\label{e22}
q = \frac{4}{15}r^3.
\end{equation}
The electric field strength at the surface of the drop has magnitude $q/r^2$, that
is,
\begin{equation}\label{e23}
E = \frac{4}{15}r.
\end{equation}
For $r = 0.1$ cm, it is $E = 0.0267$ esu/cm$^2$.

\item Let the ring be of radius $R$. Align the axes such that the origin coincides
with the centre, the ring is in the $xy$ plane so that the $z$ axis is perpendicular
to the ring. Before proceeding, 
\begin{itemize}
\item What is the field at $z = 0$?
\item What is it at $z \rightarrow \infty$?
\item Can you guess where the maximum will be?
\end{itemize}

Now let's do the maths. Fix a point $(0, 0, z)$ on the $z$ axis. Let a line joining
it and a point on the ring make an angle $\alpha$ with the $z$ axis. The electric
field due to an element of charge at that point will be along this line and make 
and angle $\alpha$ with the $z$ axis. Its sine component will be cancelled by the
field due to a diametrically opposite element. Therefore, it's only contribution
will be a cosine component, aligned along the $z$ axis. Thus,
\[
d\vec{E} = \frac{dq}{r^2}\cos\alpha\uv{z}.
\]
$r^2 = z^2 + R^2$, $dq = \lambda d\theta$, where $\lambda = Q/(2\pi R)$ is the
charge density. Thus,
\begin{equation}\label{e24}
\vec{E} = \frac{2\pi R\lambda}{R^2 + z^2}\cos\alpha\uv{z} = \frac{Qz}{(R^2 + z^2)^{3/2}}\uv{z}
\end{equation}
Therefore,
\[
\td{E}{z} = \frac{Q}{(R^2 + z^2)^{3/2}} - \frac{3}{2}\frac{Qz(2z)}{(R^2 + z^2)^{5/2}}
= Q\frac{R^2 - 2z^2}{(R^2 + z^2)^{5/2}}.
\]
The extremum is at $z = R/\sqrt{2}$ or $\cos\alpha = z/R = 1/\sqrt{2}$ or $\alpha=
\pi/4$.

\item For a point at distance $r \le a$, consider a spherical surface whose centre
coincides with that of the charge distribution and whose radius is $r$. If $\rho$
is the constant charge density then the surface encloses a charge
\[
q(r) = \frac{4\pi}{3}r^3 \rho
\]
By Gauss law, 
\[
E \times 4\pi r^2 = 4\pi \times \frac{4\pi}{3}r^3 \rho
\]
so that
\begin{equation}\label{e25}
\vec{E} = \frac{4\pi\rho}{3}r\uv{r} = \frac{4\pi\rho}{3}\vec{r}.
\end{equation}

For a point at a distance $r > a$, the charge enclosed is a constant
\[
Q = \frac{4\pi\rho}{3}a^3
\]
and
\begin{equation}\label{e26}
\vec{E} = \frac{Q}{r^2}\uv{r}.
\end{equation}
At $r = a$, \eqref{e26} is
\[
\vec{E}(a) = \frac{Q}{a^3}\vec{r} = \frac{4\pi\rho}{3}\vec{r},
\]
which is same as \eqref{e25}.

\item The field at $B$ will be radially outwards and will have a magnitude
\[
E = \frac{Q}{a^2},
\]
where
\[
Q = \frac{4\pi\rho}{3}a^3 - \frac{4\pi\rho}{3}\frac{a^3}{8} = 
\frac{7}{8}\frac{4\pi\rho}{3}a^3,
\]
Before carving out the cavity, the field at $A$ was zero. We can consider it to
be a sum of field in the matter in the carved out ball plus the field due to the
rest. The field due to the carved out ball is radially outwards from the centre
of the cavity and has magnitude 
\[
E_c = \frac{Q_c}{(a/2)^2} = \frac{4\pi\rho}{3}\frac{a}{2}.
\]
If $\vec{R}$ is the centre of the cavity then the radially outward direction at
a point $\vec{r}$ is $\vec{r} - \vec{R}$. Thus,
\[
\vec{E}_c = \frac{4\pi\rho}{3}\frac{a}{2}\frac{\vec{r} - \vec{R}}{|\vec{r} - \vec{R}|}.
\]
If the origin is chosen to coincide with the point $A$ then
\[
\vec{E}_c(A) = -\frac{4\pi\rho}{3}\frac{a}{2}\hat{R}.
\]

\item When the charge is at the centre of the cube, by Gauss law,
\[
\oint\vec{E}\cdot d\vec{a} = 4\pi q.
\]
The symmetry of the cube suggests that there is an equal flux through each face.
Therefore, through any one of them, it is
\[
\int\vec{E}\cdot d\vec{a} = \frac{2\pi q}{3}.
\]

When the charge is placed at a corner of a cube, assemble eight similar cubes so
that the charge is now at the centre of them. The flux through any one of them is
$4\pi q/8 = \pi q/2$. The field will be parallel to the faces at whose intersection
$q$ is. Therefore, there will be no flux out of them. The remaining three faces
are symmetric and will carry the flux equally. Therefore, the flux through any
one of them is $\pi q/6$.

\item The field due to each plate is
\[
\vec{E} = \begin{cases}
 \sigma\un & \text{ on one side.}\\
-\sigma\un & \text{ on the other side.}
\end{cases}
\]
where $\un$ is the same vector in each case.

When both plates are parallel, say to the $xy$-plane, the fields in the intervening
portion are $6 \uv{z}$ esu/cm$^{2}$ due to the lower plate and $(-4)(-\uv{z}) = 
4\uv{z}$ esu/cm$^{2}$  due to the upper one. The net field is $10 \uv{z}$ esu/cm
$^{2}$.

When the plates are perpendicular to each other, let one of them, say the 
positively charged one, be along the $y$ axis and the other one along the $x$
axis. Therefore, the fields in the four quadrants are
\[
\vec{E} = \begin{cases}
 6\uv{x} - 4 \uv{y} \text{ in the 1st quadrant} \\
-6\uv{x} - 4 \uv{y} \text{ in the 2nd quadrant} \\
-6\uv{x} + 4 \uv{y} \text{ in the 3rd quadrant} \\
 6\uv{x} + 4 \uv{y} \text{ in the 4th quadrant}
\end{cases}
\]

\item Let the infinite surface be aligned along the $y$ axis. At any point in
the $-x$ direction, Gauss law gives,
\[
\oint\vec{E}\cdot d\vec{a} = 4\pi(\sigma A + \rho Ad)
\]
or
\[
\vec{E} = -4\pi(\sigma + \rho d)\uv{x}
\]
At $x >= d$, by similar reasoning, $\vec{E} = 4\pi(\sigma + \rho d)\uv{x}$. If
$0 < x < d$, $\vec{E} = 4\pi\sigma\uv{x}$. The charge inside the volume charge
does not contribute anything. This can be seen by taking two Gaussian surfaces
ending at the field point, one extending leftwards and the other rightwards.

\item Align the $z$ axis along the cylinder's principle axis. At any point inside
the cylinder, consider a small cylindrical surface of length $l$, coaxial
with the charged cylinder. Since it does not enclose any charge, the electric
field at it is zero. 

For a point outside the cylinder, consider a similar gaussian surface. By symmetry,
the field is normal to its closed surface. If the point is at a distance $s$ from
the $z$ axis and if the cylinder has length $l$ then
\[
\oint\vec{E}\cdot d\vec{a} = 4\pi q \Rightarrow E(2\pi sl) = 4\pi\sigma(2\pi s_0l),
\]
where $s_0$ is the radius of the charged cylinder. Thus,
\[
E = \frac{4\pi\sigma s_0}{s}
\]
and
\[
\vec{E} = \frac{4\pi\sigma s_0}{s}\un,
\]
where $\un$ is the normal to the gaussian surface, directed outwards. Now, $2\pi
\sigma s_0$ is also the charge per unit length of the cylinder. If we denote it 
by $\lambda$, then
\[
\vec{E} = \frac{2\lambda}{s}\un,
\]
which is same as the expression for the electric field due to an infinitely long
wire carrying a charge with uniform linear density $\lambda$.

Now consider an pipe of a square cross section. The field inside continues to be
zero for the same reason. The field will be zero inside but the field outside will
not be depend only on $s$, the distance from the axis but will also depend on the
azimuthal angle. Therefore, the field will \emph{not} be line a uniformly charge
long thin wire.

Some remarks:
\begin{itemize}
\item A similar thing happens when we consider the field outside a non-spherical
body. The field is no longer that due to the entire charged being concentrated at 
the origin. In fact, it has factors with power $r^{-n}$, where $n > 2$.
\item This is true for any field for which Gauss law is valid. For example, the
slight deviation of the sun from perfect sphere makes it field have factors other
than those with $r^{-2}$ and they lead to measurable effects on planetary orbits.
\end{itemize}


\item The electric field at any point in the atom is
\[
\vec{E} = \vec{E}_e + \vec{E}_n,
\]
where
\[
\vec{E}_n = \frac{q_e}{r^2}\uv{r},
\]
is the nuclear electric field and $\vec{E}_e$ is the electronic electric field.
Here we denote the electronic charge by $q_e$ because we will also see the irrational
number $e$ in our expressions. Since,
\[
\int_0^\infty \rho dV = -q_e,
\]
we have
\[
\int_0^r dr \int_0^\pi d\theta \int_0^{2\pi}d\phi Cr^2\sin\theta  e^{-2r/a} = 
4\pi C\int_0^\infty dr r^2 e^{-2r/a} = 4\pi C\frac{a^3}{4} = -q_e
\]
so that 
\[
C = -\frac{q_e}{\pi a^3}.
\]
Thus, the electronic charge density is
\[
\rho = -\frac{q_e}{\pi a^3}e^{-2r/a}.
\]
The charge in a sphere of radius $a$ is
\begin{eqnarray*}
q &=& \int_0^a dr\int_0^\pi d\theta \int_0^{2\pi}d\phi r^2\sin\theta \rho \\
 &=& -\frac{4\pi q_e}{\pi a^3}\int_0^a r^2 e^{-2r/a}dr \\
 &=& -\frac{4q_e}{a^3}\left(\frac{e^2 - 5}{4e^2}a^3\right) \\
 &=& -q_e\left(\frac{e^2 - 5}{4e^2}\right)
\end{eqnarray*}
By Gauss law,
\[
E_e(4\pi a^2) = -q_e\left(\frac{e^2 - 5}{4e^2}\right)
\]
so that
\[
\vec{E} = \frac{q_e}{a^2}\left(1 - \left(\frac{e^2 - 5}{16\pi e^2}\right)\right)\uv{r}
= 0.99357\frac{q_e}{a^2}\uv{r}.
\]

\item The electric fields due to the sheets are:
\begin{itemize}
\item $-(2\pi)4\uv{z}$ on the upper side and $(2\pi)4\uv{z}$ on the lower side of $A$,
\item $(2\pi)7\uv{z}$ on the upper side and $-(2\pi)7\uv{z}$ on the lower side of $B$,
\item $-(2\pi)3\uv{z}$ on the upper side and $(2\pi)3\uv{z}$ on the lower side of $C$,
\end{itemize}
where we aligned the axes such that the $z$ axis is normal to each of them.

We will now compute the pressure on each sheet.
\begin{itemize}
\item To compute pressure on $A$, consider a small patch of area $a$ on it. The 
total charge on the patch is $-4a$. The electric field on it is $14\pi\uv{z}$
due to chage on $B$ and $-8\pi\uv{z}$ due to $C$, or a net field $8\pi\uv{z}$.
Therefore, the total force is $32\pi a\uv{z}$. The electric pressure is $32\pi$
dynes/cm$^{-2}$.
\item The electric field on upper side of $B$ is $8\pi\uv{z}$ and on its lower 
side it is $-6\pi\uv{z}$. For a patch of area $a$, the electric force is $56\pi a
\uv{z}$ on the upper side and $-42\pi a\uv{z}$ on the lower side. The electric
pressure is $14\pi$ dynes/cm$^{-2}$.
\item Similarly, the pressure on the sheet $C$ is $18\pi$ dynes/cm$^{-2}$.
\end{itemize}

\item The electric field of the shell is
\[
\vec{E} = \begin{cases}
\frac{Q}{r^2}\uv{r} & \text{ for } r \ge R \\
0 & \text{ for } r < R.
\end{cases}
\]
The energy density of the field is 
\[
u_E = \frac{E^2}{8\pi} = \begin{cases} \frac{Q^2}{8\pi r^4} & \text{ for } r \ge R \\
0 & \text{ for } r < R.
\end{cases}
\]
Therefore, the energy in a sphere of radius $S$ is
\[
\int_0^S u_EdV = \int_R^S \frac{Q^2}{8\pi r^4} = \frac{Q^2}{2}\int_R^S\frac{dr}{r^2}
= \frac{Q^2}{2}\left(\frac{1}{R} - \frac{1}{S}\right).
\]
If $S = \infty$ the total energy is $Q^2/(2R)$. We what to find out $S$ for 
which the energy is $0.9Q^2/(2R)$. Thus,
\[
0.9\frac{Q^2}{2R} = \frac{Q^2}{2}\left(\frac{1}{R} - \frac{1}{S}\right)
\Rightarrow 0.9 = 1 - \frac{R}{S} \Rightarrow S = 10R.
\]

\item At the point $A$, the magnitude of the field is
\[
dE = \frac{dq}{x^2},
\] W
where $dq$ is the charge between $x$ and $x + dx$. Since the charge is uniformly
distributed, $dq = \lambda dx$, where $\lambda = Q/10$ is the linear charge density.
Therefore,
\[
E = \int_3^{10}\frac{\lambda dx}{x^2} = \frac{Q}{10}\left(\frac{1}{3} - \frac{1}{10}\right)
= \frac{7Q}{300}.
\]

At the point $B$,
\[
dE = \frac{dq}{r^2}\cos\theta,
\]
where $r$ is the distance of $B$ from an element of charge $dq$ between $x$ and
$x + dx$ and $\theta$ is the angle between the line joining $B$ with the element
and the perpendicular from $B$ to the rod. Thus,
\[
\cos\theta = \frac{L}{r}, \tan\theta = \frac{x}{L},
\]
$L$ being the distance of $B$ from the centre of the rod. Thus, $dx = L\sec^2\theta
d\theta$ and
\[
dE = \frac{\lambda L \sec^2\theta \cos^2\theta}{L^2}\cos\theta d\theta =
\frac{\lambda}{L}\cos\theta d\theta.
\]
If $\sin\alpha = a/L$, $2a$ being the length of the rod, then the limits of 
integration are from $-\alpha$ to $\alpha$ so that
\[
E = \frac{\lambda}{L}(2\sin\alpha) = \frac{2a\lambda}{L^2} = \frac{Q}{L^2}.
\]

\item The magnitude of the electric field at the centre because of the two lines
is
\[
E = \frac{1}{4\pi\epsilon_0}\frac{4\lambda}{r}.
\]
Use 
\[
r = \frac{3}{2}, \frac{1}{4\pi\epsilon_0} = 9 \times 10^9
\]
to get $\lambda = 6.25 \times 10^{-7}$ C/m. Therefore, the charge on a $1$ km
line is $6.25 \times 10^{-4}$ C.

$1$ C is a very large amount of charge.

\item The contribution of the an element of charge on the semi-circle is
\begin{eqnarray*}
d\vec{E}_s &=& \frac{dq}{b^2}(\cos(2\pi - \theta)\uv{x} + \sin(2\pi - \theta)\uv{y}) \\
 &=& \frac{dq}{b^2}(\cos\theta\uv{x} - \sin\theta\uv{y}) \\
 &=& \frac{\lambda bd\theta}{b^2}(\cos\theta\uv{x} - \sin\theta\uv{y}) \\
 &=& \frac{\lambda}{b}(\cos\theta\uv{x} - \sin\theta\uv{y})d\theta
\end{eqnarray*}
Therefore,
\[
\vec{E}_s = \int_0^\pi dE_s = -\frac{2\lambda}{b}\uv{y}.
\]
The contribution of charge $dq = \lambda dy$ on the left arm of the rod is
\[
d\vec{E}_l = \frac{dq}{r^2}(cos\theta\uv{x} + \sin\theta\uv{y}) 
= \frac{\lambda dy}{r^2}(cos\theta\uv{x} + \sin\theta\uv{y}).
\]
Since $y = r\sin\theta$, $dy = r\cos\theta d\theta$ and
\[
d\vec{E}_l = \frac{\lambda\cos\theta}{r}(cos\theta\uv{x} + \sin\theta\uv{y})d\theta
= \frac{\lambda}{b}(cos\theta\uv{x} + \sin\theta\uv{y})d\theta
\]
so that
\[
\vec{E}_l = \int_0^{\pi/2}d\vec{E}_l = \frac{\lambda}{b}(\uv{x} + \uv{y})
\]
Similarly,
\begin{eqnarray*}
\vec{E}_r &=& \int_{\pi/2}^\pi \frac{\lambda}{b}(cos(\pi/2 + \theta)\uv{x} + \sin(\pi/2 +\theta)\uv{y})d\theta \\
 &=& \int_{\pi/2}^\pi \frac{\lambda}{b}(-\sin\theta\uv{x} + \cos\pi/2\uv{y})d\theta \\
\end{eqnarray*}


\item 
A few points to ponder:
\begin{itemize}
\item Does it take energy to pluck a charge from this lattice?
\item What if all charges were of the same size?
\end{itemize}

Consider a $3 \times 3$ lattice of alternating positive and negative charges
with a positive charge at the centre. If the lattice is in the $xy$ plane with the
positive charge at the origin then the potential energy of a unit charge at a point
$(0, 0, d)$ is
\[
u_1 = -\frac{4}{\sqrt{d^2 + s^2}} + \frac{4}{\sqrt{d^2 + 2s^2}}.
\]
If the lattice is extended to have a size $5 \times 5$  then there is an additional
contribution of
\[
u_2 = -\frac{4}{\sqrt{d^2 + 3s^2}} + \frac{4}{\sqrt{d^2 + 4s^2}}.
\]
In general, an $(2n + 1) \times (2n + 1)$ lattice has an additional contribution
of 
\[
u_n =  -\frac{4}{\sqrt{d^2 + (2n - 1)s^2}} + \frac{4}{\sqrt{d^2 + (2n)s^2}}
\]
over the $(2n) \times (2n)$ lattice. Thus, the total energy due to a $(2n + 1) 
\times (2n + 1)$ is
\[
U_n = \sum_{k=1}^n\left(-\frac{4}{\sqrt{d^2 + (2n - 1)s^2}} + \frac{4}{\sqrt{d^2 + (2n)s^2}}\right).
\]
We expect $U_n$ to be negative because when the central positive charge is plucked
from the lattice, it has an excess of negative charge which will attract the 
plucked charge. Therefore, we will have to supply energy to take that plucked charge to
infinity.

We can examine the convergence of the series $U_n$ numerically using the code:
\begin{verbatim}
import matplotlib.pyplot as plt
import numpy as np

def nth_term(n: int) -> float:
    d = 2
    s = 1
    x = 4*n*(1/np.sqrt(d**2 + (2*n)**2) - 1/np.sqrt(d**2 + (2*(n - 1))**2))
    return x


def show_plots(X: np.array,
               contribs: np.array,
               U: np.array) -> None:
    plt.plot(X, U, label=r'$U(n)$')
    plt.plot(X, contribs, label=r'$\delta U(n)$')    
    plt.xlabel('n')
    plt.title(r'Potential energy of 2-d lattice')
    plt.legend(loc='best')
    plt.savefig('purcell_c1p27.png')


def convergence(n: int) -> None:
    template = 'At n={0:.0e}, the additional contribution to U is {1:f}.'
    print(template.format(n, nth_term(n)))


def main():
    N = 1000
    contribs = [nth_term(n) for n in range(1, N + 1)]
    U = np.cumsum(contribs)
    X = [n for n in range(1, N + 1)]
    show_plots(X, contribs, U)
    convergence(1e6)
    convergence(1e23)
    

if __name__ == '__main__':
    main()
\end{verbatim}
\begin{figure}[!ht]
\center
\includegraphics[scale=0.6]{purcell_c1p27}
\caption{Potential energy of a 2-d lattice.}
\label{c1f2}
\end{figure}
The convergence is slow. However, for realistic lattice sizes the number of charges
is of the order $10^{23}$. The contribution of additional terms at this size is 
practically zero.

\item Figure \ref{c1f3} shows a regular octahedron.
\begin{figure}
\center
\includegraphics[scale=0.6]{octahedron}
\caption{A regular octabedron.}
\label{c1f3}
\end{figure}
If $a$ is the length of each side of the octahedron, the distance between $A$ and $C$
(and $B$ and $D$) is $d\sqrt{2}$. Similarly, the distance between $E$ and $F$ is
twice the distance between $O$ and $F$, where $O$ is the centre of the square $ABCD$.
Distance between $A$ and $O$ is $d\sqrt{2}/2 = d/\sqrt{2}$ so that distance 
between $O$ and $F$ is
\[
\sqrt{d^2 - \frac{d^2}{2}} = \frac{d}{\sqrt{2}}
\]
and distance between $E$ and $F$ is $d\sqrt{2}$. Thus, the distance between all 
non-adjacent points is the same.

We will, therefore, consider the following configurations:
\begin{itemize}
\item Positive charges at $E$ and $F$, negative charges at $A$ and $C$, a positive
charge at $D$ and a negative charge at $B$. This arrangement tries to keep like 
charges as far away as possible. The energy is a sum of
\begin{table}[ht!]
\center
\begin{tabular}{ll}
Positions & Energy  \\
\hline
AB & $1/d$ \\
AC & $1/(d\sqrt{2})$ \\
AD & -$1/d$ \\
AE & -$1/d$ \\
AF & -$1/d$ \\
BC & $1/d$ \\
BD & -$1/(d\sqrt{2})$ \\
BE & -$1/d$ \\
BF & -$1/d$ \\
CD & -$1/d$ \\
CE & -$1/d$ \\
CF & -$1/d$ \\
DE & $1/d$ \\
DF & $1/d$ \\
EF & $1/(d\sqrt{2})$
\end{tabular}
\caption{A -, B -, C -, D +, E +, F +}
\end{table}
so that
\[
U_1 = \left(-\frac{4}{d} + \frac{1}{d\sqrt{2}}\right) = \frac{e^2}{d}\left(-4 + \frac{1}{\sqrt{2}}\right)
= -3.2929\frac{e^2}{d}.
\]

\item Positive charges are A, B, F and negative charges elsewhere.
\begin{table}[ht!]
\center
\begin{tabular}{ll}
Positions & Energy  \\
\hline
AB & $1/d$ \\
AC & -$1/(d\sqrt{2})$ \\
AD & -$1/d$ \\
AE & -$1/d$ \\
AF & -$1/d$ \\
BC & -$1/d$ \\
BD & -$1/(d\sqrt{2})$ \\
BE & -$1/d$ \\
BF & $1/d$ \\
CD & $1/d$ \\
CE & $1/d$ \\
CF & -$1/d$ \\
DE & $1/d$ \\
DF & -$1/d$ \\
EF & -$1/(d\sqrt{2})$
\end{tabular}
\caption{A +, B +, C -, D -, E -, F +}
\end{table}
so that
\[
U_2 = e\left(-\frac{3}{d\sqrt{2}}\right) = -2.1213\frac{e^2}{d}.
\]
\end{itemize}
We can confirm that these are the only two states using the code:
\begin{verbatim}
import numpy as np
import pandas as pd
import math

from itertools import combinations
from typing import Dict, Set

positions = {_ for _ in 'ABCDEF'}
M = np.array(
    [[0, 1, np.sqrt(2), 1, 1, 1],
    [1, 0, 1, np.sqrt(2), 1, 1],
    [np.sqrt(2), 1, 0, 1, 1, 1],
    [1, np.sqrt(2), 1, 0, 1, 1],
    [1, 1, 1, 1, 0, np.sqrt(2)],
    [1, 1, 1, 1, np.sqrt(2), 0]])
labels = [_ for _ in positions]
distances = pd.DataFrame(data=M, index=labels, columns=labels)

def U(positives: Set[str], negatives: Set[str]) -> float:
    x = 0
    for p in positives:
        for q in positives:
            if p != q:
                x += 1/distances[p][q]

    for n in negatives:
        for m in negatives:
            if n != m:
                x += 1/distances[n][m]
    x /= 2 # To account for double counting
    
    for p in positives:
        for n in negatives:
            if p != n:
                x -= 1/distances[p][n]
    # No double counting happened in the previous loops.

    return x


def config_name(positives: Set[str], negatives: Set[str]) -> str:
    return  ''.join([_ + '+' for _ in positives]) + \
            ''.join([_ + '-' for _ in negatives])


def get_all_microstates() -> Dict[str, float]:
    microstates = {}
    for c in combinations(positions, 3):
        positives = {_ for _ in c}
        negatives = positions - positives    
        name = config_name(positives, negatives)
        microstates[name] = U(positives, negatives)

    return microstates


def show_all_microstates(ms: Dict[str, float]) -> None:
    keys = list(ms.keys())
    keys.sort()

    for k in keys:
        print('U({0}) = {1}'.format(k, np.round(ms[k], 4)))


def main():
    ms = get_all_microstates()
    show_all_microstates(ms)


if __name__ == '__main__':
    main()    
\end{verbatim}

\item Very close to any point on the spherical shell, the field is $4\pi\sigma\un$,
where $\sigma$ is the surface charge density. Note that there is no field inside
the shell. Therefore, the Gauss law applied to a small pillbox gives,
\[
\oint\vec{E}\cdot d\vec{a} = 0 + EA = 4\pi\sigma A.
\]
To achieve the effect of a hole, we attach a patch of surface density $-\sigma$.
The patch will have a field $\vec{E} = -2\pi\sigma\un$ where $\un$ is always the
outward pointing normal. After placing it on the shell, the net field at its 
centre is $4\pi\sigma\un - 2\pi\sigma\un = 2\pi\sigma\un$, pointing outwards.
We can also add the zero field inside the shell to $-2\pi\sigma\un$, with $\un$
now pointing inside the shell so that the net field is once again $2\pi\sigma\un$.

\item The electric field due to the smaller sphere in the region between the
two is
\[
E(r) = \frac{Q}{r^2} \;\; a \le r \le b
\]
so that the energy stored in the cavity is
\[
U = \int_a^b\int_0^\pi\int_0^{2\pi} E^2 r^2\sin\theta drd\theta d\phi =
4\pi Q^2\left(\frac{1}{a} - \frac{1}{b}\right).
\]

\item The field just outside the bubble is
\[
\vec{E} = \frac{Q}{R^2}\uv{r}
\]
and that inside is $0$ so that the average field is
\[
\vec{E}_s = \frac{Q}{2R^2}\uv{r}.
\]
The force on the surface is 
\[
\vec{F} = Q\vec{E}_s = \frac{Q^2}{2R^2}\uv{r}
\]
and the pressure on it is 
\[
P = \frac{Q^2}{8\pi R^4}.
\]
The force on a small area $d\vec{a}$ on the upper hemisphere is 
\[
d\vec{f} = Pd\vec{a}
\]
and its projection \textbf{normal to the} $xy$ plane is 
\[
df = P\cos\theta d\vec{a} = PR^2\sin\theta\cos\theta d\theta d\phi
\]
so that the total pressure on the $xy$ plane is
\[
f = \int_0^{2\pi} d\phi \int_0^{\pi/2} d\theta PR^2\sin\theta\cos\theta =
PR^2 \times 2\pi \times \frac{1}{2} = \pi R^2 P = \frac{Q^2}{8R^2}.
\]
This is the force exerted by the upper hemisphere on the lower one.

\item Align the axes such that the origin coincides with the centre of the circle,
one of the charges is at $(1, 0)$, the other at $(0, 1)$ and the third one, whose
magnitude is different from the previous two is the point $(-1/\sqrt{2}, 
-1/\sqrt{2})$, if the radius of the circle is $1$. Let $q$ be the magnitude of
the first two charges and $Q$ that of the third one. 

The force on the charge at $r\uv{x}$ is
\[
\vec{F} = \frac{qQ}{s_1^3}\vec{s}_1 + \frac{q^2}{2\sqrt{2}}(\uv{x} + \uv{y})
\]
where
\[
\vec{s}_1 = \left(1 + \frac{1}{\sqrt{2}}\right)\uv{x} + \frac{1}{\sqrt{2}}\uv{y}
\]
At equilibrium, $\vec{F} = 0$. Therefore, since the magnitude of the $y$-component
is also zero, we get
\[
\frac{Q}{q} = \frac{2^{3/4}(1 + \sqrt{2})^{3/2}}{\sqrt{2}} = 
(1 + \sqrt{2})\left(1 + \frac{1}{\sqrt{2}}\right)^{1/2} \approx 3.15432.
\]

\item Two protons are to be embedded in the electron jelly so that the net force
on each one of them is zero. The force on a proton will be that from the electron
jelly and the other proton. If the proton is at a distance $r$ from the centre 
then the proton will ``feel'' the force only due to the charge within a sphere of
radius r. Furthermore, the force will be as if the entire charge were concentrated
at the origin. If $\rho$ is the density of the electron jelly then the force on 
the proton is
\[
\vec{F}_- = \frac{e}{r^2}\rho\frac{4\pi}{3}r^3\uv{r} = \frac{4\pi\rho e}{3}r\uv{r}.
\]
To nullify this force, the other proton must be along the same line. If it is also
at the same distance $r$ from the centre, but on the other side, then
\[
\vec{F}_+ = \frac{e^2}{4r^2}\uv{r}.
\]
Now 
\[
\rho = \frac{-2e}{4\pi/3 a^3} = -\frac{3e}{2\pi a^3}
\]
so that $\vec{F}_- + \vec{F}_+ = 0$ implies
\[
-\frac{2e^2r}{a^3} + \frac{e^2}{4r^2} = 0 \Rightarrow r = \frac{a}{2}.
\]

\item Wherever it is possible, one should prefer to work with scalar functions.
The energy of the charge configuration is
\[
U(\theta) = \frac{Qq}{a} + \frac{Qq}{a} + \frac{Q^2}{2a\cos\theta} + \frac{q^2}{2a\sin\theta}
\]
so that
\[
U\op = -\frac{q^2}{2a}\frac{\cos\theta}{\sin^2\theta} + 
\frac{Q^2}{2a}\frac{\sin\theta}{\cos^2\theta}
\]
and therefore, the extremum corresponds to
\[
\tan^3\theta = \frac{q^2}{Q^2}.
\]

\item Align the coordinates axes such that one of the protons is at the origin
and the other is at $(0, 0, b)$. At a point $\vec{r}$, the electric fields due
to the two protons will be
\begin{eqnarray*}
\vec{E}_1 &=& \frac{e}{r^2}\uv{r} \\
\vec{E}_2 &=& \frac{e}{|\vec{r} - b\uv{z}|^3}(\vec{r} - b\uv{z})
\end{eqnarray*}
so that
\[
\vec{E}_1\cdot\vec{E}_2 = e^2\frac{r - b\cos\theta}{r^2|\vec{r} - b\uv{z}|^3}.
\]
Now $|\vec{r} - b\uv{z}|^2 = (\vec{r} - b\uv{z})\cdot(\vec{r} - b\uv{z}) = r^2 + 
b^2 - 2rb\cos\theta$ so that
\[
\vec{E}_1\cdot\vec{E}_2 = e^2\frac{r - b\cos\theta}{r^2(r^2 + b^2 - 2rb\cos\theta)^{3/2}}.
\]
Let
\begin{eqnarray*}
I &=& \int \vec{E}_1\cdot\vec{E}_2 dv \\
  &=& \int_0^\infty\int_0^{\pi}\int_0^{2\pi}\vec{E}_1\cdot\vec{E}_2 r^2\sin\theta drd\theta d\phi \\
  &=& 2\pi\int_0^\infty\int_0^\pi e^2\frac{r - b\cos\theta}{(r^2 + b^2 - 2rb\cos\theta)^{3/2}}\sin\theta drd\theta
\end{eqnarray*}
Let $u = r^2 + b^2 - 2rb\cos\theta$ so that $du = 2(r - b\cos\theta)dr$ and then $r=0$, $u=b^2$ so that
\begin{eqnarray*}
I &=& 2\pi\int_{b^2}^\infty \int_0^\pi \frac{e^2}{2}\frac{du}{u^{3/2}} \sin\theta d\theta \\
  &=& \pi e^2 \frac{2}{b}\int_0^\pi \sin\theta d\theta \\
  &=& \frac{4\pi e^2}{b}
\end{eqnarray*}
so that 
\[
\frac{1}{4\pi}\int \vec{E}_1\cdot\vec{E}_2 dv = \frac{e^2}{b}.
\]
\end{enumerate}
\end{document}
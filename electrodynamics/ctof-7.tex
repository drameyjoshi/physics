\chapter{The propagation of light}\label{c7}
\begin{enumerate}
\item A plane wave is the one for which the fields are a function only of $t - x/c$
is they are propagating along the $x$ axis. Its direction of propagation and amplitude
are the same everywhere.

\item Oftentimes, one can treat arbitrary waves as plane waves in a small enough
region of space. This is possible only if the amplitude and direction remain 
practically the same over a single wavelength. It this is indeed true then one 
can introduce the idea of a \emph{wave surface} or a \emph{wave front}. It is an
imaginary surface at all points of which the wave has a constant phase. Wave fronts
of plane waves are indeed planes.

In a tiny region of space a wave propagates along the normal to the wave front.
The normals of consecutive wave fronts create a \emph{ray}. Alternatively, a ray
is a curve whose tangent at every point coincides with the normal to the wave
front.

\item When the dimensions of the region of interest are large compared to the
wavelength, we can study the propagation of waves by paying attention only to
the rays. This is the domain of \emph{geometrical optics}. It is an approximation
in which we ignore the wave properties and consider only rays. Geometrical optics
works best for waves whose wavelengths are very small.

\item If $f$ is any quantity describing the wave then for a plane monochromatic 
wave,
\begin{equation}\label{c7e1}
f = a\exp(-i(k_\mu x^\mu + \alpha)),
\end{equation}
where $\alpha$ is the constant phase of the wave. The 4-vector $k^\mu$ is defined
by equation \eqref{c6e54}. Strictly speaking, we should have written \eqref{c7e1}
as
\[
f = a\re{\exp(-i(k_\mu x^\mu + \alpha))},
\]
but we will now assume that every equation with complex exponentials is 
interpreted by ignoring its imaginary part.

If 
\begin{equation}\label{c7e2}
\Psi = -k_\mu x^\mu + \alpha
\end{equation}
then $f = a\exp(i\Psi)$. It is called the \emph{eikonal}.

\item If we are dealing with waves whose wave fronts are not planes then their
amplitude is a function of $x^\mu$ and the expression for $\Psi$ is not as 
simple as in \eqref{c7e2}.

\item Whatever be the form of $\Psi$, the approximation of geometrical optics
is applicable only if $\Psi$ is practically constant over a single wave-length, or
\begin{equation}\label{c7e3}
\frac{\lambda}{\Psi} \rightarrow 0.
\end{equation} 
In regions whose extent is of the order of magnitude of $\lambda$, one can write
the eikonal as 
\begin{equation}\label{c7e4}
\Psi = \Psi(0) + x^\mu\pdt{\Psi}{x^\mu}.
\end{equation}
Comparing with \eqref{c7e5}, we see that
\begin{equation}\label{c7e5}
k_\mu = -\pdt{\Psi}{x^\mu}.
\end{equation}
From \eqref{c6e57}, $k_\mu k^\mu = 0$ so that we also have
\begin{equation}\label{c7e6}
\pdt{\Psi}{x^\mu}\pdt{\Psi}{x_\mu} = 0.
\end{equation}
Equation \eqref{c7e6} is called the \emph{eikonal equation}.

\item Since $f = a\exp(i\Psi)$ satisfies the wave equation, 
\[
\frac{\partial^2 f}{\partial x_\mu \partial x^\mu} = 0.
\]
Now,
\[
\pdt{f}{x_\mu} = \pdt{a}{x_\mu}\exp(i\Psi) + ia\exp(i\Psi)\pdt{\Psi}{x^\mu} =
\pdt{a}{x_\mu}\exp(i\Psi) + if\pdt{\Psi}{x_\mu}
\]
so that
\begin{eqnarray*}
\frac{\partial^2 f}{\partial x_\mu \partial x^\mu} &=& 
    \frac{\partial^2 a}{\partial x_\mu \partial x^\mu}e^{i\Psi} + 
    i\pdt{a}{x_\mu}\exp(i\Psi)\pdt{\Psi}{x^\mu} + i\pdt{f}{x^\mu}\pdt{\Psi}{x^\mu} + \\
 & & if\frac{\partial^2 \Psi}{\partial x_\mu \partial x^\mu} \\
 &=& \frac{\partial^2 a}{\partial x_\mu \partial x^\mu}e^{i\Psi} + 
    2i\pdt{a}{x_\mu}\exp(i\Psi)\pdt{\Psi}{x^\mu} -f\pdt{\Psi}{x^\mu}\pdt{\Psi}{x_\mu} + \\
 & &  if\frac{\partial^2 \Psi}{\partial x_\mu \partial x^\mu}
\end{eqnarray*}
The real and imaginary parts of this quantity should vanish independently. 
Therefore,
\begin{eqnarray}
\frac{1}{a}\frac{\partial^2 a}{\partial x_\mu \partial x^\mu} &=& 
  \pdt{\Psi}{x^\mu}\pdt{\Psi}{x_\mu} \label{c7e7} \\
\frac{2}{a}\pdt{a}{x_\mu}\exp(i\Psi)\pdt{\Psi}{x^\mu} &=& 
\frac{\partial^2 \Psi}{\partial x_\mu \partial x^\mu} \label{c7e8} 
\end{eqnarray}
In the domain of geometrical optics we also assume that the second derivative of
$a$ with respect to $x^\mu$ can be ignored so that the first of the previous pair
of equations gives the eikonal equation.

The book mentions that largeness of $\Psi$ leads one to that conclusion but I 
unable to see it that way.

Alternatively, since a small patch of an arbitrary wave front can be considered
to be a plane, and since for a plane wave $a$ is constant, the eikonal equation
follows immediately from \eqref{c7e7}.

\end{enumerate}

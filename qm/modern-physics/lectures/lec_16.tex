\documentclass{beamer}
\usefonttheme[onlymath]{serif}
\usepackage{graphicx, hyperref}
\hypersetup{colorlinks, urlcolor=blue}
\usepackage{subcaption}
\usetheme{Boadilla}

\newcommand{\pvec}[1]{\vec{#1}\mkern2mu\vphantom{#1}}

\title{Solid State Physics - 2}
\author{Amey Joshi}
\date{\today}

\begin{document}
\begin{frame}
\titlepage
\end{frame}

\begin{frame}
\frametitle{Band theory of solids}
\begin{enumerate}
\item When two H atoms come together, their 1s orbitals combine to form two 
molecular orbitals. One of them is symmetric in space variables and the other
anti-symmetric. The former is favoured because of its lower energy. Two atomic
orbitals combine to form two molecular orbitals.
\item If $N \sim O(10^{23})$ atoms come together, their atomic orbitals combine
to create those many `solid state' orbitals. Since the number is so huge, the 
orbitals are so closely spaced that they can be considered to form a continuous
band of allowed energies.
\item The valence electrons form the valence band. The higher oribitals form the
higher bands. Of these the one right above the valence band is called the 
conduction band.
\item Sometimes the bands overlap, sometimes they do not.
\end{enumerate}
\end{frame}

\begin{frame}
\frametitle{Consequences of band structure}
\begin{enumerate}
\item When the valence band and the conduction band overlap the electrons close
to the Fermi level have enough states available to get into when they acquire
energy from the applied electric field. Such solids are good conductors of 
electricity.
\item When the valence and the conduction band are separated, we say that there
is a band gap.
\item If the band gap is more than 1.2 eV, the solid is an insulator at room
temparature and below.
\item If the band gap is less than 1.2 eV, the solid is a semiconductor at room
temparature and above. It is an insulator at lower temperatures.
\item The band structure is usually very complicated. Bands may overlap in one
direction but not in another.
\end{enumerate}
\end{frame}

\begin{frame}
\frametitle{Optical properties of solids}
\begin{enumerate}
\item The size of the band gap determines if a crystal is opaque or transparent.
\item Visible light has energy range from 1.5 eV to 3eV.
\item Metals are opaque because there is no band gap. Photons of visible light
are readily aborbed because there are enough states available for the absorbing
electrons.
\item Insulators with a band gap in excess of 3 eV are transparent to visible
light. They may be opaque to ultra-violet or X-rays.
\item Semiconductors are opaque to visible light but transparent to infra-red.
\item These are properties of single crystals. Polycrystalline materials may be
opaque because of scattering of light at the crystal boundaries.
\end{enumerate}
\end{frame}

\begin{frame}
\frametitle{Electrons and holes - 1}
\begin{enumerate}
\item The relation between energy and wave vector of a free electron is
\begin{equation}\label{e1}
E = \frac{\hslash^2 k^2}{2m}
\end{equation}
from which we readily get
\begin{equation}\label{e2}
m = \left[\frac{1}{\hslash^2}\frac{d^2E}{dk^2}\right]^{-1}.
\end{equation}
\item For an electron in a crystal, $E$ is a complicated function of $\vec{k}$.
Such an electron is considered to have an `effective mass'
\begin{equation}\label{e3}
m^\ast_{ij} = \left[\frac{1}{\hslash^2}
  \frac{\partial^2E}{\partial k_i \partial k_j}\right]^{-1}
\end{equation}
\item It is a tensor and may have positive or negative components.
\end{enumerate}
\end{frame}

\begin{frame}
\frametitle{Electrons and holes - 2}
\begin{enumerate}
\item Holes are electrons with a negative effective mass.
\item In an applied electric field if electrons move in one direction the holes
go the other way.
\item The movement of holes appears like that of a positive charge. In reality
the electrons are moving.
\item Holes are the simplest examples of how electron's behaviour can be 
drastically altered when it is in a lattice.
\item Holes are not confined to semiconductors alone. Holes are responsible for
electrical conduction in Be, Mg, In, Al and Zn.
\end{enumerate}
\end{frame}

\begin{frame}
\frametitle{Band structure of semiconductors}
\begin{enumerate}
\item A pure semiconductor crystal is called an intrinsic semiconductor. Its 
band gap is small enough to permit excitation of electrons from the valence band
to the conduction band.
\item The electrons in the conduction band and the holes in the valence band 
contribute to the small current. 
\item The Fermi level is in the middle of the band gap.
\item If one adds a group V impurity like P, As, Sb then additional donor levels
are created just below the conduction band. The Fermi level shifts upward. 
These are n-type semiconductors.
\item If one adds group III impurity like B, Ga, Al then additional acceptor 
levels are created just above the valence band. The Fermi level shifts downward.
These are p-type semiconductors.
\item Solid state electronics and photonics are a result of the semiconductors'
peculiar band structure.
\end{enumerate}
\end{frame}

\begin{frame}
\frametitle{Ohm's law}
\begin{enumerate}
\item The resistance to electron's motion under the influence of an applied 
electric field comes from its scattering by the vibrating ions.
\item Lattice imperfections like impurities and crystal boundaries also 
contribute.
\item As temperature rises so do lattice vibrations and therefore the 
resistance.
\item Resistance is directly proportional to the length and inversely 
proportional to the area of cross section of the conductor.
\item Resistance of `good conductors' like Cu and Ag drops at low temperatures
but is still non-zero at 0K. On the other hand, moderately good conductors like
Hg and insulators like ceramics end up having zero resistance at low enough 
temperatures.
\item Electrical resistance is not a boring quantity as it appeared so far.
\end{enumerate}
\end{frame}

\begin{frame}
\frametitle{Superconductors - 1}
\begin{enumerate}
\item H. Kammerlingh Onnes discovered in 1911 that the resistance of Hg dropped
to zero below 4.15 K. Many other metals showed the same phenomenon.
\item The resistance is not just low. It is zero. Current loops set up in such
materials last for years.
\item If a superconductor is exposed to a magnetic field exceeding a certain 
value it loses superconductivity. The critical value of the magnetic field 
depends on temperature.
\item In Type I superconductor the superconductivity is lost entirely. In Type
II superconductors there are filaments in the crystal where superconductivity
exists even beyond the critical magnetic field. But there is another value of 
magnetic field beyond which even the residual superconductivity is lost.
\end{enumerate}
\end{frame}

\begin{frame}
\frametitle{Superconductors - 2}
\begin{enumerate}
\item Superconductivity in metallic superconductors is explained by the 
formation of a pair of electrons of opposite spins. It is called a Cooper pair
after its discoverer.
\item Electrons pair up in spite of having like charges. Yet another peculiarity
of electrons in a solid. An electron - phonon - electron interaction appears 
like an `attraction' between two electrons.
\item In the absence of a current, all Cooper pairs are in the ground state.
\item When an external potential is applied they all start moving together like
one giant system.
\item This is true only for certain solids called `conventional superconductors'.
\item Certain ceramics show superconductivity at much higher temperatures. That
phenomenon cannot be explained by Cooper pairs. High temperature 
superconductivity is an unsolved problem in solid state physics.
\end{enumerate}
\end{frame}
\end{document}
\chapter{Electromagnetic Waves}\label{c6}
\begin{enumerate}
\item Maxwell equations in vacuum are:
\begin{eqnarray}
\dive\vec{E} &=& 0 \label{c6e1} \\
\curl\vec{E} &=& -\frac{1}{c}\pdt{\vec{H}}{t} \label{c6e2} \\
\dive\vec{H} &=& 0 \label{c6e3} \\
\curl\vec{H} &=& \frac{1}{c}\pdt{\vec{E}}{t} \label{c6e4}
\end{eqnarray}
If the fields are also time independent, then we will have $\dive\vec{E} = 0,
\curl\vec{E} = 0, \dive\vec{H} = 0$ and $\curl\vec{H} = 0$. From equations
\eqref{c5e16} (Coulomb's law) and \eqref{c5e140} (Biot-Savart's law), we get
$\vec{E} = 0$ and $\tav{H} = 0$. Therefore, we assume that the fields depend
on time.

\item We will soon show that equations \eqref{c6e1} to \eqref{c6e4} have a
non-zero solution in vacuum. That is, electromagnetic fields can exist in vacuum.
We will also show that they satisfy the wave equation. Solutions of Maxwell
equations in vacuum are called electromagnetic waves.

\item As usual, it is easier to analyse the potentials. Let us choose them such
that $\varphi = 0$ and $\dive\vec{A} = 0$. Equation \eqref{c6e2} becomes
\[
-\frac{1}{c}\curl\pdt{\vec{A}}{t} = 
-\frac{1}{c}\frac{\partial}{\partial t}\curl\vec{A},
\]
an identity. While, equation \eqref{c6e4} becomes
\[
\curl\curl\vec{A} = -\frac{1}{c^2}\spdt{\vec{A}}{t}.
\]
The lhs of this equation is $\grad\dive\vec{A} - \nabla^2\vec{A}$. Since we chose
$\dive{\vec{A}} = 0$, we get
\begin{equation}\label{c6e5}
\nabla^2\vec{A} - \frac{1}{c^2}\spdt{\vec{A}}{t} = 0.
\end{equation}
The choice $\dive{\vec{A}} = 0$ is called the Coulomb gauge and that it can be
exercised without violating physics is explained in the point leading to 
\eqref{c5e134}. Taking the curl of this equation gives
\begin{equation}\label{c6e6}
\nabla^2\vec{H} - \frac{1}{c^2}\spdt{\vec{H}}{t} = 0.
\end{equation}
Taking a partial derivative of \eqref{c6e5} and recalling that we chose $\varphi
= 0$, gives
\begin{equation}\label{c6e7}
\nabla^2\vec{E} - \frac{1}{c^2}\spdt{\vec{E}}{t} = 0.
\end{equation}

\item We can repeat this analysis using the Maxwell equations written in terms of
the electromagnetic field tensor. From \eqref{c4e37},
\[
\pdt{F^{\mu\nu}}{x^\nu} = -\frac{4\pi}{c}j^\mu.
\]
In the absence of sources, it becomes
\[
\pdt{F^{\mu\nu}}{x^\nu} = 0.
\]
We now use the definition 
\[
F^{\mu\nu} = \pdt{A^\nu}{x_\mu} - \pdt{A^\mu}{x_\nu}
\]
to get
\[
\frac{\partial}{\partial x^\nu}\pdt{A^\nu}{x_\mu} - \frac{\partial}{\partial x^\mu}\pdt{A^\nu}{x_\nu} = 0
\Rightarrow
\frac{\partial}{\partial x_\mu}\pdt{A^\nu}{x^\nu} - \frac{\partial}{\partial x^\mu}\pdt{A^\nu}{x_\nu} = 0.
\]
We now choose $A^\mu$ such that
\begin{equation}\label{c6e8}
\pdt{A^\nu}{x^\nu} = 0.
\end{equation}
In terms of coordinates, it is
\begin{equation}\label{c6e9}
\frac{1}{c}\pdt{\varphi}{t} - \dive\vec{A} = 0.
\end{equation}
Equations \eqref{c6e8} and \eqref{c6e9} are called the \emph{Lorenz gauge}. Unlike
the condition of Coulomb gauge, that of the Lorenz gauge is invariant under Lorentz 
transformation. We are thus left with
\begin{equation}\label{c6e10}
\frac{\partial}{\partial x^\nu}\pdt{A^\mu}{x_\nu} = 
g^{\nu\rho}\frac{\partial}{\partial x^\nu}\pdt{A^\mu}{x^\rho} = 0.
\end{equation}
It is equivalent to \eqref{c6e5}.

\item The nine equations \eqref{c6e5}, \eqref{c6e6} and \eqref{c6e7} are all of
the form
\[
\nabla^2 f - \frac{1}{c^2}\spdt{f}{t} = 0 \Rightarrow 
\spdt{f}{t} - c^2\nabla^2 f = 0,
\]
where $f$ is one of the components of $\vec{A}, \vec{E}$ or $\vec{H}$. If the 
function $f$ depends only on $x$ and $t$, that is, if it is independent of $y$ and
$z$, then the solution $f$ is called a plane wave. In this case, the equation 
simplifies to
\begin{equation}\label{c6e11}
\spdt{f}{t} - c^2\spdt{f}{x} = 0.
\end{equation}
To solve this equation, write it as
\begin{equation}\label{c6e12}
\left(\frac{\partial}{\partial t} - c\frac{\partial}{\partial x}\right)
\left(\frac{\partial}{\partial t} + c\frac{\partial}{\partial x}\right)f = 0.
\end{equation}
Introduce the variables
\begin{eqnarray}
\xi &=& t - \frac{x}{c} \label{c6e13} \\
\eta &=& t + \frac{x}{c} \label{c6e14}
\end{eqnarray}
so that
\begin{eqnarray}
\frac{\partial}{\partial t} &=& \frac{\partial}{\partial\xi} = 
 \frac{\partial}{\partial\eta} \label{c6e15} \\
\frac{\partial}{\partial x} &=& -\frac{1}{c}\frac{\partial}{\partial\xi} =
\frac{1}{c}\frac{\partial}{\partial\eta} \label{c6e16}
\end{eqnarray}
so that
\begin{eqnarray}
\frac{\partial}{\partial\eta} &=& \frac{1}{2}
	\left(\frac{\partial}{\partial t} + c\frac{\partial}{\partial x}\right) 
	\label{c6e17} \\
\frac{\partial}{\partial\xi} &=& \frac{1}{2}
	\left(\frac{\partial}{\partial t} - c\frac{\partial}{\partial x}\right) 
	\label{c6e18}
\end{eqnarray}
This pair of relations allows us to write \eqref{c6e12} as
\begin{equation}\label{c6e19}
\frac{\partial^2 f}{\partial\xi\partial\eta} = 0.
\end{equation}
Its solution can be written as
\begin{equation}\label{c6e20}
f(\xi, \eta) = f_1(\xi) + f_2(\eta).
\end{equation}
In terms of the original variables,
\begin{equation}\label{c6e21}
f(x, t) = f_1\left(t - \frac{x}{c}\right) + f_2\left(t + \frac{x}{c}\right).
\end{equation}

\item If $f_2 = 0$ then
\[
f(x, t) = f_1\left(t - \frac{x}{c}\right)
\]
Any reasonably function of $t - x/c$ is a solution of \eqref{c6e11}. For a fixed
$x$, the value of the function changes with $t$. Likewise, for a fixed $t$, the
function changes value with $x$. Further the value of the function is the same
for a given value of its argument, namely, $t - x/c$ or $ct - x$. If we focus 
our attention on one such value, we will see it travelling along the $x$ axis 
with a speed $x/t = c$, the speed of light.

If $f_1 = 0$ and $f = f_2$ then a similar interpretation is applicable except
that a point with a certain value of $f$ travels along the negative $x$ axis
with the speed $c$.

\item How do we use this information to derive the forms of $\vec{E}$ and $\vec{H}$
for plane waves? We will start by deriving an expression for the vector potential.
Since the components of $\vec{A}$ individually satisfy \eqref{c6e11}, none of them
will be functions of $y$ and $z$. Further, $\dive\vec{A} = 0$ implies that
\[
\pdt{A_x}{x} = 0.
\]
Therefore,
\[
\spdt{A_x}{t} - c^2\spdt{A_x}{x} = 0 \Rightarrow \spdt{A_x}{t} = 0
\Rightarrow \pdt{A_x}{t} = \text{const.}
\]
This, in turn, implies that $E_x$ is a constant. If this constant is non-zero
then it results in a constant electric field in the direction of propagation
of the wave, also called the longitudnal direction. Being a constant, it is
not of the form $f(t \pm x/c)$ and therefore unrelated to the electromagnetic
wave. We can ignore it by setting $A_x = 0$. We are thus left with
\begin{equation}\label{c6e22}
\vec{A} = A_y(x)\uv{y} + A_z(x)\uv{z}
\end{equation}
and hence
\begin{eqnarray}
\vec{E} &=& -\frac{1}{c}\pdt{\vec{A}}{t} \label{c6e23} \\
\vec{H} &=& \curl\vec{A} \label{c6e24}
\end{eqnarray}
From equation \eqref{c6e15},
\[
\pdt{\vec{A}}{t} = \pdt{\vec{A}}{\xi}
\]
so that if we denote rhs of the above equation as $\vec{A}^\op$ then equation 
\eqref{c6e23} becomes
\begin{equation}\label{c6e25}
\vec{E} = -\frac{1}{c}\vec{A}^\op.
\end{equation}
Now,
\begin{equation}\label{c6e26}
\curl\vec{A} = -\uv{y}\pdt{A_z}{x} + \uv{z}\pdt{A_y}{x} = 
\frac{1}{c}\uv{y}\pdt{A_z}{\xi} - \frac{1}{c}\uv{z}\pdt{A_y}{\xi}
\end{equation}
where we used \eqref{c6e16}. If $\un$ is the direction of wave propagation then
$\un = \uv{x}$ and 
\begin{equation}\label{c6e27}
\un\times\vec{A}^\op = \uv{z}A^\op_y - \uv{y}A^\op_z
\end{equation}
and hence from \eqref{c6e26} and \eqref{c6e27} we get
\begin{equation}\label{c6e28}
\vec{H} = -\frac{1}{c}\un\times\vec{A}^\op.
\end{equation}
From \eqref{c6e25} and \eqref{c6e28} it immediately follows that
\begin{equation}\label{c6e29}
\vec{H} = \un\times\vec{E}.
\end{equation}
Thus both fields are perpendicular to $\un$, the direction of propagation. As a 
result, the plane electromagnetic waves in vacuum are transverse in nature. 
Equation \eqref{c6e29} also assures that magnitude of the two fields is the same.

\item The energy flux is
\begin{equation}\label{c6e30}
\vec{S} = \frac{c}{4\pi}\vec{E}\times\vec{H} = 
\frac{c}{4\pi}\vec{E}\times(\un\times\vec{E}) = 
\frac{c}{4\pi}E^2\un = \frac{c}{4\pi}H^2\un.
\end{equation}
We can as well write it as 
\begin{equation}\label{c6e31}
2W = \frac{c}{4\pi}(E^2 + H^2)\un
\end{equation}
so that
\begin{equation}\label{c6e32}
\vec{S} = cW\un
\end{equation}
where we used \eqref{c4e50}, where $W$ where is the energy density of the field.
The components of $\vec{S}/c$ are the components 
\[
T^{01}, T^{02}, T^{03}
\]
of the energy-momentum tensor \eqref{c4e90}. We also argued in point 23 of chapter
\ref{c4} that $T^{01}/c, T^{02}/c, T^{03}/c$ are the components of momentum 
density of the system. Thus, we can interpret $\vec{S}/c^2$ as the momentum flux
in the system. From \eqref{c6e32} we see that electromagnetic waves carry momentum
in the direction of propagation.

The relation between energy density and momentum density of electromagentic 
waves is similar to that of relativistic particle of zero mass.

\item If the $x$-axis is chosen to be along the direction of propagation of
the wave then $\vec{E} = E\uv{y}$ and $\vec{H} = \uv{z}$. The components of the
energy momentum tensor evaluated below \eqref{c4e88} are
\begin{eqnarray}
T^{00} &=& \frac{E^2 + H^2}{8\pi} \label{c6e33} \\
T^{11} &=& -\frac{E^2 + H^2}{8\pi} = -\sigma_{xx} \label{c6e34},
\end{eqnarray}

\item Lorentz transformation for $W$ follows from \eqref{c1e88}.
\begin{equation}\label{c6e35}
W = \frac{W^\op + 2\beta S^\op_x/c + \beta^2T_{11}^\op}{1 - \beta^2}
 = \frac{W^\op + 2\beta S^\op_x/c - \beta^2\sigma_{xx}^\op}{1 - \beta^2}.
\end{equation}
If $\alpha^\op$ is the angle between $\bm{\beta}$ and the direction of 
propagation of the wave then
\[
S_x^\op = cW^\op\cos\alpha^\op
\]
and 
\begin{equation}\label{c6e36}
\sigma_{xx}^\op = -W^\op\cos^2\alpha^\op,
\end{equation}
{\color{red}To do: justify \eqref{c6e36}.} so that
\begin{equation}\label{c6e37}
W = \frac{W^\op(1 + \beta\cos\alpha^\op)^2}{1 - \beta^2}.
\end{equation}

\item A wave whose time dependence is of the form $\cos(\omega t + \alpha)$, where
$\omega$ and $\alpha$ are constants is called a mono-chromatic wave. If $f(\vec{r},
t)$ is the field quantity and if its time dependence of mono-chromatic then
\begin{equation}\label{c6e38}
\nabla^2 f = -\frac{\omega^2}{c^2} f.
\end{equation}
If the wave is propagating along the positive $x$ axis then $f$ is a function of
$t - x/c$ alone. If it is also mono-chromatic then,
\[
f(x, t) = f_0\cos\left(\omega t - \omega\frac{x}{c} + \alpha\right) = 
\re\left\{f_0\exp\left(-i\left(\omega t - \omega\frac{x}{c} + \alpha\right)\right)\right\}.
\]
The constant $\alpha$ can be absorbed in $f_0$ so that we can as well write
\begin{equation}\label{c6e39}
f(x, t) = \re\left\{f_0\exp\left(-i\left(\omega t - \omega\frac{x}{c}\right)\right)\right\}.
\end{equation}
Now,
\[
\cos\left(\omega t - \frac{\omega}{c}\left(x + \frac{2\pi c}{\omega}\right)\right)
= \cos\left(\omega t - \frac{\omega}{c}x + 2\pi\right) = 
\cos\left(\omega t - \frac{\omega}{c}x\right)
\]
so that the quantity
\begin{equation}\label{c6e40}
\lambda = \frac{2\pi c}{\omega}
\end{equation}
is called the \emph{wavelength}. The vector,
\begin{equation}\label{c6e41}
\vec{k} = \frac{\omega}{c}\un
\end{equation}
is called the \emph{wave-vector}. In terms of $\vec{k}$ we can generalise 
\eqref{c6e39} to
\begin{equation}\label{c6e42}
f(\vec{r}, t) = \re\left\{f_0\exp\left(i\left(\vec{k}\cdot\vec{r} - \omega t\right)\right)\right\}.
\end{equation}
If we perform only linear operations, we can write \eqref{c6e42} as
\begin{equation}\label{c6e43}
f(\vec{r}, t) = f_0e^{i(\vec{k}\cdot\vec{r} - \omega t)}
\end{equation}
with the understanding that only the real part is physically significant.

\item If, for a plane, mono-chromatic wave,
\begin{equation}\label{c6e44}
\vec{A}(\vec{r}, t) = \vec{A}_0e^{i(\vec{k}\cdot\vec{r} - \omega t)}
\end{equation}
then
\begin{equation}\label{c6e45}
\vec{E} = -\frac{1}{c}\pdt{\vec{A}}{t} = i\frac{\omega}{c}\vec{A} = ik\vec{A},
\end{equation}
where we used \eqref{c6e41} in the last step. Further,
\begin{equation}\label{c6e46}
\vec{H} = \curl\vec{A} = i\vec{k}\times\vec{A}.
\end{equation}

\item Let 
\begin{equation}\label{c6e47}
\vec{E} = \vec{E}_0 e^{i(\vec{k}\cdot\vec{r} - \omega t)},
\end{equation}
where
\begin{equation}\label{c6e48}
\vec{E}_0 = \vec{b}e^{-i\alpha}
\end{equation}
and
\begin{equation}\label{c6e49}
\vec{b} = \vec{b}_1 + i\vec{b}_2,
\end{equation}
so that $\vec{E}_0$ is a complex vector and $\vec{b}_1, \vec{b}_2$ are real 
vectors. If we insist that
\[
|\vec{E}_0|^2 = \vec{E}_0\cdot\vec{E}_0^\ast = \vec{b}\cdot\vec{b}
\]
is real then
\[
\vec{b}\cdot\vec{b} = \vec{b}_1\cdot\vec{b}_1 - \vec{b}_2\cdot\vec{b}_2 + 
2i\vec{b}_1\cdot\vec{b}_2
\]
must be real, which requires us to have
\begin{equation}\label{c6e50}
\vec{b}_1\cdot\vec{b}_2 = 0.
\end{equation}
If the wave propagates along the $x$ direction, we can choose $\vec{b}_1$ to be
along the $y$ direction and $\vec{b}_2$ along the $z$ direction. We can then 
write
\begin{eqnarray}
E_y &=& b_1\cos(\vec{k}\cdot\vec{r} - \omega t - \alpha) \label{c6e51} \\
E_z &=& \pm b_2\sin(\vec{k}\cdot\vec{r} - \omega t - \alpha) \label{c6e52}
\end{eqnarray}
There is a $\sin$ factor in the second expression because of $i$ as a multiple
in the second term of \eqref{c6e49}. We can combine \eqref{c6e51} and 
\eqref{c6e52} to
\begin{equation}\label{c6e53}
\frac{E_y^2}{b_1^2} + \frac{E_z^2}{b_2^2} = 1.
\end{equation}
The vector $\vec{E} = \vec{E}_y\uv{y} + \vec{E}_z\uv{z}$ thus lies on the ellipse
defined by \eqref{c6e53}. As time goes by, the tip of the vector moves along the
ellipse. Such a wave is called \emph{elliptically polarised}. It is called \emph{
circularly polarised} if $b_1 = b_2$ and plane polarised of one of $b_1$ or $b_2$
is zero.

\item If we introduce a 4-vector
\begin{equation}\label{c6e54}
k^\mu = \left(\frac{\omega}{c}, \vec{k}\right)
\end{equation}
then $x^\mu = (ct, \vec{r})$ implies
\begin{equation}\label{c6e55}
k_\mu x^\mu = \left(\frac{\omega}{c}, -\vec{k}\right)\cdot(ct, \vec{r})
= \omega t =\vec{k}\cdot\vec{r}.
\end{equation}
This allows us to write the solution \eqref{c6e44} of the wave equation as 
\begin{equation}\label{c6e56}
\vec{A} = \vec{A}_0\exp(ik_\mu x^\mu).
\end{equation}
We also observe that
\begin{equation}\label{c6e57}
k_\mu k^\mu = \frac{\omega^2}{c^2} - k^2 = 0,
\end{equation}
by \eqref{c6e41}. In the pseudo-Euclidean geometry of the space-time a norm of a
vector can be zero without the vector being zero.

\item The energy-momentum tensor of the electromagnetic field of a plane wave has
the components,
\begin{eqnarray*}
T^{00} &=& \frac{E^2 + H^2}{8\pi} = W \\
T^{01} &=& \frac{E_yH_z - E_zH_y}{4\pi} = k^2A^2\\
T^{02} &=& \frac{E_zH_x - E_xH_z}{4\pi} = 0 \\
T^{03} &=& \frac{E_xH_y - E_yH_x}{4\pi} = 0 \\
T^{11} &=& \frac{E_x^2 - E_y^2 - E_z^2 + H_x^2 - H_y^2 - H_z^2}{8\pi} = W \\
T^{12} &=& -\frac{E_xE_y + H_xH_y}{4\pi} = 0\\
T^{13} &=& -\frac{E_xE_z + H_xH_z}{4\pi} = 0\\
T^{22} &=& \frac{-E_x^2 + E_y^2 - E_z^2 - H_x^2 + H_y^2 - H_z^2}{8\pi} = 0\\
T^{23} &=& -\frac{E_yE_z + H_yH_z}{4\pi} = 0 \\
T^{33} &=& \frac{-E_x^2 - E_y^2 + E_z^2 - H_x^2 - H_y^2 + H_z^2}{8\pi} = 0
\end{eqnarray*}
where we used the fact that $\vec{k} = k\uv{x}$, $\vec{H} = i\vec{k}\times\vec{A}
= i(-kA_z\uv{y} + kA_y\uv{z})$, $\vec{E} = \un\times\vec{H} = \uv{x}\times\vec{H}
= i(-kA_z\uv{z} - kA_y\uv{y})$, for a plane wave. Since, for these expressions,
\[
E^2 + H^2 = 2k^2A^2,
\]
we can write
\begin{equation}\label{c6e58}
T^{01} = k^2A^2 = \frac{E^2 + H^2}{8\pi} = W.
\end{equation}
Since $\vec{k} = k\uv{x}$, $k^\mu = (\omega/c, k, 0, 0)$ and
\begin{equation}\label{c6e59}
T^{00} = T^{01} = T^{11} = W \Rightarrow T^{\mu\nu} = 
\frac{Wc^2}{\omega^2}k^\mu k^\nu
\end{equation}

\item The Lorentz transformation for $k^\mu$ is, according to \eqref{c1e46},
\begin{eqnarray}
k^0 &=& \gamma(\bar{k}^0 + \beta\bar{k}^1) \label{c6e60} \\
k^1 &=& \gamma(\bar{k}^1 + \beta\bar{k}^0) \label{c6e61} \\
k^2 &=& \bar{k}^2 \label{c6e62} \\
k^3 &=& \bar{k}^3 \label{c6e63}
\end{eqnarray}
where $\bar{k}^\mu$ is the wave vector measured in a frame $\bar{K}$ moving
at velocity $v\uv{x}/c$ with respect to $K$. From \eqref{c6e54}, $k^0 = \omega/c$
so that \eqref{c6e60} becomes
\begin{equation}\label{c6e64}
\omega = \gamma(\bar{\omega} + c\bar{k}^1)
\end{equation}
If the wave propagates at an angle $\alpha$ with respect to the $x$ axis in 
$\bar{K}$ frame then $\bar{k}^1 = k\cos\alpha$. Using \eqref{c6e57}, we have
\begin{equation}\label{c6e65}
\bar{k}^1 = \frac{\bar{\omega}}{c}\cos\alpha
\end{equation}
so that \eqref{c6e64} becomes
\[
\omega = \gamma(\bar{\omega} + \bar{\omega}\cos\alpha) = \gamma\bar{\omega}(1 + \cos\alpha)
\]
or
\begin{equation}\label{c6e66}
\bar{\omega} = \frac{\omega\sqrt{1 - \beta^2}}{1 + \cos\alpha}.
\end{equation}
If $\alpha = \pi/2$, $\bar{\omega} = \omega\sqrt{1 - \beta^2} < \omega$. This is 
the relativistic red-shift.

\item The spectral resolution of a wave is an expression of the wave field as a
superposition of monochromatic waves. It can be done in two ways:
\begin{enumerate}
\item Express the periodic function as
\begin{equation}\label{c6e67}
f(t) = \sum_{n=-\infty}^\infty f_n e^{-i\omega_0 nt},
\end{equation}
where $\omega_0 = 2\pi/T$ and $T$ is such that $f(t + T) = f(t)$. The superposition
is written as a sum of waves of frequency $\omega_0$ and its harmonics. We can 
get the amplitudes $f_n$ using
\begin{equation}\label{c6e68}
f_n = \frac{1}{T}\int_{-T/2}^{T/2} f(t)e^{i\omega_0 nt},
\end{equation}
from which it is evident that
\begin{equation}\label{c6e69}
f_n = f_{-n}.
\end{equation}
The squared modulus of \eqref{c6e67} is
\begin{equation}\label{c6e70}
|f|^2 = \sum_{m, n = -\infty}^\infty f_n f_m^\ast e^{i\omega_0(m - n)t}.
\end{equation}
Since,
\begin{equation}\label{c6e71}
\int_{-T/2}^{T/2}e^{i\omega_0(m - n)t}dt = T\delta_{mn},
\end{equation},
\begin{equation}\label{c6e71}
\overline{|f|^2} = \frac{1}{T}\int_{-T/2}^{T/2}|f|^2 dt = 
\sum_{m, n = -\infty}^\infty f_n f_m^\ast \delta_{mn} = 
\sum_{n=-\infty}^\infty |f_n|^2.
\end{equation}
In view of \eqref{c6e69} we also have
\begin{equation}\label{c6e72}
\overline{|f|^2} = 2\sum_{n=0}^\infty |f_n|^2.
\end{equation}

\item If $f(t) \rightarrow 0$ as $t \rightarrow \infty$, one can also write
\begin{equation}\label{c6e73}
f(t) = \int_{-\infty}^\infty \hat{f}(\omega) e^{-i\omega t}\frac{d\omega}{2\pi}.
\end{equation}
The inverse of this relation is
\begin{equation}\label{c6e74}
\hat{f}(\omega) = \int_{-\infty}^\infty f(t)e^{i\omega t}dt.
\end{equation}
From \eqref{c6e74} it is immediately evident that
\begin{equation}\label{c6e75}
\hat{f}^\ast(\omega) = \hat{f}(-\omega).
\end{equation}
From \eqref{c6e74},
\[
|f(t)|^2 = \frac{1}{4\pi^2}
\iint_{-\infty}^\infty \hat{f}(\omega)\hat{f}^\ast(\omega^\op)e^{-i(\omega-\omega^\op)t}d\omega d\omega^\op
\]
and
\[
\int_{-\infty}^\infty |f(t)|^2dt = \frac{1}{4\pi^2}
\int_{-\infty}^\infty\iint_{-\infty}^\infty \hat{f}(\omega)\hat{f}^\ast(\omega^\op)e^{-i(\omega-\omega^\op)t}d\omega d\omega^\op
dt
\]
Since
\begin{equation}\label{c6e77}
\frac{1}{2\pi}\int_{-\infty}^\infty e^{-i(\omega - \omega^\op)}dt = \delta(\omega - \omega^\op)
\end{equation}
we get
\[
\int_{-\infty}^\infty |f(t)|^2dt = \frac{1}{2\pi}
\iint_{-\infty}^\infty \hat{f}(\omega)\hat{f}^\ast(\omega^\op)\delta(\omega-\omega^\op)d\omega d\omega^\op
\]
or
\begin{equation}\label{c6e78}
\int_{-\infty}^\infty |f(t)|^2dt = \frac{1}{2\pi}\int_{-\infty}^\infty |\hat{f}(\omega)|^2 d\omega.
\end{equation}
\end{enumerate}

\item A purely monochromatic wave extends all the way to infinity. There are no
such waves. Almost monochromatic waves have a frequency in a small band around a
certain value, say $\omega$. Unlike the pure monochromatic wave, whose amplitude
is a constant $\vec{E}_0$, the amplitude of an approximate monochromatic wave
is a slow varying quantity $\vec{E}_0(t)$. A pure monochromatic wave is polarised 
because $\vec{E}_0$ is constant. An approximate monochromatic wave is said to
be partially polarised.

\item Experiments studying polarisation involve measurement of intensities of
waves transmitted through media like a Nicol prism. Therefore, one studies the
quadratic functions of electric field. Functions like $E_iE_j$ or $E_i^\ast
E_j^\ast$ have a phase factor because
\begin{equation}\label{c6e79}
\vec{E}(t) = \vec{E}_0(t)e^{-i\omega t}.
\end{equation}
On the other hand, $E_iE_j^\ast$ or $E_i^\ast E_j$ will not have them. The 
long-time averages of these quantities will typically be non-zero and will be
easily observable in an experiment. Therefore, the polarisation properties
of electromagnetic waves are determined by the tensor
\begin{equation}\label{c6e80}
J_{ij} = \overline{E_{0i}{E_{0j}^\ast}},
\end{equation}
where $E_{0i}$ is the $i$-th component of the amplitude vector $\vec{E}_0(t)$.

\item The vector $\vec{E}$ is confined to a plane in the case of a plane 
electromagnetic wave. If the wave propagates along the $z$ axis then the
electric field is restricted to the $xy$-plane. Therefore, the tensor in
equation \eqref{c6e80} has only two dimensions. If we define,
\begin{equation}\label{c6e81}
J = J_{ii} = \overline{\vec{E}_0\cdot\vec{E}_0^\ast}
\end{equation}
then it is clear that $J$ is the intensity of the wave. It is not related
to its polarisation properties. We therefore factor it out from $J_{ij}$ and
define the polarisation tensor
\begin{equation}\label{c6e82}
\rho_{ij} = \frac{J_{ij}}{J}.
\end{equation}
From the definition of $J_{ij}$ it is clear that
\begin{equation}\label{c6e83}
J_{ij} = J_{ji}^\ast \text{ and } \rho_{ij} = \rho_{ji}^\ast,
\end{equation}
that is, the tensors $J_{ij}$ and $\rho_{ij}$ are hermitian and their eigen-values
are real. Note that $J$ defined in \eqref{c6e81} is also the trace of the tensor. 
Therefore,
\begin{equation}\label{c6e84}
\rho_{11} + \rho_{22} = \frac{J_{11}}{J} + \frac{J_{22}}{J} = 1.
\end{equation}
Hermiticity of $\rho_{ij}$ requires that
\begin{equation}\label{c6e85}
\rho_{21} = \rho_{12}^\ast.
\end{equation}
Equations \eqref{c6e84} and \eqref{c6e85} suggest that the tensor $\rho_{ij}$ is
of the form
\begin{equation}\label{c6e86}
\rho_{ij} = \begin{bmatrix}a & b + ic \\ b - ic & 1 - a \end{bmatrix},
\end{equation}
where $a, b, c \in \mathbb{R}$. This means that the polarisation state of a plane
electromagnetic wave can be described by three real numbers.

\item We will now examine the polarisation tensor for a variety of situations:
\begin{enumerate}
\item In the case of pure monochromatic light, $\vec{E}_0(t)$ is the constant
$\vec{E}_0$ and hence
\begin{equation}\label{c6e87}
\rho = \begin{bmatrix}|E_{01}|^2 & E_{01}E_{02}^\ast \\ 
                      E_{01}^\ast E_{02} & |E_{02}|^2 
       \end{bmatrix}
\end{equation}
In this case, 
\begin{equation}\label{c6e88}
\det\rho = |E_{01}|^2|E_{02}|^2 - E_{01}E_{02}^\ast E_{01}^\ast E_{02} = 0.
\end{equation}
This is also the case of complete polarisation.

\item The other extreme is completely unpolarised light. In this case, all
directions are equivalent and the polarisation tensor is isotropic. To ensure
that its trace is $1$, we must have
\begin{equation}\label{c6e89}
\rho_{ij} = \frac{1}{2}\delta_{ij}.
\end{equation}

\item These two extremes suggest that we can define a quantity $P$ called the
degree of polarisation as
\begin{equation}\label{c6e90}
\det\rho = \frac{1}{4}(1 - P^2)
\end{equation}
so that $P = 1$ for fully polarised, monochromatic light and $P = 0$ for 
unpolarised light.
\end{enumerate}

\item A tensor like $\rho_{ij}$ can be split into a symmetric and an anti-symmetric
part such that
\begin{equation}\label{c6e91}
\rho_{ij} = S_{ij} + A_{ij},
\end{equation}
where
\begin{eqnarray}
S_{ij} &=& \frac{1}{2}(\rho_{ij} + \rho{ji}) \label{c6e92} \\
A_{ij} &=& \frac{1}{2}(\rho_{ij} - \rho{ji}). \label{c6e93}
\end{eqnarray} 
The hermiticity of $\rho$ results in 
\begin{equation}\label{c6e94}
A_{ij} = \begin{bmatrix}0 & \rho_{12} - \rho_{21} \\
-(\rho_{21} - \rho_{12}) & 0
\end{bmatrix} = (\rho_{12} - \rho_{12}^\ast)\begin{bmatrix}0 & 1 \\ -1 & 0 \end{bmatrix}
\end{equation}
Now, $(\rho_{12} - \rho_{12}^\ast)$ is pure imaginary. Let 
\begin{equation}\label{c6e95}
A = i(\rho_{12} - \rho_{12}^\ast)
\end{equation}
so that we can write \eqref{c6e91} as
\begin{equation}\label{c6e96}
\rho_{ij} = S_{ij} - \frac{i}{2}Ae_{ij},
\end{equation}
where
\begin{equation}\label{c6e97}
e_{ij} = \begin{bmatrix}0 & 1 \\ -1 & 0 \end{bmatrix}
\end{equation}
is the fully anti-symmetric tensor.
\begin{enumerate}
\item For a circularly polarised wave, $b_1 = b_2$ in \eqref{c6e53} so that from
equations \eqref{c6e51} and \eqref{c6e52} we get $E_{02} = \pm i E_{01}$. The 
polarisation tensor is
\begin{equation}\label{c6e98}
\rho_{ij} = \begin{bmatrix}|E_{01}|^2 & \mp i |E_{01}|^2 \\
\pm i |E_{01}|^2 & |E_{01}|^2
\end{bmatrix}\frac{1}{|E_{01}|^2} = \frac{1}{2}\delta_{ij} - \frac{\pm i}{2}e_{ij}
\end{equation}
so that $S_{ij} = \delta_{ij}/2$ and $A = \pm 1$.

\item If the wave is linearly polarised, one of $b_1$ or $b_2$ in \eqref{c6e53}
is zero. From equations \eqref{c6e51} and \eqref{c6e52} it is evident that we can
write $\vec{E}_0$ as a real vector. Therefore, $A = 0$.

\item This suggests that the constant $A$ can be viewed as a degree of circularity
in the polarisation. $A \in [-1, 1]$, taking extreme values for left and right 
circular polarised waves and middle value for plane polarised ones.
\end{enumerate}

\item A real symmmetric matrix $S_{ij}$ can be diagonalised. Let $\lambda_1$ and
$\lambda_2$ be the two eigenvalues and let the corresponding eigenvectors 
correspond to directions $\un^{(1)}$ and $\un^{(2)}$. Then we can write
\begin{equation}\label{c6e99}
S_{ij} = \lambda_1 n^{(1)}_in^{(1)}_j + \lambda_2 n^{(2)}_in^{(2)}_j,
\end{equation}
and $0 \le \lambda_1, \lambda_1 \le 1$. Each of the two terms is a product of
quantities related to one of the eigen-vectors. If $A = 0$, then $\rho_{ij} = 
S_{ij}$, which means that each term of $\rho_{ij}$ is a sum of two terms, each
one of which being a product of one of the two eigen-vectors. The two parts can
be treated as independent of each other. Such a wave is called \emph{incoherent}.

\item Let $\phi$ be the angle between $\un^{(1)}$ and the $x$-axis. Then we can 
write
\begin{eqnarray}
\un^{(1)} &=& (\cos\phi, \sin\phi) \label{c6e100} \\
\un^{(2)} &=& (-\sin\phi, \cos\phi) \label{c6e101}
\end{eqnarray}
If $\lambda_1 > \lambda_2$, let
\begin{equation}\label{c6e102}
l = \lambda_1 - \lambda_2 > 0.
\end{equation}
The matrix $S$ can now be written as
\begin{equation}\label{c6e103}
S = \frac{1}{2}\begin{bmatrix}
1 + l\cos(2\phi) & l\sin(2\phi) \\
l\sin(2\phi) & 1 - l\cos(2\phi)
\end{bmatrix}
\end{equation}
The three numbers $A, l$ and $\phi$ describe the polarisation state of the wave.
One can replace them with
\begin{eqnarray}
\xi_1 &=& l\sin(2\phi) \label{c6e104} \\
\xi_2 &=& A \label{c6e105} \\
\xi_3 &=& l\cos(2\phi) \label{c6e106}
\end{eqnarray}
so that the polarisation tensor can be written as
\begin{equation}\label{c6e107}
\rho = \frac{1}{2}\begin{bmatrix}
1 + \xi_3 & \xi_1 - i\xi_2 \\
\xi_1 + i\xi_2 & 1 - \xi_3
\end{bmatrix}.
\end{equation}
The three numbers $\xi_1, \xi_2, \xi_3$ are called Stokes parameters. In this form,
\begin{equation}\label{c6e108}
\det\rho = \frac{1}{4}(1 - \xi_1^2 - \xi_2^2 - \xi_3^2),
\end{equation}
so that the degree of polarisation $P$ defined in \eqref{c6e90} becomes
\begin{equation}\label{c6e109}
P = \sqrt{\xi_1^2 + \xi_2^2 + \xi_3^2}.
\end{equation}
\end{enumerate}
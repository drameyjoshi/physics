\documentclass{report}
\usepackage{amsmath, amssymb}
\usepackage{graphicx}
\usepackage{tensor}
\usepackage{hyperref}
\usepackage{xcolor}
\usepackage{mathabx}
\usepackage{bm}

\renewcommand{\vec}{\mathbf}
\newcommand{\uv}[1]{\hat{\mathbf{#1}}}
\newcommand{\un}{\hat{\mathbf{n}}}

%\newcommand{\uv}[1]{\hat{#1}}
%\newcommand{\un}{\hat{n}}

\newcommand{\abs}[1]{\left\vert{#1}\right\vert}
\newcommand{\td}[2]{\frac{d{#1}}{d{#2}}}
\newcommand{\pdt}[2]{\frac{\partial{#1}}{\partial{#2}}}
\newcommand{\spdt}[2]{\frac{\partial^2{#1}}{\partial{#2}^2}}
\newcommand{\op}{{\,\prime}}
\newcommand{\tp}{{\,\prime\prime}}
\newcommand{\ta}[1]{\left\langle{#1}\right\rangle}
\newcommand{\tav}[1]{\ta{\vec{#1}}}
\DeclareMathOperator{\grad}{grad}
\DeclareMathOperator{\dive}{div}
\DeclareMathOperator{\curl}{curl}
\DeclareMathOperator{\Arg}{arg}
\DeclareMathOperator{\re}{Re}
\DeclareMathOperator{\im}{Im}
\DeclareMathOperator{\sech}{sech}

%\newcommand{\fint}{\mbox{--}\mkern-16mu\int}

\begin{document}
\tableofcontents
\chapter{The Special Theory of Relativity}\label{c1}
\begin{enumerate}
\item A frame of reference is a set of cartesian coordinate axes and a clock.

\item An inertial frame is the one in which Newton's first law is valid. The 
existence of such frames was inferred from the experimental observations of the
seventeenth and the eighteenth centuries.

\item Experiments also showed that a frame fixed to the earth is not inertial
but the one fixed with respect to the distant stars was. If $K$ is an inertial
frame then so is $K^\op$ if it moves at a constant relative velocity with respect
to $K$. Thus, there are an infinite number of inertial frames.

\item Suppose that two events happen in an inertial frame $K$: one at a point
$\vec{a}$, and at time $t$ and another one at a point $\vec{b}$ and at time 
$t + \delta t$. When viewed from another inertial frame $K^\op$ moving at a 
velocity $\vec{v}$ with respect to $K$, the events occured at points $\vec{r}_0 
+ \vec{a} + \vec{v}t$ and $\vec{r}_0 + \vec{b} +\vec{v}(t + \delta t)$. Here 
$\vec{r}_0$ is the point in $K$ where $K^\op$'s origin was at $t=0$. An observer 
in $K$ reports that the two events occured at times $\delta t$ apart and at points 
$\vec{b} - \vec{a}$ apart. An observer in $K^\op$ reports that the two events 
happened at a distance $\vec{b} - \vec{a} + \vec{v}\delta t$ apart. However, 
he does not dispute that the same time $\delta t$ elapsed between the two events. 
If $\delta t$ were zero then the two events will be simultaneous in \emph{all} 
inertial frames of reference although they could have happened at different 
points as seen by different observers. This is a consequence of the implicit 
assumption of an absolute time in classical physics.

\item This idea can be corroborated by a simple example. An person on a platform may
see a passenger lifting a tea cup as the coach entered the platform and take it
to his lips as the coach exited it. The two events thus occurred, from his 
perspective, at different points. However, a co-passenger will report them to
be happening at the same spot. The observers will agree on the duration between
the events.

\item The special theory of relativity has its roots in two facts borne out 
of experiments:
\begin{itemize}
\item The laws of physics are the same in all inertial frames of reference and
\item Changes in the state of a system are propagated with a finite speed.
\end{itemize}

\item It is also confirmed by experiments that the changes in the state of system 
are propagated at a speed not exceeding
\begin{equation}\label{c1e1}
c = 2.998 \times 10^8 \;\text{cm/s}.
\end{equation}
No material body can move at a speed exceeding $c$ because if it could then it
can be used to signal the change of state of another body. Further, $c$ is the
maximum speed in \emph{all} inertial frames. For if it were not then the laws of
physics will not be identical across frames. Thus, $c$ is a universal constant.
It is also the speed of light in vacuum.

\item The idea of an absolute, universal time is in conflict with the experimental 
observation of a finite speed of propagation of light in vacuum. For if the
speed is $c$ in a frame $K$ then its speed will be $3c/2$ in a frame $K^\op$
approaching $K$ with a speed $c/2$ in a direction opposite to that of propagation
of light. That this is not true was confirmed by Michelson and Morley's experiment.
Time elapses differently in different systems and therefore a value of a time
difference must be accompanied by a specification of the frame in which it was 
measured.

\item Consider a frame $K$ with a source of light at its origin and two detectors at
points $(-a, 0)$ and $(a, 0)$. A spherical light front will reach the two 
detectors simultaneously when observed from $K$. Let us consider the experiment
replicated in another frame $K^\op$ that moves with a velocity $v\uv{x}$ with
respect to $K$. Suppose that their origins coincided then the light signal was 
emitted in $K$. Since the speed of light is the same in $K$ and $K^\op$ an observer
in $K^\op$ also sees a spherical wavefront propagating isotropically. However, he
will observe the wavefront reaching $(a, 0)$ sooner than it reaches $(-a, 0)$.
Events that are simultaneous in $K$ are not so in $K^\op$.

\item An event is described by the point at which occurred and the time when it
occurred. The four numbers describing an event can be interpreted as points in a
four-dimensional space. They are called the \emph{world points}. World points of
a system move along curves in the four-dimensional space called the \emph{world 
lines}.

\item A material body at a point $\vec{r}$ to which nothing happens travels along
a world line that is parallel to the $t$ axis. In the four-dimensional space, nothing
is still. If a material body moves along the $x$ axis with a uniform speed $v$ then
its world line is a straight line making an angle $\tan^{-1}(v)$ with the $t$ axis.
If it accelerates (decelerates) then the world line will be a curve turning towards 
(away from) the $x$-axis. 
\begin{figure}
\includegraphics[scale=0.8]{ex1}
\caption{World lines}
\label{c1f1}
\end{figure}

\item Consider an experiment in an inertial frame in which light took time 
$\delta t$ to travel between points $(x, y, z)$ and $(x + \delta x, y + 
\delta y, z + \delta z)$. The same experiment was observed from another inertial 
frame in which the light pulse travelled from $(x^\op, y^\op, z^\op)$ to $(x^\op 
+ \delta x^\op, y^\op + \delta y^\op, z^\op + \delta z^\op)$ in time $\delta t^\op$.
Since the speed of light is the same in two frames,
\begin{eqnarray}
c^2\delta t^2 &=& \delta x^2 + \delta y^2 + \delta z^2 \label{c1e2} \\
c^2\delta {t^\op}^2 &=& \delta {x^\op}^2 + \delta {y^\op}^2 + \delta {z^\op}^2 \label{c1e3}
\end{eqnarray}
From these equations, we conclude that
\begin{equation}\label{c1e4}
c^2\delta t^2 - \delta x^2 - \delta y^2 - \delta z^2 = 
c^2\delta {t^\op}^2 - \delta {x^\op}^2 - \delta {y^\op}^2 - \delta {z^\op}^2
\end{equation}
This suggests that the quantity 
\begin{equation}\label{c1e5}
\delta s^2 = c^2\delta t^2 - \delta x^2 - \delta y^2 - \delta z^2
\end{equation}
is invariant across all inertial frame references. The quantity $\delta s$ is
called an interval between two world points.

\item If $\delta t = 0$ then equation \eqref{c1e4} suggests that $\delta {t^\op}^2$
need not be zero. Events simultaneous in $K$ need not be so in $K^\op$. Further, if
$\delta t = 0$ then $\delta s^2 < 0$. Intervals between world points for which 
$\delta s^2 < 0$ are called \emph{space like}. Two points separated by a space-
like interval will always have different ``space'' coordinates.

\item If $\delta x^2 + \delta y^2 + \delta z^2 = 0$ then $\delta s^2 > 0$. In this
case $\delta t$ can never be zero. Intervals between world points for which 
$\delta s^2 > 0$ are called \emph{time like}. Two points separated by a time-
like interval will always have different ``time'' coordinates.

\item It follows from the definitions of time-like and space-like intervals that
\begin{itemize}
\item If two world points are separated by a time-like interval then there exists
an inertial frame in which the events occur at the same space-point, that is, at
same values of $(x, y, z)$.
\item If two world points are separated by a space-like interval then there exists
an inertial frame in which the events occur at the same time.
\end{itemize}

\item An interval for which $\delta s = 0$ is called ``light-like''.

\item Since the nature of an interval depends on an invariant quantity like
$\delta s^2$, an interval that is space-like (or time-like or light-like) in one
inertial frame is space-like (or time-like or light-like) in all inertial frames.

\item A causal relationship between events can exists only their world-points
are separated by a time-like interval.

\item In the four-dimensional space with coordinates $x, y, z$ and $t$, the equation
\begin{equation}\label{c1e6}
c^2t^2 = x^2 + y^2 + z^2
\end{equation}
defines a double cone with origin as the vertex and the $t$ axis as its principle
axis. It is called the \emph{light cone}. 
World points inside the cone are separated by a time-like interval. World
points outside it are separated by space-like interval. If two world points $A$
and $B$ are inside the cone then in every inertial frame the difference between
their time coordinates will be non-zero. Likewise, if they were outside the cone
then in every inertial frame their space coordinates will not all be the same.
The part of the cone above (below) the origin is called the absolute future 
(past) of the origin.

\item A time interval measured with a single clock is called
\emph{proper time}. Let an interval $\delta t$ be measured in an inertial frame
$K$ with a clock at point $(x, y, z)$. In another inertial frame, $K^\op$ it will
be measured by two (synchronised) clocks at two different locations. One of the
clocks will measure start of the interval and by the time the event at the end 
of the interval happens, a different clock will go past what is $(x, y, z)$ in
$K$. The relation between the two measurements follow from equation \eqref{c1e4}
with $\delta x = \delta y = \delta z = 0$. Thus,
\begin{equation}\label{c1e7}
\delta t^2 = \delta {t^\op}^2 - \frac{\delta {x^\op}^2 + \delta {y^\op}^2 + \delta {z^\op}^2}{c^2}
= \delta {t^\op}^2\left(1 - \frac{1}{c^2}\left(\frac{\delta r^\op}{\delta t^\op}\right)^2\right),
\end{equation}
where
\begin{equation}\label{c1e8}
\delta {r^\op}^2 = \delta {x^\op}^2 + \delta {y^\op}^2 + \delta {z^\op}^2.
\end{equation}
Now $\delta r^\op/\delta t^\op$ is the speed at which and observer in $K^\op$
moves with respect to the point $(x, y, z)$ in $K$. It is precisely the relative
speed of $K^\op$ with respect to $K$. If we call it $u$, we can write equation
\eqref{c1e7} as
\begin{equation}\label{c1e9}
\delta t^2 = \delta {t^\op}^2\left(1 - \frac{u^2}{c^2}\right).
\end{equation}
Since $u < c$, $\delta t < \delta t^\op$. Thus, the proper time interval is the
smallest among the intervals measured in all inertial frames.

\item In the previous point, an observer in $K^\op$ will notice that the clock
in $K$ is moving relative to him and it reports an interval $\delta t$ while his
clocks reported $\delta t^\op$. From equation \eqref{c1e9} the moving clock
reported a shorter time interval. That is why we say that moving clocks go slow.

\item Among all clocks travelling between world points $A$ and $B$ that differ 
only in their $t$-coordinate, the clock that is stationary measures the maximum
time. Consider two clocks, one of which remains stationary and the other whose
world line curves arbitrarily between $A$ and $B$. The time measured by each of
them is their proper time. For the stationary clock, the time interval is
\begin{equation}\label{c1e10}
\delta t_1 = \frac{1}{c}\int_A^B ds_1
\end{equation}
while that measured by the moving clock is
\begin{equation}\label{c1e11}
\delta t_2 = \frac{1}{c}\int_A^B ds_2.
\end{equation}
Now, $ds_1 = cdt_1$ while $ds_2 = \sqrt{c^2dt_2^2 - dx_2^2 - dy_2^2 - dz_2^2}$.
If for all intervals along the second clock's path, $ds_1$ will be greater
than or equal to $ds_2$, then $t_1 \ge t_2$.

\item We will now derive the transformation between the coordinates of world
points in two inertial frames of reference. Let $(x, y, z, t)$ be the coordinates
of a world point in $K$. Let the coordinates of the same points be $(x^\op, y^\op,
z^\op, t^\op)$ in $K^\op$. If we consider a pulse of light travelling from the 
origin to the point being considered then from \eqref{c1e4} we have
\begin{equation}\label{c1e12}
c^2t^2 - x^2 - y^2 - z^2 = c^2{t^\op}^2 - {x^\op}^2 - {y^\op}^2 - {z^\op}^2.
\end{equation}
We can consider each side of \eqref{c1e12} the ``distance'' of the two world 
points from the origin in the respective inertial frame. 

One way in which equation \eqref{c1e12} will be valid is when the two inertial
frames have their origins displaced by a constant. However, this is not what
happens when the frames are moving relative to each other.

We next check if we can treat the motion as a ``rotation'' in the four-dimensional
space. There are six mutually orthogonal planes in four dimensions. If $K$ and
$K^\op$ are such that their $x$-axes coincide while $y$ and $z$ axes remain parallel
then the ``rotation'' can happen only in the $xt$-plane. Further, $y = y^\op$ and
$z = z^\op$ so that
\begin{equation}\label{c1e13}
c^2t^2 - x^2 = c^2{t^\op}^2 - {x^\op}^2.
\end{equation}
We can readily verify that a transformation of the form
\begin{eqnarray}
 x  &=&  x^\op\cosh\psi + c t^\op\sinh\psi \label{c1e14} \\
c t &=&  x^\op\sinh\psi + c t^\op\cosh\psi \label{c1e15}
\end{eqnarray}
satisfies \eqref{c1e12}. If we focus only on the motion of the origin of $K^\op$
in $K$ then we have $x^\op = 0$ and equations \eqref{c1e14} and \eqref{c1e15}
give
\begin{equation}\label{c1e16}
\frac{x}{ct} = \tanh\psi.
\end{equation}
If the relative speed of $K^\op$ with respect to $K$ is $u$ then $u = x/t$ and
hence
\begin{equation}\label{c1e17}
\tanh\psi = \frac{u}{c}.
\end{equation}
The factor $u/c$ occurs quite frequently in the theory of relativity. We denote 
it by
\begin{equation}\label{c1e18}
\frac{u}{c} = \beta.
\end{equation}
Since $1 - \tanh^2\psi = \sech^2\psi$ we have
\begin{eqnarray}
\cosh\psi &=& \frac{1}{\sqrt{1 - \beta^2}} \label{c1e19} \\
\sinh\psi &=& \frac{\beta}{\sqrt{1 - \beta^2}} \label{c1e20}
\end{eqnarray}
Substituting these in \eqref{c1e14} and \eqref{c1e15} we get 
\begin{eqnarray}
x &=& \frac{x^\op + ut^\op}{\sqrt{1 - \beta^2}} \label{c1e21} \\
t &=& \frac{t^\op + ux^\op/c^2}{\sqrt{1 - \beta^2}}. \label{c1e22} 
\end{eqnarray}
We get a more symmetrical form if we define $\tau = ct$ so that
\begin{eqnarray}
x &=& \frac{x^\op + \beta\tau^\op}{\sqrt{1 - \beta^2}} \label{c1e23} \\
\tau &=& \frac{\tau^\op + \beta x^\op}{\sqrt{1 - \beta^2}}. \label{c1e24} 
\end{eqnarray}
Equations \eqref{c1e21} and \eqref{c1e22} (or equivalently \eqref{c1e23}
and \eqref{c1e24}) are called \emph{Lorentz transformation}.

\item It is not clear from the above analysis why a Lorentz transformation is
called a ``rotation'' in the 4-dimensional space-time. We can write equations 
\eqref{c1e14} and \eqref{c1e15} in matrix form as
\begin{equation}\label{c1e25}
\begin{bmatrix}x \\ ct \end{bmatrix} = \begin{bmatrix} \cosh\psi & \sinh\psi \\
\sinh\psi & \cosh\psi \end{bmatrix}\begin{bmatrix}x^\op \\ ct^\op \end{bmatrix}.
\end{equation}
We can as well write it as
\begin{equation}\label{c1e26}
\begin{bmatrix}x \\ ict \end{bmatrix} = \begin{bmatrix} \cosh\psi & -i\sinh\psi \\
i\sinh\psi & \cosh\psi \end{bmatrix}\begin{bmatrix}x^\op \\ ict^\op \end{bmatrix}.
\end{equation}
Since $\cos(i\psi) = \cosh\psi$ and $\sin(i\psi) = i\sinh\psi$, we have
\begin{equation}\label{c1e27}
\begin{bmatrix}x \\ ict \end{bmatrix} = \begin{bmatrix} \cos(i\psi) & -\sin(i\psi) \\
\sin(i\psi) & \cos(i\psi) \end{bmatrix}\begin{bmatrix}x^\op \\ ict^\op \end{bmatrix}.
\end{equation}
In the early years of special relativity, the world points were considered to have
coordinates $(x, y, z, ict)$ and the distance between two world points mimicked the
usual Euclidean formula. In such a space, a Lorentz transformation did indeed look
like a ``rotation'' by an imaginary angle in the $tx$ plane.

\item From equation \eqref{c1e24}
\begin{eqnarray}
\tau_1 &=& \frac{\tau_1^\op + \beta x_1^\op}{\sqrt{1 - \beta^2}} \label{c1e28} \\
\tau_2 &=& \frac{\tau_2^\op + \beta x_2^\op}{\sqrt{1 - \beta^2}} \label{c1e29}
\end{eqnarray}
so that
\begin{equation}\label{c1e30}
\tau_2 - \tau_1 = \frac{\tau_2^\op - \tau_1^\op + \beta(x_2^\op - x_1^\op)}{\sqrt{1 - \beta^2}}.
\end{equation}
If $x_2^\op = x_1^\op$ then $\delta\tau^\op = \tau_2^\op - \tau_1^\op$ is the 
proper time. Equation \eqref{c1e30} becomes
\begin{equation}\label{c1e31}
\delta\tau = \frac{\delta\tau^\op}{\sqrt{1 - \beta^2}}.
\end{equation}
Thus, $\delta\tau^\op < \delta\tau$ as we could have expected.

\item Likewise, from equation \eqref{c1e23},
\begin{eqnarray}
x_1 &=& \frac{x^\op_1 + u\tau^\op_1}{\sqrt{1 - \beta^2}} \label{c1e32} \\
x_2 &=& \frac{x^\op_2 + u\tau^\op_2}{\sqrt{1 - \beta^2}} \label{c1e33}
\end{eqnarray}
so that
\begin{equation}\label{c1e34}
x_2 - x_1 = \frac{x^\op_2 - x^\op_1 + u(\tau^\op_2 - \tau^\op_1)}{\sqrt{1 - \beta^2}}.
\end{equation}
If a rod with ends at $x_1$ and $x_2$ is at rest in a frame $K$ then to
measure its length in the moving frame we have to measure its ends at the same
time. That is $\tau^\op_2 = \tau^\op_1$ so that equation \eqref{c1e34} becomes
\begin{equation}\label{c1e35}
\delta x = \frac{\delta x^\op}{\sqrt{1 - \beta^2}}.
\end{equation}
The length $\delta x$ of the rod measured in the frame in which it was at rest
is called its \emph{proper length}. From \eqref{c1e35} we conclude that $\delta x^\op
< \delta x$. The rod was at rest in $K$ but appeared to move in $K^\op$. Since
$\delta x^\op < \delta x$ we conclude that moving rods appear to be shortened.

\item From equations \eqref{c1e21} and \eqref{c1e22}, along with the fact that
$y = y^\op$ and $z = z^\op$, we get
\begin{eqnarray}
dx &=& \frac{dx^\op + udt^\op}{\sqrt{1 - \beta^2}} \label{c1e36} \\
dy &=& dy^\op \label{c1e37} \\
dz &=& dz^\op \label{c1e38} \\
dt &=& \frac{dt^\op + udx^\op/c^2}{\sqrt{1 - \beta^2}} \label{c1e39}
\end{eqnarray}
so that
\begin{eqnarray}
\frac{dx}{dt} &=& \frac{dx^\op + udt^\op}{dt^\op + udx^\op/c^2} \label{c1e40} \\
\frac{dy}{dt} &=& \frac{dy^\op}{dt^\op + udx^\op/c^2} \label{c1e41} \\
\frac{dz}{dt} &=& \frac{dz^\op}{dt^\op + udx^\op/c^2}. \label{c1e42}
\end{eqnarray}
We also have
\[
v_x = \frac{dx}{dt}\;;\;v_y = \frac{dy}{dt}\;;\;v_z = \frac{dz}{dt}\;;\;
v_x^\op = \frac{dx^\op}{dt^\op}\;;\;v_y^\op = \frac{dy^\op}{dt^\op}
\;;\;v_z^\op = \frac{dz^\op}{dt^\op}
\]
so that equations \eqref{c1e40} to \eqref{c1e42} can be written as
\begin{eqnarray}
v_x &=& \frac{v_x^\op + u}{1 + uv_x^\op/c^2} \label{c1e43} \\
v_y &=& \frac{v_y^\op}{1 + uv_x^\op/c^2} \label{c1e44} \\
v_z &=& \frac{v_z^\op}{1 + uv_x^\op/c^2}. \label{c1e45} 
\end{eqnarray}
These are the formulae for transformation of velocity components. We
quickly confirm that if $(v_x^\op, v_y^\op, v_z^\op) = (c, 0, 0)$ then
equations \eqref{c1e43} to \eqref{c1e44} give $(v_x, v_y, v_z) = (c, 0, 0)$.

\item The coordinates of an event $(ct, x, y, z)$ can be considered to be
components of a four-dimensional vector, or a 4-vector. The components are
denoted by $x^i$ and we have $x^0 = ct, x^1 = x, x^2 = y, x^3 = z$. The 
components of the same vector in another inertial frame travelling with a 
relative velocity $u\uv{x}$ are, by equations \eqref{c1e23} and \eqref{c1e24},
\[
x^0 = \frac{\bar{x}^0 + \beta\bar{x}^1}{\sqrt{1 - \beta^2}};\;
x^1 = \frac{\bar{x}^1 + \beta\bar{x}^0}{\sqrt{1 - \beta^2}};\;
x^2 = \bar{x}^2; x^3 = \bar{x}^3.
\]
Any set of four quantities $A^0, A^1, A^2, A^3$ which transform in a similar
way form a 4-vector. Thus, the transformation equations for these components
are
\begin{equation}\label{c1e46}
A^0 = \frac{\bar{A}^0 + \beta\bar{A}^1}{\sqrt{1 - \beta^2}};\;
A^1 = \frac{\bar{A}^1 + \beta\bar{A}^0}{\sqrt{1 - \beta^2}};\;
A^2 = \bar{A}^2; A^3 = \bar{A}^3.
\end{equation}
The numbers $A^0, A^1, A^2, A^3$ are called the \emph{contravariant} components
of the vector. The four related quantities
\begin{equation}\label{c1e47}
A_0 = A^0; A_1 = -A^1; A_2 = -A^2; A_3 = -A^3
\end{equation}
are called the \emph{covariant} components of the same vector. The magnitude of
the vector is $A^\mu A_\mu$, where we have used the summation convention. We will 
sometimes write a 4-vector as $A^\mu = (A^0, \vec{A})$ and $A_\mu = (A^0, -\vec{A})$.

\item If $A^\mu$ and $B^\mu$ are two 4-vectors then their scalar product is $A^\mu 
B_\mu$, which is same as $A_\mu B^\mu$. The resulting quantity is called a 4-scalar. 

\item The relation between contravariant and covariant components of a tensor
can be expressed as
\begin{equation}\label{c1e48}
A^\mu = g^{\mu\nu}A_\nu\;\text{ and }\; A_\mu = g_{\mu\nu}A^\nu,
\end{equation}
where $g^{\mu\nu}$ and $g_{\mu\nu}$ are components of the \emph{metric tensor}. 
They are identical and are given by
\begin{equation}\label{c1e49}
g^{\mu\nu} = g_{\mu\nu} = \begin{bmatrix}1 & 0 & 0 & 0 \\
0 & -1 & 0 & 0 \\
0 & 0 & -1 & 0 \\
0 & 0 & 0 & -1
\end{bmatrix}.
\end{equation}

\item Right now we will only introduce 4-tensors of second order in terms of their
contravariant components $\tensor{T}{^{\mu\nu}}$ or covariant components 
$\tensor{T}{_{\mu\nu}}$ or mixed components $\tensor{T}{^\mu_\nu}$ or 
$\tensor{T}{_\mu^\nu}$. We will not bother to specify their transformation properties.
These components are related to each other through the multiplication by the metric
tensor. Thus, 
\begin{equation}\label{c1e50}
\tensor{T}{_{\mu\nu}} = g_{\mu\alpha}\tensor{T}{^\alpha_\nu} =
g_{\mu\alpha}\tensor{T}{_\nu^\alpha} = g_{\mu\alpha}g_{\nu\beta}T^{\alpha\beta}.
\end{equation}
Note that the components $\tensor{T}{^\mu_\nu}$ and $\tensor{T}{_\mu^\nu}$ are 
different.

\item A unit 4-tensor is defined as
\begin{equation}\label{c1e51}
\tensor{\delta}{^\mu_\nu} = \tensor{\delta}{_\mu^\nu} = \begin{cases}
1 \text{  if  } \mu = \nu \\
0 \text{  if  } \mu \ne \nu.
\end{cases}
\end{equation}

\item The tensors $g_{\mu\nu}, g^{\mu\nu}, \tensor{\delta}{^\mu_\nu}$ and 
$\tensor{\delta}{_\mu^\nu}$ have the same components in all inertial frames. The 
competely asymmetric unit tensor of rank 4 also has the same property. It is
defined as
\begin{equation}\label{c1e52}
e^{\mu\nu\rho\sigma} = \begin{cases}
1 \text{  if  } \mu\nu\rho\sigma \text{  is an even permutation.} \\
-1 \text{  if  } \mu\nu\rho\sigma \text{  is an odd permutation.} \\
0 \text {  otherwise.}
\end{cases}
\end{equation}
If we set
\begin{equation}\label{c1e53}
e_{\mu\nu\rho\sigma} = -e^{\mu\nu\rho\sigma}
\end{equation}
then $e_{\mu\nu\rho\sigma}e^{\mu\nu\rho\sigma}$ is just the negative of the total 
number of non-zero components of $e_{\mu\nu\rho\sigma}$, which is the same as the
number of permutation of the four indices. Thus,
\begin{equation}\label{c1e54}
e_{\mu\nu\rho\sigma}e^{\mu\nu\rho\sigma} = -4! = -24.
\end{equation}
We get negative sign becaue $e^{0123} = 1, e_{0123} = -1$ so that their product 
is $-1$.
\end{enumerate}

\chapter{Relativistic Mechanics}\label{c2}
\begin{enumerate}
\item In the previous chapter, we argued that the expression
\begin{equation}\label{c2e1}
\delta t = \frac{1}{c}\int_A^B ds
\end{equation}
has a maximum for a particle if it is stationary. In this expression, $A$ and 
$B$ are world points and 
\[
ds = \sqrt{dx^\nu dx_\nu}.
\]
It follows immediately that, among all particles that start their journey at 
$A$ and end it at $B$, the expression
\begin{equation}\label{c2e2}
-\frac{1}{c}\int_A^B ds,
\end{equation}
has a minimum for the particle that is stationary.

\item We also know that for a classical system, there exists an action integral
$S$ whose value is an extremum for a path followed by the system in the 
configuration space. It is a minimum over an infinitesiml length of the path.

\item If we have to extend this idea to relativistic mechanics then the integral
must be invariant under Lorentz transformation. Therefore, it must be a true scalar.
One example of such an integral is given by \eqref{c2e2}. We can mildly generalise
it to
\begin{equation}\label{c2e3}
S = -\alpha \int_A^B ds,
\end{equation}
where $\alpha$ is a constant. From the discussion around equations \eqref{c2e1}
and \eqref{c2e3}, we infer that $\alpha > 0$.

\item We would like to write the action integral for relativistic systems as
\begin{equation}\label{c2e4}
S = \int_{t_1}^{t_2}Ldt,
\end{equation}
in analogy for the classical systems. From equations \eqref{c2e3} and 
\eqref{c2e4}, we get
\begin{equation}\label{c2e5}
L = -\alpha c\sqrt{1 - \frac{v^2}{c^2}},
\end{equation}
where
\[
v^i = \td{x^i}{t}
\]
is the 3-velocity of the particle. Can we guess $\alpha$?

\item As $v/c \rightarrow 0$, $L$ of \eqref{c2e5} should go over to the classical
Lagrangian for a free particle, which is just $mv^2/2$. For small $v/c$, we can
write \eqref{c2e5} as
\[
L = -\alpha c\left(1 - \frac{1}{2}\frac{v^2}{c^2}\right) = 
-\alpha c + \frac{\alpha}{2}\frac{v^2}{c}.
\]
A constant term in a Lagrangian can always be ignored and we infer that
\begin{equation}\label{c2e6}
\alpha = mc.
\end{equation}
Therefore, we guess the correct form of the Lagrangian for a free relativistic
particle to be
\begin{equation}\label{c2e7}
L = -mc^2\sqrt{1 - \frac{v^2}{c^2}}.
\end{equation}

\item The generalised momentum corresponding to this Lagrangian is
\begin{equation}\label{c2e8}
p^i = \pdt{L}{v^i} = (-mc^2)\frac{1}{2}\left(1 - \frac{v^2}{c^2}\right)^{-1/2}\frac{-2v^i}{c^2}
= \frac{mv^i}{\sqrt{1 - v^2/c^2}}.
\end{equation}
If
\begin{eqnarray}
\vec\beta &=& \frac{\vec{v}}{c} \\ \label{c2e9}
\gamma &=& \frac{1}{\sqrt{1 - \beta^2}} \label{c2e10}
\end{eqnarray}
then
\begin{equation}\label{c2e11}
\td{p^i}{t} = m\gamma \td{v^i}{t} + m\td{\gamma}{t}v^i.
\end{equation}
This equation is quite different from the one used in classical mechanics.

\item The energy of the particle is
\[
\mathcal{E} = p^i v_i - L = mv^2\gamma + \frac{mc^2}{\gamma} = m\gamma\left(v^2 + \frac{c^2}{\gamma^2}\right)
\]
so that
\begin{equation}\label{c2e12}
\mathcal{E} = m\gamma\left(v^2 + c^2\left(1 - \frac{v^2}{c^2}\right)\right) 
= mc^2\gamma
\end{equation}

\item From \eqref{c2e12} we see that the energy of the particle is not zero in
the relativistic framework even if the particle is resting. When $v = 0$, $\gamma
= 1$ and the particle's energy is
\begin{equation}\label{c2e13}
\mathcal{E} = mc^2.
\end{equation}
For small velocities, we can approximate \eqref{c2e12} to
\begin{equation}\label{c2e14}
\mathcal{E} = mc^2\left(1 + \frac{1}{2}\frac{v^2}{c^2} + O(\beta^4)\right) = 
mc^2 + \frac{1}{2}mv^2 + O(\beta^4).
\end{equation}

\item Since $\beta < 1$ for material bodies, $\gamma > 1$ and hence, from 
\eqref{c2e12} $\mathcal{E} \ge mc^2 > 0$. Thus energy of a free particle is always positive.
This is also true in classical mechanics for an elementary particle, that is a
particle without internal degrees of freedom. For a composite body, in classical
mechanics, energy can be negative and is determined to within a constant. However,
in the relativistic regime this is not so, $\mathcal{E} > 0$ is always true.

\item For a composite body at rest, the energy $\mathcal{E} = Mc^2$ is not the same as 
\[
\sum_{i=1}^N m_ic^2,
\]
where $m_i, i = 1, \ldots, N$ are the rest masses of the constituent particles.
The rest energy will also include the energy of interaction between the particles.
Since 
\[
M \ne \sum_{i=1}^N m_i
\]
conservation of mass is not true in relativistic mechanics.

\item From \eqref{c2e8},
\[
p^2 = p_ip^i = m^2v^2\gamma^2
\]
and from \eqref{c2e12}
\[
\mathcal{E}^2 = m^2c^4\gamma^2 \Rightarrow \mathcal{E}^2 - \mathcal{E}^2\beta^2 
= m^2c^4 \Rightarrow \mathcal{E}^2 - p^2c^2  = m^2c^4,
\]
so that
\begin{equation}\label{c2e15}
\mathcal{E}^2 = p^2c^2 + m^2c^4.
\end{equation}
Since $\mathcal{E}$ is expressed in terms of $p$, we can as well write the Hamiltonian as
\begin{equation}\label{c2e16}
\mathcal{H} = c\sqrt{p^2 + m^2c^2}
\end{equation}

\item Since $p^i = mv^i\gamma$ and $\mathcal{E} = mc^2\gamma$, we also have
\begin{equation}\label{c2e17}
p^i = \frac{\mathcal{E}}{c^2}v^i.
\end{equation}
If $v \rightarrow c$, $\gamma \rightarrow \infty$. In this limit, both $\mathcal{E}$ and
$p^i$ blow up unless $m = 0$. In that case, the two are related by
\begin{equation}\label{c2e18}
\mathcal{E} = pc.
\end{equation}

\item From equations \eqref{c2e4}, \eqref{c2e5} and \eqref{c2e6}, we can write
the principle of least action as
\begin{equation}\label{c2e19}
\delta S = -mc\delta\int_{t_1}^{t_2} ds = 0.
\end{equation}
Since $ds = \sqrt{dx^\mu dx_\mu}$, 
\begin{eqnarray*}
\delta ds &=& \frac{1}{2}\frac{(\delta dx^\mu)dx_\mu + dx^\mu (\delta dx_\mu)}{ds} \\
 &=& \frac{1}{2}\frac{(\delta dx^\mu)dx_\mu + dx_\mu (\delta dx^\mu)}{ds} \\
 &=& \frac{dx_\mu(\delta dx^\mu)}{ds} \\
 &=& u_\mu \delta(dx^\mu) \\
 &=& u_\mu d(\delta x^\mu),
\end{eqnarray*}
where we used \eqref{c1e78}. Thus, equation \eqref{c2e19} becomes
\begin{equation}\label{c2e20}
\delta S = -mc\int_{t_1}^{t_2}u_\mu d(\delta x^\mu)
\end{equation}
Integrating by parts,
\begin{equation}\label{c2e21}
\delta S = -mc u_\mu \delta x^\mu\Big|_{t_1}^{t_2} + 
mc\int_{t_1}^{t_2}\delta x^\mu \td{u_\mu}{s}ds.
\end{equation}
Since variations in trajectories vanish at the end points, the first term on the
right hand side is zero and hence
\begin{equation}\label{c2e22}
\delta S = mc\int_{t_1}^{t_2}\delta x^\mu \td{u_\mu}{s}ds.
\end{equation}
$\delta S = 0$ for all possible variations $\delta x^\mu$ therefore implies that 
for a free particle,
\begin{equation}\label{c2e23}
\td{u_\mu}{s} = 0.
\end{equation}
The particle travels with a uniform velocity, unsurprisingly.

\item Now fix $t_1$ and let $t_2$ be varied for true trajectories, that is the
ones which occur in reality and for which \eqref{c2e23} is valid. Then we have 
from
\eqref{c2e21},
\begin{equation}\label{c2e24}
\delta S = -mcu_\mu \delta (x^\mu)_{t_2},
\end{equation}
so that
\begin{equation}\label{c2e25}
mcu_\mu = -\pdt{S}{x^\mu}.
\end{equation}
The 4-vector $mcu^\mu$, from equations \eqref{c1e79} to \eqref{c1e82} is
\[
u^\mu = \left(\gamma, \frac{v^i}{c}\gamma\right)
\]
so that
\[
mcu^\mu = \left(mc\gamma, mv^i\gamma\right) = \left(\frac{\mathcal{E}}{c}, p^i\right)
\]
We define the 4-momemtum $p^\mu$ as
\begin{equation}\label{c2e26}
p^\mu = \left(\frac{\mathcal{E}}{c}, p^i\right)
\end{equation}
so that \eqref{c2e25} becomes
\begin{equation}\label{c2e27}
p_\mu = -\pdt{S}{x^\mu}.
\end{equation}
Note that if $p^\mu$ is given by \eqref{c2e26} then
\begin{equation}\label{c2e28}
p_\mu = \left(\frac{\mathcal{E}}{c}, -p^i\right).
\end{equation}
The transformation equations for $p^\mu$ are given by
\begin{equation}\label{c2e29}
\frac{\mathcal{E}}{c} = \gamma\left(\frac{\bar{\mathcal{E}}}{c} + \beta\bar{p}^1\right);
p^1 = \gamma\left(\bar{p}^1 + \beta\frac{\bar{\mathcal{E}}}{c}\right); p^2 = \bar{p}^2;
p^3 = \bar{p}^3.
\end{equation}
We also have
\begin{equation}\label{c2e30}
p^\mu p_\mu = \frac{\mathcal{E}^2}{c^2} - p^2 = m^2c^2.
\end{equation}
We now define the 4-force as 
\begin{equation}\label{c2e31}
g^\mu = \td{p^\mu}{s}.
\end{equation}
Since $ds = \sqrt{c^2(dt)^2 - dx_i dx^i}$,
\begin{equation}\label{c2e32}
ds = cdt\sqrt{1 - \frac{v^2}{c^2}} = \frac{c}{\gamma}dt.
\end{equation}
Therefore,
\[
g^\mu = \frac{\gamma}{c}\left(\frac{1}{c}\td{\mathcal{E}}{t}, \td{\vec{p}}{t}\right).
\]
Since $\dot{\mathcal{E}} = \vec{f}\cdot\vec{v}$, we can write this as
\begin{equation}\label{c2e33}
g^\mu = \left(\frac{\gamma}{c^2}\vec{f}\cdot\vec{v}, \frac{\gamma}{c}\td{\vec{p}}{t}\right).
\end{equation}

\item From equations \eqref{c2e27} and \eqref{c2e30},
\[
\left(-\pdt{S}{x^\mu}\right)\left(-\pdt{S}{x_\mu}\right) = m^2c^2
\]
or
\begin{equation}\label{c2e34}
\pdt{S}{x^\mu}\pdt{S}{x_\mu} = g^{\mu\nu}\pdt{S}{x^\mu}\pdt{S}{x^\nu} = m^2c^2
\end{equation}
is the relativistic Hamilton-Jacobi equation. Expanding the sum
\begin{equation}\label{c2e35}
\frac{1}{c^2}\left(\pdt{S}{t}\right)^2 - \left(\pdt{S}{x}\right)^2 
- \left(\pdt{S}{y}\right)^2 - \left(\pdt{S}{z}\right)^2 = m^2c^2.
\end{equation}

\item We now consider the transformation of a distribution function in momentum
space. If $f$ is such a function then $f(\vec{p})dp_xdp_ydp_z$ is the number of
particles with momenta in the range $(p_x, p_y, p_z)$ to $(p_x + dp_x, p_y + dp_y,
p_z + dp_z)$. To understand the properties of the ``volume'' element $dp_xdp_ydp_z$,
we consider the 4-dimensional energy-momentum space whose axes determine the
coordinates of 4-momentum $p^\mu$. From equation \eqref{c2e30}, the 4-momentum is
a constant. Therefore, the motion of the system happens on the hypersurface with
equation \eqref{c2e30}, that is, $p_\mu p^\mu = m^2c^2$. An element on the 
hypersurface is a 4-vector normal to it, the way an element on the surface of a 
3-sphere is normal to it. From equations \eqref{c1e71} and \eqref{c1e72}, this
vector is
\begin{equation}\label{c2e36}
dS^\mu = -\frac{1}{6}\epsilon^{\mu\nu\rho\sigma}dS_{\nu\rho\sigma}.
\end{equation}
so that $dS^0 = dp_xdp_ydp_z$. Further, the way the element on the 2-surface of
a 3-sphere at $\vec{r}$ is parallel to $\vec{r}$, $dS^\mu$ too is parallel to 
$p^\mu$. Therefore, there exists a constant, say $k$, such that $dS^\mu = kp^\mu$.
In particular, for $\mu = 0$, the ratio
\begin{equation}\label{c2e37}
\frac{dS^0}{p^0} = \frac{dp_xdp_ydp_z}{\mathcal{E}} = k,
\end{equation}
is an invariant under a Lorentz transformation. The number of particles $f(\vec{p})
dp_xdp_ydp_z$ is also an invariant. If we write
\[
f(\vec{p})dp_xdp_ydp_z = f(\vec{p})\mathcal{E}\frac{dp_xdp_ydp_z}{\mathcal{E}}
\]
then we see that
\[
f(\vec{p})dp_xdp_ydp_z = f^\op(\vec{p}^\op)dp_x^\op dp_y^\op dp_z^\op
\]
implies, from the invariance of $dS^0/p^0$ that
\begin{equation}\label{c2e38}
f^\op(\vec{p}^\op) = \frac{f(\vec{p})\mathcal{E}}{\mathcal{E}^\op}.
\end{equation}
In ``spherical'' momentum coordinates $dp_xdp_ydp_z = p^2dpdo$, where $do$ is the
solid angle in momentum space about $p^i$. Putting this in equation \eqref{c2e37}
we get the invariance of
\[
\frac{dp_xdp_ydp_z}{\mathcal{E}} = \frac{p^2dpdo}{\mathcal{E}} 
\]
Since $\mathcal{E}^2 = p^2c^2 + m^2c^4$, $\mathcal{E}d\mathcal{E} = c^2 pdp$, therefore,
\[
\frac{p^2dpdo}{\mathcal{E}} = \frac{pd\mathcal{E}do}{c^2}
\]
is also an invariant. In particular $pd\mathcal{E}do$ is also invariant.

\item We will now look at the distribution function over phase space. Thus,
$f(\vec{r}, \vec{p})dxdydzdp_xdp_ydp_z$ is the number of particles in the phase
space volume element $dxdydzdp_xdp_ydp_z$ around a point $(\vec{r}, \vec{p})$ in
a reference frame $K$. We want to find $f^\op(\vec{r}^\op, \vec{p}^\op)$ to which
$f$ transforms in another frame $K^\op$.

Let $K_0$ be the frame in which the particles in volume element are at rest. It
is possible to find such a frame if we choose a small enough volume element. If
$dV_0$ is the volume of (position) space in $K_0$ then
\begin{equation}\label{c2e39}
dV = \frac{dV_0}{\gamma(v)}; dV^\op = \frac{dV_0}{\gamma(v^\op)}
\end{equation}
are the same volumes as measured in $K$ and $K^\op$ respectively. From these 
equations we have
\begin{equation}\label{c2e40}
\frac{dV}{dV^\op} = \frac{\gamma(v^\op)}{\gamma(v)}.
\end{equation}
If $\mathcal{E}_0, \mathcal{E}, \mathcal{E}^\op$ are the values of energy in the 
three frames then we also have
\begin{equation}\label{c2e41}
\mathcal{E} = \mathcal{E}_0\gamma(v); \mathcal{E}^\op = \mathcal{E}_0\gamma(v^\op)
\end{equation}
because $p_0^i = 0$. Using \eqref{c2e41} in \eqref{c2e40} we get
\begin{equation}\label{c2e42}
\frac{dV}{dV^\op}  = \frac{\mathcal{E}^\op}{\mathcal{E}}.
\end{equation}
From \eqref{c2e37} we also have
\begin{equation}\label{c2e43}
\frac{dp_xdp_ydp_z}{dp_x^\op dp_y^\op dp_z^\op} = \frac{\mathcal{E}}{\mathcal{E}^\op}.
\end{equation}
The previous two equations give
\begin{equation}\label{c2e44}
dVdp_xdp_ydp_z = dV^\op dp_x^\op dp_y^\op dp_z^\op
\end{equation}
that is, the phase space volume is invariant under a Lorentz transformation. 
Therefore, we also have
\begin{equation}\label{c2e45}
f(\vec{r}, \vec{p}) = f(\vec{r}^\op, \vec{p}^\op).
\end{equation}

\item Before proceeding further, we note that for any vector $A^\mu$, the differential
$d^4A = dA^0dA^1dA^2dA^3$ is Lorentz invariant. This is because, $d^4A^\op = 
\det J  d^4A$, where $J$ is the Jacobian for the transformation. Since a Lorentz
transformation is a rotation in 4-space, $\det J = 1$. We can quickly confirm it
by writing the transformation in matrix form as
\[
\begin{bmatrix}x^0 \\ x^1 \\ x^2 \\ x^3\end{bmatrix} = 
\begin{bmatrix} \gamma & \beta\gamma & 0 & 0 \\
\beta\gamma & \gamma & 0 & 0 \\
0 & 0 & 1 & 1 \\
0 & 0 & 0 & 1\end{bmatrix}
\begin{bmatrix}\bar{x}^0 \\ \bar{x}^1 \\ \bar{x}^2 \\ \bar{x}^3\end{bmatrix}.
\]
Clearly,
\[
\det J = \gamma^2(1 - \beta^2) = 1.
\]

\item We now consider the problem of writing an expression for the cross-section
of collisions. Let a beam of particles with number density $n_1$ collide another
beam with number density $n_2$. Then the number of collisions $d\nu$ between them
in a time $dt$ and a volume $dV$ depends on $n_1, n_2, dV, dt$ and $v_r$, the 
relative speed of one beam with respect to the other. Thus,
\[
d\nu \propto n_1n_2v_r dVdt
\]
or
\begin{equation}\label{c2e46}
d\nu = \sigma n_1n_2v_r dVdt,
\end{equation}
where $\sigma$ is called the collision cross section. Its dimensions are that of
an area. Since $dVdt = dx^\mu/c$, it is invariant under a Lorentz transformation. 
$d\nu$ is invariant because it is a scalar. Therefore, the quantity $\sigma v_r 
n_1n_2$ is also an invariant.

However, this expression is developed only in the frame of the reference in which
one of the beams is stationary. In order to generalise it to an arbitrary frame,
we surmise that equation \eqref{c2e46} is generalised to
\begin{equation}\label{c2e47}
d\nu = An_1n_2dVdt,
\end{equation}
where $A = \sigma v_r$ in the frame of reference in which one of the beams is
stationary. 

The number of particles in a volume $dV$ is $ndV$ and is invariant. In another
frame of reference, it would be $n^\op dV^\op$. If $dV$ is the volume in measured
in a frame in which it is at rest, $dV^\op = dV/\gamma$. Thus,
\begin{equation}\label{c2e48}
n^\op dV^\op = n dV \Rightarrow n^\op\frac{dV}{\gamma} = ndV \Rightarrow n^\op
= n\gamma.
\end{equation}
From equation \eqref{c2e12}, 
\begin{equation}\label{c2e49}
n^\op = \frac{n\mathcal{E}}{mc^2}.
\end{equation}
Invariance of $An_1n_2$ thus follows from that of $A\mathcal{E}_1\mathcal{E}_2$. 
The dot product of $p_1^\mu$ and $p_2^\mu$ is a scalar and therefore invariant.
Therefore, the quantity
\[
\frac{A\mathcal{E}_1\mathcal{E}_2}{p_1^\mu {p_2}_\mu}
\]
is also an invariant. In the rest frame of beam 2, it becomes $\sigma v_r$, so that
\begin{equation}\label{c2e50}
A = \sigma v_r\frac{p_1^\mu {p_2}_\mu}{\mathcal{E}_1\mathcal{E}_2}.
\end{equation}
In the rest frame of particle 2, $p_2^\mu = (m_2c, 0, 0, 0)$, so that 
\[
p_1^\mu {p_2}_\mu = m_1\gamma(v_r) m_2 c^2 = 
\frac{m_1m_2c^2}{\sqrt{1 - v_r^2/c^2}} \Rightarrow 1 - \frac{v_r^2}{c^2} = 
\frac{m_1^2m_2^2c^4}{(p_1^\mu {p_2}_\mu)^2},
\]
or that
\begin{equation}\label{c2e51}
v_r = c\sqrt{1 - \frac{m_1^2m_2^2c^4}{(p_1^\mu {p_2}_\mu)^2}}.
\end{equation}
Now, 
\[
p_1^\mu {p_2}_\mu = m_1m_2\gamma(v_1)\gamma(v_2)c^2\left(1 - \frac{\vec{v}_1\cdot\vec{v}_2}{c^2}\right).
\]
so that
\begin{eqnarray*}
v_r &=& c\sqrt{1 - \frac{1}{\gamma^2(v_1)\gamma^2(v_2)\left(1 - \frac{\vec{v}_1\cdot\vec{v}_2}{c^2}\right)^2}} \\
 &=& \frac{1}{\gamma(v_1)\gamma(v_2)\left(1 - \frac{\vec{v}_1\cdot\vec{v}_2}{c^2}\right)}
 \sqrt{\gamma^2(v_1)\gamma^2(v_2)\left(1 - \frac{\vec{v}_1\cdot\vec{v}_2}{c^2}\right)^2 - 1} \\
 &=& \frac{1}{\left(1 - \frac{\vec{v}_1\cdot\vec{v}_2}{c^2}\right)}
 \sqrt{\left(1 - \frac{\vec{v}_1\cdot\vec{v}_2}{c^2}\right)^2 - \gamma^{-2}(v_1)\gamma^{-2}(v_2)} \\
 &=& \frac{1}{\left(1 - \frac{\vec{v}_1\cdot\vec{v}_2}{c^2}\right)}
 \sqrt{\left(1 - \frac{\vec{v}_1\cdot\vec{v}_2}{c^2}\right)^2 - \left(1 - \frac{v_1^2}{c^2}\right)
 \left(1 - \frac{v_2^2}{c^2}\right)}
\end{eqnarray*}
We now simplify the notation a bit, and let
\begin{eqnarray}
\vec{\beta}_1 &=& \frac{\vec{v}_1}{c} \label{c2e52} \\
\vec{\beta}_2 &=& \frac{\vec{v}_2}{c} \label{c2e53}
\end{eqnarray}
so that
\begin{eqnarray*}
v_r &=& \frac{\sqrt{(1 - \vec{\beta}_1\cdot\vec{\beta}_2)^2 - (1 - \beta_1^2 - \beta_2^2 + \beta_1^2\beta_2^2)}}
{1 - \vec{\beta}_1\cdot\vec{\beta}_2} \\
 &=& \frac{\sqrt{1 - 2\vec{\beta}_1\cdot\vec{\beta}_2 + \beta_1^2\beta_2^2\cos^2\theta - 
 (1 - \beta_1^2 - \beta_2^2 + \beta_1^2\beta_2^2)}}{1 - \vec{\beta}_1\cdot\vec{\beta}_2} \\
 &=& \frac{\sqrt{\beta_1^2 - 2\vec{\beta}_1\cdot\vec{\beta}_2 + \beta_2^2 - \beta_1^2\beta_2^2\sin^2\theta }}
 {1 - \vec{\beta}_1\cdot\vec{\beta}_2}
\end{eqnarray*}
Finally, we get
\begin{equation}\label{c2e54}
v_r = \frac{\sqrt{(\vec{\beta}_1 - \vec{\beta}_2)^2 - (\vec{\beta}_1 \times \vec{\beta}_2)^2}}
  {1 - \vec{\beta}_1\cdot\vec{\beta}_2}
\end{equation}
In terms of $\vec{\beta}_i$,
\begin{equation}\label{c2e55}
p_1^\mu {p_2}_\mu = m_1m_2\gamma(v_1)\gamma(v_2)c^2\left(1 - \vec{\beta}_1\cdot\vec{\beta}_2\right).
\end{equation}
Substituting \eqref{c2e54} and \eqref{c2e55} in \eqref{c2e50}, we get
\begin{equation}\label{c2e56}
A = \sigma \sqrt{(\vec{\beta}_1 - \vec{\beta}_2)^2 - (\vec{\beta}_1 \times \vec{\beta}_2)^2},
\end{equation}
where we also used the fact that $\mathcal{E}_i = m_ic^2\gamma$. Therefore, equation \eqref{c2e47} gives
\begin{equation}\label{c2e57}
d\nu = \sigma \sqrt{(\vec{\beta}_1 - \vec{\beta}_2)^2 - (\vec{\beta}_1 \times \vec{\beta}_2)^2}
	n_1n_2dVdt.
\end{equation}

\item We now consider elastic collisions between particles. Let a particle of
mass $m_1$, energy $\mathcal{E}_1$ and momentum $\vec{p}_1$ collide with a particle of mass
$m_2$, energy $\mathcal{E}_2$ and momentum $\vec{p}_2$. Let $\mathcal{E}_1^\op, 
\vec{p}_1^\op$ ($\mathcal{E}_2^\op, \vec{p}_2^\op$) be the energy and momenta after 
collision. By the conservation of 4-momentum,
\begin{equation}\label{c2e58}
p_1^\mu + p_2^\mu = p_1^{\op\mu} + p_2^{\op\mu}.
\end{equation}
Then $p_1^\mu + p_2^\mu - p_1^{\op\mu} = p_2^{\op\mu}$. Squaring each side, that
is, taking a product with the covariant components of each side, we get
\begin{equation}\label{c2e59}
m_1^2c^2 + p_{1\mu}p_2^\mu - p_{1\mu}p_1^{\op\mu} - p_{2\mu}p_1^{\op\mu} = 0,
\end{equation}
where we used the fact that $p_\mu p^\mu = \mathcal{E}^2/c^2 - p^2 = 
(\mathcal{E}^2 - p^2c^2)/c^2 = m^2c^2$.
Likewise, squaring $p_1^\mu + p_2^\mu - p_2^{\op\mu} = p_1^{\op\mu}$ gives
\begin{equation}\label{c2e60}
m_2^2c^2 + p_{1\mu}p_2^\mu - p_{1\mu}p_2^{\op\mu} - p_{2\mu}p_2^{\op\mu} = 0.
\end{equation}
If $m_2$ was at rest in the laboratory frame of reference then $p_2^\mu = (\mathcal{E}_2/c,
0)$. Using it in \eqref{c2e59} gives
\begin{equation}\label{c2e61}
m_1^2c^2 + \frac{\mathcal{E}_1\mathcal{E}_2}{c^2} - p_{1\mu}p_1^{\op\mu} 
- \frac{\mathcal{E}_1^\op \mathcal{E}_2}{c^2} = 0.
\end{equation}
Since $\mathcal{E}_2 = m_2c^2$ and $p_{1\mu}p_1^{\op\mu} = 
\mathcal{E}_1\mathcal{E}_1^\op/c^2 - \vec{p}_1\cdot\vec{p}_1^\op$, \eqref{c2e61} becomes
\begin{equation}\label{c2e62}
m_1^2c^2 + m_2\mathcal{E}_1 - \mathcal{E}_1\mathcal{E}_1^\op/c^2 + 
\vec{p}_1\cdot\vec{p}_1^\op - m_2E_1^\op = 0,
\end{equation}
or
\[
p_1p_1^\op\cos\theta_1 = \mathcal{E}_1^\op(\mathcal{E}_1/c^2 + m_2) - E\mathcal{E}_1m_2 - m_1^2c^2
\]
so that
\begin{equation}\label{c2e63}
\cos\theta_1 = \frac{\mathcal{E}_1^\op(\mathcal{E}_1/c^2 + m_2) - \mathcal{E}_1m_2 - m_1^2c^2}{p_1p_1^\op}
\end{equation}
Using $p_2^\mu = (\mathcal{E}_2/c, 0)$ in \eqref{c2e60} we get
\[
m_2^2c^2 + \frac{\mathcal{E}_1\mathcal{E}_2}{c^2} - \frac{\mathcal{E}_1\mathcal{E}_2^\op}{c^2} + 
\vec{p}_1\cdot\vec{p}_2^\op - \frac{\mathcal{E}_2\mathcal{E}_2^\op}{c^2} = 0.
\]
Since $\mathcal{E}_2 = m_2c^2$, this equation can be written as
\[
m_2^2c^2 + m_2\mathcal{E}_1 - \frac{\mathcal{E}_1\mathcal{E}_2^\op}{c^2} + 
\vec{p}_1\cdot\vec{p}_2^\op - m_2\mathcal{E}_2^\op = 0.
\]
Reorganising it, we get
\begin{equation}\label{c2e64}
\cos\theta_2 = \frac{(\mathcal{E}_2^\op - m_2c^2)(\mathcal{E}_1/c^2 + m_2)}{p_1p_2^\op}.
\end{equation}

An experiment records $\theta_1$ and $\theta_2$ for controlled values of $E_1$
and fixed masses $m_1$ and $m_2$. Equations \eqref{c2e63} and \eqref{c2e64}
can be inverted, in principle, to get $\mathcal{E}_1^\op$ and $\mathcal{E}_2^\op$ in terms of
the scattering angles $\theta_1$ and $\theta_2$.

\item We will examine an elastic collision considered in the previous point in 
the centre of mass reference frame. Here, the momenta of the two particles retain
their magnitudes but change their directions. They rotate by an angle, say $\chi$.
We will denote the quantities in this frame with a bar. Thus,
\begin{equation}\label{c2e65}
\bar{p}_{1\mu}\bar{p}_1^{\op\mu} = \frac{\bar{\mathcal{E}}_1\bar{\mathcal{E}}_1^\op}{c^2} - 
\bar{\vec{p}}_1\cdot\bar{\vec{p}}+1^\op = \frac{\bar{\mathcal{E}}_1\bar{\mathcal{E}}_1^\op}{c^2} - 
\bar{p}^2_1\cos\chi,
\end{equation}
because only the direction of momentum changes in this frame. Further, in an
elastic collision, $\bar{\mathcal{E}}_1 = \bar{\mathcal{E}}_1^\op$ so that
\begin{equation}\label{c2e66}
\bar{p}_{1\mu}\bar{p}_1^{\op\mu} = \frac{\bar{\mathcal{E}}_1^2}{c^2} - \bar{p}_1^2\cos\chi.
\end{equation}
Since $\bar{\mathcal{E}}_1^2 = m_1^2c^4 + \bar{p}^2c^2$, 
\begin{equation}\label{c2e67}
\bar{p}_{1\mu}\bar{p}_1^{\op\mu} = m_1^2c^2 + \bar{p}_1^2(1 - \cos\chi).
\end{equation}
If the particle with mass $m_2$ was at rest in the laboratory frame,
\begin{eqnarray}
p_{1\mu}p_2^\mu &=& \mathcal{E}_1m_2c^2 \label{c2e68} \\
p_{2\mu}p_1^{\op\mu} &=& m_2c^2 \mathcal{E}_1^\op \label{c2e69}
\end{eqnarray}
so that
\begin{equation}\label{c2e70}
m_2c^2(\mathcal{E}_1^\op - \mathcal{E}_1) = p_{2\mu}p_1^{\op\mu} - p_{1\mu}p_2^\mu.
\end{equation}
Since scalars are invariant under a transformation to centre of frame reference
we can as well use \eqref{c2e66} to write
\begin{equation}\label{c2e71}
p_{1\mu}p_1^{\op\mu} = m_1^2c^2 + \bar{p}_1^2(1 - \cos\chi).
\end{equation}
Substituting \eqref{c2e70} and \eqref{c2e71} into \eqref{c2e59}, we get
\begin{equation}\label{c2e72}
\mathcal{E}_1^\op - \mathcal{E}_1 = -\frac{\bar{p}_1^2}{m_2c^2}(1 - \cos\chi).
\end{equation}
We next use the fact that $p_{1\mu}p_2^\mu = \bar{p}_{1\mu}\bar{p}_2^\mu$ so that,
using \eqref{c2e68},
\begin{equation}\label{c2e73}
m_2c^2\mathcal{E}_1 = \bar{\mathcal{E}}_1\bar{\mathcal{E}}_2 - \bar{\vec{p}}_1\cdot\bar{\vec{p}}_2 = 
\bar{\mathcal{E}}_1\bar{\mathcal{E}}_2 + \bar{p}_1^2,
\end{equation}
as the two momenta are equal and opposite in the centre of mass system. Further, 
$\bar{\mathcal{E}}_1 = \sqrt{\bar{p}_1^2 + m_1^2c^4}, \bar{\mathcal{E}}_2 = \sqrt{\bar{p}_2^2 + m_2^2c^4}
= \sqrt{\bar{p}_1^2 + m_2^2c^4}$ so that equation \eqref{c2e72} becomes
\begin{equation}\label{c2e74}
m_2c^2\mathcal{E}_1 - \bar{p}_1^2 = \sqrt{\bar{p}_1^2 + m_1^2c^4}\sqrt{\bar{p}_1^2 + m_2^2c^4},
\end{equation}
whose solution is
\begin{equation}\label{c2e75}
\bar{p}_1^2 = \frac{m_2c^2(\mathcal{E}_1^2 - m_1^2c^4)}{m_1^2 + m_2^2 + 2m_2\mathcal{E}_1}.
\end{equation}
Substituting \eqref{c2e75} in \eqref{c2e72} we get
\begin{equation}\label{c2e76}
\mathcal{E}_1^\op = \mathcal{E}_1 - 
\frac{m_2(\mathcal{E}_1^2 - m_1^2c^4)}{m_1^2 + m_2^2 + 2m_2\mathcal{E}_1}(1 - \cos\chi).
\end{equation}
This equation gives the energy of the scatterd particle in terms of its initial
energy and angle of scattering in the centre of mass frame.

Energy conservation gives us $\mathcal{E}_1 + m_2c^2 = \mathcal{E}_1^\op + \mathcal{E}_2^\op$ so that
\[
\mathcal{E}_2^\op = \mathcal{E}_1 - \mathcal{E}_1^\op - m_2c^2
\]
so that
\begin{equation}\label{c2e77}
\mathcal{E}_2^\op = m_2c^2 + 
\frac{m_2(\mathcal{E}_1^2 - m_1^2c^4)}{m_1^2 + m_2^2 + 2m_2\mathcal{E}_1}(1 - \cos\chi).
\end{equation}

\item Let $x^\mu$ be the world-coordinates of particle. Under an infinitesimal 
rotation of the reference frame, the coordinates now become $x^{\op\mu}$. The two
are so close that one can express the relation between them as
\begin{equation}\label{c2e78}
x^{\op\mu} = x^\mu + x_\nu\delta\Omega^{\nu\mu}.
\end{equation}
Since the norm of this vector remains unchanged, $x^\op_\mu x^{\op\mu} = x_\mu 
x^\mu$. Now,
\[
x^\op_\mu = g_{\mu\nu}x^{\op\nu} = g_{\mu\nu}(x^\nu + x_\rho \delta\Omega^{\rho\nu})
= x_\mu + x_\rho g_{\mu\nu}\delta\Omega^{\rho\nu},
\]
so that
\begin{eqnarray*}
x^\op_\mu x^{\op\mu} &=& (x_\mu + x_\rho g_{\mu\nu}\delta\Omega^{\rho\nu})(x^\mu + 
x_\nu\delta\Omega^{\nu\mu}) \\
 &=& x_\mu x^\mu + x_\mu x_\nu \delta\Omega^{\nu\mu} + x_\rho g_{\mu\nu}x^\mu 
 \delta\Omega^{\rho\nu} \\
 &=& x_\mu x^\mu + x_\mu x_\nu \delta\Omega^{\nu\mu} + x_\rho x_\nu\delta
 \Omega^{\rho\nu} \\
 &=& x_\mu x^\mu + x_\mu x_\nu \delta\Omega^{\nu\mu} + x_\mu x_\nu\delta
 \Omega^{\mu\nu} \\
 &=& x_\mu x^\mu + x_\mu x_\nu (\delta\Omega^{\nu\mu} + \delta \Omega^{\mu\nu}).
\end{eqnarray*}
Here we have ignored terms quadratic in $\delta\Omega^{\mu\nu}$.
If $x^\op_\mu x^{\op\mu} = x_\mu x^\mu$ then we must have
\begin{equation}\label{c2e79}
\delta\Omega^{\nu\mu} = -\delta \Omega^{\mu\nu},
\end{equation}
that is, the tensor $\delta\Omega^{\mu\nu}$ us anti-symmetric.

\item For an infinitesimal change of coordinates, the change in action is
\begin{equation}\label{c2e80}
\delta S = \sum_{a} p_{\mu a}\delta x^{\mu a},
\end{equation}
where we have used equations \eqref{c2e24} and \eqref{c2e26} and the summation
is over all particles of the system. From equation \eqref{c2e78},
\begin{equation}\label{c2e81}
\delta S = \sum_{a} p_{\mu a}x_{\nu a}\delta\Omega^{\mu\nu}.
\end{equation}
The infinitesimal transformation $\delta\Omega^{\mu\nu}$ is the same for all
particles in the system. Therefore,
\begin{equation}\label{c2e82}
\delta S = \delta\Omega^{\mu\nu}\sum_{a} p_{\mu a}x_{\nu a}.
\end{equation}
We write the tensor $p_{\mu a}x_{\nu a}$ as a sum of symmmetric and anti-symmetric
tensors
\begin{equation}\label{c2e83}
p_{\mu a}x_{\nu a} = \frac{p_{\mu a}x_{\nu a} + p_{\nu a}x_{\mu a}}{2} +
\frac{p_{\mu a}x_{\nu a} - p_{\nu a}x_{\mu a}}{2}
\end{equation}
Since $\delta\Omega^{\mu\nu}$ is anti-symmetric,
\begin{eqnarray*}
\delta\Omega^{\mu\nu}\frac{p_{\mu a}x_{\nu a} + p_{\nu a}x_{\mu a}}{2} &=&
\frac{\delta\Omega^{\mu\nu}p_{\mu a}x_{\nu a}}{2} + 
\frac{\delta\Omega^{\mu\nu}p_{\nu a}x_{\mu a}}{2} \\
 &=& \frac{\delta\Omega^{\mu\nu}p_{\mu a}x_{\nu a}}{2} + 
\frac{\delta\Omega^{\nu\mu}p_{\mu a}x_{\nu a}}{2} \\
 &=& \frac{1}{2}(\delta\Omega^{\mu\nu} + \delta\Omega^{\nu\mu})p_{\mu a}x_{\nu a}.
\end{eqnarray*}
Because of the anti-symmetric nature of $\delta\Omega^{\mu\nu}$, the right hand 
side is zero. Thus, only the anti-symmetric part of \eqref{c2e83} survives in
the summation over $\mu\nu$ in \eqref{c2e82}. We, therefore, write \eqref{c2e82}
as
\begin{equation}\label{c2e84}
\delta S = \frac{1}{2}\delta\Omega^{\mu\nu}\sum_{a}p_{\mu a}x_{\nu a} - p_{\nu a}x_{\mu a}.
\end{equation}
Equivalently,
\begin{equation}\label{c2e85}
\delta S = \frac{1}{2}\delta\Omega_{\mu\nu}\sum_{a}p^{\mu a}x^{\nu a} - p^{\nu a}x^{\mu a}.
\end{equation}
The tensor
\begin{equation}\label{c2e86}
L^{\mu\nu} = x^{\mu}p^{\nu} - x^{\nu}p^{\mu}
\end{equation}
is called the angular momentum 4-tensor. Since $x^\mu = (ct, x^1, x^2, x^3), 
p^\mu = (\mathcal{E}/c, p^1, p^2, p^3)$,
\[
x^\mu p^\nu = \begin{bmatrix}t\mathcal{E} & p^1ct & p^2ct & p^3ct \\
x^1\mathcal{E}/c & x^1p^1 & x^1p^2 & x^1p^3 \\
x^2\mathcal{E}/c & x^2p^1 & x^2p^2 & x^2p^3 \\
x^3\mathcal{E}/c & x^3p^1 & x^3p^2 & x^3p^3
\end{bmatrix}
\]
and hence
\[
L^{\mu\nu} = \begin{bmatrix}t\mathcal{E} & p^1ct & p^2ct & p^3ct \\
x^1\mathcal{E}/c & x^1p^1 & x^1p^2 & x^1p^3 \\
x^2\mathcal{E}/c & x^2p^1 & x^2p^2 & x^2p^3 \\
x^3\mathcal{E}/c & x^3p^1 & x^3p^2 & x^3p^3
\end{bmatrix} - \begin{bmatrix}tE & x^1\mathcal{E}/c & x^2\mathcal{E}/c & x^3\mathcal{E}/c \\
p^1ct & x^1p^1 & x^2p^1 & x^3p^1 \\
p^2ct & x^1p^2 & x^2p^2 & x^3p^2 \\
p^3ct & x^1p^3 & x^2p^3 & x^3p^3
\end{bmatrix}
\]
or
\begin{equation}\label{c2e87}
L^{\mu\nu} = \begin{bmatrix}0 & p^1ct - x^1\mathcal{E}/c & p^2ct - x^2\mathcal{E}/c & p^3ct - x^3\mathcal{E}/c\\
p^1ct - x^1\mathcal{E}/c & 0 & x^1p^2 - x^2p^1 & x^1p^3 - x^3p^1 \\
p^2ct - x^2\mathcal{E}/c & x^2p^1 - x^1p^2 & 0 & x^2p^3 - x^3p^2 \\
p^2ct - x^3\mathcal{E}/c & x^3p^1 - x^3p^1 & x^3p^2 - x^2p^3 & 0
\end{bmatrix}
\end{equation}
This tensor is very similar to the electromagnetic field tensor with the electric
part given by the vector
\begin{equation}\label{c2e88}
\vec{L}_e = \vec{p}ct - \vec{x}\frac{\mathcal{E}}{c}
\end{equation}
and the magnetic part by the vector
\begin{equation}\label{c2e89}
\vec{L}_m = -(\vec{x} \times \vec{p}).
\end{equation}
If $L^{\mu\nu}$ is conserved then so are $\vec{L}_e$ and $\vec{L}_m$ individually.
For a system of multiple particles, it means that the quantity
\begin{equation}\label{c2e90}
\sum_a \left(\vec{p}_act - \vec{x}_a\frac{\mathcal{E}_a}{c}\right) = \text{constant.}
\end{equation}
Equivalently,
\begin{equation}\label{c2e91}
\frac{\sum_a\vec{p}_ac^2t}{\sum_a \mathcal{E}_a} - \frac{\sum_a\vec{x}_a \mathcal{E}_a}{\sum \mathcal{E}_a} = \text{constant.}
\end{equation}
Thus a pointl
\begin{equation}\label{c2e92}
\vec{R} = \frac{\sum_a\vec{x}_a \mathcal{E}_a}{\sum \mathcal{E}_a}
\end{equation}
travels with uniform velocity
\begin{equation}\label{c2e93}
\vec{V} = c^2\frac{\sum_a\vec{p}_a}{\sum_a \mathcal{E}_a}.
\end{equation}
Equation \eqref{c2e92} is analogous to the centre of mass vector of classical 
mechanics. The formula merely replaces mass in classical mechanics with energy.
\end{enumerate}

\chapter{Charges in Electromagnetic Fields}\label{c3}
\begin{enumerate}
\item Unless they are in touch with each other, particles interact with each
other via fields.

\item Disturbances in fields propagate information. It is a postulate of the 
special theory of relativity that the changes in the state of a system are
progagated at a finite speed not exceeding $c$, the speed of light in vacuum.

\item This restriction rejects the hypothesis of an ideal rigid body in relativity.
For suppose a force is applied to one end of a rigid body. Then in order that the
inter-particle distance is unchanged, the force has to be propagated instantaneously
to the other end. Therefore, in elementary analysis of the theory of relativity
all particles are assumed to be points.

\item The principle of action for a free particle is given by equation \eqref{c2e4}
and \eqref{c2e7},
\begin{equation}\label{c3e1}
S = -\int_{A}^{B}mcds,
\end{equation}
where $A$ and $B$ are world points.
For a particle in an electromagnetic field, the right hand side of this equation
should be augmented with another terms that represents the interaction of the
particle with the field. It should comprise of quantities describing the particle
as well as the field.

The particle's interaction with the field is described by a single quantity, its
charge $e$ and the field by its 4-potential $A_\mu$. The complete action is
\begin{equation}\label{c3e2}
S = -\int_A^B \left(mcds + \frac{e}{c}A_\mu dx^\mu\right).
\end{equation}

The 4-potential can be written as 
\begin{equation}\label{c3e3}
A_\mu = (\varphi, \vec{A})
\end{equation}
so that we can write \eqref{c3e2} as
\begin{equation}\label{c3e4}
S = \int_A^B\left(-mcds - e\varphi dt + 
\frac{e}{c}\vec{A}\cdot d\vec{x}\right).
\end{equation}
Since $ds = c\sqrt{1 - \beta^2}dt$ and $d\vec{x} = \vec{v}dt$,
\begin{equation}\label{c3e5}
S = \int_{t_1}^{t_2}\left(-mc^2\sqrt{1 - \beta^2} - e\varphi + 
\frac{e}{c}\vec{A}\cdot\vec{v}\right)dt.
\end{equation}
Thus, the Lagrangian for a particle in an electromagnetic field is
\begin{equation}\label{c3e6}
L = -mc^2\sqrt{1 - \beta^2} - e\varphi + \frac{e}{c}\vec{A}\cdot\vec{v}
= -mc^2\sqrt{1 - \beta^2} - e\varphi + \frac{e}{c}A^i v_i.
\end{equation}
From the Lagrangian, we get the generalised momentum
\begin{equation}\label{c3e7}
p^i = \pdt{L}{v^i} = m\gamma v^i + \frac{e}{c}A^i.
\end{equation}
From equation \eqref{c2e8}, the first term is just the generalised momentum for
the Lagrangian of \eqref{c2e7}. The Hamiltonian is given be the Legendre
transform
\begin{eqnarray*}
\mathcal{H} &=& p^iv_i - L \\
  &=& m\gamma v^iv_i + \frac{e}{c}A^iv_i + mc^2\sqrt{1 - \beta^2} + e\varphi - 
  \frac{e}{c}A^iv_i \\
  &=& \frac{mv^2}{\sqrt{1 - \beta^2}} + mc^2\sqrt{1 - \beta^2} + e\varphi \\
  &=& \frac{mc^2}{\sqrt{1 - \beta^2}} + e\varphi
\end{eqnarray*}
or that
\begin{equation}\label{c3e8}
\mathcal{H} = mc^2\gamma + e\varphi.
\end{equation}
Stricly speaking \eqref{c3e8} is not the correct form of the Hamiltonian because 
it is expressed in terms of the generalised velocity. To remedy this defect, we 
write
\eqref{c3e8} as
\[
\frac{(\mathcal{H} - e\varphi)^2}{c^2} = m^2c^2 \gamma^2.
\]
Now,
\begin{equation}\label{c3e9}
\gamma^2 = \frac{1}{1 - \beta^2} = 1 + \frac{\beta^2}{1 - \beta^2} = 
1 + \beta^2\gamma^2
\end{equation}
so that
\[
\frac{(\mathcal{H} - e\varphi)^2}{c^2} = m^2c^2(1 + \beta^2\gamma^2) 
\]
or that
\[
\frac{(\mathcal{H} - e\varphi)^2}{c^2} = m^2c^2 + m^2\gamma^2 v^2,
\]
From \eqref{c3e7} the last term on the rhs can be written as
\begin{equation}\label{c3e10}
\frac{(\mathcal{H} - e\varphi)^2}{c^2} = m^2c^2 + \left(p^i - \frac{e}{c}A^i\right)
\left(p_i - \frac{e}{c}A_i\right)
\end{equation}
Thus,
\begin{equation}\label{c3e11}
\mathcal{H} = e\varphi + \sqrt{m^2c^4 + c^2 
\left(p^i - \frac{e}{c}A^i\right)\left(p_i - \frac{e}{c}A_i\right)}
\end{equation}
is the correct form of the Hamiltonian. For speeds small as compared to $c$,
\[
\sqrt{1 - \beta^2} = 1 - \frac{1}{2}\beta^2
\]
so that the Lagrangian of \eqref{c3e6} can be written as
\begin{equation}\label{c3e12}
L = \frac{1}{2}mv^2 - e\varphi + \frac{e}{c}\vec{A}\cdot\vec{v},
\end{equation}
where we ignored the constant terms $-mc^2$. At non-relativistic speeds, $\gamma$
can be well-approximated by $1$ so that the generalised momentum is
\begin{equation}\label{c3e13}
\vec{p} = m\vec{v} + \frac{e}{c}\vec{A}
\end{equation}
and the Hamiltonian is
\begin{equation}\label{c3e14}
\mathcal{H} = \frac{1}{2m}\left(\vec{p} - \frac{e}{c}\vec{A}\right)^2 + e\varphi.
\end{equation}

From \eqref{c2e27},
\[
p^\nu = g^{\mu\nu}p_\mu = -g^{\mu\nu}\pdt{S}{x^\mu}
\]
so that
\[
p^0 = -\frac{1}{c}\pdt{S}{t}, \vec{p} = \grad S
\]
and hence the Hamilton-Jacobi equation is, from \eqref{c3e10},
\[
\frac{1}{c^2}\left(\pdt{S}{t} + e\varphi\right)^2 = m^2c^2 + 
\left(\grad S - \frac{e}{c}\vec{A}\right)^2
\]
where we replaced $\mathcal{H}$ in \eqref{c3e10} with $-(1/c)\partial_t S$ because
it is the energy of the particle. We can reexpress the relativistic Hamilton-Jacobi
equation as 
\begin{equation}\label{c3e15}
\left(\grad S - \frac{e}{c}\vec{A}\right)^2 - 
\frac{1}{c^2}\left(\pdt{S}{t} + e\varphi\right)^2 + m^2c^2 = 0.
\end{equation}

\item The Lagrangian of \eqref{c3e12} allows us to derive the equation of motion.
\begin{eqnarray*}
\grad L &=& -e\grad\varphi + \frac{e}{c}(\vec{A}\cdot\grad\vec{v} + 
 \vec{v}\cdot\grad\vec{A} + \vec{A}\times\curl\vec{v} + \vec{v}\times\curl\vec{A})\\ 
 &=& -e\grad\varphi + \frac{e}{c}(\vec{v}\cdot\grad\vec{A} + \vec{v}\times\curl\vec{A})\\ 
\grad_{\vec{v}}L &=& m\vec{v} + \frac{e}{c}\vec{A}.
\end{eqnarray*}
The equation for $\grad L$ simplifies because $\vec{v}$ is independent of $\vec{x}$.
Therefore, $\grad\vec{v}$ and $\curl\vec{v}$ are both zero.
\begin{equation}\label{c3e16}
\frac{d}{dt}\left(m\vec{v} + \frac{e}{c}\vec{A}\right) = 
-e\grad\varphi + \frac{e}{c}(\vec{v}\cdot\grad\vec{A} + \vec{v}\times\curl\vec{A})
\end{equation}
is the equation of motion. Now.
\begin{equation}\label{c3e17}
\td{\vec{A}}{t} = \pdt{\vec{A}}{t} + \vec{v}\cdot\vec{A}
\end{equation}
so that \eqref{c3e16} is simplified to
\begin{equation}\label{c3e18}
\td{\vec{p}}{t} = e\left(-\grad\varphi - \frac{1}{c}\pdt{\vec{A}}{t}\right)
+ \frac{e}{c}\vec{v}\times\curl\vec{A}
\end{equation}
We define
\begin{eqnarray}
\vec{E} &=& -\grad\varphi - \frac{1}{c}\pdt{\vec{A}}{t} \label{c3e19} \\
\vec{H} &=& \curl\vec{A} \label{c3e20}
\end{eqnarray}
and call $\vec{E}$ the electric field and $\vec{H}$ the magnetic field. The
quantity
\begin{equation}\label{c3e21}
\vec{F} = e\left(\vec{E} + \frac{\vec{v}}{c}\times\vec{H}\right)
\end{equation}
is called the Lorentz force. This equation depends on the form \eqref{c3e12} of
the Lagrangian and \eqref{c3e13} of the momentum both of which are valid only in
the non-relativistic limit. As a result, \eqref{c3e18} to \eqref{c3e21} are 
applicable only in the non-relativistic limit. We derive the equation of motion
that is valid at all speeds later in equation \eqref{c3e74}. Interestingly, the
relativistic equation takes the same form as the non-relativistic one.

\item Recall that 
\begin{equation}\label{c3e22}
p^\mu = mcu^\mu = (mc\gamma, mv^i\gamma) = \left(\frac{\mathcal{E}}{c}, \vec{p}\right)
\end{equation}
so that the energy of the particle is
\begin{equation}\label{c3e23}
\mathcal{E} = mc^2\gamma
\end{equation}
so that
\begin{equation}\label{c3e24}
\td{\mathcal{E}}{t} = mc^2\td{\gamma}{t} = 
-\frac{mc^2}{2}\gamma^3\left(\frac{-2\vec{v}}{c^2}\cdot\td{\vec{v}}{t}\right)
= m\gamma^3 \vec{v}\cdot\td{\vec{v}}{t}
\end{equation}
and
\[
\td{\vec{p}}{t} = m\gamma\td{\vec{v}}{t} + m\gamma^3\frac{\vec{v}}{c^2}\vec{v}\cdot\td{\vec{v}}{t} = 
m\gamma^3\left(m\gamma^{-2}\td{\vec{v}}{t} + \frac{\vec{v}}{c^2}\vec{v}\cdot\td{\vec{v}}{t}\right)
\]
or, since
\[
\gamma = \frac{1}{\sqrt{1 - v^2/c^2}},
\]
we have,
\[
\td{\vec{p}}{t} = m\gamma^3\left(\left(1 - \frac{v^2}{c^2}\right)\td{\vec{v}}{t} + 
\frac{\vec{v}}{c^2}\vec{v}\cdot\td{\vec{v}}{t}\right)
\]
Therefore,
\begin{equation}\label{c3e25}
\vec{v}\cdot\td{\vec{p}}{t} = m\gamma^3\left(\left(1 - \frac{v^2}{c^2}\right)\vec{v}\cdot\td{\vec{v}}{t}
+ \frac{v^2}{c^2}\vec{v}\cdot\td{\vec{v}}{t}\right) = m\gamma^3\vec{v}\cdot\td{\vec{v}}{t}
\end{equation}
From equations \eqref{c3e24} and \eqref{c2e25},
\begin{equation}\label{c3e26}
\td{\mathcal{E}}{t} = \vec{v}\cdot\td{\vec{p}}{t}.
\end{equation}
If we apply it to the equation of motion \eqref{c3e18} and use the definitions 
of the fields, we get
\begin{equation}\label{c3e27}
\td{\mathcal{E}}{t} = e\vec{v}\cdot\vec{E}.
\end{equation}

Here we have used the fact that
\begin{equation}\label{c3e28}
\td{v^2}{t} = \td{(\vec{v}\cdot\vec{v})}{t^2} = 2\vec{v}\cdot\td{\vec{v}}{t}.
\end{equation}

\item The equation of motion
\begin{equation}\label{c3e29}
\td{\vec{p}}{t} = e\left(\vec{E} + \frac{\vec{v}}{c}\times\vec{H}\right)
\end{equation}
is invariant under time reversal if we also reverse the direction of the magnetic
field.

\item A 4-potential $A_\mu$ gives a unique combination of electric and magnetic
fields. Hut the relationship is many-to-one. This is because, the potential
\begin{equation}\label{c3e30}
A_\mu^\op = A_\mu - \pdt{f}{x^\mu}
\end{equation}
results in
\begin{eqnarray}
\varphi^\op &=& \varphi - \frac{1}{c}\pdt{f}{t} \label{c3e31} \\
\vec{A}^\op &=& \vec{A} + \grad f \label{c3e32}
\end{eqnarray}
and the fields
\begin{eqnarray}
\vec{E}^\op &=& -\grad{\varphi^\op} - \frac{1}{c}\pdt{\vec{A}^\op}{t} \nonumber \\
 &=& -\grad\varphi + \frac{1}{c}\grad\pdt{f}{t} - 
      \frac{1}{c}\pdt{\vec{A}}{t} - \frac{1}{c}\grad{\pdt{f}{t}} \nonumber \\
 &=& -\grad\varphi - \frac{1}{c}\pdt{\vec{A}}{t} \nonumber \\
 &=& \vec{E} \label{c3e33} \\
\vec{H}^\op &=& \curl\vec{A}^\op \nonumber \\
 &=& \curl\vec{A} \nonumber \\
 &=& \vec{H} \label{c3e34}
\end{eqnarray}
Equation \eqref{c3e30} is called the gauge transformation and an invariance of
functions under it is called gauge invariance. All physically meaningful quantities
must be gauge invariant.

\item If $A_\mu$ is independent of time then so are the fields. The electric field
depends only on $\varphi$. Further, since the Lagrangian is independent of time, the
energy of the system is a constant and it coincides with the Hamiltonian. From
\eqref{c3e8}, it is
\begin{equation}\label{c3e35}
\mathcal{E} (= \mathcal{H}) = mc^2\gamma + e\varphi.
\end{equation} 
The effect of the fields is just to add a term $e\varphi$ to the energy. This term
is called the \emph{potential energy}. It is independent of $\vec{A}$ which means
that the magnetic field has no influence on the particle's energy.

If the fields are constant then one can easily verify that
\begin{eqnarray}
\varphi &=& -\vec{E}\cdot\vec{x} \label{c3e36} \\
\vec{A} &=& \frac{1}{2}\vec{H} \times \vec{x} \label{c3e37}
\end{eqnarray}

\item We will now consider the motion of a charge particle in a constant electric
field. From the equation of motion \eqref{c3e29},
\begin{equation}\label{c3e38}
\td{\vec{p}}{t} = e\vec{E}.
\end{equation}
If $\vec{E} = E\uv{x}$ then we have
\begin{eqnarray}
p_x(t) &=& p_x(0) + eEt \label{c3e39} \\
p_y(t) &=& p_y(0) \label{c3e40}.
\end{eqnarray}
Let us choose the zero of time at an instant when $p_x(0) = 0$ and $p_y(0) = p_0$,
the initial momentum of the particle. Since $\vec{p} = m\gamma\vec{v}$ and 
$\mathcal{E} = mc^2$, we have 
\begin{equation}\label{c3e41}
\vec{v} = \frac{\vec{p}c^2}{\mathcal{E}}
\end{equation}
so that
\begin{eqnarray}
\dot{x}(t) &=& \frac{ec^2Et}{\mathcal{E}} \label{c3e42} \\
\dot{y}(t) &=& \frac{c^2p_0}{\mathcal{E}} \label{c3e43}
\end{eqnarray}
A particle in an electric field gains energy. Therefore, $\mathcal{E}$ is not a
constant and hence \eqref{c3e42} cannot be integrated as is. However, we know 
that $\mathcal{E}^2 = m^2c^4 + p^2c^2 = m^2c^4 + p_0^2c^2 + e^2E^2t^2c^2$ so that
equation \eqref{c3e42} becomes
\begin{equation}\label{c3e44}
\dot{x}(t) = \frac{ec^2Et}{\sqrt{m^2c^4 + p_0^2c^2 + c^2e^2E^2t^2}}.
\end{equation}
and hence
\[
x(t) = \int \frac{ec^2Et dt}{\sqrt{m^2c^4 + p_0^2c^2 + c^2e^2E^2t^2}} + x(0).
\]
If $ceEt = u, dt = du/(ceE)$ and
\[
x(t) = \frac{1}{eE}\int\frac{u du}{\sqrt{\mathcal{E}_0^2 + u^2}} + x(0)
\]
If $u = \mathcal{E}_0\sinh v$, $du = \mathcal{E}_0\cosh  vdv$ and
\begin{eqnarray*}
x(t) &=& \frac{1}{eE}\int\frac{\mathcal{E}_0^2\sinh v\cosh v dv}{\mathcal{E}_0\cosh v} + x(0) \\
 &=& \frac{\mathcal{E}_0}{eE}\int\sinh v dv + x(0) \\
 &=& \frac{\mathcal{E}_0}{eE}\cosh v + x(0) \\
 &=& \frac{1}{eE}\sqrt{\mathcal{E}_0^2 + u^2} + x(0)
\end{eqnarray*}
or
\begin{equation}\label{c3e45}
x(t) = \frac{1}{eE}\sqrt{\mathcal{E}_0^2 + e^2c^2E^2t^2} + x(0)
\end{equation}
Analogous to \eqref{c3e44}
\[
\dot{y}(t) = \frac{p_0c^2}{\sqrt{\mathcal{E}_0^2 + e^2E^2c^2t^2}}
\]
so that
\[
y(t) = \int \frac{p_0c^2}{\sqrt{\mathcal{E}_0^2 + e^2E^2c^2t^2}} dt + y(0).
\]
Put $ecEt = \mathcal{E}_0\sinh u$ so that $ecEdt = \mathcal{E}_0\cosh u du$
and
\[
y(t) = \frac{p_0c}{eE}\int \frac{\mathcal{E}_0\cosh u du}{\mathcal{E}_0\cosh u}
 + y(0) = \frac{p_0c}{eE} u + y(0) 
\]
that is
\begin{equation}\label{c3e46}
y(t) = \frac{p_0c}{eE} \sinh^{-1}\left(\frac{ecEt}{\mathcal{E}_0}\right) + y(0).
\end{equation}
We choose the origin such that $x(0) = 0, y(0) = 0$ so that from \eqref{c3e46}
we have
\[
ecEt = \mathcal{E}_0\sinh\left(\frac{eEy}{p_0c}\right)
\]
Substituting this in \eqref{c3e45} we get the equation of the orbit
\begin{equation}\label{c3e47}
x(y) = \frac{\mathcal{E}_0}{eE}\cosh\left(\frac{eEy}{p_0c}\right).
\end{equation}
It describes a catenary symmetric about the $x$-axis. When $eEy \ll p_0c$, we
can approximate \eqref{c3e47} as
\begin{equation}\label{c3e48}
x(y) = \frac{\mathcal{E}_0}{eE}\left(1 + \frac{1}{2}\frac{e^2E^2y^2}{p_0^2c^2}\right),
\end{equation}
which is an equation of a parabola.

\item When a charged particle is subjected to a constant magnetic field, the
equation of motion is
\begin{equation}\label{c3e49}
\td{\vec{p}}{t} = e\frac{\vec{v}}{c} \times \vec{H}.
\end{equation}
Once again, we use the fact that $\mathcal{E} = m\gamma c^2$ and $\vec{p} = 
m\gamma\vec{v}$ to get
\[
\vec{p} = \frac{\mathcal{E}}{c^2}\vec{v}
\]
so that \eqref{c2e48} becomes
\[
\td{\vec{v}}{t} = \frac{ec}{\mathcal{E}}\vec{v} \times \vec{H}.
\]
In a magnetic field, the energy of a particle remains unchanged. Therefore, we 
could pull $\mathcal{E}$ out of the integral. Let us align the $z$ axis along
the magnetic field so that $\vec{H} = H\uv{z}$ and $\vec{v} \times \vec{H} = 
\uv{x}(v_yH - 0) + \uv{y}(-v_xH) + \uv{z}(0)$. Therefore,
\begin{eqnarray*}
\dot{v}_x(t) &=& \frac{ecH}{\mathcal{E}}v_y \\
\dot{v}_y(t) &=& -\frac{ecH}{\mathcal{E}}v_x \\
\dot{v}_z(t) &=& 0
\end{eqnarray*}
From these equations, we get
\begin{eqnarray*}
\ddot{v}_x(t) &=& -\omega^2 v_x \\
\ddot{v}_y(t) &=& -\omega^2 v_y \\
\ddot{v}_z(t) &=& 0
\end{eqnarray*}
where
\begin{equation}\label{c3e50}
\omega = \frac{ecH}{\mathcal{E}},
\end{equation}
is called the cyclotron frequency. At low speeds, $\gamma \approx 1$, $\mathcal{E}
= mc^2$ and the cyclotron frequency is $(eH)/(mc)$. Note that $\omega$ is a 
constant. We can write solutions to the differential equations of velocity
components as
\begin{eqnarray*}
v_x(t) &=& A_1\omega\cos\omega t + A_2\omega\sin\omega t \\
v_y(t) &=& A_3\omega\cos\omega t + A_4\omega\sin\omega t \\
v_z(t) &=& v_z(0)
\end{eqnarray*}
here $A_1, A_2, A_3, A_4$ are the constants of integration. If the initial
conditions are $v_x(0) = v_{\perp}, v_y(0) = 0$ then $A_1 = v_{\perp}$ and 
$A_3 = 0$ so that
\begin{eqnarray*}
v_x(t) &=& v_\perp\cos\omega t + A_2\omega\sin\omega t \\
v_y(t) &=& A_4\omega\sin\omega t \\
v_z(t) &=& v_z(0)
\end{eqnarray*}
from which we get
\begin{eqnarray*}
x(t) &=& \frac{v_\perp}{\omega}\sin\omega t + A_2\cos\omega t \\
y(t) &=& -A_4\cos\omega t \\
z(t) &=& z(0) + v_z(0)t
\end{eqnarray*}
If $x(0) = 0, y(0) = r$ then $A_2 = 0$ and $A_4 = -r$ so that
\begin{eqnarray}
x(t) &=& \frac{v_\perp}{\omega}\sin\omega t \label{c3e51} \\
y(t) &=& r\cos\omega t \label{c3e52} \\
z(t) &=& z(0) + v_z(0)(t) \label{c3e53}
\end{eqnarray}
The constants $v_\perp$ and $r$ are not unrelated to each other. Using them in
the equations of velocity components we get
\begin{eqnarray*}
v_x(t) &=& v_\perp\cos\omega t \\
v_y(t) &=& -r\omega\sin\omega t
\end{eqnarray*}
so that $\dot{v}_x(t) = v_\perp\omega\sin\omega t$. From the equation of motion
the right hand side is $\omega v_y = -r\omega^2\sin\omega t$ so that we get
\begin{equation}\label{c3e54}
v_\perp = -r\omega.
\end{equation}
The equations of the trajectory therefore simplify to
\begin{eqnarray}
x(t) &=& -r\sin\omega t \label{c3e55} \\
y(t) &=& r\cos\omega t \label{c3e56} \\
z(t) &=& z(0) + v_z(0)(t) \label{c3e57}
\end{eqnarray}
This is an equation of an helix.

\item Now consider a charged particle in the presence of a constant magnetic field
$\vec{H} = H\uv{z}$ and a constant electric field $\vec{E} = E_y\uv{y} + E_z\uv{z}$.
We restrict ourselves to non-relativistic regime so that $\vec{p} = m\vec{v}$ and
the equation of motion \eqref{c3e29} becomes
\begin{eqnarray}
m\dot{v}_x &=& \frac{e}{c}v_yH \label{c3e58} \\
m\dot{v}_y &=& eE_y - \frac{e}{c}v_xH \label{c3e59} \\
m\dot{v}_z &=& eE_z \label{c3e60}
\end{eqnarray}
The last equation immediately gives
\[
v_z = \frac{e}{m}E_zt + v_z(0)
\]
and
\begin{equation}\label{c3e61}
z(t) = \frac{e}{2m}E_zt^2 + v_z(0)t + z(0).
\end{equation}
In order to solve the coupled equations \eqref{c3e57} and \eqref{c3e58}, we can 
either differentiate them once more and decouple them or use a different trick.
We demonstrated the former in the previous point. Here, we will use the latter.
Let $\tilde{v} = v_x + iv_y$ so that
\[
m\td{\tilde{v}}{t} = \frac{e}{c}v_yH + ieE_y - \frac{e}{c}{iv_xH}
= ieE_y -i\frac{e}{c}(H(v_x + iv_y)) = ieE_y - i\frac{eH}{c}\tilde{v}
\]
or
\begin{equation}\label{c3e62}
\td{\tilde{v}}{t} = -i\frac{eH}{mc}\tilde{v} + i\frac{e}{m}E_y,
\end{equation}
It can be readily integrated to
\[
\ln\left(\tilde{v} - \frac{eE_y}{m\omega}\right) = -i\omega t + \ln\tilde{v}(0)
\]
or
\begin{equation}\label{c3e63}
\tilde{v} = \tilde{v}(0)e^{-i\omega t} + \frac{eE_y}{m\omega},
\end{equation}
where we approximated \eqref{c3e50} for non-relativistic speeds. Equating the
real and imaginary parts,
\begin{eqnarray}
v_x(t) &=& v_x(0)\cos\omega t  + \frac{e}{m\omega}E_y\label{c3e64} \\
v_y(t) &=& -v_y(0)\sin\omega t \label{c3e65}
\end{eqnarray}
From these equations, we readily get
\begin{eqnarray}
\langle v_x(t) \rangle &=& \frac{e}{m\omega}E_y \label{c3e66} \\
\langle v_y(t) \rangle &=& 0 \label{c3e67}
\end{eqnarray}
Thus, the motion remains bounded on the $y$ axis but not so on the other axes. 
Equations \eqref{c3e65} and \eqref{c3e66} can be readily integrated to
\begin{eqnarray}
x(t) &=& \frac{v_x(0)}{\omega}\sin\omega t + \frac{e}{m\omega}E_yt + x(0) \label{c3e68} \\
y(t) &=& \frac{v_y(0)}{\omega}\cos\omega t + y(0) \label{c3e69}
\end{eqnarray}
Equations \eqref{c3e61}, \eqref{c3e68} and \eqref{c3e69} are the equations of the
trajectory.

\item We derived the equations of motion in point (4) from the Lagrangian which
was obtained after writing the principle of least action as an integral over time.
We can, instead, apply the variational method to equation \eqref{c3e2}. Thus,
\begin{eqnarray*}
\delta S &=& -\delta\int_A^B\left(mcds + \frac{e}{c}A_\mu dx^\mu\right) \\
 &=& -\int_A^B\left(mc\delta ds + \frac{e}{c}\delta(A_\mu dx^\mu)\right) \\
 &=& -\int_A^B\left(mc\delta\sqrt{dx_\mu dx^\mu} + \frac{e}{c}(\delta(A_\mu) dx^\mu + A_\mu d\delta x^\mu)\right) \\
 &=& -\int_A^B\left(mc\frac{1}{2}\frac{dx_\mu d\delta x^\mu + d(\delta x_\mu)dx^\mu}{ds}
     + \frac{e}{c}(\delta(A_\mu) dx^\mu + A_\mu d\delta x^\mu)\right)
\end{eqnarray*}
Now $d(\delta x_\mu)dx^\mu = d(\delta x^\mu)dx_\mu = dx_\mu d(\delta x^\mu)$ so that
\begin{equation}\label{c3e70}
\delta S = -\int_A^B\left(mc\frac{dx_\mu d(\delta x^\mu)}{ds}
     + \frac{e}{c}(\delta(A_\mu) dx^\mu + A_\mu d\delta x^\mu)\right)
\end{equation}    
Using \eqref{c1e78},
\begin{equation}\label{c3e71}
\delta S = -\int_A^B\left((mcu_\mu + \frac{e}{c}A_\mu)d\delta x^\mu + \frac{e}{c}\delta(A_\mu) dx^\mu\right).
\end{equation}
We can integrate first term by parts,
\[
\delta S = -\left(mcu_\mu + \frac{e}{c}A_\mu\right)\delta x^\mu\big|_A^B + 
 \int_A^B \left(\delta x^\mu d\left(mcu_\mu + \frac{e}{c}A_\mu\right) - 
 \frac{e}{c}\delta(A_\mu) dx^\mu\right)
\]
Since the variations $\delta x^\mu$ vanish at the end points,
\[
\delta S = \int_A^B\left(\delta x^\mu\left(mcdu_\mu + \frac{e}{c}dA_\mu\right) - 
 \frac{e}{c}\delta(A_\mu) dx^\mu\right)
\]
We now write
\[
dA_\mu = \pdt{A_\mu}{x^\nu}dx^\nu;\; \delta A_\mu = \pdt{A_\mu}{x^\nu}\delta x^\nu dx^\mu
\]
so that
\begin{eqnarray*}
\delta S &=& \int_A^B\left(mcdu_\mu\delta x^\mu + \frac{e}{c}\pdt{A_\mu}{x^\nu}dx^\nu\delta x^\mu
 - \frac{e}{c}\pdt{A_\mu}{x^\nu}\delta x^\nu dx^\mu\right) \\
 &=& \int_A^B\left(mcdu_\mu\delta x^\mu + \frac{e}{c}\pdt{A_\mu}{x^\nu}dx^\nu\delta x^\mu
 - \frac{e}{c}\pdt{A_\nu}{x^\mu}\delta x^\mu dx^\nu\right) \\
 &=& \int_A^B\left(mcdu_\mu - \frac{e}{c}\left(\pdt{A_\nu}{x^\mu}
 - \pdt{A_\mu}{x^\nu}\right)dx^\nu\right)\delta x^\mu \\
 &=& \int_A^B\left(mc\td{u_\mu}{s}ds - \frac{e}{c}\left(\pdt{A_\nu}{x^\mu}
 - \pdt{A_\mu}{x^\nu}\right)u^\nu ds\right)\delta x^\mu \\
 &=& \int_A^B\left(mc\td{u_\mu}{s} - \frac{e}{c}\left(\pdt{A_\nu}{x^\mu}
 - \pdt{A_\mu}{x^\nu}\right)u^\nu \right)ds\delta x^\mu
\end{eqnarray*}
$\delta S = 0$ implies that
\[
\int_A^B\left(mc\td{u_\mu}{s} - \frac{e}{c}\left(\pdt{A_\nu}{x^\mu}
 - \pdt{A_\mu}{x^\nu}\right)u^\nu \right)ds\delta x^\mu = 0.
\]
Since this is true for all $\delta x^\mu = 0$,
\[
\int_A^B\left(mc\td{u_\mu}{s} - \frac{e}{c}\left(\pdt{A_\nu}{x^\mu}
 - \pdt{A_\mu}{x^\nu}\right)u^\nu \right)ds = 0.
\]
This integral is always zero, we must have
\begin{equation}\label{c3e72}
mc\td{u_\mu}{s} = \frac{e}{c}\left(\pdt{A_\nu}{x^\mu} - \pdt{A_\mu}{x^\nu}\right)u^\nu
\end{equation}
Define the electromagnetic field tensor
\begin{equation}\label{c3e73}
F_{\mu\nu} = \pdt{A_\nu}{x^\mu} - \pdt{A_\mu}{x^\nu}
\end{equation}
so that the equation of motion becomes
\[
mc\td{u_\mu}{s} = \frac{e}{c}F_{\mu\nu}u^\nu.
\]
equivalently,
\begin{equation}\label{c3e74}
mc\td{u^\mu}{s} = \frac{e}{c}F^{\mu\nu}u_\nu.
\end{equation}
Recall that $dx^\mu = (cdt, dx^1, dx^2, dx^3)$ so that $dx_\mu = (cdt, -dx^1, 
-dx^2, -dx^3)$. Further, $A_\mu = (\varphi, -A_1, -A_2, -A_3)$ so that the 
components of $F_{\mu\nu}$ are
\[
\begin{bmatrix}
0 & -c^{-1}\partial_t A_1 - \partial_{x^1}\varphi & -c^{-1}\partial_t A_2 - \partial_{x^2}\varphi & -c^{-1}\partial_t A_3 - \partial_{x^3}\varphi \\
c^{-1}\partial_t A_1 - \partial_{x^1}\varphi & 0 & -\partial_{x^1}A_2+\partial_{x^2}A^1 & -\partial_{x^1}A_3+\partial_{x^3}A^3\\
c^{-1}\partial_t A_2 - \partial_{x^2}\varphi & \partial_{x^1}A_2-\partial_{x^2}A_1 & 0 & -\partial_{x^2}A_3+\partial_{x^3}A_2\\
c^{-1}\partial_t A_3 - \partial_{x^3}\varphi & \partial_{x^1}A_3-\partial_{x^3}A_1 & \partial_{x^3}A_2-\partial_{x^3}A_2 & 0
\end{bmatrix}
\]
From equations \eqref{c3e19} and \eqref{c3e20},
\begin{equation}\label{c3e75}
F_{\mu\nu} = 
\begin{bmatrix}
0 & E_1 & E_2 & E_3 \\
-E_1 & 0 & -H_3 & H_2\\
-E_2 & H_3 & 0 & -H_1\\
-E_3 & -H_2 & H_1 & 0
\end{bmatrix}
\end{equation}
In order to transform this into a contravariant tensor, we use the equation
\begin{equation}\label{c3e76}
F^{\mu\nu} = g^{\mu\sigma}F_{\sigma\rho}g^{\rho\nu}.
\end{equation}
The factors are written in a manner that mimics the matrix multiplication. The matrix
representing the metric is $\text{diag}(1, -1, -1, -1)$ so that the result is
\begin{equation}\label{c3e77}
F^{\mu\nu} = 
\begin{bmatrix}
0 & -E_1 & -E_2 & -E_3 \\
E_1 & 0 & -H_3 & H_2\\
E_2 & H_3 & 0 & -H_1\\
E_3 & -H_2 & H_1 & 0
\end{bmatrix}
\end{equation}
Note that $E_i, H_i$ are not space components of a 4-vector. They should be treated
as scalars. That is $E_i = E^i$ and $H_i = H^i$.


\item Let us write \eqref{c3e74} in components form. Since
\[
u^\mu = \left(\gamma, -\frac{\gamma}{c}v^1, -\frac{\gamma}{c}v^2, -\frac{\gamma}{c}v^3\right),
\]
we have
\[
c\td{u^\mu}{s} = \left(c\td{\gamma}{s}, -\td{(\gamma v^1)}{s}, 
-\td{(\gamma v^2)}{s}, -\td{(\gamma v^3)}{s}\right)
\]
and
\[
u_\mu = \left(\gamma, \frac{\gamma}{c}v_1, \frac{\gamma}{c}v_2, \frac{\gamma}{c}v_3\right),
\]
and hence
\begin{eqnarray*}
mc\td{\gamma}{s} &=& \frac{e}{c}(F^{00}\gamma - F^{01}\gamma \beta_1 - F^{02}\gamma \beta_2 -F^{03}\gamma \beta_3) \\
-m\frac{d}{ds}(\gamma v^1)&=&\frac{e}{c}(F^{10}\gamma + F^{11}\gamma \beta_1 + F^{12}\gamma \beta_2 + F^{13}\gamma \beta_3) \\
-m\frac{d}{ds}(\gamma v^2)&=&\frac{e}{c}(F^{20}\gamma + F^{21}\gamma \beta_1 + F^{22}\gamma \beta_2 + F^{23}\gamma \beta_3) \\
-m\frac{d}{ds}(\gamma v^3)&=&\frac{e}{c}(F^{30}\gamma + F^{31}\gamma \beta_1 + F^{32}\gamma \beta_2 + F^{33}\gamma \beta_3) 
\end{eqnarray*}
Since $v^i = -v_i$ and $ds = cdt/\gamma$,
\begin{eqnarray*}
mc\td{\gamma}{t} &=& e(E_1 \beta_1 + E_2 \beta_2 + E_3 \beta_3) \\
m\frac{d}{dt}(\gamma v_1) &=& e(E_1 + H_3 \beta_2 - H_2 \beta_3) \\
m\frac{d}{dt}(\gamma v_2) &=& e(E_2 - H_3 \beta_1 + H_1 \beta_3) \\
m\frac{d}{dt}(\gamma v_3) &=& e(E_3 + H_2 \beta_1 - H_1 \beta_2).
\end{eqnarray*}
The last three equations can be written in vector form as
\[
\frac{d}{dt}(m\gamma\vec{v}) = \td{\vec{p}}{t} = e(\vec{E} + \vec{\beta} \times \vec{H})
\]
which is indeed the same as \eqref{c3e29}. The first equation can be written as
\[
mc^2\td{\gamma}{t} = \frac{d}{dt}(mc^2\gamma) = \td{\mathcal{E}}{t} = e\vec{E}\cdot\vec{v}
\]
This is the ``relativistic'' form of the work-energy theorem of \eqref{c3e27}.

\item From the equation of motion \eqref{c3e74} we also have
\[
mcu_\mu\td{u^\mu}{s} = \frac{e}{c}u_\mu F^{\mu\nu}u_\nu.
\]
The left hand side is zero because of \eqref{c1e86} while the right hand side 
vanishes because of anti-symmetry of the electromagnetic field tensor. This means
that the four equations of motion in the previous point are not independent.

\item We go back to equation \eqref{c3e71}
\[
\delta S = -\int_A^B\left((mcu_\mu + \frac{e}{c}A_\mu)d\delta x^\mu + \frac{e}{c}\delta(A_\mu) dx^\mu\right).
\]
and its immediate consequence,
\begin{eqnarray*}
\delta S &=& -\left(mcu_\mu + \frac{e}{c}A_\mu\right)\delta x^\mu\big|_A^B + \\
  & & \int_A^B \left(\delta x^\mu d\left(mcu_\mu + \frac{e}{c}A_\mu\right) - 
 \frac{e}{c}\delta(A_\mu) dx^\mu\right)
\end{eqnarray*}
The first term is a difference of a quantity at two fixed points while the second
one is a variation over all paths between those points. Since the variations 
vanish at the fixed points the first term is zero while the second term, after some
manipulations, gives the equations of motion.

If we consider only the actual paths starting from the world-point $A$ and let $H$
be any other point on it, then the second term will necessarily be zero while the 
first one will not be zero. The equation then gives,
\[
\frac{\delta S}{\delta x^\mu} = -\left(mcu_\mu + \frac{e}{c}A_\mu\right).
\]
In the limit $\delta x^\mu \rightarrow 0$ this equation becomes
\begin{equation}\label{c3e78}
-\pdt{S}{x^\mu} = mcu_\mu + \frac{e}{c}A_\mu.
\end{equation}
From equation \eqref{c2e26}, the left hand side is the generalised momentum
\begin{equation}\label{c3e79}
p_\mu = mcu_\mu + \frac{e}{c}A_\mu.
\end{equation}
This agrees with \eqref{c3e7}.

\item From equation \eqref{c1e46}, the Lorentz transformation of the 4-potential
gives
\begin{equation}\label{c3e80}
\varphi = \gamma(\bar{\varphi} + \beta\bar{A}_1);
\; A_1 = \gamma(\bar{A}_1 + \beta\bar{\varphi}); A_2 = \bar{A}_2;\; A_3 = \bar{A}_3.
\end{equation}
Using the solution to problem 2 in chapter \ref{c1}, we have $F^{01} = 
\bar{F}^{01}$ (from \eqref{c1e98}) so that 
\begin{equation*}
E_1 = \bar{E}_1.
\end{equation*}
From \eqref{c1e90}
\[
F^{02} = \frac{\bar{F}^{02} + \beta\bar{F}^{12}}{1 - \beta^2}
\]
so that
\begin{equation*}
E_2 = \frac{\bar{E}_2 + \beta\bar{H}_3}{1 - \beta^2}.
\end{equation*}
From \eqref{c1e91}
\[
F^{03} = \frac{\bar{F}^{03} + \beta\bar{F}^{13}}{1 - \beta^2}
\]
so that
\begin{equation*}
E_3 = \frac{\bar{E}_3 - \beta\bar{H}_2}{1 - \beta^2}.
\end{equation*}
From \eqref{c1e93}
\[
F^{12} = \frac{\bar{F}^{12} + \beta\bar{F}^{02}}{1 - \beta^2}
\]
so that
\begin{equation*}
-H_3 = \frac{-\bar{H}_3 - \beta\bar{E}_2}{1 - \beta^2} \Rightarrow 
H_3 = \frac{\bar{H}_3 + \beta\bar{E}_2}{1 - \beta^2}
\end{equation*}
From \eqref{c1e94}
\[
F^{13} = \frac{\bar{F}^{13} + \beta\bar{F}^{03}}{1 - \beta^2}
\]
so that
\begin{equation*}
H_2 = \frac{\bar{H}_2 - \beta\bar{E}_3}{1 - \beta^2}.
\end{equation*}
Finally, $F^{23} = \bar{F}^{23}$ implies
\begin{equation*}
H_1 = \bar{H}_1.
\end{equation*}
We collect these formulae together as
\begin{eqnarray}
E_1 &=& \bar{E}_1 \\ \label{c3e81}
E_2 &=& \frac{\bar{E}_2 + \beta\bar{H}_3}{1 - \beta^2} \\ \label{c3e82}
E_3 &=& \frac{\bar{E}_3 - \beta\bar{H}_2}{1 - \beta^2} \\ \label{c3e83}
H_1 &=& \bar{H}_1 \\ \label{c3e84}
H_2 &=& \frac{\bar{H}_2 - \beta\bar{E}_3}{1 - \beta^2} \\ \label{c3e85}
H_3 &=& \frac{\bar{H}_3 + \beta\bar{E}_2}{1 - \beta^2} \label{c3e86}
\end{eqnarray}

\item At low speeds, $1 - \beta^2 \approx 1$ and the transformation equations
become
\begin{eqnarray}
E_1 &=& \bar{E}_1 \\ \label{c3e87}
E_2 &=& \bar{E}_2 + \beta\bar{H}_3 \\ \label{c3e88}
E_3 &=& \bar{E}_3 - \beta\bar{H}_2 \\ \label{c3e89}
H_1 &=& \bar{H}_1 \\ \label{c3e90}
H_2 &=& \bar{H}_2 - \beta\bar{E}_3 \\ \label{c3e92}
H_3 &=& \bar{H}_3 + \beta\bar{E}_2 \label{c3e93}
\end{eqnarray}
Since $\beta = v/c$ or really $v_1/c$, we can write these equations in vector form as
\begin{eqnarray}
\vec{E} = \vec{E}^\op + \frac{1}{c}\vec{H}^\op \times \vec{v} \label{c3e94} \\
\vec{H} = \vec{H}^\op - \frac{1}{c}\vec{E}^\op \times \vec{v}, \label{c3e95}
\end{eqnarray}
where $\vec{E}^\op = (\bar{E}_1, \bar{E}_2, \bar{E}_3)$ and $\vec{H}^\op = 
(\bar{H}_1, \bar{H}_2, \bar{H}_3)$, These equations tell that even if $\vec{H}^\op
= 0$ ($\vec{E}^\op = 0$), $\vec{H}$ ($\vec{E}$) is not. Further, it is clear that
$\vec{E}$ is perpendicular to $\vec{H}$ in both cases.

\item Suppose $\vec{E}^\op$ and $\vec{H}^\op$ are perpendicular to each other.
If $\vec{v} = c\vec{H}^\op/E^\op$ then, from \eqref{c3e93} and \eqref{c3e94},
\begin{eqnarray}
\vec{E} &=& \vec{E}^\op \label{c3e96} \\
\vec{H} &=& 0 \label{c3e97}
\end{eqnarray}
Similarly, if we choose $\vec{v} = c\vec{E}^\op/H^\op$, we will get $\vec{E} = 0$
and $\vec{H} = \vec{H}^\op$. Thus, if $\vec{E}^\op$ and $\vec{H}^\op$ are 
perpendicular to each other then one can find a frame of reference in which one
of the fields is zero.

\item For any tensor, $A^{\mu\nu}$, $A_{\mu\nu}A^{\mu\nu}$ and 
$\epsilon^{\mu\rho\nu\sigma}A^{\mu\rho}A^{\nu\sigma}$ are scalars, and therefore
invariants under Lorentz transformation. In the case of electromagnetic field
tensor, the first of these is
\[
F_{\mu\nu}F^{\mu\nu} = -E_1^2 - E_2^2 - E_3^2 + H_3^2 + H_2^2 + H_1^2.
\]
Thus, the quantity 
\begin{equation}\label{c3e98}
H^2 - E^2
\end{equation}
is an invariant. To find the next invariant, we
need the product
\begin{eqnarray*}
\epsilon^{\mu\rho\nu\sigma}F^{\mu\rho}F^{\nu\sigma} &=& F^{01}F^{23} - F^{01}F^{32} - F^{02}F^{13} + F^{02}F^{31} \\
 & & F^{03}F^{12} - F^{03}F^{21} - F^{10}F^{23} + F^{10}F^{32} \\
 & & F^{12}F^{03} - F^{12}F^{30} - F^{13}F^{02} + F^{13}F^{20} \\
 & & F^{20}F^{13} - F^{20}F^{31} - F^{21}F^{03} + F^{21}F^{30} \\
 & & F^{23}F^{01} - F^{23}F^{10} - F^{30}F^{12} + F^{30}F^{21} \\
 & & F^{31}F^{02} - F^{31}F^{20} - F^{32}F^{01} + F^{32}F^{10} \\
 &=& E_1H_1 + E_1H_1 + E_2H_2 + E_2H_2 \\
 & & E_3H_3 + E_3H_3 + E_1H_1 + E_1H_1 \\
 & & E_3H_3 + E_3H_3 + E_2H_2 + E_2H_2 \\
 & & E_2H_2 + E_2H_2 + E_3H_3 + E_3H_3 \\
 & & E_1H_1 + E_1H_1 + E_3H_3 + E_3H_3 \\
 & & E_2H_2 + E_2H_2 + E_1H_1 + E_1H_1 \\
 &=& 8\vec{E}\cdot\vec{H}
\end{eqnarray*}
Thus, \begin{equation}\label{c3e99}
\vec{E}\cdot\vec{H}
\end{equation}
is another invariant.

From the invariants of \eqref{c3e97} and \eqref{c3e98} we have
\begin{enumerate}
\item If the electric and magnetic fields are
perpendicular to each other in one frame of reference then they are so in all 
other inertial frames of reference.

\item If $E > H$ ($E < H$) in one frame of reference then it is so in all other
frames.

\item If the angle between $\vec{E}$ and $\vec{H}$ is acute (obtuse) in one frame
of reference then it is so in all other frames.

\item If $\vec{E}\cdot\vec{H} \ne 0$ then we can find a frame of reference
in which the electric and magnetic fields are parallel to each other. In that
frame ${H^\op}^2 - {E^\op}^2 = H^2 - E^2$ and $\vec{E}\cdot\vec{H} = E^\op H^\op$.

\item if $\vec{E}\cdot\vec{H} = 0$ and if $H^2 - E^2 > 0$ then we can find
a frame in which $E^\op = 0$ and ${H^\op}^2 = H^2 - E^2 > 0$. Likewise, if $H^2
- E^2 < 0$ then we can find a frame in which $H^\op = 0$ and $-{E^\op}^2 = H^2 - 
E^2 < 0$. 
\end{enumerate}
\end{enumerate}

\section{Problems}
\begin{enumerate}
\item Find the 3-acceleration of a particle in electric and magnetic fields.
\item[Solution:] The equation of motion follows from \eqref{c3e18}, \eqref{c3e19}
and \eqref{c3e20}. It is
\begin{equation}\label{c3e100}
\td{\vec{p}}{t} = e\vec{E} + e\frac{\vec{v}}{c}\times\vec{H}.
\end{equation}
Now, $p^i = m\gamma v^i$ and $\mathcal{E} = m\gamma c^2$ so that 
\[
p^i = \frac{\mathcal{E} v^i}{c^2}.
\]
Therefore,
\[
c^2\td{\vec{p}}{t} = \td{\mathcal{E}}{t}\vec{v} + \mathcal{E}\td{\vec{v}}{t}.
\]
From equations \eqref{c3e27} and \eqref{c3e99} we have
\[
ec^2\vec{E} + ec\vec{v}\times\vec{H} = e\vec{v}\cdot\vec{E}\vec{v} + \mathcal{E}\td{\vec{v}}{t}.
\]
or
\[
\td{\vec{v}}{t} = \frac{ec^2}{\mathcal{E}}
\left(\vec{E} + \frac{\vec{v}}{c}\times\vec{H} - \frac{\vec{v}\cdot\vec{E}}{c^2}\vec{v}\right)
= \frac{e}{m\gamma}
\left(\vec{E} + \frac{\vec{v}}{c}\times\vec{H} - \frac{\vec{v}\cdot\vec{E}}{c^2}\vec{v}\right)
\]
or
\begin{equation}\label{c3e101}
\td{\vec{v}}{t} = \frac{e}{m}\sqrt{1 - \frac{v^2}{c^2}}
\left(\vec{E} + \frac{\vec{v}}{c}\times\vec{H} - \frac{\vec{v}\cdot\vec{E}}{c^2}\vec{v}\right).
\end{equation}
In terms of $\vec{\beta}$, this equation is
\begin{equation}\label{c3e102}
\td{\vec{v}}{t} = \frac{e}{m}\sqrt{1 - \beta^2}
\left(\vec{E} + \vec{\beta}\times\vec{H} - (\vec{\beta}\cdot\vec{E})\vec{\beta}\right).
\end{equation}
Unlike the non-relativistic case, we cannot put acceleration as the ratio of 
force to mass. In the expression for momentum, $\gamma$ is not a constant. Instead of
finding an expression for $d\gamma/dt$, we replace $m\gamma$ by $\mathcal{E}/c^2$
and use the expression for $d\mathcal{E}/dt$.

\item Maupertuis' principle for a particle in an electromagnetic field.
\item[Solution:] Maupertuis' principle states that if the energy of the particle is
conserved then 
\[
\delta\int\vec{p}\cdot d\vec{r} = 0,
\]
where $\vec{p}$ is the generalised momentum. In the case of a charged particle in
an electromagnetic field,
\[
\vec{p} = m\vec{v} + \frac{e}{c}\vec{A}
\]
so that the variational equation becomes
\[
\delta\int\left(m\vec{v} + \frac{e}{c}\vec{A}\right)\cdot d\vec{r} = 0 \Rightarrow
\delta\int m\vec{v}\cdot d\vec{r} + \frac{e}{c}\delta\int\vec{A}\cdot d\vec{r} = 0.
\]
The energy momentum relation for a free particle is $\mathcal{E}^2 = p^2c^2 + 
m^2c^4$. For a free particle, $p = |m\vec{v}|$ and $\mathcal{E}$ is the kinetic
energy so that
\[
|m\vec{v}|^2 = \frac{\mathcal{E}^2}{c^2} - m^2c^2.
\]
For a charged particle in an electromagnetic field, the kinetic energy is, $mc^2
\gamma = \mathcal{E} - e\varphi$, according to \eqref{c3e35}. Therefore, the energy
momentum relation is
\[
|m\vec{v}|^2 = \frac{(\mathcal{E} - e\varphi)^2}{c^2} - m^2c^2.
\]
Finally, note that $\vec{p}\cdot d\vec{r} = pdl$, where $dl$ is the length of the
infinitesimal portion of the trajectory. This is because $d\vec{r}$ is parallel to
the momemtum. The variational principle, thus, becomes,
\[
\delta\int \left(\frac{(\mathcal{E} - e\varphi)^2}{c^2} - m^2c^2\right)^{1/2}dl + 
\frac{e}{c}\delta\int\vec{A}\cdot d\vec{r} = 0.
\]

\item A charged, isotropic oscillator forced by the magnetic field.
\item[Solution:] Consider an oscillator with natural frequency $\omega_0$ and 
suppose that it obeys the equations of motion
\begin{eqnarray*}
\ddot{x} + \omega_0^2 x &=& 0 \\
\ddot{y} + \omega_0^2 y &=& 0 \\
\ddot{z} + \omega_0^2 z &=& 0
\end{eqnarray*}
Assume that the oscillator carries a charge $e$. An oscillating charge will emit
radiation and I am not sure if the these equations remain valid. Hut let's assume
that the frequency of oscillation is low enough to make radiation loss ignorable.
(I think the radiation loss can be accounted for a friction like term.)

Now suppose that there is a magnetic field in the $z$ axis. It exerts a force
\[
\vec{F} = \frac{e}{c}(\dot{y} H\uv{x} - \dot{x}H\uv{y})
\]
so that the equations of motion are modified to
\begin{eqnarray*}
\ddot{x} + \omega_0^2 x &=& \frac{eH}{mc}\dot{y} \\
\ddot{y} + \omega_0^2 y &=& -\frac{eH}{mc}\dot{x}\\
\ddot{z} + \omega_0^2 z &=& 0
\end{eqnarray*}
Let $\zeta = x + iy$ and write the above equations as
\begin{eqnarray*}
\ddot{x} + \omega_0^2 x &=& -\frac{ieH}{mc}\dot{(iy)} \\
\ddot{y} + \omega_0^2 y &=& -i\frac{ieH}{mc}\dot{x} \\
\ddot{(iy)} + \omega_0^2 (iy) &=& \frac{ieH}{mc}\dot{x}
\end{eqnarray*}
Adding the first and the third equations, we get
\[
\ddot{\zeta} + \frac{ieH}{mc}\zeta + \omega_0^2\zeta = 0.
\]
Let the trial solution be $\zeta = e^{\lambda t}$ so that 
\[
\lambda^2 + \frac{ieH}{mc}\lambda + \omega_0^2 = 0,
\]
the solution of which is
\[
\lambda = \left(-\frac{ieH}{mc} \pm \sqrt{-\frac{e^2H^2}{m^2c^2} - 4\omega_0^2}\right)
\frac{1}{2} = -i\left(\frac{eH}{2mc} \pm \sqrt{\omega_0^2 + \frac{e^2H^2}{4m^2c^2}}\right).
\]
Therefore, the frequency of the oscillator is
\[
-i\lambda = \omega = \sqrt{\omega_0^2 + \frac{e^2H^2}{4m^2c^2}} \pm \frac{eH}{2mc}.
\]
If the field is weak, that is if
\[
H \ll \frac{2mc}{e},
\]
we can ignore the second term under the square root and
\[
\omega \approx \omega_0 \pm \frac{eH}{2mc}.
\]

\item Relativistic motion of a charged particle in parallel, uniform electric
and magnetic fields.
\item[Solution:] Let $\vec{E} = E\uv{z}, \vec{H} = H\uv{z}$ so that the equations
of motion are
\[
\td{\vec{p}}{t} = eE\uv{z} + \frac{e}{c}\vec{v}\times\vec{H}
\]
that is,
\begin{eqnarray}
\td{p_x}{t} &=& \frac{e}{c}v_yH \label{c3e103} \\
\td{p_y}{t} &=& -\frac{e}{x}v_xH \label{c3e104} \\
\td{p_z}{t} &=& eE \label{c3e105} 
\end{eqnarray}
Equation \eqref{c3e104} can be readily integrated to
\[
p_z(t) = eEt + p_z(0).
\]
Since $\vec{p} = m\gamma\vec{v}$, we have
\begin{equation}\label{c3e106} 
v_z = \frac{eE}{m\gamma}t + \frac{p_z(0)}{m\gamma}.
\end{equation}
We choose $t = 0$ such that $p_z(0) = 0$ and use the relation, 
$\mathcal{E} = m\gamma c^2$ so that
\begin{equation}\label{c3e107}
v_z = \frac{eEc^2}{\mathcal{E}}t.
\end{equation} 
From equation \eqref{c3e27},
\[
\td{\mathcal{E}}{t} = ev_zE
\]
so that, using \eqref{c3e106}, we have 
\[
\td{\mathcal{E}}{t} = (eEc)^2\frac{t}{\mathcal{E}} \Rightarrow
\frac{\mathcal{E}^2}{2} = \frac{(eEct)^2}{2} + \frac{\mathcal{E}_0^2}{2},
\]
so that, we have
\begin{equation}\label{c3e108}
\mathcal{E} = \sqrt{\mathcal{E}_0^2 + (eEct)^2}.
\end{equation}
Substituting this in \eqref{c3e106},
\[
\dot{z} = c\frac{eEct}{\sqrt{\mathcal{E}_0^2 + (eEct)^2}}
\]
To solve this equation, put $eEct = \mathcal{E}_0\sinh u$ so that $eEcdt = 
\mathcal{E}_0\cosh udu$ and
\[
dz = \frac{\mathcal{E}_0}{eE}\sinh udu \Rightarrow
z(t) = z(0) + \frac{\mathcal{E}_0}{eE}\sinh u = z(0) + \frac{\mathcal{E}_0}{eE}
\sqrt{1 + \frac{(eEc)^2}{\mathcal{E}_0^2}t^2}
\]
If $z(0) = 0$ then 
\begin{equation}\label{c3e109}
z(t) = \frac{\mathcal{E}_0}{eE}\sqrt{1 + \frac{(eEc)^2}{\mathcal{E}_0^2}t^2}.
\end{equation}
We now turn our attention to the $xy$ plane. From equations \eqref{c3e102} and
\eqref{c3e103},
\[
\frac{d}{dt}(p_x + ip_y) = \frac{eH}{c}(v_y - iv_x) = -\frac{ieH}{c}(v_x + iv_y).
\]
Since $\vec{p} = (\mathcal{E}/c^2)\vec{v}$, the above equation becomes
\begin{equation}\label{c3e110}
\frac{d}{dt}(p_x + ip_y) = -\frac{ieHc}{\mathcal{E}}(p_x + ip_y).
\end{equation}
Now introduce a variable $\phi$ such that
\begin{equation}\label{c3e111}
d\phi = \frac{eHc}{\mathcal{E}}dt,
\end{equation}
so that the solution of \eqref{c3e109} can be written as
\begin{equation}\label{c3e112}
p_x + ip_y = p_t e^{-i\phi},
\end{equation}
where $p_t$ is a constant of integration. Putting \eqref{c3e107} in \eqref{c3e110},
we get
\begin{equation}\label{c3e113}
\phi = eHc\frac{dt}{\sqrt{\mathcal{E}_0^2 + (eEct)^2}} = \frac{H}{E}\sinh^{-1}
\left(\frac{eEct}{\mathcal{E}_0}\right),
\end{equation}
which can be written as
\begin{equation}\label{c3e114}
ct = \frac{\mathcal{E}_0}{eE}\sinh\left(\frac{E}{H}\phi\right).
\end{equation}
Since $\vec{p} = (\mathcal{E}/c^2)\vec{v}$, we can write \eqref{c3e111} as
\[
(\mathcal{E}/c^2)(v_x + iv_y) = p_t e^{-i\phi} \Rightarrow 
(\mathcal{E}/c^2)\frac{d}{dt}(x + iy) = p_te^{-i\phi}
\]
From \eqref{c3e110}, we get
\[
d(x + iy) = \frac{cp_t}{eH}e^{-i\phi}d\phi \Rightarrow x + iy = \frac{icp_t}{eH}e^{-i\phi}
\]
or
\[
x + iy = \frac{cp_t}{eH}\sin\phi + i\frac{cp_t}{eH}\cos\phi
\]
from which we get
\begin{eqnarray}
x(\phi) &=& \frac{cp_t}{eH}\sin\phi \label{c3e115} \\
y(\phi) &=& \frac{cp_t}{eH}\cos\phi \label{c3e116}
\end{eqnarray}
The equation of the trajectory is
\begin{eqnarray}
x(t) &=& \frac{cp_t}{eH}\sin\left(\frac{H}{E}\sinh^{-1}\left(\frac{eEct}{\mathcal{E}_0}\right)\right) \label{c3e117} \\
y(t) &=& \frac{cp_t}{eH}\cos\left(\frac{H}{E}\sinh^{-1}\left(\frac{eEct}{\mathcal{E}_0}\right)\right) \label{c3e118} \\
z(t) &=& \frac{\mathcal{E}_0}{eE}\sqrt{1 + \frac{(eEc)^2}{\mathcal{E}_0^2}t^2}. \label{c3e119}
\end{eqnarray}
\end{enumerate}


\chapter{The Electromagnetic Field Equations}\label{c4}
\begin{enumerate}
\item From equations \eqref{c3e19} and \eqref{c3e20} it follows that
\begin{eqnarray}
\curl\vec{E} &=& -\frac{1}{c}\pdt{\vec{H}}{t} \label{c4e1} \\
\dive\vec{H} &=& 0 \label{c4e2}
\end{eqnarray}
These equations can be easily cast in integral form as 
\begin{eqnarray}
\oint\vec{E}\cdot d\vec{l} &=& -\frac{1}{c}\frac{\partial}{\partial t}\int\vec{H}\cdot d\vec{f} \label{c4e3} \\
\oint\vec{H}\cdot d\vec{f} &=& 0 \label{c4e4}
\end{eqnarray}
Note that, in equation \eqref{c4e3}, the surface integral on the rhs is over the
surface bounded by the countour along which the line integral of lhs is calculated.
The line integral is over a closed curve but the surface integral is not over a 
closed surface. The equation states that the circulation of the electric field
is equal to $-1/c$ times the flux of $\vec{H}$ through the surface bound by the
countour. On the other hand, \eqref{c4e4} tells that flux of $\vec{H}$ through
any closed surface is zero.

\item Equations \eqref{c4e1} and \eqref{c4e2} form the first pair of Maxwell
equations. They do not describe the electromagnetic field completely because
they involve the time derivative of $\vec{H}$ alone and not $\vec{E}$.

\item It is possible to write equations \eqref{c4e1} and \eqref{c4e2} in terms
of the electromagnetic field tensor $F_{\mu\nu}$. Since
\[
F_{\mu\nu} = \pdt{A_\nu}{x^\mu} - \pdt{A_\mu}{x^\nu},
\]
we have
\begin{eqnarray*}
\pdt{F_{\mu\nu}}{x^\rho} &=& \pdt{A^2_\nu}{x^\mu x^\rho} - \pdt{A^2_\mu}{x^\nu x^\rho} \\
\pdt{F_{\nu\rho}}{x^\mu} &=& \pdt{A^2_\rho}{x^\nu x^\mu} - \pdt{A^2_\nu}{x^\rho x^\mu} \\
\pdt{F_{\rho\mu}}{x^\nu} &=& \pdt{A^2_\mu}{x^\rho x^\nu} - \pdt{A^2_\rho}{x^\mu x^\nu}
\end{eqnarray*}
from which we get
\begin{equation}\label{c4e5}
\pdt{F_{\mu\nu}}{x^\rho} + \pdt{F_{\nu\rho}}{x^\mu} + \pdt{F_{\rho\mu}}{x^\nu} = 0.
\end{equation}

\item We will examine \eqref{c4e5} in greater details.
\begin{enumerate}
\item $\mu = 0, \nu = 1, \rho = 2$:
\[
\pdt{F_{01}}{x^2} + \pdt{F_{12}}{x^0} + \pdt{F_{20}}{x^1} = 0 \Rightarrow
\pdt{E_y}{x} - \pdt{E_x}{y} = - \frac{1}{c}\pdt{H_z}{t}.
\]

\item $\mu = 0, \nu = 1, \rho = 3$:
\[
\pdt{F_{01}}{x^3} + \pdt{F_{13}}{x^0} + \pdt{F_{30}}{x^1} = 0 \Rightarrow
\pdt{E_x}{z} - \pdt{E_z}{x} = - \frac{1}{c}\pdt{H_y}{t}.
\]

\item $\mu = 0, \nu = 2, \rho = 3$:
\[
\pdt{F_{02}}{x^3} + \pdt{F_{23}}{x^0} + \pdt{F_{30}}{x^2} = 0 \Rightarrow
\pdt{E_z}{y} - \pdt{E_y}{z} = - \frac{1}{c}\pdt{H_x}{t}.
\]

\item $\mu = 1, \nu = 2, \rho = 3$:
\[
\pdt{F_{12}}{x^3} + \pdt{F_{23}}{x^1} + \pdt{F_{31}}{x^2} = 0 \Rightarrow
-\pdt{H_z}{z} - \pdt{H_x}{x} - \pdt{H_y}{y} = 0.
\]
\end{enumerate}
Thus, \eqref{c4e5} encodes the two Maxwell equations \eqref{c4e1} and \eqref{c4e2}.
We also note that
\begin{enumerate}
\item If any two indices in \eqref{c4e5} are equal then we get the identity $0=0$.
\item If all three indices are equal then each term on lhs of \eqref{c4e5} is
zero.
\item Let us examine what happens when we consider other combinations of indices.
If we swap the values of $\mu$ and $\nu$ in case (a), that is, if $\mu = 1, \nu = 0,
\rho = 2$ then we have
\[
\pdt{F_{10}}{x^2} + \pdt{F_{02}}{x^1} + \pdt{F_{21}}{x^0} = 0 \Rightarrow
-\pdt{E_x}{y} + \pdt{E_y}{x} - \frac{1}{c}\pdt{H_z}{t} = 0,
\]
which is same as the conclusion of case (a). We can similarly show that all swaps
give 
\[
\curl\vec{E} = -\frac{1}{c}\pdt{\vec{H}}{t}.
\]
\end{enumerate}

\item Now consider the expression,
\begin{equation}\label{c4e6}
C^\mu = \epsilon^{\mu\nu\rho\sigma}\pdt{F_{\rho\sigma}}{x^\nu}.
\end{equation}
Then
\[
C^0 = \pdt{F_{23}}{x^1} - \pdt{F_{32}}{x^1} - \pdt{F_{13}}{x^2} + \pdt{F_{31}}{x^2}
 + \pdt{F_{12}}{x^3} - \pdt{F_{21}}{x^3}
\]
which, owing to anti-symmetric nature of $F_{\mu\nu}$ is
\[
C^0 = 2\left(\pdt{F_{23}}{x^1} + \pdt{F^{31}}{x^2} + \pdt{F_{12}}{x^3}\right)
\]
From equation \eqref{c4e5}, it is $C^0 = 0$. Similarly we can show that the other
components too vanish. Therefore, equation \eqref{c4e5} is equivalent to
\begin{equation}\label{c4e7}
\epsilon^{\mu\nu\rho\sigma}\pdt{F_{\rho\sigma}}{x^\nu} = 0.
\end{equation}

\item We next develop the action function for a system consisting of particles
in an electromagnetic field. From equations \eqref{c2e3} and \eqref{c2e6}, the
action for a free particle is
\begin{equation}\label{c4e8}
S_m = -mc\int ds.
\end{equation}
If there are many particles, we can write it as
\begin{equation}\label{c4e9}
S_m = -\sum_i m_ic\int ds.
\end{equation}
The subscript `m' indicates that the action depends only of the mass of the 
particles and therefore pertains to free particles alone. When the particles
interact with the field, we need an additional term,
\begin{equation}\label{c4e10}
S_{mf} = -\sum_i \frac{e_i}{c}\int A_\mu dx^\mu.
\end{equation}
It is a mild generalisation of the second term on the rhs of \eqref{c3e2}.
We need a term $S_f$ that determines the behaviour of the fields in absence of
matter so that the action of the system is
\begin{equation}\label{c4e11}
S = S_m + S_{mf} + S_f.
\end{equation}

It is an experimental fact that the electromagnetic fields obey the principle of
superposition. Therefore, the differential equations describing them must be linear.
Since the equations are derived from the variational principle, we require that the
integrand of the action integral must be quadratic in the field tensor. Furthermore,
we require the action integral to be a scalar. Therefore, we also require the 
integrand to be a scalar. The only scalar, quadratic quantity that depends on the
field tensor alone is $F_{\mu\nu}F^{\mu\nu}$. We propose that
\begin{equation}\label{c4e12}
S_f = a\int F_{\mu\nu}F^{\mu\nu} d\Omega.
\end{equation}

The quantity $\epsilon^{\mu\rho\nu\sigma}F^{\mu\rho}F^{\nu\sigma}$ is not 
considered to be part of $S_f$ because it is a pseudo-scalar. We showed in 
\eqref{c3e98} that it is $\vec{E}\cdot\vec{H}$. Since $\vec{H}$ is an axial vector,
the dot product is a pseudo-scalar.

Equation \eqref{c3e97} gave us
\begin{equation}\label{c4e13}
F_{\mu\nu}F^{\mu\nu} = 2(H^2 - E^2).
\end{equation}
In terms of the potentials,
\begin{equation}\label{c4e14}
F_{\mu\nu}F^{\mu\nu} = |\curl\vec{A}|^2 - |\grad\varphi|^2 + 
2\grad\varphi\pdt{\vec{A}}{t} - \left|\pdt{\vec{A}}{t}\right|^2.
\end{equation}
This shows that $F_{\mu\nu}F^{\mu\nu}$ can be made arbitrarily negative if $\vec{A}$
can be made to vary arbitrarily quickly with $t$. In that case, we cannot minimise
the action integral of \eqref{c4e12}. To prevent that from happening, we require
the constant $a$ to be negative. Its value depends on our choice of units. In
gaussian units, it is 
\[
a = -\frac{1}{16\pi c}
\]
so that the action for the field is
\begin{equation}\label{c4e15}
S_f = -\frac{1}{16\pi c}\int F_{\mu\nu}F^{\mu\nu} d\Omega.
\end{equation}
Using \eqref{c4e13},
\begin{equation}\label{c4e16}
S_f = \frac{1}{8\pi}\int (E^2 - H^2) dt dV,
\end{equation}
where we used the fact that $d\Omega = cdtdV$. We can also write it as
\begin{equation}\label{c4e17}
S_f = \int dt \left(\frac{1}{8\pi}\int dV (E^2 - H^2)\right) = \int dt L,
\end{equation}
where
\begin{equation}\label{c4e18}
L_f = \frac{1}{8\pi}\int dV (E^2 - H^2) = \int dV \mathcal{L}_f.
\end{equation}
The quantity,
\begin{equation}\label{c4e19}
\mathcal{L}_f = \frac{1}{8\pi}(E^2 - H^2)
\end{equation}
is called the Lagrangian density of the field and $L_f$ the Lagrangian of the 
field.

\item It is convenient to introduce charge density and treat discrete charges in
terms of it. For example,
\begin{equation}\label{c4e20}
\rho = \sum_a e_a\delta(\vec{r} - \vec{r}_a)
\end{equation}
and
\begin{equation}\label{c4e21}
de = \rho dV.
\end{equation}
From equation \eqref{c4e21} we have
\begin{equation}\label{c4e22}
de dx^\mu = \rho dV dx^\mu = \rho dV dt \td{x^\mu}{t}.
\end{equation}
We call the quantity,
\begin{equation}\label{c4e23}
j^\mu = \rho\td{x^\mu}{t}
\end{equation}
as the \emph{current density}. Clearly, as 
\[
\td{x^\mu}{t} = (c, \vec{v})
\]
we have
\begin{equation}\label{c4e24}
j^\mu = (\rho c, \rho\vec{v}) = (\rho c, \vec{j})
\end{equation}
where
\begin{equation}\label{c4e25}
\vec{j} = \rho\vec{v}
\end{equation}
is the usual current density 3-vector.

\item This is a bit tricky. Consider the surface integral,
\[
\int j^\mu dS_\mu.
\]
If the surface its normal along the $x^0$ axis then $j^\mu dS_\mu = j^0 dV$ and
\[
\int j^\mu dS_\mu = \int j^0 dV = c\int \rho dV = ce,
\]
where $e$ is the total charge enclosed by the hypersurface. From this equation
we can write
\begin{equation}\label{c4e26}
\int\rho dV = \frac{1}{c}\int j^\mu dS_\mu,
\end{equation}
where the surface integral is over a hypersurface whose normal is along the $x^0$
axis. For an arbitrary hypersurface,
\[
\frac{1}{c}\int j^\mu dS_\mu
\]
is just the sum of charges whose world lines pass through it.

\item From equations \eqref{c4e9}, \eqref{c4e10}, \eqref{c4e11} and \eqref{c4e15},
the action of the system of charged particles in electromagnetic fields is
\[
S = -\sum_i m_ic\int ds - \sum_i\frac{e_i}{c}\int A_\mu dx^\mu - \frac{1}{16\pi c}
\int F_{\mu\nu} F^{\mu\nu} d\Omega.
\]
The second term is
\[
S_{mf} = -\frac{1}{c}\sum_i e_i\left(\int A_0dx^0 + \cdot + \int A_3dx^3\right).
\]
For a continuous charge distribution, we replace $\sum_i e_i$ by $\int \rho dV$
so that
\begin{eqnarray*}
S_{mf} &=& -\frac{1}{c}\int dV \rho\left(\int A_0dx^0 + \cdots + \int A_3dx^3\right) \\
 &=& -\frac{1}{c}\left(\int dVdt \rho\td{x^0}{t}A_0 + \cdots + \int dVdt \rho\td{x^3}{t}A_3\right) \\
 &=& -\frac{1}{c}\left(\int dVdt j^0A_0 + \cdots + \int dVdt j^3A_3\right)
\end{eqnarray*}
where we used equation \eqref{c4e23} in the last equation. Finally, we write it
as
\begin{equation}\label{c4e27}
S_{mf} = -\frac{1}{c^2}\int j^\mu A_\mu d\Omega.
\end{equation}
Thus, the action can be written as
\begin{equation}\label{c4e28}
S = -\sum_i m_ic\int ds - \frac{1}{c^2}\int j^\mu A_\mu d\Omega - 
\frac{1}{16\pi c}\int F_{\mu\nu} F^{\mu\nu} d\Omega.
\end{equation}

\item It is an experimental fact that electric charge is conserved. It is expressed
as an equation of continuity,
\begin{equation}\label{c4e29}
\frac{\partial}{\partial t}\int\rho dV = -\oint \rho\vec{v}\cdot d\vec{f}.
\end{equation}
The left hand side, if it is positive, is the rate of increase of charge per unit
time in the region of integration. The right hand side is the rate of inflow of
charges in the region of integration. The negative sign is needed because $d\vec{f}$
always points outwards so that if $\vec{v}$ points inwards then the dot product 
is negative. To get an overall positive sign, we negate the surface integral.
In differential form, \eqref{c4e29} is
\begin{equation}\label{c4e30}
\pdt{\rho}{t} + \dive\vec{j} = 0.
\end{equation}
We can also write it as
\[
\pdt{\rho}{t} + \pdt{j^1}{x^1} + \pdt{j^2}{x^2} + \pdt{j^3}{x^3} = 0.
\]
Since $j^\mu = (c\rho, \vec{j})$, this equation is equivalent to
\begin{equation}\label{c4e31}
\pdt{j^\mu}{x^\mu} = 0.
\end{equation}
This is the four-dimensional form of the equation of continuity.

\item Recall equation \eqref{c4e26} in which
\[
\frac{1}{c}\int j^\mu dS_\mu
\]
is the total charge at a time $x^0/c$ and the integral is over a hypersurface 
normal to the $x^0$ axis at that time. Consider the integral
\[
\frac{1}{c}\oint j^\mu dS_\mu,
\]
where the closed surface consists of the hypersurface of instant of time, a
similar hypersurface at a later instant of time and the two joined by surface
parallel to $x^0$ axis far away from the charge distribution. The integral over
this latter surface is zero because $j^\mu$ is zero on it. Therefore,
\begin{equation}\label{c4e32}
\frac{1}{c}\oint j^\mu dS_\mu = 0.
\end{equation}
Using the divergence theorem, we can transform the surface integral to a volume
integral so that
\[
\frac{1}{c}\int \pdt{j^\mu}{x^\mu} = 0.
\]
Since this is true for any two points on the $x^0$ axis, the divergence itself
is zero. This is another way of proving \eqref{c4e31}.

\item We showed in chapter \ref{c3} that the transformation described by equation
\eqref{c3e30},
\[
A_\mu^\op = A_\mu - \pdt{f}{x^\mu},
\]
also called the gauge transformation, leaves the fields unchanged. Let us make 
this transformation in the second term, \eqref{c4e27}, of the action \eqref{c4e28}.
Then,
\[
S_{mf} \rightarrow -\frac{1}{c^2}\int j^\mu A_\mu d\Omega + \frac{1}{c^2}
\int j^\mu\pdt{f}{x^\mu} d\Omega.
\]
Now,
\[
\frac{\partial}{\partial x^\mu}(fj^\mu) = f\pdt{j^\mu}{x^\mu} + j^\mu\pdt{f}{x^\mu}.
\]
If the equation of continuity, \eqref{c4e31}, is valid then
\begin{equation}\label{c4e33}
\frac{\partial}{\partial x^\mu}(fj^\mu) = j^\mu\pdt{f}{x^\mu}
\end{equation}
and
\[
S_{mf} \rightarrow -\frac{1}{c^2}\int j^\mu A_\mu d\Omega + \frac{1}{c^2}
\int \frac{\partial}{\partial x^\mu}(fj^\mu) d\Omega.
\]
Using divergence theorem on the second term,
\[
S_{mf} \rightarrow -\frac{1}{c^2}\int j^\mu A_\mu d\Omega + \frac{1}{c^2}
\int (fj^\mu) dS_\mu.
\]
We can always choose the surface to be large enough that $j^\mu$ is zero on it,
in which case, the integral vanishes. This manipulation critically depends on the
validity of \eqref{c4e33}, which in turn depends on the equation of continuity.
This shows that the gauge invariance of the electromagnetic fields is closely
related to the equation of continuity, in other words, to conservation of 
electric charge. Electric charge must be conserved if gauge invariance must be
true.

\item In finding the equations of motion from variational principle we assumed
that the fields are fixed while varying the generalised coordinates. Now, we will
assume that the generalised coordinates are fixed and will vary the fields instead.
The resulting equations are the second pair of Maxwell equations. The expression
for total action in \eqref{c4e28} has fields only in the second and the third
terms. Therefore,
\begin{equation}\label{c4e34}
\delta S = -\frac{1}{c^2}\int j^\mu \delta A_\mu d\Omega - \frac{1}{16\pi c}
\int\delta(F_{\mu\nu}F^{\mu\nu})d\Omega.
\end{equation}
Now 
\begin{eqnarray*}
\delta(F_{\mu\nu}F^{\mu\nu}) &=& (\delta F_{\mu\nu})F^{\mu\nu} + F_{\mu\nu} (\delta F^{\mu\nu}) \\
 &=& 2F^{\mu\nu}\delta F_{\mu\nu} \\
 &=& 2F^{\mu\nu}\left(\pdt{\delta A_\nu}{x^\mu} - \pdt{\delta A_\mu}{x^\nu}\right) \\
 &=& 2F^{\mu\nu}\pdt{\delta A_\nu}{x^\mu} - 2F^{\mu\nu}\pdt{\delta A_\mu}{x^\nu} \\
 &=& 2F^{\nu\mu}\pdt{\delta A_\mu}{x^\nu} - 2F^{\mu\nu}\pdt{\delta A_\mu}{x^\nu} \\
 &=& -2F^{\mu\nu}\pdt{\delta A_\mu}{x^\nu} - 2F^{\mu\nu}\pdt{\delta A_\mu}{x^\nu} \\
 &=& -4F^{\mu\nu}\pdt{\delta A_\mu}{x^\nu}
\end{eqnarray*}
so that equation \eqref{c4e34} becomes
\begin{equation}\label{c4e35}
\delta S = -\frac{1}{c^2}\int j^\mu \delta A_\mu d\Omega + \frac{1}{4\pi c}
\int F^{\mu\nu}\pdt{\delta A_\mu}{x^\nu} d\Omega
\end{equation}
Since 
\[
\frac{\partial}{\partial x^\nu}(F^{\mu\nu} \delta A_\mu) =
F^{\mu\nu}\pdt{\delta A_\mu}{x^\nu} + \delta A_\mu\pdt{F^{\mu\nu}}{x^\nu},
\]
\begin{eqnarray}
\delta S &=& -\frac{1}{c^2}\int j^\mu \delta A_\mu d\Omega + 
\frac{1}{4\pi c}\int \frac{\partial}{\partial x^\nu}(F^{\mu\nu} \delta A_\mu) d\Omega - \nonumber \\
 & & \frac{1}{4\pi c}\int \delta A_\mu\pdt{F^{\mu\nu}}{x^\nu} d\Omega \nonumber \\
 &=& -\frac{1}{c}\left(\frac{1}{c}\int j^\mu \delta A_\mu d\Omega + 
    \frac{1}{4\pi} \int\delta A_\mu\pdt{F^{\mu\nu}}{x^\nu}d\Omega\right) + \nonumber \\
 & & \frac{1}{4\pi c}\int F^{\mu\nu}\delta A_\mu dS_\mu, \nonumber 
\end{eqnarray}
where we used the generalisation of Gauss' theorem to 4-dimensions to convert a
volume integral to a surface integral. We can always choose the surface to be so
large that there are no fields on it. As a result the last term can be ignored
and we are left with
\begin{equation}\label{c4e36}
\delta S = -\frac{1}{c}\int\left(\frac{j^\mu}{c} +
 \frac{1}{4\pi}\pdt{F^{\mu\nu}}{x^\nu}\right)\delta A_\mu d\Omega
\end{equation}
If $\delta S = 0$ for arbitrary variations of the fields then we must have
\begin{equation}\label{c4e37}
\pdt{F^{\mu\nu}}{x^\nu} = -\frac{4\pi}{c}j^\mu.
\end{equation}
This equation includes the remaining two Maxwell equations. For $\mu=0$, it is
\[
\pdt{F^{00}}{x^0} + \pdt{F^{01}}{x^1} + \pdt{F^{02}}{x^2} + \pdt{F^{03}}{x^3}
= -\frac{4\pi}{c}j^0 = -4\pi\rho.
\]
From \eqref{c3e77}, $F^{01} = -E_1, F^{02} = -E_2, F^{03} = -E_3$ so that the
above equation is
\begin{equation}\label{c4e38}
\dive\vec{E} = 4\pi\rho.
\end{equation}
For $\mu = 1$, \eqref{c4e37} becomes,
\[
\pdt{F^{10}}{x^0} + \pdt{F^{11}}{x^1} + \pdt{F^{12}}{x^2} + \pdt{F^{13}}{x^3}
= -\frac{4\pi}{c}j^1 = -\frac{4\pi}{c}j^1
\]
that is,
\[
\frac{1}{c}\pdt{E_1}{t} + 0 - \pdt{H_3}{x_2} + \pdt{H_2}{x_3} = -\frac{4\pi}{c}j^1
\]
or
\begin{equation}\label{c4e39}
(\curl\vec{H})_x = \frac{4\pi}{c}j_x + \frac{1}{c}\pdt{E_x}{t}.
\end{equation}
The other components can be derived similarly. We thus get,
\begin{equation}\label{c4e40}
\curl\vec{H} = \frac{4\pi}{c}\vec{j} + \frac{1}{c}\pdt{\vec{E}}{t}.
\end{equation}
Equations \eqref{c4e1}, \eqref{c4e2}, \eqref{c4e38} and \eqref{c4e40}, are the
four Maxwell equations. In terms of field vectors they are
\begin{eqnarray*}
\curl\vec{E} &=& -\frac{1}{c}\pdt{\vec{H}}{t} \\
\dive\vec{H} &=& 0 \\
\dive\vec{E} &=& 4\pi\rho \\
\curl\vec{H} &=& \frac{4\pi}{c}\vec{j} + \frac{1}{c}\pdt{\vec{E}}{t}.
\end{eqnarray*}
They can be equivalently written in terms of the electromagnetic field tensor as
equations \eqref{c4e5} and \eqref{c4e37}.
\begin{eqnarray*}
\pdt{F_{\mu\nu}}{x^\rho} + \pdt{F_{\nu\rho}}{x^\mu} + \pdt{F_{\rho\mu}}{x^\nu} &=& 0 \\
\pdt{F^{\mu\nu}}{x^\nu} &=& -\frac{4\pi}{c}j^\mu
\end{eqnarray*}

\item Equation \eqref{c4e38} in integral form is
\begin{equation}\label{c4e41}
\oint\vec{E}\cdot d\vec{f} = 4\pi\int\rho dV.
\end{equation}
The total electric flux over a closed surface is equal to $4\pi$ times the total 
charge enclosed in it. This is \emph{Gauss law}.

\item Equation \eqref{c4e40} in integral form is
\begin{equation}\label{c4e42}
\oint\vec{H}\cdot d\vec{l} = \frac{4\pi}{c}\left(\int\vec{j}\cdot d\vec{f} + 
\frac{1}{4\pi}\int\pdt{\vec{E}}{t}\cdot d\vec{f}\right).
\end{equation}
The first term in the bracket on the rhs is the true current while the second one
is called the `displacement current'. The term,
\[
\frac{1}{4\pi}\pdt{\vec{E}}{t}
\]
is called the \emph{displacement current density}.

\item From \eqref{c4e40}, one immediately gets, after taking divergence of both sides
and using \eqref{c4e38},
\[
\dive\vec{j} + \pdt{\rho}{t} = 0.
\]
Thus the equation of continuity and charge conversation are encoded in 
\eqref{c4e37}. It can also be easily derived taking the derivative of \eqref{c4e37}
with respect to $x^\mu$,
\[
\frac{\partial^2 F^{\mu\nu}}{\partial x^\mu \partial x^\nu} = -\frac{4\pi}{c}\pdt{j^\mu}{x^\mu}.
\]
The lhs can be written as
\[
\frac{\partial^2 F^{\mu\nu}}{\partial x^\mu \partial x^\nu} = 
\frac{\partial^2 F^{\nu\mu}}{\partial x^\nu \partial x^\mu}
= -\frac{\partial^2 F^{\mu\nu}}{\partial x^\nu \partial x^\mu} = 
-\frac{\partial^2 F^{\mu\nu}}{\partial x^\mu \partial x^\nu}.
\]
The last step follows from the equality of mixed partial derivatives and the 
second last from the anti-symmetry of $F^{\mu\nu}$. Thus the lhs has to be zero
so that, we have,
\[
\pdt{j^\mu}{x^\mu} = 0,
\]
which is indeed the equation of continuity, \eqref{c4e31}, in 4-vector form.

\item Taking dot product of \eqref{c4e1} with $\vec{H}$ and that of \eqref{c4e40}
with $\vec{E}$, we get
\begin{eqnarray*}
\vec{H}\cdot\curl\vec{E} &=& -\frac{1}{2c}\pdt{H^2}{t} \\
\vec{E}\cdot\curl\vec{H} &=& \frac{4\pi}{c}\vec{j}\cdot\vec{E} + \frac{1}{2c}\pdt{E^2}{t}
\end{eqnarray*}
Subtracting the first equation from the second,
\begin{equation}\label{c4e43}
\vec{E}\cdot\curl\vec{H} - \vec{H}\cdot\curl\vec{E} = \frac{4\pi}{c}\vec{j}\cdot\vec{E}
+ \frac{1}{2c}\frac{\partial}{\partial t}(E^2 + H^2).
\end{equation}
We can rearrange it as
\begin{equation}\label{c4e44}
\frac{\partial}{\partial t}\left(\frac{E^2 + H^2}{8\pi}\right) = 
-\vec{j}\cdot\vec{E} - 
\frac{c}{4\pi}(\vec{H}\cdot\curl\vec{E} - \vec{E}\cdot\curl\vec{H}).
\end{equation}
The \emph{Poynting vector}, is defined as
\begin{equation}\label{c4e45}
\vec{S} = \frac{c}{4\pi}(\vec{E} \times \vec{H})
\end{equation}
so that \eqref{c4e44} can be written as
\begin{equation}\label{c4e46}
\frac{\partial}{\partial t}\left(\frac{E^2 + H^2}{8\pi}\right) = 
-\vec{j}\cdot\vec{E} - \dive\vec{S}.
\end{equation}

\item From equation \eqref{c3e27}, we have
\[
\td{\mathcal{E}}{t} = e\vec{v}\cdot\vec{E}
\]
so that for a collection of charges, we have
\[
\sum_a e_a\vec{v}_a\cdot\vec{E} = \td{\mathcal{E}}{t}
\]
or, in the continuum limit,
\begin{equation}\label{c4e47}
\int\vec{j}\cdot\vec{E}dV = \td{\mathcal{E}}{t}.
\end{equation}
If we integrate \eqref{c4e46} over a volume, $V$, we have
\begin{equation}\label{c4e48}
\frac{d}{dt}\int\frac{E^2 + H^2}{8\pi}dV + 
\int \vec{j}\cdot\vec{E}dV = -\oint d\vec{f}\cdot\vec{S}.
\end{equation}
If the volume of integration is so large that $\vec{S}$ is effectively zero on
its bounding surface then, using \eqref{c4e47},
\begin{equation}\label{c4e49}
\frac{d}{dt}\left(\int\frac{E^2 + H^2}{8\pi}dV + \mathcal{E}\right) = 0.
\end{equation}
This equation suggests that for a closed system of charges and currents,
the quantity
\[
\int\frac{E^2 + H^2}{8\pi}dV + \mathcal{E}
\]
is conserved. Since $\mathcal{E}$ is the kinetic energy of the charges,
the integral must be the energy of the fields. We all the quantity,
\begin{equation}\label{c4e50}
W = \frac{E^2 + H^2}{8\pi}
\end{equation}
the energy density of the field. For a small enough volume for which the rhs of
\eqref{c4e48} does not vanish, we interpret the rhs as the flux of energy escaping
the bounding surface. The Poynting vector $\vec{S}$ is thus the flux density of
energy passing through a surface.

\item Consider an arbitrary relativistic field with generalised coordinates 
$q_\alpha$ and described by a Lagrangian density,
\[
\Lambda\left(q_\alpha, \pdt{q_\alpha}{x^\mu}\right)
\]
so that its action is
\begin{equation}\label{c4e51}
S = \int \Lambda\left(q_\alpha, \pdt{q_\alpha}{x^\mu}\right)dV dt = \frac{1}{c}
\int\Lambda\left(q_\alpha, \pdt{q_\alpha}{x^\mu}\right)d\Omega.
\end{equation}
Let us introduce the notation
\begin{equation}\label{c4e52}
q_{\alpha; \mu} = \pdt{q_\alpha}{x^\mu}
\end{equation}
to simplify our expressions. A variation of the action is
\begin{equation}\label{c4e53}
\delta S = \frac{1}{c}\int\left(\pdt{\Lambda}{q_\alpha}\delta q_\alpha +
\pdt{\Lambda}{q_{\alpha;\mu}}\delta q_{\alpha;\mu}\right)d\Omega.
\end{equation}
Now,
\[
\frac{\partial}{\partial x^\mu}\left(\pdt{\Lambda}{q_{\alpha;\mu}} \delta q_\alpha\right) = 
\pdt{\Lambda}{q_{\alpha;\mu}}\delta q_{\alpha;\mu} + \delta q_\alpha\frac{\partial}
{\partial x^\mu}\pdt{\Lambda}{q_{\alpha;\mu}}
\]
so that \eqref{c4e53} becomes,
\begin{equation}\label{c4e54}
\delta S = \frac{1}{c}\int\left(\left(\pdt{\Lambda}{q_\alpha} - \frac{\partial}
{\partial x^\mu}\pdt{\Lambda}{q_{\alpha;\mu}}\right)\delta q_\alpha +
\frac{\partial}{\partial x^\mu}\left(\pdt{\Lambda}{q_{\alpha;\mu}} \delta q_\alpha
\right)\right)d\Omega.
\end{equation}
The second term in the integrand is a 4-divergence. The volume integral of a 4-
divergence can be written as a surface integral. If we choose the surface large 
enough so that the field is zero on it, the surface integral vanishes and we are
left with
\begin{equation}\label{c4e55}
\delta S = \frac{1}{c}\int \left(\pdt{\Lambda}{q_\alpha} - \frac{\partial}
{\partial x^\mu}\pdt{\Lambda}{q_{\alpha;\mu}}\right)\delta q_\alpha d\Omega.
\end{equation}
Since this equation is true for all variations $\delta q_\alpha$, we have the 
Euler-Lagrange equations
\begin{equation}\label{c4e56}
\pdt{\Lambda}{q_\alpha} = \frac{\partial}{\partial x^\mu}\pdt{\Lambda}{q_{\alpha;\mu}}
\end{equation}
Recall that $\Lambda$ is a function of $q_\alpha$ and $q_{\alpha;\mu}$ so that
\[
d\Lambda = \pdt{\Lambda}{q_\alpha}dq_\alpha + \pdt{\Lambda}{q_{\alpha;\mu}}dq_{\alpha;\mu}
\]
and
\begin{equation}\label{c4e57}
\pdt{\Lambda}{x^\nu} = \pdt{\Lambda}{q_\alpha}\pdt{q_\alpha}{x^\nu} + 
\pdt{\Lambda}{q_{\alpha;\mu}}\pdt{q_{\alpha;\mu}}{x^\nu}
\end{equation}
Since,
\[
\pdt{q_{\alpha;\mu}}{x^\nu} = \frac{\partial^2 q_\alpha}{\partial x^\mu \partial x^\nu}
= \frac{\partial^2 q_\alpha}{\partial x^\nu \partial x^\mu} =\pdt{q_{\alpha;\nu}}{x^\mu},
\]
equation \eqref{c4e57} becomes
\[
\pdt{\Lambda}{x^\nu} = \pdt{\Lambda}{q_\alpha}\pdt{q_\alpha}{x^\nu} + 
\pdt{\Lambda}{q_{\alpha;\mu}}\pdt{q_{\alpha;\nu}}{x^\mu}.
\]
Using the Euler-Lagrange equation \eqref{c4e56} in the first term on the rhs,
\[
\pdt{\Lambda}{x^\nu} = \frac{\partial}{\partial x^\mu}\pdt{\Lambda}{q_{\alpha;\mu}}
q_{\alpha;\nu} + \pdt{\Lambda}{q_{\alpha;\mu}}\pdt{q_{\alpha;\nu}}{x^\mu} =
\frac{\partial}{\partial x^\mu}\left(\pdt{\Lambda}{q_{\alpha;\mu}}q_{\alpha;\nu}\right)
\]
We can write the lhs as
\[
\pdt{\Lambda}{x^\nu} = \delta^\mu_\nu\pdt{\Lambda}{x^\mu} = 
\frac{\partial}{\partial x^\mu}\left(\delta^\mu_\nu \Lambda\right)
\]
so that we have
\begin{equation}\label{c4e58}
\frac{\partial}{\partial x^\mu}
\left(\pdt{\Lambda}{q_{\alpha;\mu}}q_{\alpha;\nu} - \delta^\mu_\nu \Lambda\right) = 0.
\end{equation}
If
\begin{equation}\label{c4e59}
\tensor{T}{_\nu^\mu} = q_{\alpha;\nu}\pdt{\Lambda}{q_{\alpha;\mu}} - 
\tensor{\delta}{_\nu^\mu} \Lambda,
\end{equation}
then we can write \eqref{c4e58} as
\begin{equation}\label{c4e60}
\pdt{\tensor{T}{_\nu^\mu}}{x^\mu} = 0.
\end{equation}
Since $T^{\nu\mu} = g^{\mu\nu}\tensor{T}{_\nu^\mu}$ and $g^{\mu\nu}$ being a constant tensor,
equation \eqref{c4e60} can also be written as
\begin{equation}\label{c4e61}
\pdt{T^{\nu\mu}}{x^\mu} = 0.
\end{equation}
Since
\[
\int \pdt{T^{\nu\mu}}{x^\mu} d\Omega = \int T^{\nu\mu} dS_\mu,
\]
equation \eqref{c4e61} also gives us
\begin{equation}\label{c4e62}
\oint T^{\nu\mu} dS_\mu = 0.
\end{equation}
If we choose the surface to be such that it is composed of a hyperplane 
perpendicular to $x^0$ at one instant of time, another hyperplane 
perpendicular to $x^0$ at another instant of time and an arbitrary surface
joining these hyperplanes at a great distance from the fields, then \eqref{c4e62}
indicates that the vector
\begin{equation}\label{c4e63}
p^\nu = a\int T^{\nu\mu}dS_\mu,
\end{equation}
where $a$ is a constant, has the same value at the two instants of $x^0$ involved
in constructing the closed surface. In other words, $p^\nu$ is a constant. For an
appropriately chosen value of $a$, $p^\nu$ becomes the 4-momentum density of the
system.

\item If $\nu = 0$, \eqref{c4e63} becomes
\[
p^0 = a\int T^{0\mu}dS_\mu,
\]
If the integration is over the hyperplane $x^0$ then $T^{0\mu}dS_\mu = T^{00}dV$. Thus,
\[
p^0 = a\int T^{00}dV.
\]
Now, 
\[
T^{00} = g^{00}\tensor{T}{_0^0} = q_{\alpha;0}\pdt{\Lambda}{q_{\alpha;0}} - 
\tensor{\delta}{_0^0} \Lambda
\]
as $g^{00} = 1$. Furthermore, 
\[
q_{\alpha; 0} = \pdt{q_\alpha}{x^0} = \frac{1}{c}\dot{q}_\alpha
\]
so that
\[
T^{00} = \dot{q}_\alpha\pdt{\Lambda}{\dot{q}_\alpha} - \tensor{\delta}{_0^0} \Lambda.
\]
Clearly, $T^{00}$ is the energy density of the system. In analogy with \eqref{c2e26},
the constant $a$ can be chosen to be $1/c$. Thus, the correct expression for the 
4-momentum density is
\begin{equation}\label{c4e64}
p^\nu = \frac{1}{c}\int T^{\nu\mu}dS_\mu.
\end{equation}

\item Consider the tensor
\begin{equation}\label{c4e65}
U^{\mu\nu} = T^{\mu\nu} + \pdt{\psi^{\mu\nu\sigma}}{x^\sigma}
\end{equation}
where $\psi^{\mu\nu\sigma} = -\psi^{\mu\sigma\nu}$, that is $\psi^{\mu\nu\sigma}$
is an anti-symmetric tensor of rank 3. Because of anti-symmetry,
\[
\frac{\partial^2\psi^{\mu\nu\sigma}}{\partial x^\mu \partial x^\sigma} = 
\frac{\partial^2\psi^{\mu\nu\sigma}}{\partial x^\sigma \partial x^\mu} = 
\frac{\partial^2\psi^{\mu\sigma\nu}}{\partial x^\sigma \partial x^\mu} = 
-\frac{\partial^2\psi^{\mu\nu\sigma}}{\partial x^\mu \partial x^\sigma}
\]
so that
\begin{equation}\label{c4e66}
\frac{\partial^2\psi^{\mu\nu\sigma}}{\partial x^\mu \partial x^\sigma} = 0.
\end{equation}
Now consider,
\begin{equation}\label{c4e67}
\int U^{\mu\nu} dS_\mu = \int T^{\mu\nu} dS_\mu + 
\int\pdt{\psi^{\mu\nu\sigma}}{x^\sigma} dS_\mu.
\end{equation}
We manipulate the second term on the rhs using equation \eqref{c1e77} for an 
anti-symmetric tensor to get
\begin{equation}\label{c4e68}
\int U^{\mu\nu} dS_\mu = \int T^{\mu\nu} dS_\mu + 
\frac{1}{2}\int df^\ast_{\mu\sigma}\psi^{\mu\nu\sigma}.
\end{equation}
The first two integrals in \eqref{c4e68} are over three dimensional hypersurfaces,
that is ordinary volumes while the last integral is over a two dimensional surface.
If we let the surface be large enough that the fields on it are zero then the
last term will vanish and we are left with
\begin{equation}\label{c4e69}
\int U^{\mu\nu} dS_\mu = \int T^{\mu\nu} dS_\mu,
\end{equation}
that is, the 4-momentum density is the same for both energy-momentum tensors. Thus
the energy-momentum tensor is defined to withing the partial derivative of an
anti-symmetric tensor of rank 3.

\item From equation \eqref{c2e86} the angular momentum of the field can be written
as
\begin{equation}\label{c4e70}
L^{\mu\nu} = \int (x^\mu dp^\nu - x^\nu dp^\mu).
\end{equation}
From \eqref{c4e64}, we have $dp^\mu = T^{\mu\nu}dS_\nu$ so that we can write 
\eqref{c4e70} as 
\begin{equation}\label{c4e71}
L^{\mu\nu} = \frac{1}{c}\int (x^\mu T^{\nu\rho} - x^\nu T^{\mu\rho})dS_\rho.
\end{equation}
If the integration is taken such that $dS_\rho$ is normal to the hypersurface of
all space coordiates at at a certain value of $x^0$ then $L^{\mu\nu}$ is the
angular momentum of the system at time $x^0/c$. Now,
\[
L^{\mu\nu}(t_1) - L^{\mu\nu}(t_2) = 
\frac{1}{c}\oint(x^\mu T^{\nu\rho} - x^\nu T^{\mu\rho})dS_\rho,
\]
where the integral is over the closed hypersurface comprising of:
\begin{itemize}
\item The hypersurface normal to the $x^0$ axis at time $t_1$,
\item The hypersurface normal to the $x^0$ axis at time $t_2$ and
\item A hypersurface at a great distance from the fields joining these hyperplanes.
\end{itemize}
From the 4-dimensional Gauss theorem of \eqref{c1e74},
\begin{equation}\label{c4e72}
L^{\mu\nu}(t_1) - L^{\mu\nu}(t_2) = 
\frac{1}{c}\int\frac{\partial}{\partial x^\rho}(x^\mu T^{\nu\rho} - x^\nu T^{\mu\rho})d\Omega.
\end{equation}
The integrand is
\begin{eqnarray*}
\frac{\partial}{\partial x^\rho}(x^\mu T^{\nu\rho} - x^\nu T^{\mu\rho}) &=&
\tensor{\delta}{_\rho^\mu}T^{\nu\rho} + x^\mu\pdt{T^{\nu\rho}}{x^\rho} - 
\tensor{\delta}{_\rho^\nu}T^{\mu\rho} - x^\nu\pdt{T^{\mu\rho}}{x^\rho} \\
 &=& T^{\nu\mu} + x^\mu\pdt{T^{\nu\rho}}{x^\rho} - T^{\mu\nu} -  x^\nu\pdt{T^{\mu\rho}}{x^\rho}
\end{eqnarray*}
From equation \eqref{c4e61},
\[
\pdt{T^{\nu\rho}}{x^\rho} = 0; \;\; \pdt{T^{\mu\rho}}{x^\rho} = 0
\]
so that equation \eqref{c4e72} is
\begin{equation}\label{c4e73}
L^{\mu\nu}(t_1) - L^{\mu\nu}(t_2) = \frac{1}{c}\int (T^{\nu\mu} - T^{\mu\nu})d\Omega.
\end{equation}
If $L^{\mu\nu}(t_1) = L^{\mu\nu}(t_2)$ then we must have $T^{\mu\nu} = T^{\nu\mu}$. 

\item We once again choose to the surface of integration in \eqref{c4e64} as the
one whose normal is along the $x^0$ axis so that
\[
p^\nu = \frac{1}{c}\int T^{\nu\mu}dS_0 = \int{1}{c}\int T^{\nu 0}dV.
\]
Since $p^\nu$ is the 4-momentum of the system, we call $W = T^{00}$ the energy
density and $T^{01}/c, T^{02}/c, T^{03}/c$ as the momentum density of the system.

\item Consider the conservation equation \eqref{c4e61} for $\nu = 0$. It is
\[
\frac{1}{c}\pdt{T^{00}}{t} + \pdt{T^{0i}}{x^i} = 0.
\]
If we integrate it over a volume $V$, 
\[
\frac{1}{c}\frac{\partial}{\partial t}\int_V T^{00}dV = -\int\pdt{T^{0i}}{x^i}dV
\]
Using the Gauss theorem in 3-dimensions on the rhs we get
\[
\frac{\partial}{\partial t}\int_V T^{00}dV = -c\oint T^{0i}df_i.
\]
Thus, the components $cT^{0i}$ are the also the flux of energy through the closed
surface bounding the volume $V$. In the previous point we identifier $T^{0i} = 
cp^i$. Thus, $c^2p^i$ can be identified as components of flux of energy.

\item Now consider the conservation equation \eqref{c4e62} for the space
components,
\[
\frac{1}{c}\pdt{T^{0i}}{t} + \pdt{T^{ij}}{x^j} = 0.
\]
Repeating the manipulations in the previous point, we get
\[
\frac{\partial}{\partial t}\int_V \frac{T^{0i}}{c} dt = -\oint T^{ij}df_j.
\]
However, $T^{0i}/c = p^i$, so that
\[
\frac{\partial}{\partial t}\int_V p^i dt = -\oint T^{ij}df_j.
\]
The lhs is the rate of change of momentum in the volume. Therefore, the rhs must
be the total force on it. The force is expressed as a surface integral. Therefore,
we identify the integrand as the negative of the stress tensor $\sigma^{ij}$. The
stress tensor is thus a flux of momentum. We can write the above equation as
\begin{equation}\label{c4e74}
\frac{\partial}{\partial t}\int_V p^i dt = \oint \sigma^{ij}df_j.
\end{equation}

$T^{ij}$ is the flux of $p^i$ through surface perpendicular to the $x^j$ axis. In
similar vain, $\sigma^{ij}$ is the normal force on a surface perpendicular to the
$x^i$ axis and the $i$-th component of the shear force acting in the $x^j$ 
direction.

The physical nature of the energy-momentum tensor reveals itself if we write it as
\begin{equation}\label{c4e75}
T^{\mu\nu} = \begin{bmatrix}
W & p_x & p_y & p_z \\
p_x & -\sigma_{xx} & -\sigma_{xy} & \sigma_{xz} \\
p_y & -\sigma_{yx} & -\sigma_{yy} & \sigma_{yz} \\
p_z & -\sigma_{zx} & -\sigma_{yy} & \sigma_{zz}
\end{bmatrix}.
\end{equation}
Note that, $p_x, p_y, p_z$ are components of the 3-momentum for which we do not
distinguish between covariant and contravariant components. Likewise for the
components of the stress tensor in three dimensions.

\item We will now apply the theory developed for a general Lagrangian density
in points 19 onwards to the electromagnetic field for which
\begin{equation}\label{c4e76}
\Lambda(A^\mu) = -\frac{1}{16\pi}F^{\mu\nu}F_{\mu\nu}.
\end{equation}
To get an expression for the stress tensor of \eqref{c4e59}, we need the 
derivative of $\Lambda$ with respect to 
\[
A_{\alpha; \mu} = \pdt{A_\alpha}{x^\mu}.
\]
A tedious and error prone method could be write down all $16$ terms on the rhs of
\eqref{c4e78}. A more convenient method is to get a variation of $\Lambda$. That is,
\begin{equation}\label{c4e77}
\delta\Lambda = -\frac{1}{16\pi}\left((\delta F^{\mu\nu})F_{\mu\nu} + 
F^{\mu\nu}(\delta F_{\mu\nu})\right) = -\frac{1}{8\pi}F^{\mu\nu}\delta F_{\mu\nu}.
\end{equation}
However,
\begin{equation}\label{c4e78}
\delta F_{\mu\nu} = \delta(\partial_\mu A_\nu - \partial_\nu A_\mu)
= \delta\partial_\mu A_\nu - \delta\partial_\nu A_\mu.
\end{equation}
so that
\begin{eqnarray}
\delta\Lambda &=& -\frac{1}{8\pi}F^{\mu\nu}(\delta\partial_\mu A_\nu - \delta\partial_\nu A_\mu) \nonumber \\
8\pi\delta\Lambda &=& -F^{\mu\nu}\delta\partial_\mu A_\nu + F^{\mu\nu}\delta\partial_\nu A_\mu \nonumber \\
 &=& -F^{\mu\nu}\delta\partial_\mu A_\nu + F^{\nu\mu}\delta\partial_\mu A_\nu \nonumber \\
 &=& -F^{\mu\nu}\delta\partial_\mu A_\nu - F^{\mu\nu}\delta\partial_\mu A_\nu \nonumber \\
 &=& -2F^{\mu\nu}\delta\partial_\mu A_\nu \label{c4e79}
\end{eqnarray}
so that
\begin{equation}\label{c4e80}
\pdt{\Lambda}{A_{\nu;\mu}} = -\frac{1}{4\pi}F^{\mu\nu}.
\end{equation}
Using this equation in \eqref{c4e59}, we get
\begin{equation}\label{c4e81}
\tensor{T}{_\nu^\mu} = -\frac{1}{4\pi}A_{\alpha;\nu}F^{\mu\alpha} +
\frac{1}{16\pi}\tensor{\delta}{_\nu^\mu} F_{\alpha\beta}F^{\alpha\beta}.
\end{equation}
Multiplying by $g^{\nu\rho}$,
\begin{eqnarray}
g^{\nu\rho}\tensor{T}{_\nu^\mu} &=& -\frac{1}{4\pi}g^{\nu\rho}A_{\alpha;\nu}F^{\mu\alpha} +
\frac{1}{16\pi}g^{\nu\rho}\tensor{\delta}{_\nu^\mu} F_{\alpha\beta}F^{\alpha\beta} \nonumber \\
T^{\rho\mu} &=& -\frac{1}{4\pi}\pdt{A_\alpha}{x_\rho}F^{\mu\alpha} + 
\frac{1}{16\pi}\tensor{g}{^\rho^\mu} F_{\alpha\beta}F^{\alpha\beta} \nonumber \\
 &=& -\frac{1}{4\pi}\pdt{A^\alpha}{x_\rho}\tensor{F}{^\mu_\alpha} + 
\frac{1}{16\pi}\tensor{g}{^\rho^\mu} F_{\alpha\beta}F^{\alpha\beta} \label{c4e82}
\end{eqnarray}
From equation \eqref{c4e37}, after multiplying by an appropriate metric tensor,
\[
\pdt{\tensor{F}{^\mu_\alpha}}{x_\alpha} = -\frac{4\pi}{c}j^\mu.
\]
In absence of charges, it becomes
\[
\pdt{\tensor{F}{^\mu_\alpha}}{x_\alpha} = 0
\]
so that
\begin{equation}\label{c4e83}
\frac{1}{4\pi}\frac{\partial}{\partial x_\alpha}(A^\rho\tensor{F}{^\mu_\alpha}) = 
\frac{1}{4\pi}\pdt{A^\rho}{x_\alpha}\tensor{F}{^\mu_\alpha}
\end{equation}
From equations \eqref{c4e65} and \eqref{c4e69}, we can always add a term like 
the lhs of above equation to the stress tensor without altering its physical 
content. We will add the term to make the stress tensor symmmetric. Thus, we 
write \eqref{c4e82} as
\begin{eqnarray*}
T^{\rho\mu} &=& -\frac{1}{4\pi}\pdt{A^\alpha}{x_\rho}\tensor{F}{^\mu_\alpha} + 
\frac{1}{16\pi}\tensor{g}{^\rho^\mu} F_{\alpha\beta}F^{\alpha\beta} +
\frac{1}{4\pi}\frac{\partial}{\partial x_\alpha}(A^\rho\tensor{F}{^\mu_\alpha}) \\
 &=& -\frac{1}{4\pi}\pdt{A^\alpha}{x_\rho}\tensor{F}{^\mu_\alpha} + 
\frac{1}{16\pi}\tensor{g}{^\rho^\mu} F_{\alpha\beta}F^{\alpha\beta} +
\frac{1}{4\pi}\pdt{A^\rho}{x_\alpha}\tensor{F}{^\mu_\alpha} \\
 &=& \frac{1}{4\pi}\left(\pdt{A^\rho}{x_\alpha} - \pdt{A^\alpha}{x_\rho}\right)
 \tensor{F}{^\mu_\alpha} + 
 \frac{1}{16\pi}\tensor{g}{^\rho^\mu} F_{\alpha\beta}F^{\alpha\beta} \\
 &=& \frac{1}{4\pi}F^{\alpha\rho}\tensor{F}{^\mu_\alpha} +
  \frac{1}{16\pi}\tensor{g}{^\rho^\mu} F_{\alpha\beta}F^{\alpha\beta}
\end{eqnarray*}
so that
\begin{equation}\label{c4e84}
T^{\rho\mu} = \frac{1}{4\pi}\left(-F^{\rho\alpha}\tensor{F}{^\mu_\alpha} +
\frac{1}{4}g^{\rho\mu}F_{\alpha\beta}F^{\alpha\beta}\right)
\end{equation}
Equation \eqref{c4e84} is the appropriate form of the energy-momentum tensor for
the electromagnetic field.
From equations \eqref{c3e75} and \eqref{c3e77}
\begin{eqnarray*}
F_{\mu\nu} &=& \begin{bmatrix} 0 & E_x & E_y & E_z \\
-E_x & 0 & -H_z & H_y \\
-E_y & H_z & 0 & -H_x \\
-E_z & -H_y & H_x & 0
\end{bmatrix} \\
F^{\mu\nu} &=& \begin{bmatrix} 0 & -E_x & -E_y & -E_z \\
E_x & 0 & -H_z & H_y \\
E_y & H_z & 0 & -H_x \\
E_z & -H_y & H_x & 0
\end{bmatrix}
\end{eqnarray*}
Since $\tensor{F}{^\mu_\nu} = F^{\mu\rho}g_{\rho\nu}$,
\begin{eqnarray}
\tensor{F}{^\mu_\nu} &=& \begin{bmatrix} 0 & E_x & E_y & E_z \\
-E_x & 0 & -H_z & H_y \\
-E_y & H_z & 0 & -H_x \\
-E_z & -H_y & H_x & 0
\end{bmatrix}\begin{bmatrix}1 & 0 & 0 & 0 \\
0 & -1 & 0 & 0 \\
0 & 0 & -1 & 0 \\
0 & 0 & 0 & -1
\end{bmatrix} \nonumber \\
 &=& \begin{bmatrix}0 & -E_x & -E_y & -E_z \\
 -E_x & 0 & H_z & -H_y \\
 -E_y & -H_z & 0 & H_x \\
 -E_z & H_y & -H_x & 0 
 \end{bmatrix} \label{c4e85}
\end{eqnarray}
so that
\[
F^{\rho\alpha}\tensor{F}{^\mu_\alpha} = \begin{bmatrix}
0 & -E_x & -E_y & -E_z \\
E_x & 0 & -H_z & H_y \\
E_y & H_z & 0 & -H_x \\
E_z & -H_y & H_x & 0
\end{bmatrix}\begin{bmatrix}0 & -E_x & -E_y & -E_z \\
 -E_x & 0 & H_z & -H_y \\
 -E_y & -H_z & 0 & H_x \\
 -E_z & H_y & -H_x & 0 
 \end{bmatrix}
\]
or
\begin{equation}\label{c4e86}
F^{\rho\alpha}\tensor{F}{^\mu_\alpha} =
\begin{bmatrix} E^2 & E_yH_z - E_zH_y & -E_xH_z + E_zH_x & E_xH_y  - E_yH_x \\
 E_yH_z - E_zH_y & -E_x^2 + H_z^2 + H_y^2 & -E_xE_y -H_xH_y & -E_xE_z -H_xH_z \\
 E_zH_x - E_xH_z & -E_xE_y - H_xH_y & -E_y^2 + H_z^2 + H_x^2 & -E_yE_z - H_yH_z \\
 E_xH_y - E_yH_x & -E_xE_z -H_xH_z & -E_yE_z - H_yH_z & -E_z^2 + H_y^2 + H_x^2
 \end{bmatrix}
\end{equation}
Similarly,
\[
F_{\mu\nu}F^{\mu\nu} = \begin{bmatrix} 0 & E_x & E_y & E_z \\
-E_x & 0 & -H_z & H_y \\
-E_y & H_z & 0 & -H_x \\
-E_z & -H_y & H_x & 0
\end{bmatrix}\begin{bmatrix} 0 & -E_x & -E_y & -E_z \\
E_x & 0 & -H_z & H_y \\
E_y & H_z & 0 & -H_x \\
E_z & -H_y & H_x & 0
\end{bmatrix}
\]
or
\begin{equation}\label{c4e87}
F^{\mu\nu}F_{\mu\nu} = 2(-E^2 + H^2).
\end{equation}
and
\begin{equation}\label{c4e88}
\frac{1}{4}F^{\mu\nu}F_{\mu\nu} = \frac{H^2}{2} - \frac{E^2}{2}.
\end{equation}
and \eqref{c4e84} becomes (note that the second term in \eqref{c4e84} is a 
diagonal matrix)
\begin{eqnarray*}
T^{00} &=& \frac{E^2 + H^2}{8\pi} \\
T^{01} &=& \frac{E_yH_z - E_zH_y}{4\pi} \\
T^{02} &=& \frac{E_zH_x - E_xH_z}{4\pi} \\
T^{03} &=& \frac{E_xH_y - E_yH_x}{4\pi} \\
T^{11} &=& \frac{E_x^2 - E_y^2 - E_z^2 + H_x^2 - H_y^2 - H_z^2}{8\pi} \\
T^{12} &=& -\frac{E_xE_y + H_xH_y}{4\pi} \\
T^{13} &=& -\frac{E_xE_z + H_xH_z}{4\pi} \\
T^{22} &=& \frac{-E_x^2 + E_y^2 - E_z^2 - H_x^2 + H_y^2 - H_z^2}{8\pi} \\
T^{23} &=& -\frac{E_yE_z + H_yH_z}{4\pi} \\
T^{33} &=& \frac{-E_x^2 - E_y^2 + E_z^2 - H_x^2 - H_y^2 + H_z^2}{8\pi}
\end{eqnarray*}
The remaining terms need not be computed because of the symmetry of $T^{\mu\nu}$.
From \eqref{c4e50}, we observe that $T^{00} = W$, the energy density and from
\eqref{c4e45}, $T^{0i} = S_i/c$. Further, if
\begin{equation}\label{c4e89}
\sigma_{ij} = \frac{1}{4\pi}\left(E_iE_j + H_iH_j - \frac{\delta_{ij}}{2}(E^2+H^2)\right).
\end{equation}
is the Maxwell stress tensor then
\begin{equation}\label{c4e90}
T^{\mu\nu} = \begin{bmatrix}
W & S_x/c & S_y/c & S_z/c \\
S_x/c & -\sigma_{xx} & -\sigma_{xy} & -\sigma_{xz} \\
S_y/c & -\sigma_{yx} & -\sigma_{yy} & -\sigma_{yz} \\
S_z/c & -\sigma_{zx} & -\sigma_{zy} & -\sigma_{zz}
\end{bmatrix}
\end{equation}

\item Analogously to \eqref{c4e20}, we define the mass density as
\begin{equation}\label{c4e91}
\rho_m = \sum_a m_a\delta(\vec{r} - \vec{r}_a)
\end{equation}
so that the 4-momentum density is $\rho_m c u^\mu$. This is also $T^{0\mu}/c$,
where $T^{\mu\nu}$ is the energy-momentum tensor of a system of particles. Thus,
\begin{equation}\label{c4e92}
T^{0\mu} = \rho_m c^2u^\mu.
\end{equation}
Now, $dx^\nu = (cdt, dx^1, dx^2, dx^3)$ so that
\begin{equation}\label{c4e93}
c = \td{x^0}{t}
\end{equation} 
and hence, from \eqref{c4e92}
\begin{equation}\label{c4e94}
T^{0\mu} = \rho_m c\td{x^\mu}{s}\td{x^0}{t}.
\end{equation}
We generalise this to
\[
T^{\mu\nu} =  \rho_m c\td{x^\mu}{s}\td{x^\nu}{t} = 
\rho_m c\td{x^\mu}{s}\td{x^\nu}{s}\td{s}{t}
\]
or
\begin{equation}\label{c4e95}
T^{\mu\nu} = \rho_m cu^\mu u^\nu \td{s}{t}.
\end{equation}
This tensor is symmetric and it is the energy-momentum tensor for non-interacting
particles.

\item What does energy/momentum conservation mean for the tensor of \eqref{c4e95}?
To understand that, we first write \eqref{c4e95} as
\begin{equation}\label{c4e96}
T^{\mu\nu} = \rho_m c u^\mu\td{x^\nu}{t}
\end{equation}
so that
\begin{eqnarray*}
\pdt{T^{\mu\nu}}{x^\nu} &=& cu^\mu\frac{\partial}{\partial x^\nu}\left(\rho_m\td{x^\nu}{t}\right)
+ \rho_m c\td{x^\nu}{t}\pdt{u^\mu}{x^\nu} \\
 &=& cu^\mu\frac{\partial}{\partial x^\nu}\left(\rho_m\td{x^\nu}{t}\right) + \rho_m c\td{u^\mu}{t}
\end{eqnarray*}
The term
\[
\frac{\partial}{\partial x^\nu}\left(\rho_m\td{x^\nu}{t}\right)
\]
must vanish if the masses are conserved. In this case, since we assumed to be
non-interacting, they are indeed conserved so that
\begin{equation}\label{c4e97}
\pdt{T^{\mu\nu}}{x^\nu} = \rho_m c\td{u^\mu}{t}
\end{equation}

\item We will repeat this exercise for the tensor of \eqref{c4e84}.
\[
T^{\rho\mu} = \frac{1}{4\pi}\left(-F^{\rho\alpha}\tensor{F}{^\mu_\alpha} +
\frac{1}{4}g^{\rho\mu}F_{\alpha\beta}F^{\alpha\beta}\right)
\]
which is same as
\[
\tensor{T}{^\rho_\mu} = \frac{1}{4\pi}\left(-F^{\rho\alpha}F_{\mu\alpha} +
\frac{1}{4}\tensor{g}{^\rho_\mu}F_{\alpha\beta}F^{\alpha\beta}\right).
\]
Therefore,
\[
4\pi\pdt{\tensor{T}{^\rho_\mu}}{x^\rho} = -\pdt{F^{\rho\alpha}}{x^\rho}F_{\mu\alpha}
- F^{\rho\alpha}\pdt{F_{\mu\alpha}}{x^\rho} + \frac{1}{2}F^{\alpha\beta}\pdt{F_{\alpha\beta}}{x^\rho}
\]
Using \eqref{c4e5} in the last term,
\[
4\pi\pdt{\tensor{T}{^\rho_\mu}}{x^\rho} = -\pdt{F^{\rho\alpha}}{x^\rho}F_{\mu\alpha}
- F^{\rho\alpha}\pdt{F_{\mu\alpha}}{x^\rho} - \frac{\tensor{g}{^\rho_\mu}}{2}
 F^{\alpha\beta}\left(\pdt{F_{\beta\rho}}{x^\alpha} + \pdt{F_{\rho\alpha}}{x^\beta}\right)
\]
Using \eqref{c4e37} in the first term,
\[
4\pi\pdt{\tensor{T}{^\rho_\mu}}{x^\rho} = -\frac{4\pi}{c}j^\alpha F_{\mu\alpha}
- F^{\rho\alpha}\pdt{F_{\mu\alpha}}{x^\rho} - \frac{\tensor{g}{^\rho_\mu}}{2}F^{\alpha\beta}
\left(\pdt{F_{\beta\rho}}{x^\alpha} + \pdt{F_{\rho\alpha}}{x^\beta}\right)
\]
or
\[
4\pi\pdt{\tensor{T}{^\rho_\mu}}{x^\rho} = -\frac{4\pi}{c}j^\alpha F_{\mu\alpha}
- F^{\rho\alpha}\pdt{F_{\mu\alpha}}{x^\rho} + \frac{\tensor{g}{^\rho_\mu}}{2}
\left(F^{\alpha\beta}\pdt{F_{\rho\beta}}{x^\alpha} + F^{\beta\alpha}\pdt{F_{\rho\alpha}}{x^\beta}\right).
\]
Interchange the bound indices $\alpha$ and $\beta$ in the last term,
\[
4\pi\pdt{\tensor{T}{^\rho_\mu}}{x^\rho} = -\frac{4\pi}{c}j^\alpha F_{\mu\alpha}
- F^{\rho\alpha}\pdt{F_{\mu\alpha}}{x^\rho} + \frac{\tensor{g}{^\rho_\mu}}{2}
\left(F^{\alpha\beta}\pdt{F_{\rho\beta}}{x^\alpha} + F^{\alpha\beta}\pdt{F_{\rho\beta}}{x^\alpha}\right).
\]
That is,
\[
4\pi\pdt{\tensor{T}{^\rho_\mu}}{x^\rho} = -\frac{4\pi}{c}j^\alpha F_{\mu\alpha}
- F^{\rho\alpha}\pdt{F_{\mu\alpha}}{x^\rho} + \tensor{g}{^\rho_\mu}
F^{\alpha\beta}\pdt{F_{\rho\beta}}{x^\alpha}
\]
In the second term on the rhs, rename $\rho \rightarrow \alpha, \alpha \rightarrow
\beta$ so that
\[
4\pi\pdt{\tensor{T}{^\rho_\mu}}{x^\rho} = -\frac{4\pi}{c}j^\alpha F_{\mu\alpha}
- F^{\alpha\beta}\pdt{F_{\mu\beta}}{x^\alpha} + \tensor{g}{^\rho_\mu}
F^{\alpha\beta}\pdt{F_{\rho\beta}}{x^\alpha}
\]
Since $\tensor{g}{^\rho_\mu} F_{\rho\beta} = F_{\mu\beta}$,
\[
4\pi\pdt{\tensor{T}{^\rho_\mu}}{x^\rho} = -\frac{4\pi}{c}j^\alpha F_{\mu\alpha}
- F^{\alpha\beta}\pdt{F_{\mu\beta}}{x^\alpha} + F^{\alpha\beta}\pdt{F_{\mu\beta}}{x^\alpha}
\]
or
\begin{equation}\label{c4e98}
\pdt{\tensor{T}{^\rho_\mu}}{x^\rho} = -\frac{1}{c}j^\alpha F_{\mu\alpha}
\end{equation}

\item We can generalise the equation of motion \eqref{c3e74} for continua as
\begin{equation}\label{c4e99}
\rho_m c\td{u^\mu}{s} = \frac{\rho}{c}F^{\mu\nu}u_\nu
\end{equation}
that is,
\[
\rho_m c\td{u^\mu}{t}\td{s}{t} = \frac{\rho}{c}F^{\mu\nu}u_\nu.
\]
Combining this with \eqref{c4e97}, we get
\begin{equation}\label{c4e100}
\pdt{T^{(m)\mu\nu}}{x^\nu}\td{s}{t} = \frac{\rho}{c}F^{\mu\nu}u_\nu,
\end{equation}
where we added the superscript $(m)$ to differentiate the energy momentum tensor
for a system of masses from that of the electromagnetic field. Recall that
\[
u^\nu = \td{x^\nu}{s} = \td{x^\nu}{t}\td{s}{t},
\]
so that \eqref{c4e100} becomes
\begin{equation}\label{c4e101}
\pdt{T^{(m)\mu\nu}}{x^\nu} = \frac{\rho}{c}F^{\mu\nu}\td{x_\nu}{t}.
\end{equation}
Using \eqref{c4e23},
\begin{equation}\label{c4e102}
\pdt{T^{(m)\mu\nu}}{x^\nu} = \frac{1}{c}F^{\mu\nu}j_\nu.
\end{equation}
An equivalent form of this equation is
\begin{equation}\label{c4e103}
\pdt{\tensor{T}{^{(m)\rho}_\mu}}{x^\rho} = \frac{1}{c}F_{\mu\alpha}j^\alpha.
\end{equation}
Adding equation \eqref{c4e98} and \eqref{c4e103} gives
\begin{equation}\label{c4e104}
\frac{\partial}{\partial x^\mu}\left(\tensor{T}{^{(f)\rho}_\mu} + \tensor{T}{^{(m)\rho}_\mu}\right) = 0.
\end{equation}
Thus the combined energy momentum tensor 
\begin{equation}\label{c4e105}
\tensor{T}{^\rho_\mu} = \tensor{T}{^{(f)\rho}_\mu} + \tensor{T}{^{(m)\rho}_\mu}
\end{equation}
satisfies the conservation equation \eqref{c4e61}. We used the superscript $(f)$
to identify the electromagnetic contribution of the tensor.

\item The trace of the energy momentum tensor for a system of charged particles
in an electromagnetic field is
\begin{equation}\label{c4e106}
\tensor{T}{^\mu_\mu} = \tensor{T}{^{(m)\mu}_\mu},
\end{equation}
because the electromagnetic field's energy momentum tensor is anti-symmetric and
hence traceless. From \eqref{c4e95},
\begin{equation}\label{c4e107}
\tensor{T}{^\mu_\mu} = \rho_m cu^\mu u_\mu \td{s}{t} = \rho_m c^2\sqrt{1 - \beta^2}.
\end{equation}
If, instead of a continuous mass distribution, we have discrete particles,
\begin{equation}\label{c4e108}
\tensor{T}{^{\mu}_\mu} = \sum_a m_ac^2\sqrt{1 - \beta_a^2}\delta(\vec{r} - \vec{r}_a).
\end{equation}
From this equation, it is evident that
\begin{equation}\label{c4e109}
\tensor{T}{^\mu_\mu} \ge 0.
\end{equation}

\item From equations \eqref{c4e104} and \eqref{c4e105}, we have
\begin{equation}\label{c4e110}
\pdt{T^{\mu\nu}}{x^\mu} = 0.
\end{equation}
For the space components alone, this means
\[
\pdt{T^{\mu i}}{x^\mu} = 0,
\]
or
\begin{equation}\label{c4e111}
\frac{1}{c}\pdt{T^{0i}}{t} + \pdt{T^{ji}}{x^j} = 0.
\end{equation}
Now let's assume that this system of particles always has a small range of
positions and momenta so that all components of $T^{\mu\nu}$ vary in a limited
range. If we define the average of a function $f$ as
\begin{equation}\label{c4e112}
\langle f \rangle = \frac{1}{T}\int_0^T fdt
\end{equation}
then
\begin{equation}\label{c4e113}
\left\langle\td{f}{t}\right\rangle = \frac{1}{T} \int_0^T \td{f}{t}dt = \frac{f(T) - f(0)}{T}.
\end{equation}
In the limit $T \rightarrow \infty$, since the function $f$ takes finite
values at all times,
\begin{equation}\label{c4e114}
\left\langle\td{f}{t}\right\rangle = 0.
\end{equation}
If we take the time-average of \eqref{c4e111}, in view of \eqref{c4e114}, we get
\[
\pdt{\langle T\rangle^{ji}}{x^j} = 0
\]
or equivalently,
\begin{equation}\label{c4e115}
\pdt{\tensor{\langle T \rangle}{_i^j}}{x^j} = 0.
\end{equation}
From this equation, we also get
\[
x^i\pdt{\tensor{\langle T \rangle}{_i^j}}{x^j} = 0.
\]
Integrating this equation over all space,
\[
\int x^i\pdt{\tensor{\langle T \rangle}{_i^j}}{x^j} dV = 0 \Rightarrow 
\int \left(\frac{\partial}{\partial x^j}(x^i\tensor{\langle T \rangle}{_i^j}) - 
\pdt{x^i}{x^j}\tensor{\langle T \rangle}{_i^j}\right)dV = 0
\]
The first term can be converted into a surface integral, so that
\[
\oint x^i\tensor{\langle T \rangle}{_i^j} df_j - 
\int \tensor{\delta}{^i_j}\tensor{\langle T \rangle}{_i^j} dV = 0.
\]
Since we assumed that the system's energy, momenta and positions vary over a 
limited range, if we extend the surface integral beyond it then the first term
vanishes and we are left with
\begin{equation}\label{c4e116}
\int \tensor{\langle T \rangle}{_i^i} dV = 0.
\end{equation}
From this equation we readily conclude that
\begin{equation}\label{c4e117}
\int \tensor{\langle T \rangle}{_\mu^\mu} dV = \int\tensor{\langle T \rangle}{_0^0} dV + \int\tensor{\bar{T}}{_i^i}dV =
\int WdV = \mathcal{E}.
\end{equation}
From \eqref{c4e108}, we have
\[
\tensor{\langle T \rangle}{^{\mu}_\mu} = \sum_a m_ac^2\langle\sqrt{1 - \beta_a^2}\rangle\delta(\vec{r} - \vec{r}_a),
\]
so that substituting it on the extreme lhs of \eqref{c4e117} gives,
\[
\int \sum_a m_ac^2\langle\sqrt{1 - \beta_a^2}\rangle\delta(\vec{r} - \vec{r}_a) dV = \mathcal{E}.
\]
or
\begin{equation}\label{c4e118}
\sum_a m_ac^2\langle(1 - \beta_a^2)^{1/2}\rangle = \mathcal{E}.
\end{equation}
This is the relativistic generalisation of the classical virial theorem.

\item The calculations in the previous point should be corrected by removing the
``self-energy'' terms. These are the terms arising from field contributions of the
masses and are
\[
\sum_a \int\frac{E_a^2 + H_a^2}{8\pi}dV.
\]

\item We will not examine how the energy momentum tensor looks for a continuum.
Referring to the form in \eqref{c4e90}, in the frame of reference in which the
continuum is at rest,
\begin{equation}\label{c4e119}
\sigma_{ij} = -p\delta_{ij},
\end{equation}
where $p$ is the pressure on the body. This equation is always true for fluids
but is approximately true for solids. The energy momentum tensor then becomes
\begin{equation}\label{c4e120}
T^{\mu\nu} = \begin{bmatrix}\mathcal{E} & 0 & 0 & 0 \\
0 & p & 0 & 0 \\
0 & 0 & p & 0 \\
0 & 0 & 0 & p
\end{bmatrix}
\end{equation}
In the rest frame of reference, the 4-velocity of a particle is $u^\mu = (1, 0, 0, 0)$
so that an expression of the form
\begin{equation}\label{c4e121}
T^{\mu\nu} = (p + \mathcal{E})u^\mu u^\nu - p\delta^{\mu\nu}
\end{equation} 
reduces to \eqref{c4e120} in the rest frame. In mixed components, this equation
becomes
\begin{equation}\label{c4e122}
\tensor{T}{_\mu^\nu} = (p + \mathcal{E})u_\mu u^\nu - p\tensor{\delta}{_\mu^\nu}.
\end{equation} 

Now 
\[
u^\mu = \gamma(1, \vec{v})
\]
so that
\[
W = T^{00} = (p + \mathcal{E})u^0 u^0 - p = \gamma^2(p + \mathcal{E}) - p
= \frac{p + \mathcal{E}}{1 - \beta^2} - p
\]
or
\begin{equation}\label{c4e123}
W = \frac{\mathcal{E} + p\beta^2}{1 - \beta^2} = \gamma(\mathcal{E} + p\beta^2).
\end{equation}
Similarly,
\[
\frac{\vec{S}}{c} = \frac{(p + \mathcal{E})\bm{\beta}}{1 - \beta^2}
\]
so that
\begin{equation}\label{c4e124}
\vec{S} = \frac{(p + \mathcal{E})\vec{v}}{1 - \beta^2}.
\end{equation}
The components of the stress tensor are
\begin{equation}\label{c4e125}
\sigma_{ij} = -\gamma(p + \mathcal{E})\beta_i\beta_j - p\delta_{ij}.
\end{equation}
If the velocity of the continuum is small, \eqref{c4e124} can be approximated by
\begin{equation}
\vec{S} = (p + \mathcal{E})\vec{v}.
\end{equation}

\item From equation \eqref{c4e122},
\[
\tensor{T}{_\mu^\mu} = (p + \mathcal{E})u_\mu u^\mu - 4p
\]
From equation \eqref{c1e84}, $u_\mu u^\mu = 1$ and since $\tensor{\delta}{_\mu^\mu} 
= 4$, we get
\begin{equation}\label{c4e126}
\tensor{T}{_\mu^\mu} = \mathcal{E} - 3p.
\end{equation}
The inequality \eqref{c4e109} gives
\begin{equation}\label{c4e127}
p \le \frac{\mathcal{E}}{3}.
\end{equation}
Furtherm from \eqref{c4e108} and \eqref{c4e126} we have
\begin{equation}\label{c4e128}
\mathcal{E} - p = \sum_a m_ac^2\sqrt{1 - \beta_a^2}\delta(\vec{r} - \vec{r}_a).
\end{equation}
In the ultra-relativistic case, $\beta_a \rightarrow 1$ and we get the familiar
expression for the relation between energy density and pressure of the radiation
\begin{equation}\label{c4e129}
p = \frac{\mathcal{E}}{3}.
\end{equation}

\end{enumerate}


\chapter{Constant electromagnetic fields}\label{c5}
\begin{enumerate}
\item If the fields are independent of time then they obey
\begin{eqnarray}
\dive\vec{E} &=& 4\pi\rho \label{c5e1} \\
\curl\vec{E} &=& 0 \label{c5e2} \\
\dive\vec{H} &=& 0 \label{c5e3} \\
\curl\vec{H} &=& 4\pi\frac{\vec{J}}{c}. \label{c5e4}
\end{eqnarray}
The symmetry of the situation reveals itself if we note the correspondence $\vec{E}
\mapsto \vec{H}$ and $\rho \mapsto \vec{J}/c$ and $\dive \mapsto \curl$. Equations
\eqref{c5e2} and \eqref{c5e3} suggest that
\begin{eqnarray}
\vec{E} &=& -\grad\phi \label{c5e5} \\
\vec{H} &=& \curl\vec{A} \label{c5e6}
\end{eqnarray}
which is understandable if we note that only the time component of $J^\mu$ 
determines $\vec{E}$ and its space component determines $\vec{H}$. Therefore,
only the time component of $\vec{A}^\mu$ suffices to describe the electric
field and its space component suffices to describe the magnetic field.

\item From equations \eqref{c5e1} and \eqref{c5e5} we have
\begin{equation}\label{c5e7}
\nabla^2\phi = -4\pi\rho.
\end{equation}
In region where there are no charges, the potential satisfies Laplace's equation
\begin{equation}\label{c5e8}
\nabla^2\phi = 0.
\end{equation}
This equation immediately implies that the potential function cannot have a local
minimum or a maximum. The nature of extremum of a function of three variables is
determined by its Hessian matrix. It is a minimum (maximum) if the Hessian matrix
is positive (negative) definite. Since the trace of a matrix is the sum of its
eigenvalues, we have an equivalent condition that the trace of the Hessian matrix
is positive (negative) for an extremum to be a minimum (maximum). But the trace
of Hessian is $\nabla^2\phi$ evaluated at the extremum. If $\phi$ is a solution
of Laplace's equation \eqref{c5e8}, then the Hessian can never be positive or
negative definite.

\item The field of a point charge can be derived using Gauss' law. Let the point
charge be at $\vec{r}^\op$. In order to find the field at $\vec{r}$, we consider
a sphere centred at $\vec{r}^\op$ and passing through $\vec{r}$. As the space is
isotropic, we expect the magnitude of $\vec{E}$ to be the same at all points on the
sphere and its direction along $\vec{r} - \vec{r}^\op$. Therefore,
\[
4\pi |\vec{r} - \vec{r}^\op|^2 = q
\]
or
\begin{equation}\label{c5e9}
\vec{E} = \frac{q}{|\vec{r} - \vec{r}^\op|^3}(\vec{r} - \vec{r}^\op).
\end{equation}
If we define
\begin{equation}\label{c5e10}
\vec{R} = \vec{r} - \vec{r}^\op
\end{equation}
then we can write \eqref{c5e9} simply as
\begin{equation}\label{c5e11}
\vec{E} = \frac{q}{R^3}\vec{R} = \frac{q}{R^2}\uv{R}.
\end{equation}
Equation \eqref{c5e11} is \emph{Coulomb's law}. The potential of this field is
\begin{equation}\label{c5e12}
\phi = \frac{q}{R}.
\end{equation}
If there are discrete, point charges at $\vec{r}_a$ then the field due to all of 
them is
\begin{equation}\label{c5e13}
\vec{E} = \sum_a \frac{q_a}{R_a^2}\uv{R_a},
\end{equation}
where $\vec{R}_a = \vec{r}^\op - \vec{r}$. The potential is
\begin{equation}\label{c5e14}
\phi = \sum_a\frac{q_a}{R_a}.
\end{equation}
For a continuum of charges, \eqref{c5e11} and \eqref{c5e12} generalise to
\begin{eqnarray}
\phi(\vec{r}) &=& \int\frac{\rho(\vec{r}^\op)}{R}dV^\op \label{c5e15} \\
\vec{E}(\vec{r}) &=& \int\frac{\rho(\vec{r}^\op)}{R^2}\uv{R}dV^\op  \label{c5e16}
\end{eqnarray}

\item For a point charge, $\rho = q\delta(\vec{R})$. In this case, \eqref{c5e7}
becomes,
\[
\nabla^2\phi = 4\pi q\delta(\vec{R}).
\]
From \eqref{c5e12}, we also have
\begin{equation}\label{c5e17}
\nabla^2\left(\frac{1}{R}\right) = -4\pi\delta(\vec{R}).
\end{equation}

\item In the absence of magnetic field, the energy density is
\begin{equation}\label{c5e18}
W = \frac{1}{8\pi}\int E^2dV.
\end{equation}
From \eqref{c5e5},
\[
W = -\frac{1}{8\pi}\int\vec{E}\cdot\grad\phi dV.
\]
Since $\dive(\phi\vec{E}) = \grad\phi\cdot\vec{E} + \phi\dive\vec{E}$, we have
\begin{eqnarray*}
W &=& \frac{1}{8\pi}\int\phi\dive\vec{E}dV - \frac{1}{8\pi}\int\dive(\phi\vec{E})dV \\
  &=& \frac{1}{8\pi}\int\phi\dive\vec{E}dV - \frac{1}{8\pi}\oint \phi\vec{E}\cdot d\vec{f}.
\end{eqnarray*}
The potential goes as $O(r^{-1})$ and the field as $O(r^{-2})$ so that the surface
integral goes as $O(r^{-1})$. If we choose a large enough surface the second 
integral can be made as small as we desire. Thus, we have
\[
W = \frac{1}{8\pi}\int\phi\dive\vec{E}dV
\]
or using \eqref{c5e1},
\begin{equation}\label{c5e19}
W = \frac{1}{2}\int\rho\phi dV.
\end{equation}
If the charge density consists od point charges, equation \eqref{c5e19} becomes
\begin{equation}\label{c5e20}
W = \sum_a q_a\phi(\vec{r}_a),
\end{equation}
where $\phi(\vec{r}_a)$ is the potential due to all charges at the point 
$\vec{r}_a$ where the charge $q_a$ is located. This formula immediately runs 
into a tricky situation.

\item The potential $\phi$ due to a charge $q$ at $\vec{r}^\op$ at the same point
is infinity according to \eqref{c5e12}. As a result, from \eqref{c5e20}, a point
charge has infinite energy and therefore an infinite mass. To avoid such an absurd
result, we say that the laws of classical electrodynamics are not applicable at
extremely short distances. In fact, we cannot even ask the question if the mass of
a charge has electrodynamic origin, that is, it exists because of electrodynamic
energy.

The energy of a uniformly charged sphere of radius $s$ can be computed as follow.
We first note that its charge density is
\begin{equation}\label{c5e21}
\rho = \frac{Q}{\frac{4\pi}{3}s^3}.
\end{equation}
The field due to a sphere of radius $r$ and a uniform charge density $\rho$ at 
points on and outside it is as if the entire charge was concentrated at its 
centre. If $q(r)$ is the charge in a sphere of radius $r$ then the energy
of a charge $dq$ at a distance $r$ from it is
\[
dU = \frac{qdq}{r},
\]
where 
\[
q = \frac{4\pi}{3}r^3 \rho
\]
and $dq = 4\pi r^2 dr$. Thus,
\[
dU = \frac{16\pi^2}{3}\rho^2 r^4dr
\]
and the energy of the entire sphere is
\begin{equation}\label{c5e22}
U = \int_0^s dU = \frac{3}{5}\frac{Q^2}{s}.
\end{equation}
If we consider an elementary charge $q$ to have an infinitesimal radius $r_0$ and
mass $m$ then if the mass has electrodynamic origin,
\[
mc^2 = \frac{3}{5}\frac{q^2}{s}.
\]
Ignoring the factor of $3/5$, the ``classical radius'' of the charge $q$ is defined
to be
\begin{equation}\label{c5e23}
s = \frac{q}{m} c^2.
\end{equation}

\item Quantum effects become important at distance of the order of $\hslash/mc$. The
ratio of this distance to the classical radius of an electron is
\[
\frac{\hslash}{mc}\frac{mc^2}{e^2} = \frac{\hslash c}{e^2} = \frac{1}{\alpha}
\approx 137,
\]
where $\alpha$ is the fine-structure constant. Thus, quantum effects must be taken
into account at distances much larger than the ``classical radius'' of the electron. 
The lower limit of the applicability of classical electrodynamics is much larger 
than the classical electron radius.

\item To avoid the tricky questions of self-energy of elementary charges, we write
the potential $\phi(\vec{r}_a)$ in equation \eqref{c5e20} as
\begin{equation}\label{c5e24}
\phi(\vec{r}_a) = \sum_{b \ne a} \frac{q_b}{R_{ab}},
\end{equation}
where $R_{ab} = |\vec{r}_a - \vec{r}_b|$.

\item We next consider the field of a charge $q$ moving with a uniform velocity
$\vec{v}$. We align the axes such that $\vec{v} = v\uv{x}$ and call the frame
moving with the charge $K^\op$. Let $K$ be the laboratory frame in with axes 
parallel to those of $K^\op$ and let their origins coincide at $t = 0$. Let $P$
be a field point with coordinates $(x^\op, y^\op, z^\op)$ in $K^\op$. Then the
scalar potential at $P$ in $K^\op$ is
\begin{equation}\label{c5e25}
\phi^\op = \frac{q}{R^\op},
\end{equation}
where 
\begin{equation}\label{c5e26}
R^\op = \sqrt{{x^\op}^2 + {y^\op}^2 + {z^\op}^2}
\end{equation}
and the vector potential is
\begin{equation}\label{c5e27}
\vec{A}^\op = 0.
\end{equation}
Thus,
\begin{equation}\label{c5e28}
{A^\op}^\mu = \left(\frac{q}{R^\op}, 0, 0, 0\right).
\end{equation}
From \eqref{c1e46}, the 4-potential in $K$ is
\begin{equation}\label{c5e29}
A^\mu = \left(\gamma\frac{q}{R^\op}, \beta\gamma\frac{q}{R^\op}, 0, 0\right),
\end{equation}
where
\begin{eqnarray*}
\vec{\beta} &=& \frac{\vec{v}}{c} \\
\gamma &=& \frac{1}{\sqrt{1 - \beta^2}}.
\end{eqnarray*}
However, \eqref{c5e29} is not a correct expression because we continue to have
$R^\op$ in it. The relation between $(x, y, z)$ and $(x^\op, y^\op, z^\op)$ is
given the Lorentz transformation
\[
x^\op = \gamma(x - vt) \;;\; y^\op = y \;;\; z^\op = z
\]
so that
\begin{equation}\label{c5e30}
{R^\op}^2 = \gamma^2(x - vt)^2 + (y^2 + z^2).
\end{equation}
We define a function $R^\ast$ of the coordinates $x, y, z$ as
\begin{equation}\label{c5e31}
R^\ast(x, y, z) = \sqrt{(x - vt)^2 + (1 - \beta^2)(y^2 + z^2)}
\end{equation}
so that
\begin{equation}\label{c5e32}
\frac{R^\op(x^\op, y^\op, z^\op)}{\gamma} = R^\ast(x, y, z)
\end{equation}
and the partial Lorentz transformation of \eqref{c5e29} can be completed to
\begin{equation}\label{c5e33}
A^\mu = \left(\frac{q}{R^\ast}, \beta\frac{q}{R^\ast}, 0, 0\right)
\end{equation}
or
\begin{eqnarray}
\phi(x, y, z) &=& \frac{q}{R^\ast} \label{c5e34} \\
\vec{A}(x, y, z) &=& \frac{q\vec{v}}{cR^\ast} \label{c5e35}
\end{eqnarray}

\item The electric and magnetic fields in $K^\op$ frame are
\begin{eqnarray}
\vec{E}^\op &=& \frac{q}{{R^\op}^3}\vec{R}^\op \label{c5e36} \\
\vec{H}^\op &=& 0 \label{c5e37}
\end{eqnarray}
We use equations \eqref{c3e81} to \eqref{c3e83} to get the electric field in the
$K$ frame.
\begin{eqnarray}
E_x &=& E_x^\op \label{c5e38} \\
E_y &=& \gamma E_y^\op \label{c5e39} \\
E_z &=& \gamma E_z^\op \label{c5e40}
\end{eqnarray}
Using \eqref{c5e36} we get
\begin{eqnarray}
E_x &=& \frac{q}{{R^\op}^3}x^\op \label{c5e41} \\
E_y &=& \gamma\frac{q}{{R^\op}^3}y^\op \label{c5e42} \\
E_z &=& \gamma\frac{q}{{R^\op}^3}z^\op \label{c5e43}
\end{eqnarray}
These equations are not satisfactory because their rhs are written in terms of
the primed coordinates. Using $R^\op = \gamma R^\ast$ from \eqref{c5e32} and the
Lorentz transformation formulae
\begin{eqnarray}
E_x &=& \frac{q}{{R^\ast}^3}\frac{x - vt}{\gamma^2} \label{c5e44} \\
E_y &=& \frac{q}{{R^\ast}^3}\frac{y}{\gamma^2} \label{c5e45} \\
E_z &=& \frac{q}{{R^\ast}^3}\frac{z}{\gamma^2} \label{c5e46}
\end{eqnarray}
so that
\begin{equation}\label{c5e47}
\vec{E} = (1 - \beta^2)\frac{q}{{R^\ast}^3}\vec{R},
\end{equation}
where
\begin{equation}\label{c5e48}
\vec{R} = (x - vt)\uv{x} + y\uv{y} + z\uv{z}.
\end{equation}
If $\theta$ is the angle between $\vec{R}$ and $\vec{v}$ then
\begin{equation}\label{c5e49}
\vec{v}\cdot\vec{R} = v(x - vt) = vR\cos\theta \Rightarrow x - vt = R\cos\theta
\end{equation}
so that,
\begin{equation}\label{c5e50}
y^2 + z^2 = R^2 - (x - vt)^2 = R^2\sin^2\theta.
\end{equation}
From \eqref{c5e31}, \eqref{c5e49} and \eqref{c5e50} we get
\begin{equation}\label{c5e51}
{R^\ast}^2 = R^2\cos^2\theta + (1 - \beta^2)R^2\sin^2\theta = 
R^2(1 - \beta^2\sin^2\theta).
\end{equation}
We can now write \eqref{c5e47} solely in terms of $\vec{R}$ as
\begin{equation}\label{c5e52}
\vec{E} = \frac{q\vec{R}}{R^3} \frac{1 - \beta^2}{(1 - \beta^2\sin^2\theta)^{3/2}}.
\end{equation}
Note that $\vec{R}$ is the position of the (moving) charge in $K$ frame.

\item Lorentz transformation of $\vec{E}$ and $\vec{H}$ are a consequence of the
transformation properties of the field tensor. One should use \eqref{c3e81} to
\eqref{c3e84} instead of evaluating $-\grad\phi$ with $\phi$ given by \eqref{c5e34}.
That is, although
\[
\phi^\op = \frac{q}{R^\op} \mapsto \phi = \frac{q}{R^\ast},
\]
the mapping
\[
-\grad^\op\phi^\op \mapsto -\grad\phi
\]
does not hold.

\item In \eqref{c5e52}, $\theta$ is the angle between the radius vector of the
field point $\vec{R}$ and the direction of motion. The magnitude of the field
is
\begin{equation}\label{c5e53}
E = \frac{q}{R^2}\frac{1 - \beta^2}{(1 - \beta^2\sin^2\theta)^{3/2}}
\end{equation}
Figure \ref{c5f1} shows how the magnitude of the electric field varies with 
$\theta$ for different values of $\beta$. As the speed of the charge approaches
$c$, the electric field gets increasingly concentrated along the directions
perpendicular to the motion of the particle.
\begin{figure}[!ht]
\includegraphics[scale=0.8]{ex2}
\caption{$E(\theta)$}
\label{c5f1}
\end{figure}
For an observer in the $K$ frame, the charge in motion creates an electric current
and therefore expects to observe magnetic field. In fact, equations \eqref{c3e84}
to \eqref{c3e86} give
\begin{eqnarray}
H_x &=& 0 \label{c5e54} \\
H_y &=& -\gamma\beta E^\op_z \label{c5e55} \\
H_z &=& \gamma\beta E^\op_y \label{c5e56}
\end{eqnarray}
Since $\beta = v/c = v_x/c$, using \eqref{c5e39} and \eqref{c5e40}, we get
\begin{eqnarray}
H_x &=& 0 \label{c5e57} \\
H_y &=& -\frac{1}{c}v_xE_z \label{c5e58} \\
H_z &=& \frac{1}{c}v_xE_y. \label{c5e59}
\end{eqnarray}
These three equations can be combined as
\begin{equation}\label{c5e60}
\vec{H} = \frac{1}{c}\vec{v} \times \vec{E} = \vec{\beta} \times \vec{E}.
\end{equation}

\item We now consider the motion of a charge $q$ in the electric field produced
by another one $Q$. Let their masses by $m$ and $M$ respectively. Assume that
$m \ll M$ so that the charge $Q$ is almost stationary. The problem now reduces to
studying the motion of $q$ in a potential $\phi = Q/r$.

The energy of the charged particle is
\begin{equation}\label{c5e61}
\mathcal{E} = \sqrt{p^2c^2 + m^2c^4} + \frac{\alpha}{r},
\end{equation}
where
\begin{equation}\label{c5e62}
\alpha = qQ.
\end{equation}
The motion in a central field is confined to a plane. Align the coordinate axes
such that $Q$ is at the origin and the motion happens in $xy$ plane. Furthermore,
since the torque on the particle is zero, its angular momentum is a constant of
motion. The particle's velocity in polar coordinates is
\begin{equation}\label{c5e63}
\vec{v} = \dot{r}\uv{r} + r\dot{\theta}\uv{\theta}
\end{equation}
so that $v^2 = \dot{r}^2 + r^2\dot{\theta}^2$. Since $\vec{L} = \vec{r} \times
\vec{p} = m\gamma\vec{r} \times \vec{p}$, (using \eqref{c2e8}) we have $L = 
\gamma mr^2\dot{\theta}$. We can express $v^2$ in terms of $L$ as
\[
v^2 = \dot{r}^2 + \frac{L^2}{m^2\gamma^2 r^2}
\]
so that
\begin{equation}\label{c5e64}
p^2 = m^2\gamma^2\dot{r}^2 + \frac{L^2}{r^2} = p_r^2 + \frac{L^2}{r^2}.
\end{equation}
Therefore, \eqref{c5e61} becomes
\begin{equation}\label{c5e65}
\mathcal{E} = c\sqrt{p_r^2 + \frac{L^2}{r^2} + m^2c^2} + \frac{\alpha}{r}.
\end{equation}

\item If $\alpha > 0$ then as $r$ increases, the rhs of \eqref{c5e64} also
increases even if $p_r \rightarrow 0$. Since $\mathcal{E}$ is a constant of
motion, a drop in $r$ can be compensated by a drop in $p_r$ only to a limited
extent. This prevents the two charges from coming arbitrarily close to each 
other.

To analyse the condition $\alpha < 0$, write \eqref{c5e65} as
\[
\mathcal{E} = \frac{Lc}{r}\sqrt{\frac{r^2(p_r^2 + m^2c^2)}{L^2} + 1} - \frac{|\alpha|}{r}.
\]
so that we can approximate
\[
\mathcal{E} \approx \frac{Lc}{r} + \frac{cr}{2L}(p_r^2 + m^2c^2) - \frac{|\alpha|}{r}.
\]
for small $r$. If $Lc > |\alpha|$ then the rhs blows up as $r \rightarrow 0$. 
Therefore, to enforce constancy of $\mathcal{E}$, $r$ cannot be permitted to
decrease indefinitely. If $Lc < |\alpha|$, if $p_r = m\gamma\dot{r}$ can become
indefinitely large, $r$ may be allowed to become arbitrarily small while still
keeping $\mathcal{E}$ constant.

In the non-relativistic case, we would have had
\[
\mathcal{E} = \frac{1}{2}mv_r^2 + \frac{L^2}{2mr^2} - \frac{|\alpha|}{r}.
\]
If $L \ne 0$ then the first two terms, which are always positive, rise much faster
than the third term falls. Therefore, energy conservation does not allow the two
charges to come arbitrarily close to each other. On the other hand, if $L = 0$, 
then $v_r$ can rise enough to compensate the drop in the third term and still
keep  $\mathcal{E}$ constant. Thus, the two charges can come arbitrarily close only
if they approach to each other head-on.

\item The Hamilton-Jacobi equation \eqref{c3e15} for this problem is
\[
(\grad S)^2 - \frac{1}{c^2}\left(\pdt{S}{t} + \frac{\alpha}{r}\right)^2 + m^2c^2 = 0.
\]
Since $Q$ is stationary, $q$ experiences only the electric field and therefore 
$\vec{A} = 0$. The gradient in plane-polar coordinates is
\begin{equation}\label{c5e66}
\grad S = \pdt{S}{r}\uv{r} + \frac{1}{r}\pdt{S}{\phi}
\end{equation}
so that the Hamilton-Jacobi equation becomes
\begin{equation}\label{c5e67}
\left(\pdt{S}{r}\right)^2 + \frac{1}{r^2}\left(\pdt{S}{\phi}\right)^2 - \frac{1}{c^2}
\left(\pdt{S}{t} + \frac{\alpha}{r}\right)^2 + m^2c^2 = 0.
\end{equation}
or
\[
\left(\pdt{S}{r}\right)^2 + \frac{1}{r^2}\left(\pdt{S}{\phi}\right)^2 - \frac{1}{c^2}
\left(\pdt{S}{t}\right)^2 + 2\frac{\alpha}{rc^2}\pdt{S}{t} + \frac{\alpha^2}{c^2r^2}
+ m^2c^2 = 0.
\]
This is a non-linear, first-order pde. We try a solution of the form
\begin{equation}\label{c5e68}
S(r, \phi, t) = -\mathcal{E}t + L\phi + f(r)
\end{equation}
to get
\[
{f^\op(r)}^2 + \frac{L^2}{r^2} - \frac{\mathcal{E}^2}{c^2} - 2\frac{\alpha\mathcal{E}}{rc^2}
+ \frac{\alpha^2}{c^2r^2} + m^2c^2 = 0.
\]
or
\[
f^\op = \pm\sqrt{\frac{1}{c^2}\left(\mathcal{E} - \frac{\alpha}{r}\right)^2 - \frac{L^2}{r^2} 
-m^2c^2}
\]
The solution, thus, is
\begin{equation}\label{c5e69}
S(r, \phi, t) = -\mathcal{E}t + L\phi \pm 
\frac{1}{c}\int\sqrt{\left(\mathcal{E}-\frac{\alpha}{r}\right)^2-\frac{L^2c^2}{r^2} -m^2c^4}dr
\end{equation}
The trajectories are given by (find out why)
\begin{equation}\label{c5e70}
\pdt{S}{L} = \text{constant.}
\end{equation}
Differentiating \eqref{c5e69},
\begin{equation}\label{c5e71}
\pdt{S}{L} = \phi \mp Lc\int\frac{dr}{r\sqrt{A^2r^2 - 2Br + C^2}},
\end{equation}
where
\begin{eqnarray}
A^2 &=& \mathcal{E}^2 - m^2c^4 \label{c5e72} \\
B   &=& \alpha\mathcal{E} \label{c5e73} \\
C^2 &=& \alpha^2 - L^2c^2 \label{c5e74}
\end{eqnarray}
We now use the result
\begin{equation}\label{c5e75}
\int\frac{dx}{x\sqrt{a^2x^2 - 2bx + c^2}} = 
-\frac{1}{c}\tanh^{-1}\left(\frac{c^2 - bx}{c\sqrt{a^2x^2 - 2bx + c^2}}\right) +
\text{const}.
\end{equation}
in equation \eqref{c5e71} to get
\begin{equation}\label{c5e76}
\phi_0 = \phi \pm \frac{Lc}{C}\tanh^{-1}\left(\frac{C^2 - Br}{C\sqrt{A^2r^2 - 2Br + C^2}}\right),
\end{equation}
where $\phi_0$ is a constant of integration.

We consider the following cases:
\begin{enumerate}
\item $\alpha^2 > L^2c^2$, that is $C$ is real. Then \eqref{c5e71} can be written as
\[
\tanh\left(\frac{C}{Lc}(\phi_0 - \phi)\right) = \frac{C^2 - Br}{C\sqrt{A^2r^2 - 2Br + C^2}}.
\]
Let
\begin{equation}\label{c5e77}
X = \frac{C}{Lc}(\phi_0 - \phi) = \sqrt{\frac{\alpha^2}{L^2c^2} - 1}(\phi_0 - \phi)
\end{equation}
so that
\begin{eqnarray*}
\tanh^2 X &=& \frac{(C^2 - Br)^2}{C^2(A^2r^2 - 2Br + C^2)} \\
C^2(A^2r^2 + C^2 - 2Br)\tanh^2X &=& C^4 - 2BC^2r + B^2r^2 \\
C^2A^2r^2\tanh^2X &=& (C^4 - 2BC^2r)\sech^2X + B^2r^2 \\
C^2A^2r^2\sinh^2X &=& C^2(C^2 - 2B^2r) + B^2r^2\cosh^2X \\
C^2A^2r^2(\cosh^2X - 1) &=& C^2(C^2 - 2B^2r) + B^2r^2\cosh^2X \\
(C^2A^2 - B^2)r^2\cosh^2X &=& C^2(A^2r^2 - 2B^2r + C^2)
\end{eqnarray*}
Now,
\[
A^2C^2-B^2 = (L^2c^2 -\alpha^2)m^2c^4 -\mathcal{E}^2L^2c^2
\]
Since we assumes $\alpha^2 > L^2c^2$, the rhs of the above equation is negative.
Therefore, before taking the square-root, we flip the signs to get
\begin{eqnarray*}
(B^2 - C^2A^2)r^2\cosh^2X &=& C^2(2B^2r - A^2r^2 - C^2) \\
\sqrt{B^2 - C^2A^2}\cosh X &=& \frac{C^2}{r}\sqrt{2\frac{B^2}{C^2}r - \frac{A^2}{C^2}r^2 - 1}
\end{eqnarray*}
Thus,
\[
\pm c\sqrt{(L\mathcal{E})^2 + m^2c^2(\alpha^2 - L^2c^2)}\cosh X = 
\frac{C^2}{r}\sqrt{2\frac{B^2}{C^2}r - \frac{A^2}{C^2}r^2 - 1}
\]
Since $\cosh$ is an even function,
\begin{eqnarray}
\pm c\sqrt{(L\mathcal{E})^2 + m^2c^2(\alpha^2 - L^2c^2)}
\cosh\left((\phi - \phi_0)\sqrt{\frac{\alpha^2}{L^2c^2} - 1}\right) &=& \nonumber\\
\frac{C^2}{r}\sqrt{2\frac{B^2}{C^2}r - \frac{A^2}{C^2}r^2 - 1} \label{c5e78}
\end{eqnarray}
We can choose $\phi_0$ such that this equation reduces to
\begin{equation}\label{c5e79}
\pm c\sqrt{(L\mathcal{E})^2 + m^2c^2(\alpha^2 - L^2c^2)}
\cosh\left((\phi - \phi_0)\sqrt{\frac{\alpha^2}{L^2c^2} - 1}\right) = \frac{C^2}{r}.
\end{equation}
This can always be done because \eqref{c5e78} is of the form
\begin{equation}\label{c5e80}
A\cosh(B(x_0-x)) = C^2f(x).
\end{equation}
We can write it as
\begin{equation}\label{c5e81}
x_0 = x + \frac{1}{B}\cosh^{-1}\left(\frac{C^2f(x)}{A}\right).
\end{equation}
Choose $x_0$ such that if $f(x^\op) = 1$ then
\begin{equation}\label{c5e82}
x_0 = x^\op + \frac{1}{B}\cosh^{-1}\left(\frac{C^2}{A}\right).
\end{equation}

\item $\alpha^2 < L^2c^2$. In this case,
\begin{eqnarray*}
\cosh\left((\phi - \phi_0)\sqrt{\frac{\alpha^2}{L^2c^2} - 1}\right) 
 &=& \cosh\left(i(\phi - \phi_0)\sqrt{1 - \frac{\alpha^2}{L^2c^2}}\right) \\
 &=& \cos\left((\phi - \phi_0)\sqrt{1 - \frac{\alpha^2}{L^2c^2}}\right)
\end{eqnarray*}
and \eqref{c5e79} becomes
\begin{equation}\label{c5e83}
\pm c\sqrt{(L\mathcal{E})^2 - m^2c^2(L^2c^2 - \alpha^2)}
\cos\left((\phi - \phi_0)\sqrt{1 - \frac{\alpha^2}{L^2c^2}}\right) = \frac{C^2}{r}.
\end{equation}

Note that, when $\phi \mapsto \phi + 2\pi$ increases by $2\pi$, 
\begin{eqnarray*}
\cos\left((\phi - \phi_0)\sqrt{1 - \frac{\alpha^2}{L^2c^2}}\right)  
 &\mapsto& \cos\left((\phi - \phi_0)\sqrt{1 - \frac{\alpha^2}{L^2c^2}} + 2\pi\sqrt{1 - \frac{\alpha^2}{L^2c^2}}\right) \\
 &\ne& \cos\left((\phi - \phi_0)\sqrt{1 - \frac{\alpha^2}{L^2c^2}}\right)
\end{eqnarray*}
so that the orbits described by \eqref{c5e83} are not closed curves like 
ellipses but are rosettes.

\item $\alpha^2 = L^2c^2$. In this case, \eqref{c5e71} becomes
\begin{equation}\label{c5e84}
\phi_0 = \phi \mp Lc\int\frac{dr}{r\sqrt{A^2r^2 - 2Br}} =  
\phi \mp \frac{Lc}{B}\frac{\sqrt{A^2r^2 - 2Br}}{r}.
\end{equation}
Squaring both sides,
\[
(\phi - \phi_0)^2 = \frac{Lc}{B}\frac{A^2r^2 - 2Br}{r^2}
\]
Using the expressions for $A$ and $B$ in \eqref{c5e72} and \eqref{c5e73}, we get
\[
(\phi - \phi_0)^2 = \frac{L^2c^2}{\alpha^2\mathcal{E}^2}
\frac{(\mathcal{E}^2 - m^2c^4)r^2 - 2\alpha\mathcal{E}r}{r^2}
\]
or
\[
\left(\frac{\mathcal{E}\alpha}{Lc}\right)^2(\phi - \phi_0)^2 = 
\mathcal{E}^2 - m^2c^4 - 2\frac{\alpha\mathcal{E}}{r}
\]
Choosing initial conditions such that $\phi_0 = 0$, we finally get

\begin{equation}\label{c5e85}
2\frac{\alpha\mathcal{E}}{r} = \mathcal{E}^2 - m^2c^4 -
\left(\frac{\mathcal{E}\alpha}{Lc}\right)^2\phi^2.
\end{equation}
\end{enumerate}

\item Consider a localised collection of discrete charges $q_a$ at locations 
$\vec{R}_a$. The potential due to all of them at a point $\vec{R}_0$ is
\begin{equation}\label{c5e86}
\varphi(\vec{R}_0) = \sum_a\frac{q_a}{|\vec{R}_0 - \vec{R}_a|}.
\end{equation}
The rhs is a sum over $\vec{R}_a$ of functions of the form $q_af(\vec{R}_0-\vec{R}_a)$.
We can expand them as
\[
f(\vec{R}_0-\vec{R}_a) = f(\vec{R}_0) - \vec{R}_a\cdot\grad f|_{\vec{R}_0}.
\]
Here we have ignored the higher order terms. In this case, $f(\vec{R}) = 1/\vec{R}$
so that
\[
\grad f = -\frac{\vec{R}}{R^3}.
\]
Thus, we can write \eqref{c5e86} as
\begin{equation}\label{c5e87}
\varphi(\vec{R}_0) = 
\sum_a\left(\frac{q_a}{R_0} + q_a\vec{R}_a\cdot\frac{\vec{R_0}}{R_0^3}\right).
\end{equation}
If
\begin{eqnarray}
Q &=& \sum_a q_a \label{c5e88} \\
\vec{p} &=& \sum_a q_a\vec{R}_a \label{c5e89}
\end{eqnarray}
then we can write \eqref{c5e87} as
\begin{equation}\label{c5e90}
\varphi(\vec{R}_0) = \frac{Q}{R_0} + \frac{\vec{p}\cdot\vec{R_0}}{R_0^3}.
\end{equation}
$Q$ is the total charge in the collection and $\vec{p}$ is called its \emph{dipole
moment}.

\item The dipole moment depends on $\vec{R}_a$ and therefore on the choice of the
origin. If we shift the origin to $\vec{b}$ then the new position vectors are
$\vec{R}_a - \vec{b}$. The new dipole moment is
\begin{equation}\label{c5e91}
\vec{p}^\op = \sum_a q_a (\vec{R}_a - \vec{b}) = \vec{p} - Q\vec{b}.
\end{equation}
If $Q = 0$ then $\vec{p}^\op = \vec{p}$. 

\item Let us denote the positive charges by $q_a^+$ and the negative ones by
$-q_a^-$. Let their positions be denoted by $\vec{R}_a^+$ and $\vec{R}_a^-$
respectively. Then,
\begin{equation}\label{c5e92}
\vec{p} =  \sum_a q_a^+\vec{R}_a^+ - \sum_a q_a^-\vec{R}_a^-.
\end{equation}
Let,
\begin{eqnarray*}
\vec{R}^+ &=& \frac{\sum_a q_a^+\vec{R}_a^+}{\sum_a q_a^+} \\
\vec{R}^- &=& \frac{\sum_a q_a^-\vec{R}_a^-}{\sum_a q_a^-}
\end{eqnarray*}
so that \eqref{c5e92} becomes
\begin{equation}\label{c5e93}
\vec{p} = \left(\sum_a q_a^+\right)\vec{R}^+ - \left(\sum_a q_a^-\right)\vec{R}^-.
\end{equation}
If
\begin{equation}\label{c5e94}
\sum_a q_a^+ = \sum_a q_a^- = Q_0,
\end{equation}
say, then equation \eqref{c5e93} becomes
\begin{equation}\label{c5e95}
\vec{p} = Q_0\vec{R}^{+-},
\end{equation}
where
\begin{equation}\label{c5e96}
\vec{R}^{+-} = \vec{R}^+ - \vec{R}^-.
\end{equation}
This shows that if the total charge of the collection is zero then we can write 
its dipole moment as a product of $Q_0$ the sum of its positive charges and 
$\vec{R}^{+-}$, which can be viewed as the separation of the ``centres of masses
`' of the charges of each kind. The term ``centre of mass'' is not inappropriate
given the similarity of the definitions of $R^+$ and $R^-$ with that of the centre
of mass of a collection of discrete particles.

\item We have been considering the case $Q = 0$ in the previous two points. It is
interesting and important because most macroscopic bodies are electrically neutral
although they are composed of positive and negative charges. Continuing to analyse
this case further, if $Q = 0$, \eqref{c5e90} simplifies to
\begin{equation}\label{c5e97}
\varphi(\vec{R}_0) = \frac{\vec{p}\cdot\vec{R}_0}{R_0^3}.
\end{equation}
The electric field at $\vec{R}_0$ is
\begin{equation}\label{c5e98}
\vec{E} = -\grad\varphi(\vec{R}_0) = -\grad\frac{\vec{p}\cdot\vec{R}_0}{R_0^3}.
\end{equation}
Now,
\[
\grad\frac{\vec{p}\cdot\vec{R}_0}{R_0^3} = \frac{\vec{p}}{R_0^3} - 
\vec{p}\cdot\vec{R}_0\grad\frac{1}{R_0^3} = \frac{\vec{p}}{R_0^3} -
\frac{3(\vec{p}\cdot\vec{R}_0)\vec{R}_0}{R_0^5}.
\]
If 
\begin{equation}\label{c5e99}
\un = \frac{\vec{R}_0}{R_0}
\end{equation}
then
\begin{equation}\label{c5e100}
\grad\frac{\vec{p}\cdot\vec{R}_0}{R_0^3} = 
\frac{\vec{p}}{R_0^3} - \frac{3(\un\cdot\vec{p})\un}{R_0^3}
\end{equation}
and \eqref{c5e98} becomes
\begin{equation}\label{c5e101}
\vec{E} = \frac{3(\un\cdot\vec{p})\un - \vec{p}}{R_0^3}.
\end{equation}
The potential of a dipole varies as $R_0^{-2}$ and the field as $R_0^{-3}$.

\item The Taylor expansion of $f(\vec{R}_0-\vec{R}_a)$ up to second order term is
\begin{equation}\label{c5e102}
f(\vec{R}_0-\vec{R}_a) = f(\vec{R}_0) - \vec{R}_a\cdot\grad f\Big|_{\vec{R}_0} +
\frac{1}{2}\sum_{i, j}R_{ai}R_{aj}
\frac{\partial^2 f}{\partial x_i\partial x_j}\Big|_{\vec{R}_0}.
\end{equation}
Here $R_{ai}, R_{aj}$ are components of of $\vec{R}_a$.
If $f = 1/|\vec{R}_0-\vec{R}_a|$ then
\[
\pdt{f}{x_i} = -\frac{x_i}{R_0^3}
\]
and
\[
\frac{\partial^2 f}{\partial x_j\partial x_i} = -\frac{\delta_{ij}}{R_0^3} + 3\frac{x_ix_j}{R_0^5}
= \frac{3x_ix_j - R_0^2\delta_{ij}}{R_0^5}.
\]
The second order term in Taylor expansion of the potential of \eqref{c5e86} is
\begin{eqnarray*}
T_2 &=& \frac{1}{2}\sum_a\sum_{ij} q_aR_{ai}R_{aj}\frac{3x_ix_j - R_0^2\delta_{ij}}{R_0^5}\Big|_{\vec{R}_0} \\
 &=& \frac{1}{2}\sum_a\sum_{ij} q_aR_{ai}R_{aj}\frac{3R_{0i}R_{0j} - R_0^2\delta_{ij}}{R_0^5} \\
 &=& \frac{1}{2}\sum_{ij}\sum_a q_a\frac{3R_{ai}R_{aj}R_{0i}R_{0j} - R_{ai}R_{aj}R_0^2\delta_{ij}}{R_0^5}
\end{eqnarray*}
Now,
\[
\sum_{ij}R_{ai}R_{aj}R_0^2\delta_{ij} = R_a^2R_0^2 = \sum_{ij}R_a^2\delta_{ij}R_{ai}R_{aj}
\]
so that
\[
T_2 = \frac{1}{2}\sum_{ij}R_{0i}\sum_a q_a(3R_{ai}R_{aj} - R_a^2\delta_{ij}R_{0j})\frac{1}{R_0^5}.
\]
We can now extend the expression in \eqref{c5e87} to
\begin{equation}\label{c5e103}
\varphi(R_0) = \frac{Q}{R_0} + \frac{\vec{p}\cdot\vec{R}_0}{R_0^3} + 
\frac{1}{2}\frac{R_{0i}D_{ij}R_{0j}}{R_0^5}.
\end{equation}
where we have used the summation convention and
\begin{equation}\label{c5e104}
D_{ij} = \sum_a q_a(3R_{ai}R_{aj} - R_a^2\delta_{ij})
\end{equation}
is the quadrupole moment of the collection of charges. $D_{ij}$ is a symmetric,
second-order 3-tensor. Further, it is also traceless, for
\[
D_{ii} = \sum_a q_a(3R_{ai}R_{ai} - R_a^2\delta_{ii}) = \sum_a q_a(3R_a^2 - 3R_a^2) = 0.
\]
Therefore, only five of its components are independent. Using \eqref{c5e99}, we
can further simplify \eqref{c5e103} to
\begin{equation}\label{c5e105}
\varphi(R_0) = \frac{Q}{R_0} + \frac{d_in_i}{R_0^2} + 
\frac{1}{2}\frac{n_iD_{ij}n_j}{R_0^3}.
\end{equation}

\item A symmetric axis can be diagonalised. If the charge distribution has a 
symmetry axis then align the $z$-axis with it. The symmetry about the $z$ axis
makes $D_{11} = D_{22}$. Further, since the trace of $D_{ij}$ is zero, we have 
$D_{11} + D_{22} + D_{33} = 0$ or, taking since $D_{11} = D_{22}$,
\begin{equation}\label{c5e106}
D_{11} = D_{22} = -\frac{D_{33}}{2} = -\frac{D}{2},
\end{equation}
where defined $D = D_{33}$. Therefore,
\[
n_i D_{ij} n_j = n_1^2D_{11} + n_2^2D_{22} + n_3^2D_{33} = 
D\left(n_3^2 - \frac{n_1^2 + n_2^2}{2}\right).
\]
If $\un$ has polar coordinates $(1, \phi, \theta)$ where the $z$-axis is the 
symmetry axis of the charge distribution then $n_3 = \cos\theta, n_2 = \sin\theta
\sin\phi, n_1 = \sin\theta\cos\phi$ and hence,
\begin{equation}\label{c5e107}
n_iD_{ij}n_j = \frac{D}{2}(3\cos^2\theta - 1).
\end{equation}
We can then write \eqref{c5e105} as
\begin{equation}\label{c5e108}
\varphi(R_0) = \frac{Q}{R_0} + \frac{d\cos\theta}{R_0^2} + 
\frac{D}{4R_0^3}(3\cos^2\theta - 1)
\end{equation}
Since $P_1(\cos\theta) = 1, P_2(\cos\theta) = \cos\theta$ and $P_3(\cos\theta)
= (3\cos^2\theta - 1)/2$, $P_n$ being the Legendre polynomial of order $n$, we 
can write \eqref{c5e109} as
\begin{equation}\label{c5e109}
\varphi(R_0) = \frac{Q}{R_0}P_1(\cos\theta) + \frac{d}{R_0^2}P_2(\cos\theta)
+ \frac{D}{2R_0^3}P_3(\cos\theta).
\end{equation}

\item We can continue to expand the potential to get higher order terms expressed
as multiples of Legendra polynomials using
\begin{equation}\label{c5e110}
\frac{1}{|\vec{R}_0 - \vec{r}|} = \frac{1}{\sqrt{R_0^2 + r^2 - 2rR_0\cos\chi}}
= \sum_{l=0}^\infty \frac{r^l}{R_0^{l+1}}P_l(\cos\chi).
\end{equation}
If the spherical angles of $\vec{R}_0$ are $\Theta$ and $\Phi$ and those of
$\vec{r}_0$ are $\theta$ and $\phi$ then
\begin{equation}\label{c5e111}
P_l(\cos\chi) = \sum_{m=-l}^l\frac{(l -|m|)!}{(l + |m|)!}
P_l^{|m|}(\cos\Theta)P_l^{|m|}(\cos\theta)e^{-im(\Phi-\phi)},
\end{equation}
where $P_l^{|m|}$ are called the associated Legendre polynomials. We next introduce
the spherical harmonics as
\begin{equation}\label{c5e112}
Y_{lm}(\theta,\phi)=(-1)^m\sqrt{\frac{2l+1}{4\pi}\frac{(l-m)!}{(l+m)!}}P_l^m(\cos\theta)e^{im\phi},
m \ge 0
\end{equation}
so that we can write \eqref{c5e111} as
\begin{equation}\label{c5e113}
P_l(\cos\chi)=\sum_{m=-l}^l\frac{4\pi}{2l+1}Y_{lm}^\ast(\Theta,\Phi)Y_{lm}(\theta,\phi)
\end{equation}
and \eqref{c5e110} as
\begin{equation}\label{c5e114}
\frac{1}{|\vec{R}_0 - \vec{r}_a|} = \sum_{l=0}^\infty\sum_{m=-l}^l\frac{r^l}{R_0^{l+1}}
\frac{4\pi}{2l+1}Y_{lm}^\ast(\Theta,\Phi)Y_{lm}(\theta_a,\phi_a).
\end{equation}
This allows us to write \eqref{c5e86} as
\begin{eqnarray*}
\varphi(\vec{R}_0) &=& \sum_a\frac{q_a}{|\vec{R}_0 - \vec{r}|} \\
 &=& \sum_a q_a\sum_{l=0}^\infty\sum_{m=-l}^l\frac{r^l}{R_0^{l+1}}\frac{4\pi}{2l+1}Y_{lm}^\ast(\Theta,\Phi)
     Y_{lm}(\theta_a,\phi_a) \\
 &=& \sum_{l=0}^\infty\frac{1}{R_0^{l+1}}\sum_{m=-l}^l\sum_a q_ar^l\frac{4\pi}{2l+1}Y_{lm}^\ast(\Theta,\Phi)
     Y_{lm}(\theta_a,\phi_a) \\
 &=& \sum_{l=0}^\infty\frac{1}{R_0^{l+1}}\sum_{m=-l}^l\sqrt{\frac{4\pi}{2l+1}}Y_{lm}^\ast(\Theta,\Phi)
     \sum_a q_ar^l\sqrt{\frac{4\pi}{2l+1}}Y_{lm}(\theta_a,\phi_a)
\end{eqnarray*}
Define
\begin{equation}\label{c5e115}
Q_m^l = \sum_a q_ar^l\sqrt{\frac{4\pi}{2l+1}}Y_{lm}(\theta_a,\phi_a)
\end{equation}
so that
\begin{equation}\label{c5e116}
\varphi(\vec{R}_0) = \sum_{l=0}^\infty\frac{1}{R_0^{l+1}}\sum_{m=-l}^l\sqrt{\frac{4\pi}{2l+1}}
Y_{lm}^\ast(\Theta,\Phi)Q_m^l.
\end{equation}
For each $l$, $m$ can take $2l + 1$ values from $-l$ to $l$. They form the 
components of the $2^l$-moment of the collection of charges.

\item Now consider a collection of charges $q_a$ at positions $\vec{r}_a$ with 
respect to an origin within the collection. Let it be exposed to an external 
electric field which varies slowly over the extent of the collection. If $\varphi$
is the potential of the field then the energy of the collection is
\begin{equation}\label{c5e117}
U = \sum_aq_a\varphi(\vec{r}_a).
\end{equation}
Expand $U$ about the origin so that
\begin{equation}\label{c5e118}
U = \sum_{n=0}^\infty U^{(n)}.
\end{equation}
Then 
\begin{equation}\label{c5e119}
U^{(0)} = \sum_a q_a\varphi(0) = Q\varphi(0).
\end{equation}
The first order term is
\[
U^{(1)} = \sum_a q_a\vec{r}_a\cdot\grad\varphi|_0 = 
-\left(\sum_a q_a\vec{r}_a\right)\cdot\vec{E}_0,
\]
where $\vec{E}_0 = -\grad\varphi|_0$ is the electric field at the origin. From 
\eqref{c5e89} we know that the term in the bracket is the dipole moment of the
distribution so that
\begin{equation}\label{c5e120}
U^{(1)} = -\vec{p}\cdot\vec{E}_0,
\end{equation}
The force on the distribution is
\begin{equation}\label{c5e121}
\vec{F} = -\grad U
\end{equation}
so that up to the first order, the force on the distribution is
\[
\vec{F} = -Q\grad\varphi|_0 + \grad(\vec{p}\cdot\vec{E})|_0
\]
or, since $\vec{p}$ is a constant,
\begin{equation}\label{c5e122}
\vec{F} = Q\vec{E}_0 + (\vec{p}\cdot\grad)\vec{E}|_0.
\end{equation}
The torque on the collection with respect to the origin, to the lowest order term,
is
\begin{equation}\label{c5e123}
\vec{N} = \sum_aq_a\vec{r_a}\times\vec{E}_0  = \vec{p}\times\vec{E}_0.
\end{equation}

\item Now consider two systems with net charge $0$ so that up to the lowest order
the energy is of the first system is
\begin{equation}\label{c5e124}
U_1 = -\vec{p}_1\cdot\vec{E},
\end{equation}
$\vec{E}$ being the external electric field \emph{at the origin within its extent}.
If $\vec{E}$ is due to the other system whose origin is at $\vec{R}$, then using
\eqref{c5e101},
\[
\vec{E} = \frac{(3\vec{R}\cdot\vec{p}_2)\vec{R} - \vec{p}_2R^2}{R^5}
\]
so that
\begin{equation}\label{c5e125}
U_1 = \frac{(\vec{p}_1\cdot\vec{p}_2)R^2 - 3(\vec{p}_1\cdot\vec{R})(\vec{p}_2\cdot\vec{R})}{R^5}.
\end{equation}

If the second system had a net charge $Q_2$ then
\[
\vec{E} = -\frac{Q}{R^3}\vec{R}
\]
and
\begin{equation}\label{c5e126}
U_1 = Q\frac{\vec{p}\cdot\vec{R}}{R^3}.
\end{equation}

\item The second order term in the expansion of \eqref{c5e118} is
\[
U^{(2)} = \frac{1}{2}\sum_a q_a r_{ai}r_{aj}\frac{\partial^2\varphi}{\partial x_i\partial x_j}\Big|_0.
\]
Since $\varphi$ is a solution of the Laplace equation,
\[
\frac{\partial^2\varphi}{\partial x_i\partial x_j} = 
(1 - \delta_{ij})\frac{\partial^2\varphi}{\partial x_i\partial x_j}
\]
so that
\[
\frac{\partial^2\varphi}{\partial x_i\partial x_j}\Big|_0 = 
(1 - \delta_{ij})\frac{\partial^2\varphi}{\partial x_i\partial x_j}\Big|_0
\]
and hence
\begin{eqnarray*}
U^{(2)} &=& \frac{1}{2}\sum_a q_a r_{ai}r_{aj}(1 - \delta_{ij}) \\
 &=& \frac{1}{2}\frac{\partial^2\varphi}{\partial x_i\partial x_j}\Big|_0\sum_a q_a r_{ai}r_{aj}(1 - \delta_{ij}) \\
 &=& \frac{1}{2}\frac{\partial^2\varphi}{\partial x_i\partial x_j}\Big|_0\sum_a q_a (r_{ai}r_{aj} - r_a^2\frac{\delta_{ij}}{3}) \\
 &=& \frac{1}{6}\frac{\partial^2\varphi}{\partial x_i\partial x_j}\Big|_0\sum_a q_a (3r_{ai}r_{aj} - r_a^2\delta_{ij})
\end{eqnarray*}
Using the definition of the quadrupole moment tensor of \eqref{c5e104},
\begin{equation}\label{c5e127}
U^{(2)} = \frac{1}{6}\frac{\partial^2\varphi}{\partial x_i\partial x_j}\Big|_0D_{ij}
\end{equation}

\item The spherical harmonics form a complete set of basis functions. Therefore,
one can write
\begin{equation}\label{c5e128}
\phi(\vec{r}) = \sum_{l=0}^\infty r^l\sum_{m=-l}^l\sqrt{\frac{4\pi}{2l+1}}a_{lm}Y_{lm}(\theta,\phi),
\end{equation}
where $a_{lm}$ are constants. Substituting this in equation \eqref{c5e117},
\begin{eqnarray*}
U &=& \sum_a q_a\sum_{l=0}^\infty r_a^l\sum_{m=-l}^l\sqrt{\frac{4\pi}{2l+1}}a_{lm}Y_{lm}(\theta,\phi) \\
  &=& \sum_{l=0}^\infty\sum_{m=-l}^l a_{lm}\left(\sum_a q_ar_a^l\sqrt{\frac{4\pi}{2l+1}}Y_{lm}(\theta,\phi)\right)
\end{eqnarray*}
Using \eqref{c5e115},
\[
U = \sum_{l=0}^\infty\sum_{m=-l}^l a_{lm}Q_m^l.
\]
from which we infer that
\begin{equation}\label{c5e129}
U^{(l)} = \sum_{m=-l}^l a_{lm}Q_m^l.
\end{equation}

\item Consider the magnetic field produced by charges in motion over a finite
portion of space and with momenta that also remain finite. This requirement is 
same under which the virial theorem was proved and under which the time averages
of time derivatives of functions is zero. Refer to definition \eqref{c4e112} 
for the time average of a function and the proof of \eqref{c4e114} for the 
vanishing of the time average of its derivative.

For such a system, we will perform the time averaging of the Maxwell equations
for the magnetic field to get
\begin{eqnarray}
\dive\ta{\vec{H}} &=& 0 \label{c5e130} \\
\curl\ta{\vec{H}} &=& \frac{4\pi}{c}\ta{\vec{J}} \label{c5e131}
\end{eqnarray}

\item One could have got similar equations had one assumed that the fields are
independent of time. We wrote \eqref{c5e3} and \eqref{c5e4} under the same 
assumption. However, one cannot consider the electric field to be time independent
of there are currents in the system. Therefore, the correct way to introduce
constant fields in this case is to assume that the $\langle\vec{E}\rangle = 0$.

\item We introduce $\langle\vec{A}\rangle$ using
\begin{equation}\label{c5e132}
\ta{\vec{H}} = \curl\ta{\vec{A}}.
\end{equation}
Substituting it in \eqref{c5e131} we get
\begin{equation}\label{c5e133}
\grad\dive\ta{\vec{A}} - \nabla^2\ta{\vec{A}} = \frac{4\pi}{c}\ta{\vec{J}}.
\end{equation}
Since the curl of a gradient is always zero, a vector potential
\[
\tav{A}^\op = \tav{A} + \grad\psi
\]
also gives the same $\tav{H}$. Now,
\[
\dive\tav{A}^\op = \dive\tav{A} + \nabla^2\psi.
\]
If we insist on choosing a $\psi$ that is a solution of the Laplace equation then
we get $\dive\tav{A}^\op = \dive\tav{A} = 0$. Choosing
\begin{equation}\label{c5e134}
\dive\tav{A} = 0
\end{equation} is equivalent of choosing $\psi$ to be a harmonic function. The 
choice of \eqref{c5e134} is called the \emph{Coulomb gauge}.

\item Under the Coulomb gauge, \eqref{c5e133} becomes
\begin{equation}\label{c5e135}
\nabla^2\ta{\vec{A}} = -\frac{4\pi}{c}\ta{\vec{J}}.
\end{equation}
Thus, $\tav{A}$ obeys the vector Poisson equation the way $\varphi$ obeys the
scalar Poisson equation of \eqref{c5e7}. The solution of \eqref{c5e135} is similar
to the solution \eqref{c5e15} of \eqref{c5e7}. Thus,
\begin{equation}\label{c5e136}
\tav{A}(\vec{r}) = \frac{1}{c}\int\frac{\tav{J}(\vec{r}^\op)}{|\vec{r} - \vec{r}^\op|}dV^\op.
\end{equation}
We can write the rhs of \eqref{c5e136} for discrete charges as
\begin{equation}\label{c5e137}
\tav{A} = \frac{1}{c}\sum_a \ta{\frac{q_a\vec{v}_a}{R_a}},
\end{equation}
where 
\begin{equation}\label{c5e138}
\vec{R}_a = \vec{r} - \vec{r}_a.
\end{equation}

\item From \eqref{c5e136}, we can immediately get
\[
\tav{H} = \frac{1}{c}\curl\int\frac{\tav{J}(\vec{r}^\op)}{|\vec{r} - \vec{r}^\op|}dV^\op.
\]
The curl is with respect to the unprimed variables. Therefore,
\[
\tav{H} = \frac{1}{c}\int\curl\frac{\tav{J}(\vec{r}^\op)}{|\vec{r} - \vec{r}^\op|}dV^\op
 = \frac{1}{c}\int\grad\frac{1}{|\vec{r} - \vec{r}^\op|} \times \tav{J}(\vec{r}^\op)dV^\op,
\]
where we used the identity
\begin{equation}\label{c5e139}
\curl(f\vec{F}) = f\curl\vec{F} + \grad f \times \vec{F}.
\end{equation}
From this we get the \emph{Biot-Savart} law,
\begin{equation}\label{c5e140}
\tav{H} = \frac{1}{c}\int\frac{\tav{J}(\vec{r}^\op) \times (\vec{r} - \vec{r}^\op)}{|\vec{r} - \vec{r}^\op|^3}dV^\op.
\end{equation}

\item We go back to equation \eqref{c5e137} which described the vector potential
due to a collection of charges. Choose the origin to be somewhere within the
positions of the charges and let the field point be $\vec{R}_0$. Expanding the
function $1/R_a$ up to the first order terms, we get
\[
\frac{1}{R_a} = \frac{1}{|\vec{R}_0 - \vec{r}_a|} = 
\frac{1}{R_0} - \vec{r}_a\cdot\grad\frac{1}{R}\Big|_{R_0}
\]
Therefore,
\begin{equation}\label{c5e141}
\tav{A} = \frac{1}{c}\sum_a\frac{q_a\tav{v}_a}{R_0} - 
\frac{1}{c}\sum_aq_a\ta{\vec{v_a}\vec{r}_a\cdot\grad\frac{1}{R}\Big|_{R_0}}.
\end{equation}
For a system of finite extent, $\tav{v}_a = 0$ so that
\begin{equation}\label{c5e142}
\tav{A} = +\frac{1}{cR_0^3}\sum_aq_a\ta{\vec{v}_a\vec{r}_a\cdot\vec{R}_0}.
\end{equation}
Now,
\[
\frac{d}{dt}\sum_a q_a\vec{r}_a(\vec{r}_a\cdot\vec{R}_0) = 
\sum_a q_a\left(\vec{v}_a(\vec{r}_a\cdot\vec{R}_0) + \vec{r}_a(\vec{v}_a\cdot\vec{R}_0)\right)
\]
so that
\[
\frac{1}{2}\sum_a q_a \vec{v}_a(\vec{r}_a\cdot\vec{R}_0) = 
\frac{1}{2}\frac{d}{dt}\sum_a q_a\vec{r}_a(\vec{r}_a\cdot\vec{R}_0) - 
\frac{1}{2}\sum_a q_a \vec{r}_a(\vec{v}_a\cdot\vec{R}_0).
\]
and hence
\begin{equation}\label{c5e143}
\ta{\frac{1}{2}\sum_a q_a \vec{v}_a(\vec{r}_a\cdot\vec{R}_0)} = 
-\ta{\frac{1}{2}\sum_a q_a \vec{r}_a(\vec{v}_a\cdot\vec{R}_0)},
\end{equation}
as the average of the time derivative vanishes. If we write
\[
\tav{A} = \frac{1}{cR_0^3}\sum_a\frac{q_a}{2}\ta{\vec{v}_a\vec{r}_a\cdot\vec{R}_0}
 + \frac{1}{cR_0^3}\sum_a\frac{q_a}{2}\ta{\vec{v}_a\vec{r}_a\cdot\vec{R}_0}.
\]
and use \eqref{c5e143} for one of the factors, we get
\begin{equation}\label{c5e144}
\tav{A} = \frac{1}{2cR_0^3}\sum_aq_a\ta{\vec{v}_a(\vec{r}_a\cdot\vec{R}_0) - 
\vec{r}_a(\vec{v}_a\cdot\vec{R}_0)}.
\end{equation}
We now introduce the magnetic dipole moment
\begin{equation}\label{c5e145}
\vec{m} = \frac{1}{2c}\sum_a q_a\vec{r}_a \times \vec{v}_a
\end{equation}
so that
\[
\vec{m} \times \vec{R}_0 = \frac{1}{2c}
\sum_a q_a(\vec{v}_a(\vec{r}_a\cdot\vec{R}_0) - \vec{r}_a(\vec{v}_a\cdot\vec{R}_0))
\]
(Note that this is $(\vec{A} \times \vec{B})\times\vec{C}$ and not $\vec{A}\times
(\vec{B}\times\vec{C})$.) and \eqref{c5e144} becomes
\begin{equation}\label{c5e146}
\tav{A} = \frac{\tav{m} \times \vec{R}_0}{R_0^3}.
\end{equation}

\end{enumerate}

\section{Problems}
%To do: change the equation labels.
\begin{enumerate}
\item Determine the quadrupole moment of a uniformly charged ellipsoid with 
respect to its centre.
\item[Solution:] We align the cartesian coordinate axes to coincide with the 
principal axes of the ellipsoid. Therefore, the tensor of \eqref{c5e104} can be
written as
\[
D_{xx} = \sum_a q_a(3x^2 - x^2 - y^2 - z^2) = \sum_a q_a(2x^2-y^2-z^2).
\]
If we generalise it to a continuous distribution, 
\begin{equation}\label{c5e117a}
D_{xx} = \int_V\rho(2x^2-y^2-z^2)dxdydz.
\end{equation}
where $V$ is the volume of the ellipsoid. If the equation of the ellipsoid is
\begin{equation}\label{c5e118a}
\frac{x^2}{a^2} + \frac{y^2}{b^2} + \frac{z^2}{c^2} = 1
\end{equation}
then introduce the transformation $x_1 = x/a, y_1 = y/b, z_1 = z/c$ so that the 
Jacobian of transformation is $abc$ and
\[
D_{xx} = \rho abc\int_{V_1}\left(2a^2x_1^2 -b^2y_1^2 - c^2z_1^2\right)dx_1dy_1dz_1
\]
and the volume of integration $V_1$ is over $x_1^2 + y_1^2 + z_1=1$, a unit sphere. 
Introduce spherical coordinates so that $x_1 = \sin\theta\cos\phi, y_1 = 
\sin\theta\sin\phi, z_1 = \cos\theta$ so that
\begin{eqnarray*}
\frac{D_{xx}}{\rho abc}
 &=& \int_0^\pi\int_0^{2\pi}(2a^2\sin^2\theta\cos^2\phi-b^2\sin^2\theta
 \sin^2\phi-c^2\cos^2\theta)\sin\theta d\theta d\phi \\
 &=& 2a^2\int_0^\pi\sin^3\theta d\theta\int_0^{2\pi}\cos^2\phi d\phi - 
      b^2\int_0^\pi\sin^3\theta d\theta\int_0^{2\pi}\sin^2\phi d\phi - \\
 & & c^2\int_0^\pi\sin\theta\cos^2\theta d\theta \int_0^{2\pi}d\phi \\
 &=& 2a^2\left(\frac{4}{3}\right)\pi - b^2\left(\frac{4}{3}\right)\pi
     - c^2\left(\frac{2}{3}\right)2\pi \\
 &=& \frac{4\pi}{3}(2a^2-b^2-c^2)
\end{eqnarray*}
so that
\[
D_{xx} = \rho\frac{4\pi}{3}abc(2a^2-b^2-c^2)
\]
Since the volume of the ellipsoids is $(4\pi/3)abc$, if $Q$ is the total charge
on the ellipsoid then
\begin{equation}\label{c5e119a}
D_{xx} = Q(2a^2-b^2-c^2).
\end{equation}
Analogous to \eqref{c5e117a} we have
\begin{equation}\label{c5e120a}
D_{yy} = \int_V(2y^2 - x^2 - z^2)dxdydz
\end{equation}
which, upon the same transformation as before, becomes
\[
D_{yy} = \rho abs\int_{V_1}(2b^2y_1^2 - a^2x_a^2 - c^2z_1^2)dx_1dy_1dz_1.
\]
This integral can be evaluated in the same way so that
\begin{equation}\label{c5e121a}
D_{yy} = Q(2b^2 - a^2 - c^2).
\end{equation}
Since the quadrupole moment tensor is traceless,
\begin{equation}\label{c5e122a}
D_{zz} = -D_{xx} - D_{yy} = Q(2c^2 - a^2 - b^2).
\end{equation}
\end{enumerate}
\chapter{Electromagnetic Waves}\label{c6}
\begin{enumerate}
\item Maxwell equations in vacuum are:
\begin{eqnarray}
\dive\vec{E} &=& 0 \label{c6e1} \\
\curl\vec{E} &=& -\frac{1}{c}\pdt{\vec{H}}{t} \label{c6e2} \\
\dive\vec{H} &=& 0 \label{c6e3} \\
\curl\vec{H} &=& \frac{1}{c}\pdt{\vec{E}}{t} \label{c6e4}
\end{eqnarray}
If the fields are also time independent, then we will have $\dive\vec{E} = 0,
\curl\vec{E} = 0, \dive\vec{H} = 0$ and $\curl\vec{H} = 0$. From equations
\eqref{c5e16} (Coulomb's law) and \eqref{c5e140} (Biot-Savart's law), we get
$\vec{E} = 0$ and $\tav{H} = 0$. Therefore, we assume that the fields depend
on time.

\item We will soon show that equations \eqref{c6e1} to \eqref{c6e4} have a
non-zero solution in vacuum. That is, electromagnetic fields can exist in vacuum.
We will also show that they satisfy the wave equation. Solutions of Maxwell
equations in vacuum are called electromagnetic waves.

\item As usual, it is easier to analyse the potentials. Let us choose them such
that $\varphi = 0$ and $\dive\vec{A} = 0$. Equation \eqref{c6e2} becomes
\[
-\frac{1}{c}\curl\pdt{\vec{A}}{t} = 
-\frac{1}{c}\frac{\partial}{\partial t}\curl\vec{A},
\]
an identity. While, equation \eqref{c6e4} becomes
\[
\curl\curl\vec{A} = -\frac{1}{c^2}\spdt{\vec{A}}{t}.
\]
The lhs of this equation is $\grad\dive\vec{A} - \nabla^2\vec{A}$. Since we chose
$\dive{\vec{A}} = 0$, we get
\begin{equation}\label{c6e5}
\nabla^2\vec{A} - \frac{1}{c^2}\spdt{\vec{A}}{t} = 0.
\end{equation}
The choice $\dive{\vec{A}} = 0$ is called the Coulomb gauge and that it can be
exercised without violating physics is explained in the point leading to 
\eqref{c5e134}. Taking the curl of this equation gives
\begin{equation}\label{c6e6}
\nabla^2\vec{H} - \frac{1}{c^2}\spdt{\vec{H}}{t} = 0.
\end{equation}
Taking a partial derivative of \eqref{c6e5} and recalling that we chose $\varphi
= 0$, gives
\begin{equation}\label{c6e7}
\nabla^2\vec{E} - \frac{1}{c^2}\spdt{\vec{E}}{t} = 0.
\end{equation}

\item We can repeat this analysis using the Maxwell equations written in terms of
the electromagnetic field tensor. From \eqref{c4e37},
\[
\pdt{F^{\mu\nu}}{x^\nu} = -\frac{4\pi}{c}j^\mu.
\]
In the absence of sources, it becomes
\[
\pdt{F^{\mu\nu}}{x^\nu} = 0.
\]
We now use the definition 
\[
F^{\mu\nu} = \pdt{A^\nu}{x_\mu} - \pdt{A^\mu}{x_\nu}
\]
to get
\[
\frac{\partial}{\partial x^\nu}\pdt{A^\nu}{x_\mu} - \frac{\partial}{\partial x^\mu}\pdt{A^\nu}{x_\nu} = 0
\Rightarrow
\frac{\partial}{\partial x_\mu}\pdt{A^\nu}{x^\nu} - \frac{\partial}{\partial x^\mu}\pdt{A^\nu}{x_\nu} = 0.
\]
We now choose $A^\mu$ such that
\begin{equation}\label{c6e8}
\pdt{A^\nu}{x^\nu} = 0.
\end{equation}
In terms of coordinates, it is
\begin{equation}\label{c6e9}
\frac{1}{c}\pdt{\varphi}{t} - \dive\vec{A} = 0.
\end{equation}
Equations \eqref{c6e8} and \eqref{c6e9} are called the \emph{Lorenz gauge}. Unlike
the condition of Coulomb gauge, that of the Lorenz gauge is invariant under Lorentz 
transformation. We are thus left with
\begin{equation}\label{c6e10}
\frac{\partial}{\partial x^\nu}\pdt{A^\mu}{x_\nu} = 
g^{\nu\rho}\frac{\partial}{\partial x^\nu}\pdt{A^\mu}{x^\rho} = 0.
\end{equation}
It is equivalent to \eqref{c6e5}.

\item The nine equations \eqref{c6e5}, \eqref{c6e6} and \eqref{c6e7} are all of
the form
\[
\nabla^2 f - \frac{1}{c^2}\spdt{f}{t} = 0 \Rightarrow 
\spdt{f}{t} - c^2\nabla^2 f = 0,
\]
where $f$ is one of the components of $\vec{A}, \vec{E}$ or $\vec{H}$. If the 
function $f$ depends only on $x$ and $t$, that is, if it is independent of $y$ and
$z$, then the solution $f$ is called a plane wave. In this case, the equation 
simplifies to
\begin{equation}\label{c6e11}
\spdt{f}{t} - c^2\spdt{f}{x} = 0.
\end{equation}
To solve this equation, write it as
\begin{equation}\label{c6e12}
\left(\frac{\partial}{\partial t} - c\frac{\partial}{\partial x}\right)
\left(\frac{\partial}{\partial t} + c\frac{\partial}{\partial x}\right)f = 0.
\end{equation}
Introduce the variables
\begin{eqnarray}
\xi &=& t - \frac{x}{c} \label{c6e13} \\
\eta &=& t + \frac{x}{c} \label{c6e14}
\end{eqnarray}
so that
\begin{eqnarray}
\frac{\partial}{\partial t} &=& \frac{\partial}{\partial\xi} = 
 \frac{\partial}{\partial\eta} \label{c6e15} \\
\frac{\partial}{\partial x} &=& -\frac{1}{c}\frac{\partial}{\partial\xi} =
\frac{1}{c}\frac{\partial}{\partial\eta} \label{c6e16}
\end{eqnarray}
so that
\begin{eqnarray}
\frac{\partial}{\partial\eta} &=& \frac{1}{2}
	\left(\frac{\partial}{\partial t} + c\frac{\partial}{\partial x}\right) 
	\label{c6e17} \\
\frac{\partial}{\partial\xi} &=& \frac{1}{2}
	\left(\frac{\partial}{\partial t} - c\frac{\partial}{\partial x}\right) 
	\label{c6e18}
\end{eqnarray}
This pair of relations allows us to write \eqref{c6e12} as
\begin{equation}\label{c6e19}
\frac{\partial^2 f}{\partial\xi\partial\eta} = 0.
\end{equation}
Its solution can be written as
\begin{equation}\label{c6e20}
f(\xi, \eta) = f_1(\xi) + f_2(\eta).
\end{equation}
In terms of the original variables,
\begin{equation}\label{c6e21}
f(x, t) = f_1\left(t - \frac{x}{c}\right) + f_2\left(t + \frac{x}{c}\right).
\end{equation}

\item If $f_2 = 0$ then
\[
f(x, t) = f_1\left(t - \frac{x}{c}\right)
\]
Any reasonably function of $t - x/c$ is a solution of \eqref{c6e11}. For a fixed
$x$, the value of the function changes with $t$. Likewise, for a fixed $t$, the
function changes value with $x$. Further the value of the function is the same
for a given value of its argument, namely, $t - x/c$ or $ct - x$. If we focus 
our attention on one such value, we will see it travelling along the $x$ axis 
with a speed $x/t = c$, the speed of light.

If $f_1 = 0$ and $f = f_2$ then a similar interpretation is applicable except
that a point with a certain value of $f$ travels along the negative $x$ axis
with the speed $c$.

\item How do we use this information to derive the forms of $\vec{E}$ and $\vec{H}$
for plane waves? We will start by deriving an expression for the vector potential.
Since the components of $\vec{A}$ individually satisfy \eqref{c6e11}, none of them
will be functions of $y$ and $z$. Further, $\dive\vec{A} = 0$ implies that
\[
\pdt{A_x}{x} = 0.
\]
Therefore,
\[
\spdt{A_x}{t} - c^2\spdt{A_x}{x} = 0 \Rightarrow \spdt{A_x}{t} = 0
\Rightarrow \pdt{A_x}{t} = \text{const.}
\]
This, in turn, implies that $E_x$ is a constant. If this constant is non-zero
then it results in a constant electric field in the direction of propagation
of the wave, also called the longitudnal direction. Being a constant, it is
not of the form $f(t \pm x/c)$ and therefore unrelated to the electromagnetic
wave. We can ignore it by setting $A_x = 0$. We are thus left with
\begin{equation}\label{c6e22}
\vec{A} = A_y(x)\uv{y} + A_z(x)\uv{z}
\end{equation}
and hence
\begin{eqnarray}
\vec{E} &=& -\frac{1}{c}\pdt{\vec{A}}{t} \label{c6e23} \\
\vec{H} &=& \curl\vec{A} \label{c6e24}
\end{eqnarray}
From equation \eqref{c6e15},
\[
\pdt{\vec{A}}{t} = \pdt{\vec{A}}{\xi}
\]
so that if we denote rhs of the above equation as $\vec{A}^\op$ then equation 
\eqref{c6e23} becomes
\begin{equation}\label{c6e25}
\vec{E} = -\frac{1}{c}\vec{A}^\op.
\end{equation}
Now,
\begin{equation}\label{c6e26}
\curl\vec{A} = -\uv{y}\pdt{A_z}{x} + \uv{z}\pdt{A_y}{x} = 
\frac{1}{c}\uv{y}\pdt{A_z}{\xi} - \frac{1}{c}\uv{z}\pdt{A_y}{\xi}
\end{equation}
where we used \eqref{c6e16}. If $\un$ is the direction of wave propagation then
$\un = \uv{x}$ and 
\begin{equation}\label{c6e27}
\un\times\vec{A}^\op = \uv{z}A^\op_y - \uv{y}A^\op_z
\end{equation}
and hence from \eqref{c6e26} and \eqref{c6e27} we get
\begin{equation}\label{c6e28}
\vec{H} = -\frac{1}{c}\un\times\vec{A}^\op.
\end{equation}
From \eqref{c6e25} and \eqref{c6e28} it immediately follows that
\begin{equation}\label{c6e29}
\vec{H} = \un\times\vec{E}.
\end{equation}
Thus both fields are perpendicular to $\un$, the direction of propagation. As a 
result, the plane electromagnetic waves in vacuum are transverse in nature. 
Equation \eqref{c6e29} also assures that magnitude of the two fields is the same.

\item The energy flux is
\begin{equation}\label{c6e30}
\vec{S} = \frac{c}{4\pi}\vec{E}\times\vec{H} = 
\frac{c}{4\pi}\vec{E}\times(\un\times\vec{E}) = 
\frac{c}{4\pi}E^2\un = \frac{c}{4\pi}H^2\un.
\end{equation}
We can as well write it as 
\begin{equation}\label{c6e31}
2W = \frac{c}{4\pi}(E^2 + H^2)\un
\end{equation}
so that
\begin{equation}\label{c6e32}
\vec{S} = cW\un
\end{equation}
where we used \eqref{c4e50}, where $W$ where is the energy density of the field.
The components of $\vec{S}/c$ are the components 
\[
T^{01}, T^{02}, T^{03}
\]
of the energy-momentum tensor \eqref{c4e90}. We also argued in point 23 of chapter
\ref{c4} that $T^{01}/c, T^{02}/c, T^{03}/c$ are the components of momentum 
density of the system. Thus, we can interpret $\vec{S}/c^2$ as the momentum flux
in the system. From \eqref{c6e32} we see that electromagnetic waves carry momentum
in the direction of propagation.

The relation between energy density and momentum density of electromagentic 
waves is similar to that of relativistic particle of zero mass.

\item If the $x$-axis is chosen to be along the direction of propagation of
the wave then $\vec{E} = E\uv{y}$ and $\vec{H} = \uv{z}$. The components of the
energy momentum tensor evaluated below \eqref{c4e88} are
\begin{eqnarray}
T^{00} &=& \frac{E^2 + H^2}{8\pi} \label{c6e33} \\
T^{11} &=& -\frac{E^2 + H^2}{8\pi} = -\sigma_{xx} \label{c6e34},
\end{eqnarray}

\item Lorentz transformation for $W$ follows from \eqref{c1e88}.
\begin{equation}\label{c6e35}
W = \frac{W^\op + 2\beta S^\op_x/c + \beta^2T_{11}^\op}{1 - \beta^2}
 = \frac{W^\op + 2\beta S^\op_x/c - \beta^2\sigma_{xx}^\op}{1 - \beta^2}.
\end{equation}
If $\alpha^\op$ is the angle between $\bm{\beta}$ and the direction of 
propagation of the wave then
\[
S_x^\op = cW^\op\cos\alpha^\op
\]
and 
\begin{equation}\label{c6e36}
\sigma_{xx}^\op = -W^\op\cos^2\alpha^\op,
\end{equation}
{\color{red}To do: justify \eqref{c6e36}.} so that
\begin{equation}\label{c6e37}
W = \frac{W^\op(1 + \beta\cos\alpha^\op)^2}{1 - \beta^2}.
\end{equation}

\item A wave whose time dependence is of the form $\cos(\omega t + \alpha)$, where
$\omega$ and $\alpha$ are constants is called a mono-chromatic wave. If $f(\vec{r},
t)$ is the field quantity and if its time dependence of mono-chromatic then
\begin{equation}\label{c6e38}
\nabla^2 f = -\frac{\omega^2}{c^2} f.
\end{equation}
If the wave is propagating along the positive $x$ axis then $f$ is a function of
$t - x/c$ alone. If it is also mono-chromatic then,
\[
f(x, t) = f_0\cos\left(\omega t - \omega\frac{x}{c} + \alpha\right) = 
\re\left\{f_0\exp\left(-i\left(\omega t - \omega\frac{x}{c} + \alpha\right)\right)\right\}.
\]
The constant $\alpha$ can be absorbed in $f_0$ so that we can as well write
\begin{equation}\label{c6e39}
f(x, t) = \re\left\{f_0\exp\left(-i\left(\omega t - \omega\frac{x}{c}\right)\right)\right\}.
\end{equation}
Now,
\[
\cos\left(\omega t - \frac{\omega}{c}\left(x + \frac{2\pi c}{\omega}\right)\right)
= \cos\left(\omega t - \frac{\omega}{c}x + 2\pi\right) = 
\cos\left(\omega t - \frac{\omega}{c}x\right)
\]
so that the quantity
\begin{equation}\label{c6e40}
\lambda = \frac{2\pi c}{\omega}
\end{equation}
is called the \emph{wavelength}. The vector,
\begin{equation}\label{c6e41}
\vec{k} = \frac{\omega}{c}\un
\end{equation}
is called the \emph{wave-vector}. In terms of $\vec{k}$ we can generalise 
\eqref{c6e39} to
\begin{equation}\label{c6e42}
f(\vec{r}, t) = \re\left\{f_0\exp\left(i\left(\vec{k}\cdot\vec{r} - \omega t\right)\right)\right\}.
\end{equation}
If we perform only linear operations, we can write \eqref{c6e42} as
\begin{equation}\label{c6e43}
f(\vec{r}, t) = f_0e^{i(\vec{k}\cdot\vec{r} - \omega t)}
\end{equation}
with the understanding that only the real part is physically significant.

\item If, for a plane, mono-chromatic wave,
\begin{equation}\label{c6e44}
\vec{A}(\vec{r}, t) = \vec{A}_0e^{i(\vec{k}\cdot\vec{r} - \omega t)}
\end{equation}
then
\begin{equation}\label{c6e45}
\vec{E} = -\frac{1}{c}\pdt{\vec{A}}{t} = i\frac{\omega}{c}\vec{A} = ik\vec{A},
\end{equation}
where we used \eqref{c6e41} in the last step. Further,
\begin{equation}\label{c6e46}
\vec{H} = \curl\vec{A} = i\vec{k}\times\vec{A}.
\end{equation}

\item Let 
\begin{equation}\label{c6e47}
\vec{E} = \vec{E}_0 e^{i(\vec{k}\cdot\vec{r} - \omega t)},
\end{equation}
where
\begin{equation}\label{c6e48}
\vec{E}_0 = \vec{b}e^{-i\alpha}
\end{equation}
and
\begin{equation}\label{c6e49}
\vec{b} = \vec{b}_1 + i\vec{b}_2,
\end{equation}
so that $\vec{E}_0$ is a complex vector and $\vec{b}_1, \vec{b}_2$ are real 
vectors. If we insist that
\[
|\vec{E}_0|^2 = \vec{E}_0\cdot\vec{E}_0^\ast = \vec{b}\cdot\vec{b}
\]
is real then
\[
\vec{b}\cdot\vec{b} = \vec{b}_1\cdot\vec{b}_1 - \vec{b}_2\cdot\vec{b}_2 + 
2i\vec{b}_1\cdot\vec{b}_2
\]
must be real, which requires us to have
\begin{equation}\label{c6e50}
\vec{b}_1\cdot\vec{b}_2 = 0.
\end{equation}
If the wave propagates along the $x$ direction, we can choose $\vec{b}_1$ to be
along the $y$ direction and $\vec{b}_2$ along the $z$ direction. We can then 
write
\begin{eqnarray}
E_y &=& b_1\cos(\vec{k}\cdot\vec{r} - \omega t - \alpha) \label{c6e51} \\
E_z &=& \pm b_2\sin(\vec{k}\cdot\vec{r} - \omega t - \alpha) \label{c6e52}
\end{eqnarray}
There is a $\sin$ factor in the second expression because of $i$ as a multiple
in the second term of \eqref{c6e49}. We can combine \eqref{c6e51} and 
\eqref{c6e52} to
\begin{equation}\label{c6e53}
\frac{E_y^2}{b_1^2} + \frac{E_z^2}{b_2^2} = 1.
\end{equation}
The vector $\vec{E} = \vec{E}_y\uv{y} + \vec{E}_z\uv{z}$ thus lies on the ellipse
defined by \eqref{c6e53}. As time goes by, the tip of the vector moves along the
ellipse. Such a wave is called \emph{elliptically polarised}. It is called \emph{
circularly polarised} if $b_1 = b_2$ and plane polarised of one of $b_1$ or $b_2$
is zero.

\item If we introduce a 4-vector
\begin{equation}\label{c6e54}
k^\mu = \left(\frac{\omega}{c}, \vec{k}\right)
\end{equation}
then $x^\mu = (ct, \vec{r})$ implies
\begin{equation}\label{c6e55}
k_\mu x^\mu = \left(\frac{\omega}{c}, -\vec{k}\right)\cdot(ct, \vec{r})
= \omega t =\vec{k}\cdot\vec{r}.
\end{equation}
This allows us to write the solution \eqref{c6e44} of the wave equation as 
\begin{equation}\label{c6e56}
\vec{A} = \vec{A}_0\exp(ik_\mu x^\mu).
\end{equation}
We also observe that
\begin{equation}\label{c6e57}
k_\mu k^\mu = \frac{\omega^2}{c^2} - k^2 = 0,
\end{equation}
by \eqref{c6e41}. In the pseudo-Euclidean geometry of the space-time a norm of a
vector can be zero without the vector being zero.

\item The energy-momentum tensor of the electromagnetic field of a plane wave has
the components,
\begin{eqnarray*}
T^{00} &=& \frac{E^2 + H^2}{8\pi} = W \\
T^{01} &=& \frac{E_yH_z - E_zH_y}{4\pi} = k^2A^2\\
T^{02} &=& \frac{E_zH_x - E_xH_z}{4\pi} = 0 \\
T^{03} &=& \frac{E_xH_y - E_yH_x}{4\pi} = 0 \\
T^{11} &=& \frac{E_x^2 - E_y^2 - E_z^2 + H_x^2 - H_y^2 - H_z^2}{8\pi} = W \\
T^{12} &=& -\frac{E_xE_y + H_xH_y}{4\pi} = 0\\
T^{13} &=& -\frac{E_xE_z + H_xH_z}{4\pi} = 0\\
T^{22} &=& \frac{-E_x^2 + E_y^2 - E_z^2 - H_x^2 + H_y^2 - H_z^2}{8\pi} = 0\\
T^{23} &=& -\frac{E_yE_z + H_yH_z}{4\pi} = 0 \\
T^{33} &=& \frac{-E_x^2 - E_y^2 + E_z^2 - H_x^2 - H_y^2 + H_z^2}{8\pi} = 0
\end{eqnarray*}
where we used the fact that $\vec{k} = k\uv{x}$, $\vec{H} = i\vec{k}\times\vec{A}
= i(-kA_z\uv{y} + kA_y\uv{z})$, $\vec{E} = \un\times\vec{H} = \uv{x}\times\vec{H}
= i(-kA_z\uv{z} - kA_y\uv{y})$, for a plane wave. Since, for these expressions,
\[
E^2 + H^2 = 2k^2A^2,
\]
we can write
\begin{equation}\label{c6e58}
T^{01} = k^2A^2 = \frac{E^2 + H^2}{8\pi} = W.
\end{equation}
Since $\vec{k} = k\uv{x}$, $k^\mu = (\omega/c, k, 0, 0)$ and
\begin{equation}\label{c6e59}
T^{00} = T^{01} = T^{11} = W \Rightarrow T^{\mu\nu} = 
\frac{Wc^2}{\omega^2}k^\mu k^\nu
\end{equation}

\item The Lorentz transformation for $k^\mu$ is, according to \eqref{c1e46},
\begin{eqnarray}
k^0 &=& \gamma(\bar{k}^0 + \beta\bar{k}^1) \label{c6e60} \\
k^1 &=& \gamma(\bar{k}^1 + \beta\bar{k}^0) \label{c6e61} \\
k^2 &=& \bar{k}^2 \label{c6e62} \\
k^3 &=& \bar{k}^3 \label{c6e63}
\end{eqnarray}
where $\bar{k}^\mu$ is the wave vector measured in a frame $\bar{K}$ moving
at velocity $v\uv{x}/c$ with respect to $K$. From \eqref{c6e54}, $k^0 = \omega/c$
so that \eqref{c6e60} becomes
\begin{equation}\label{c6e64}
\omega = \gamma(\bar{\omega} + c\bar{k}^1)
\end{equation}
If the wave propagates at an angle $\alpha$ with respect to the $x$ axis in 
$\bar{K}$ frame then $\bar{k}^1 = k\cos\alpha$. Using \eqref{c6e57}, we have
\begin{equation}\label{c6e65}
\bar{k}^1 = \frac{\bar{\omega}}{c}\cos\alpha
\end{equation}
so that \eqref{c6e64} becomes
\[
\omega = \gamma(\bar{\omega} + \bar{\omega}\cos\alpha) = \gamma\bar{\omega}(1 + \cos\alpha)
\]
or
\begin{equation}\label{c6e66}
\bar{\omega} = \frac{\omega\sqrt{1 - \beta^2}}{1 + \cos\alpha}.
\end{equation}
If $\alpha = \pi/2$, $\bar{\omega} = \omega\sqrt{1 - \beta^2} < \omega$. This is 
the relativistic red-shift.

\item The spectral resolution of a wave is an expression of the wave field as a
superposition of monochromatic waves. It can be done in two ways:
\begin{enumerate}
\item Express the periodic function as
\begin{equation}\label{c6e67}
f(t) = \sum_{n=-\infty}^\infty f_n e^{-i\omega_0 nt},
\end{equation}
where $\omega_0 = 2\pi/T$ and $T$ is such that $f(t + T) = f(t)$. The superposition
is written as a sum of waves of frequency $\omega_0$ and its harmonics. We can 
get the amplitudes $f_n$ using
\begin{equation}\label{c6e68}
f_n = \frac{1}{T}\int_{-T/2}^{T/2} f(t)e^{i\omega_0 nt},
\end{equation}
from which it is evident that
\begin{equation}\label{c6e69}
f_n = f_{-n}.
\end{equation}
The squared modulus of \eqref{c6e67} is
\begin{equation}\label{c6e70}
|f|^2 = \sum_{m, n = -\infty}^\infty f_n f_m^\ast e^{i\omega_0(m - n)t}.
\end{equation}
Since,
\begin{equation}\label{c6e71}
\int_{-T/2}^{T/2}e^{i\omega_0(m - n)t}dt = T\delta_{mn},
\end{equation},
\begin{equation}\label{c6e71}
\overline{|f|^2} = \frac{1}{T}\int_{-T/2}^{T/2}|f|^2 dt = 
\sum_{m, n = -\infty}^\infty f_n f_m^\ast \delta_{mn} = 
\sum_{n=-\infty}^\infty |f_n|^2.
\end{equation}
In view of \eqref{c6e69} we also have
\begin{equation}\label{c6e72}
\overline{|f|^2} = 2\sum_{n=0}^\infty |f_n|^2.
\end{equation}

\item If $f(t) \rightarrow 0$ as $t \rightarrow \infty$, one can also write
\begin{equation}\label{c6e73}
f(t) = \int_{-\infty}^\infty \hat{f}(\omega) e^{-i\omega t}\frac{d\omega}{2\pi}.
\end{equation}
The inverse of this relation is
\begin{equation}\label{c6e74}
\hat{f}(\omega) = \int_{-\infty}^\infty f(t)e^{i\omega t}dt.
\end{equation}
From \eqref{c6e74} it is immediately evident that
\begin{equation}\label{c6e75}
\hat{f}^\ast(\omega) = \hat{f}(-\omega).
\end{equation}
From \eqref{c6e74},
\[
|f(t)|^2 = \frac{1}{4\pi^2}
\iint_{-\infty}^\infty \hat{f}(\omega)\hat{f}^\ast(\omega^\op)e^{-i(\omega-\omega^\op)t}d\omega d\omega^\op
\]
and
\[
\int_{-\infty}^\infty |f(t)|^2dt = \frac{1}{4\pi^2}
\int_{-\infty}^\infty\iint_{-\infty}^\infty \hat{f}(\omega)\hat{f}^\ast(\omega^\op)e^{-i(\omega-\omega^\op)t}d\omega d\omega^\op
dt
\]
Since
\begin{equation}\label{c6e77}
\frac{1}{2\pi}\int_{-\infty}^\infty e^{-i(\omega - \omega^\op)}dt = \delta(\omega - \omega^\op)
\end{equation}
we get
\[
\int_{-\infty}^\infty |f(t)|^2dt = \frac{1}{2\pi}
\iint_{-\infty}^\infty \hat{f}(\omega)\hat{f}^\ast(\omega^\op)\delta(\omega-\omega^\op)d\omega d\omega^\op
\]
or
\begin{equation}\label{c6e78}
\int_{-\infty}^\infty |f(t)|^2dt = \frac{1}{2\pi}\int_{-\infty}^\infty |\hat{f}(\omega)|^2 d\omega.
\end{equation}
\end{enumerate}

\item A purely monochromatic wave extends all the way to infinity. There are no
such waves. Almost monochromatic waves have a frequency in a small band around a
certain value, say $\omega$. Unlike the pure monochromatic wave, whose amplitude
is a constant $\vec{E}_0$, the amplitude of an approximate monochromatic wave
is a slow varying quantity $\vec{E}_0(t)$. A pure monochromatic wave is polarised 
because $\vec{E}_0$ is constant. An approximate monochromatic wave is said to
be partially polarised.

\item Experiments studying polarisation involve measurement of intensities of
waves transmitted through media like a Nicol prism. Therefore, one studies the
quadratic functions of electric field. Functions like $E_iE_j$ or $E_i^\ast
E_j^\ast$ have a phase factor because
\begin{equation}\label{c6e79}
\vec{E}(t) = \vec{E}_0(t)e^{-i\omega t}.
\end{equation}
On the other hand, $E_iE_j^\ast$ or $E_i^\ast E_j$ will not have them. The 
long-time averages of these quantities will typically be non-zero and will be
easily observable in an experiment. Therefore, the polarisation properties
of electromagnetic waves are determined by the tensor
\begin{equation}\label{c6e80}
J_{ij} = \overline{E_{0i}{E_{0j}^\ast}},
\end{equation}
where $E_{0i}$ is the $i$-th component of the amplitude vector $\vec{E}_0(t)$.

\item The vector $\vec{E}$ is confined to a plane in the case of a plane 
electromagnetic wave. If the wave propagates along the $z$ axis then the
electric field is restricted to the $xy$-plane. Therefore, the tensor in
equation \eqref{c6e80} has only two dimensions. If we define,
\begin{equation}\label{c6e81}
J = J_{ii} = \overline{\vec{E}_0\cdot\vec{E}_0^\ast}
\end{equation}
then it is clear that $J$ is the intensity of the wave. It is not related
to its polarisation properties. We therefore factor it out from $J_{ij}$ and
define the polarisation tensor
\begin{equation}\label{c6e82}
\rho_{ij} = \frac{J_{ij}}{J}.
\end{equation}
From the definition of $J_{ij}$ it is clear that
\begin{equation}\label{c6e83}
J_{ij} = J_{ji}^\ast \text{ and } \rho_{ij} = \rho_{ji}^\ast,
\end{equation}
that is, the tensors $J_{ij}$ and $\rho_{ij}$ are hermitian and their eigen-values
are real. Note that $J$ defined in \eqref{c6e81} is also the trace of the tensor. 
Therefore,
\begin{equation}\label{c6e84}
\rho_{11} + \rho_{22} = \frac{J_{11}}{J} + \frac{J_{22}}{J} = 1.
\end{equation}
Hermiticity of $\rho_{ij}$ requires that
\begin{equation}\label{c6e85}
\rho_{21} = \rho_{12}^\ast.
\end{equation}
Equations \eqref{c6e84} and \eqref{c6e85} suggest that the tensor $\rho_{ij}$ is
of the form
\begin{equation}\label{c6e86}
\rho_{ij} = \begin{bmatrix}a & b + ic \\ b - ic & 1 - a \end{bmatrix},
\end{equation}
where $a, b, c \in \mathbb{R}$. This means that the polarisation state of a plane
electromagnetic wave can be described by three real numbers.

\item We will now examine the polarisation tensor for a variety of situations:
\begin{enumerate}
\item In the case of pure monochromatic light, $\vec{E}_0(t)$ is the constant
$\vec{E}_0$ and hence
\begin{equation}\label{c6e87}
\rho = \begin{bmatrix}|E_{01}|^2 & E_{01}E_{02}^\ast \\ 
                      E_{01}^\ast E_{02} & |E_{02}|^2 
       \end{bmatrix}
\end{equation}
In this case, 
\begin{equation}\label{c6e88}
\det\rho = |E_{01}|^2|E_{02}|^2 - E_{01}E_{02}^\ast E_{01}^\ast E_{02} = 0.
\end{equation}
This is also the case of complete polarisation.

\item The other extreme is completely unpolarised light. In this case, all
directions are equivalent and the polarisation tensor is isotropic. To ensure
that its trace is $1$, we must have
\begin{equation}\label{c6e89}
\rho_{ij} = \frac{1}{2}\delta_{ij}.
\end{equation}

\item These two extremes suggest that we can define a quantity $P$ called the
degree of polarisation as
\begin{equation}\label{c6e90}
\det\rho = \frac{1}{4}(1 - P^2)
\end{equation}
so that $P = 1$ for fully polarised, monochromatic light and $P = 0$ for 
unpolarised light.
\end{enumerate}

\item A tensor like $\rho_{ij}$ can be split into a symmetric and an anti-symmetric
part such that
\begin{equation}\label{c6e91}
\rho_{ij} = S_{ij} + A_{ij},
\end{equation}
where
\begin{eqnarray}
S_{ij} &=& \frac{1}{2}(\rho_{ij} + \rho{ji}) \label{c6e92} \\
A_{ij} &=& \frac{1}{2}(\rho_{ij} - \rho{ji}). \label{c6e93}
\end{eqnarray} 
The hermiticity of $\rho$ results in 
\begin{equation}\label{c6e94}
A_{ij} = \begin{bmatrix}0 & \rho_{12} - \rho_{21} \\
-(\rho_{21} - \rho_{12}) & 0
\end{bmatrix} = (\rho_{12} - \rho_{12}^\ast)\begin{bmatrix}0 & 1 \\ -1 & 0 \end{bmatrix}
\end{equation}
Now, $(\rho_{12} - \rho_{12}^\ast)$ is pure imaginary. Let 
\begin{equation}\label{c6e95}
A = i(\rho_{12} - \rho_{12}^\ast)
\end{equation}
so that we can write \eqref{c6e91} as
\begin{equation}\label{c6e96}
\rho_{ij} = S_{ij} - \frac{i}{2}Ae_{ij},
\end{equation}
where
\begin{equation}\label{c6e97}
e_{ij} = \begin{bmatrix}0 & 1 \\ -1 & 0 \end{bmatrix}
\end{equation}
is the fully anti-symmetric tensor.
\begin{enumerate}
\item For a circularly polarised wave, $b_1 = b_2$ in \eqref{c6e53} so that from
equations \eqref{c6e51} and \eqref{c6e52} we get $E_{02} = \pm i E_{01}$. The 
polarisation tensor is
\begin{equation}\label{c6e98}
\rho_{ij} = \begin{bmatrix}|E_{01}|^2 & \mp i |E_{01}|^2 \\
\pm i |E_{01}|^2 & |E_{01}|^2
\end{bmatrix}\frac{1}{|E_{01}|^2} = \frac{1}{2}\delta_{ij} - \frac{\pm i}{2}e_{ij}
\end{equation}
so that $S_{ij} = \delta_{ij}/2$ and $A = \pm 1$.

\item If the wave is linearly polarised, one of $b_1$ or $b_2$ in \eqref{c6e53}
is zero. From equations \eqref{c6e51} and \eqref{c6e52} it is evident that we can
write $\vec{E}_0$ as a real vector. Therefore, $A = 0$.

\item This suggests that the constant $A$ can be viewed as a degree of circularity
in the polarisation. $A \in [-1, 1]$, taking extreme values for left and right 
circular polarised waves and middle value for plane polarised ones.
\end{enumerate}

\item A real symmmetric matrix $S_{ij}$ can be diagonalised. Let $\lambda_1$ and
$\lambda_2$ be the two eigenvalues and let the corresponding eigenvectors 
correspond to directions $\un^{(1)}$ and $\un^{(2)}$. Then we can write
\begin{equation}\label{c6e99}
S_{ij} = \lambda_1 n^{(1)}_in^{(1)}_j + \lambda_2 n^{(2)}_in^{(2)}_j,
\end{equation}
and $0 \le \lambda_1, \lambda_1 \le 1$. Each of the two terms is a product of
quantities related to one of the eigen-vectors. If $A = 0$, then $\rho_{ij} = 
S_{ij}$, which means that each term of $\rho_{ij}$ is a sum of two terms, each
one of which being a product of one of the two eigen-vectors. The two parts can
be treated as independent of each other. Such a wave is called \emph{incoherent}.

\item Let $\phi$ be the angle between $\un^{(1)}$ and the $x$-axis. Then we can 
write
\begin{eqnarray}
\un^{(1)} &=& (\cos\phi, \sin\phi) \label{c6e100} \\
\un^{(2)} &=& (-\sin\phi, \cos\phi) \label{c6e101}
\end{eqnarray}
If $\lambda_1 > \lambda_2$, let
\begin{equation}\label{c6e102}
l = \lambda_1 - \lambda_2 > 0.
\end{equation}
The matrix $S$ can now be written as
\begin{equation}\label{c6e103}
S = \frac{1}{2}\begin{bmatrix}
1 + l\cos(2\phi) & l\sin(2\phi) \\
l\sin(2\phi) & 1 - l\cos(2\phi)
\end{bmatrix}
\end{equation}
The three numbers $A, l$ and $\phi$ describe the polarisation state of the wave.
One can replace them with
\begin{eqnarray}
\xi_1 &=& l\sin(2\phi) \label{c6e104} \\
\xi_2 &=& A \label{c6e105} \\
\xi_3 &=& l\cos(2\phi) \label{c6e106}
\end{eqnarray}
so that the polarisation tensor can be written as
\begin{equation}\label{c6e107}
\rho = \frac{1}{2}\begin{bmatrix}
1 + \xi_3 & \xi_1 - i\xi_2 \\
\xi_1 + i\xi_2 & 1 - \xi_3
\end{bmatrix}.
\end{equation}
The three numbers $\xi_1, \xi_2, \xi_3$ are called Stokes parameters. In this form,
\begin{equation}\label{c6e108}
\det\rho = \frac{1}{4}(1 - \xi_1^2 - \xi_2^2 - \xi_3^2),
\end{equation}
so that the degree of polarisation $P$ defined in \eqref{c6e90} becomes
\begin{equation}\label{c6e109}
P = \sqrt{\xi_1^2 + \xi_2^2 + \xi_3^2}.
\end{equation}
\end{enumerate}
\chapter{The propagation of light}\label{c7}
\begin{enumerate}
\item A plane wave is the one for which the fields are a function only of $t - 
x/c$ if they are propagating along the $x$ axis. Its direction of propagation 
and amplitude are the same everywhere.

\item Oftentimes, one can treat arbitrary waves as plane waves in a small enough
region of space. This is possible only if the amplitude and direction remain 
practically the same over a single wavelength. If this is indeed true then one 
can introduce the idea of a \emph{wave surface} or a \emph{wave front}. It is an
imaginary surface at all points of which the wave has a constant phase. Wave 
fronts of plane waves are indeed planes.

In a tiny region of space a wave propagates along the normal to the wave front.
The normals of consecutive wave fronts create a \emph{ray}. Alternatively, a ray
is a curve whose tangent at every point coincides with the normal to the wave
front.

\item When the dimensions of the region of interest are large compared to the
wavelength, we can study the propagation of waves by paying attention only to
the rays. This is the domain of \emph{geometrical optics}. It is an approximation
in which we ignore the wave properties and consider only rays. Geometrical optics
works best for waves whose wavelengths are very small.

\item If $f$ is any quantity describing the wave then for a plane monochromatic 
wave,
\begin{equation}\label{c7e1}
f = a\exp(-i(k_\mu x^\mu + \alpha)),
\end{equation}
where $\alpha$ is the constant phase of the wave. The 4-vector $k^\mu$ is defined
by equation \eqref{c6e54}. Strictly speaking, we should have written \eqref{c7e1}
as
\[
f = a\re{\exp(-i(k_\mu x^\mu + \alpha))},
\]
but we will now assume that every equation with complex exponentials is 
interpreted by ignoring its imaginary part.

If 
\begin{equation}\label{c7e2}
\Psi = -k_\mu x^\mu + \alpha
\end{equation}
then $f = a\exp(i\Psi)$. It is called the \emph{eikonal}.

\item If we are dealing with waves whose wave fronts are not planes then their
amplitude is a function of $x^\mu$ and the expression for $\Psi$ is not as 
simple as in \eqref{c7e2}.

\item Whatever be the form of $\Psi$, the approximation of geometrical optics
is applicable only if $\Psi$ is practically constant over a single wave-length, 
or
\begin{equation}\label{c7e3}
\frac{\lambda}{\Psi} \rightarrow 0.
\end{equation} 
In regions whose extent is of the order of magnitude of $\lambda$, one can write
the eikonal as 
\begin{equation}\label{c7e4}
\Psi = \Psi(0) + x^\mu\pdt{\Psi}{x^\mu}.
\end{equation}
Comparing with \eqref{c7e2}, we see that
\begin{equation}\label{c7e5}
k_\mu = -\pdt{\Psi}{x^\mu}.
\end{equation}
From \eqref{c6e57}, $k_\mu k^\mu = 0$ so that we also have
\begin{equation}\label{c7e6}
\pdt{\Psi}{x^\mu}\pdt{\Psi}{x_\mu} = 0.
\end{equation}
Equation \eqref{c7e6} is called the \emph{eikonal equation}.

\item Since $f = a\exp(i\Psi)$ satisfies the wave equation, 
\[
\frac{\partial^2 f}{\partial x_\mu \partial x^\mu} = 0.
\]
Now,
\[
\pdt{f}{x_\mu} = \pdt{a}{x_\mu}\exp(i\Psi) + ia\exp(i\Psi)\pdt{\Psi}{x^\mu} =
\pdt{a}{x_\mu}\exp(i\Psi) + if\pdt{\Psi}{x_\mu}
\]
so that
\begin{eqnarray*}
\frac{\partial^2 f}{\partial x_\mu \partial x^\mu} &=& 
    \frac{\partial^2 a}{\partial x_\mu \partial x^\mu}e^{i\Psi} + 
 	i\pdt{a}{x_\mu}\exp(i\Psi)\pdt{\Psi}{x^\mu} + 
	i\pdt{f}{x^\mu}\pdt{\Psi}{x^\mu} + \\
 & & if\frac{\partial^2 \Psi}{\partial x_\mu \partial x^\mu} \\
 &=& \frac{\partial^2 a}{\partial x_\mu \partial x^\mu}e^{i\Psi} + 
    2i\pdt{a}{x_\mu}\exp(i\Psi)\pdt{\Psi}{x^\mu} -
	f\pdt{\Psi}{x^\mu}\pdt{\Psi}{x_\mu} + \\
 & &  if\frac{\partial^2 \Psi}{\partial x_\mu \partial x^\mu}
\end{eqnarray*}
The real and imaginary parts of this quantity should vanish independently. 
Therefore,
\begin{eqnarray}
\frac{1}{a}\frac{\partial^2 a}{\partial x_\mu \partial x^\mu} &=& 
  \pdt{\Psi}{x^\mu}\pdt{\Psi}{x_\mu} \label{c7e7} \\
\frac{2}{a}\pdt{a}{x_\mu}\exp(i\Psi)\pdt{\Psi}{x^\mu} &=& 
\frac{\partial^2 \Psi}{\partial x_\mu \partial x^\mu} \label{c7e8} 
\end{eqnarray}
In the domain of geometrical optics we also assume that the second derivative of
$a$ with respect to $x^\mu$ can be ignored so that the first of the previous pair
of equations gives the eikonal equation.

The book mentions that largeness of $\Psi$ leads one to that conclusion but I 
unable to see it that way.

Alternatively, since a small patch of an arbitrary wave front can be considered
to be a plane, and since for a plane wave $a$ is constant, the eikonal equation
follows immediately from \eqref{c7e7}.

\item We can expand the eikonal equation \eqref{c7e6} as
\[
\frac{1}{c^2}\left(\pdt{f}{t}\right)^2 - \left(\pdt{f}{x}\right)^2 - 
\left(\pdt{f}{y}\right)^2 - \left(\pdt{f}{z}\right)^2 = 0
\]
and compare it with the Hamilton-Jacobi equation \eqref{c2e35}
\[
\frac{1}{c^2}\left(\pdt{S}{t}\right)^2 - \left(\pdt{S}{x}\right)^2 
- \left(\pdt{S}{y}\right)^2 - \left(\pdt{S}{z}\right)^2 = m^2c^2
\]
to suspect an analogy between the action $S$ of point particle with the eikonal
$f$ of a ray and the ray itself being a massless point particle. We also recall
that the action is related to the Hamiltonian and the momemtum as (refer to
\eqref{c2e27}),
\begin{eqnarray}
\mathcal{H} &=& -\pdt{S}{t} \label{c7e9} \\
\vec{p} &=& \pdt{S}{\vec{r}} \label{c7e10}
\end{eqnarray}
If we expand \eqref{c7e4} and \eqref{c7e2} we get
\begin{eqnarray*}
\Psi &=& \Psi(0) + t\pdt{\Psi}{t} - \vec{r}\cdot\pdt{\Psi}{\vec{r}} \\
\Psi &=& \alpha - \omega t + \vec{r}\cdot\vec{k}
\end{eqnarray*}
Comparing the rhs of previous two equations we get
\begin{eqnarray}
\omega &=& -\pdt{\Psi}{t} \label{c7e11} \\
\vec{k} &=& \pdt{\Psi}{\vec{r}} \label{c7e12}
\end{eqnarray}
Comparing the pair of equations \eqref{c7e9} and \eqref{c7e10} with the pair
\eqref{c7e11} and \eqref{c7e12} we extend the analogy $\Psi \leftrightarrow S$
to $\omega \leftrightarrow \mathcal{H}$ and $\vec{k} \leftrightarrow \vec{p}$.
It was left to the development of quantum mechanics to transform this analogy
into fundamental equations of physics.

\item Hamilton equations for the particles are
\begin{eqnarray*}
\td{\vec{p}}{t} &=& -\pdt{\mathcal{H}}{\vec{r}} \\
\td{\vec{r}}{t} &=& +\pdt{\mathcal{H}}{\vec{p}}.
\end{eqnarray*}
Their analogues to geometrical optics are
\begin{eqnarray}
\td{\vec{k}}{t} &=& -\pdt{\omega}{\vec{r}} \label{c7e13} \\
\td{\vec{r}}{t} &=& +\pdt{\omega}{\vec{k}} \label{c7e14}
\end{eqnarray}
In vacuum, $\omega = ck$ so that the first equation gives $\dot{\vec{k}} = 0$ 
while the second one gives $\vec{v} = \dot{\vec{r}} = c\un$, $\un$ being a unit
vector in the direction of propagation of the ray. Neither of these results are 
surprising but the analogy is quite effective when extended to propagation of 
light in material media.

\item From the analogy $\omega \leftrightarrow \mathcal{H}$ and $\vec{k} 
\leftrightarrow \vec{p}$, we can write the Lagrangian for a ray as
\[
\mathcal{L} = \vec{k}\cdot\pdt{\omega}{\vec{k}} - \omega.
\]
However, $\omega = ck$ gives $\mathcal{L} = 0$ identically. Therefore, we cannot
extend the principle of least action for particles in the form
\[
\delta\int Ldt = 0
\]
to geometrical optics. However, when the energy of a particle is constant the
principle of least action is equivalent to Maupertuis principle
\[
\delta\int\vec{p}\cdot d\vec{q} = 0.
\]
The analogue of Maupertuis principle to geometrical optics is
\begin{equation}\label{c7e15}
\delta\int\vec{k}\cdot d\vec{l} = 0.
\end{equation}
In vacuum, $\omega\un = c\vec{k}$ so that the equation simplifies to
\begin{equation}\label{c7e16}
\delta\int dl = 0,
\end{equation}
which is Euler's principle of least time.

\item Geometrical optics deals only with rays. It considers the propagation of a
bundle or a pencil of rays. The rays determine the direction of propagation and 
give no information about the intensity. If rays start from a certain region then
after a certain duration $\delta t$ the points $c\delta t$ away from the starting
region define a surface called the wave-front. An infinitesimal portion of the
wave-front has two principal radii of curvatures, say $R_1$ and $R_2$. In 
general, they are not the same. Therefore, the points on two principle circles 
of curvature appear to emerge from different points. This is the cause of the 
aberration of astigmatism.

The area of the wave-front is proportional to $R_1R_2$ while the intensity is
inversely proportional to it. Thus,
\begin{equation}\label{c7e17}
I \propto \frac{1}{R_1R_2}.
\end{equation}

Note that this formula, in particular the two radii of curvature, describe only 
the chosen infinitesimal surface. Therefore \eqref{c7e17} is not applicable to
areas illuminated by a different pencil of rays.

The intensity of light is proportional to the squared modulus of the field.
Therefore, the field itself is given by
\begin{equation}\label{c7e18}
f \propto \frac{1}{\sqrt{R_1R_2}} e^{ikR},
\end{equation}
where $R$ is any one of $R_1$ or $R_2$. Since $R_1$ and $R_2$ are fixed, the only
difference between the formulae using them is the constant phase 
$\exp(ik(R_1-R_2))$, which can be readily absorbed in the constant of 
proportionality.

In the special case of the two radii of curvature being the same, the wavefront
is a sector of a sphere and two centres of curvature coincide. 

$I$ blows up with either $R_1$ or $R_2$ vanish. Therefore the centres of 
curvature are points of infinite intensity. The locus of all centres of 
curvatures define a caustic surface. Refer to 
\href{https://physics.stackexchange.com/a/256561/10236}{a StackExchange answer} 
for an excellent explanation of the how caustics are formed and the rays are 
tangential to it.

\item \emph{Optical systems} are transparent bodies through which light travels.
In general, the direction of travel is changed when light passes through optical
systems. The laws governing the change of direction of rays are similar to those
governing the change of direction of particles when they pass through 
electromagnetic (and gravitational) fields.

\item We observed in point 7 above that the eikonal equation \eqref{c7e6} is
equivalent to \eqref{c7e7} with lhs zero (which is probably the reason why the 
differential equation $|\nabla u|^2 = 1$ is called the eikonal equation). Thus,
the eikonal $\Psi$ satisfies
\[
\pdt{\Psi}{x^\mu}\pdt{\Psi}{x_\mu} = 0
\]
or
\[
\frac{1}{c^2}\left(\pdt{\Psi}{t}\right)^2 - \left(\pdt{\Psi}{x}\right)^2 - 
\left(\pdt{\Psi}{y}\right)^2 - \left(\pdt{\Psi}{z}\right)^2 = 0.
\]
For a monochromatic wave, $\Psi = -\omega t + \psi_0(\vec{r})$ so that
\[
\pdt{\Psi}{t} = \omega
\]
and hence the eikonal equation becomes
\[
\frac{\omega^2}{c^2} = |\nabla\psi_0|^2.
\]
Define $\psi = c\psi_0/\omega$ so that the eikonal equation for monochromatic
ray becomes
\begin{equation}\label{c7e19}
|\nabla\psi|^2 = 1.
\end{equation}
The solution of \eqref{c7e19} gives the eikonal $\psi$. The negative gradient 
of the eikonal gives the direction of motion. This fact follows from the 
observation that the eikonal of a plane, monochromatic wave is $\omega t - 
\vec{k}\cdot\vec{r}$, whose gradient is $-\vec{k}$.

\item The eikonal is the phase of the rays passing through a definite point.
While studying optical systems one is more interested in the rays passing
through a pair of points. The corresponding eikonal is $\psi(\vec{r}, 
\vec{r}^\op)$ and it is the phase difference of the ray between the points
$\vec{r}$ and $\vec{r}^\op$. For a fixed $\vec{r}^\op$, the eikonal $\psi
(\vec{r}, \vec{r}^\op)$ describes all rays passing through $\vec{r}^\op$. Since
the eikonal must satisfy \eqref{c7e19}, we must have
\begin{equation}\label{c7e20}
|\nabla\psi|^2 = 1 \text{ and } |\nabla^\op\psi|^2 = 1,
\end{equation}
where $\nabla^\op$ denotes gradient with respect to the primed coordinates. 
The direction of rays getting out of $\vec{r}$ is $-\nabla\psi$ and that of
rays getting into $\vec{r}^\op$ is $-(-\nabla^\op\psi) = \nabla^\op\psi$. The
eikonal equation guarantees that these vectors have unit magnitude. They may,
therefore, be denoted as $\un$ and $\un^\op$ respectively.

\item The four vectors $\vec{r}, \vec{r}^\op, \un, \un^\op$ are related by
the requirement that $\un$ and $\un^\op$ are gradients of a certain function
of $\vec{r}$ and $\vec{r}^\op$ which also satisfy equations \eqref{c7e20}. In
order to obtain a relation between the four vectors it is convenient to 
introduce another function $\chi$, called the \emph{angular eikonal}, which is
a Legendre transform of $\psi$. Since $\psi$ is a function of $\vec{r}$ and
$\vec{r}^\op$,
\begin{eqnarray*}
d\psi &=& \nabla\psi\cdot d\vec{r} + \nabla^\op\psi\cdot d\vec{r}^\op \\
 &=& -\un\cdot d\vec{r} + \un\cdot d\vec{r}^\op \\
 &=& -d(\un\cdot d\vec{r}) + \vec{r}\cdot d\un + d(\un\cdot d\vec{r}^\op) -
     \vec{r}^\op d\un^\op
\end{eqnarray*}
so that
\[
d(\psi + \un\cdot\vec{r} - \un^\op\cdot\vec{r}^\op) = 
\vec{r}\cdot d\un - \vec{r}^\op d\un^\op.
\]
The function,
\begin{equation}\label{c7e21}
\chi(\un, \un^\op) = \psi + \un\cdot\vec{r} - \un^\op\cdot\vec{r}^\op
\end{equation}
is the angular eikonal. The function $\chi$ is not required to satisfy any
differential equation. However, its arguments have to be of unit magnitude.
Therefore, the six variables on which it depends are not independent and two
of them can be expressed in terms of others as
\begin{eqnarray}
n_x &=& \sqrt{1 - n_y^2 - n_z^2} \label{c7e22} \\
n_x^\op &=& \sqrt{1 - {n_y^\op}^2 - {n_z^\op}^2} \label{c7e23}.
\end{eqnarray}
From these equations we have
\begin{eqnarray}
dn_x &=& -\frac{n_y}{n_x}dn_y - \frac{n_z}{n_x}dn_z \label{c7e24} \\
dn_x^\op &=& -\frac{n_y^\op}{n_x^\op}dn_y^\op - 
              \frac{n_z^\op}{n_x^\op}dn_z^\op \label{c7e25}
\end{eqnarray}
so that 
\begin{eqnarray}
d\chi &=& -\left(y - \frac{n_y}{n_x}x\right) dn_y 
  -\left(z - \frac{n_z}{n_x}x\right) dn_z \nonumber \\
 & & -\left(y^\op - \frac{n_y^\op}{n_x^\op}x\right) dn_y^\op 
  -\left(z^\op - \frac{n_z^\op}{n_x^\op}x\right) dn_z^\op \label{c7e26}
\end{eqnarray}
From these equations, it follows that
\begin{eqnarray}
\pdt{\chi}{n_y} &=& -\left(y - \frac{n_y}{n_x}x\right) \label{c7e27} \\
\pdt{\chi}{n_z} &=& -\left(z - \frac{n_z}{n_x}x\right) \label{c7e28} \\
\pdt{\chi}{n_y^\op} &=& \left(y-\frac{n_y^\op}{n_x^\op}x\right) \label{c7e29}\\
\pdt{\chi}{n_z^\op} &=& \left(z-\frac{n_z^\op}{n_x^\op}x\right) \label{c7e30}
\end{eqnarray}
The angular eikonal $\chi$ depends on the properties of the optical system. For
fixed values of $\un$ and $\un^\op$ these are equations of straight lines. They
are the incident and refracted rays. Equations \eqref{c7e27} to \eqref{c7e29}
determine the path of light ray passing through an optical system described by
the angular eikonal $\chi$.

\item A bundle of rays passing through a single point is said to be 
homocentric. In general, a homocentric bundle ceases to be so after passing 
through an optical system except in the trivial case of a plane mirror. A
homocentric bundle remains approximately homocentric for sufficiently narrow
beam travelling close to a particular line for a given optical system. This 
line is called the \emph{optic axis} and the rays travelling close to it are
called \emph{paraxial rays}.

\item Even arbitrarily narrow bundles of rays are not homocentric, in general.
Consider an infinitesimally small wave surface of an arbitrarily narrow bundle
of rays. In general, it will have two radii of curvature and will be crossed by
two principle circles. Rays passing through each of them will pass through its
centre of curvature. Only when the wave surface is spherical will two centres of
curvature merge and the bundle of rays be homocentric.

\item A surface of revolution about the optic axis will have identical radii of
curvature for points on the axis. This follows from the smoothness of the
surface of revolution. A smooth surface will approximately a paraboloid in the 
neighbourhood of the optic axis. By symmetry, the two radii of curvature will be
equal. This fact can be rigorously proved using the techniques of differential
geometry.

\item We will now use equations \eqref{c7e27} to \eqref{c7e30} for determining
image formation by paraxial rays. For sake of definiteness, let us align the $x$
axis along the optic axis. Since the rays are almost along the optic axis, the
$x$ component of $\un$ and $\un^\op$ are very large as compared to the other
components. We can, therefore, expand $\chi$ around the origin. Since $\chi$ has
to describe a system symmetric about the $x$ axis, it must be even-powered in its
arguments. To lowest order in the arguments, we can express $\chi$ as
\begin{equation}\label{c7e31}
\chi = \text{const.} + \frac{g}{2}(n_y^2 + n_z^2) + f(n_yn_y^\op+n_zn_z^\op)
+ \frac{h}{2}({n_y^\op}^2 + {n_z^\op}^2).
\end{equation}
Substituting it in equations \eqref{c7e27} to \eqref{c7e30}, we get
\begin{eqnarray*}
y &=& n_y\left(\frac{x}{n_x} - g\right) - fn_y^\op \\
z &=& n_z\left(\frac{x}{n_x} - g\right) - fn_z^\op \\
y^\op &=& n_y^\op\left(\frac{x^\op}{n_x^\op} + h\right) + fn_y \\
z^\op &=& n_z^\op\left(\frac{x^\op}{n_x^\op} + h\right) + fn_z
\end{eqnarray*}
The component $n_x^\op \approx 1$ for a lens and $\approx -1$ for a 
mirror. If we consider the optical system to be a lens then the above equations 
become
\begin{eqnarray}
y &=& n_y(x - g) - fn_y^\op \label{c7e32} \\
z &=& n_z(x - g) - fn_z^\op \label{c7e33} \\
y^\op &=& n_y^\op(x^\op + h) + fn_y \label{c7e34} \\
z^\op &=& n_z^\op(x^\op + h) + fn_z \label{c7e35}
\end{eqnarray}

\item Consider a homocentric bundle of rays starting from $(x, y, z)$ and
converging at $(x^\op, y^\op, z^\op)$ after passing through the optical system.
If the group of equations \eqref{c7e32} to \eqref{c7e35} were independent then
we would get exactly one $n_y, n_z, n_y^\op, n_z^\op$ which would describe only
one ray and not a whole bundle of them. In order to accommodate the entire
bundle we assume that the equations are not independent. That is, we assume that
the ratio of coefficients of like terms are equal. This gives us,
\begin{equation}\label{c7e36}
\frac{x - g}{f} = -\frac{f}{x^\op + h} = \frac{y}{y^\op} = \frac{z}{z^\op}
\end{equation}
one consequence of which is
\begin{equation}\label{c7e37}
(x - g)(x^\op + h) = -f^2.
\end{equation}

\end{enumerate}

\nocite{*}
%\bibliographystyle{plain}
%\bibliography{ctof}
\end{document}  
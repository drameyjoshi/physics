\chapter{The Electromagnetic Field Equations}\label{c4}
\begin{enumerate}
\item From equations \eqref{c3e19} and \eqref{c3e20} it follows that
\begin{eqnarray}
\curl\vec{E} &=& -\frac{1}{c}\pdt{\vec{H}}{t} \label{c4e1} \\
\dive\vec{H} &=& 0 \label{c4e2}
\end{eqnarray}
These equations can be easily cast in integral form as 
\begin{eqnarray}
\oint\vec{E}\cdot d\vec{l} &=& -\frac{1}{c}
	\frac{\partial}{\partial t}\int\vec{H}\cdot d\vec{f} \label{c4e3} \\
\oint\vec{H}\cdot d\vec{f} &=& 0 \label{c4e4}
\end{eqnarray}
Note that, in equation \eqref{c4e3}, the surface integral on the rhs is over the
surface bounded by the countour along which the line integral of lhs is 
calculated. The line integral is over a closed curve but the surface integral 
is not over a closed surface. The equation states that the circulation of 
the electric field is equal to $-1/c$ times the flux of $\vec{H}$ through 
the surface bound by the countour. On the other hand, \eqref{c4e4} tells that 
flux of $\vec{H}$ through any closed surface is zero.

\item Equations \eqref{c4e1} and \eqref{c4e2} form the first pair of Maxwell
equations. They do not describe the electromagnetic field completely because
they involve the time derivative of $\vec{H}$ alone and not $\vec{E}$.

\item It is possible to write equations \eqref{c4e1} and \eqref{c4e2} in terms
of the electromagnetic field tensor $F_{\mu\nu}$. Since
\[
F_{\mu\nu} = \pdt{A_\nu}{x^\mu} - \pdt{A_\mu}{x^\nu},
\]
we have
\begin{eqnarray*}
\pdt{F_{\mu\nu}}{x^\rho} &=& 
\pdt{A^2_\nu}{x^\mu x^\rho} - \pdt{A^2_\mu}{x^\nu x^\rho} \\
\pdt{F_{\nu\rho}}{x^\mu} &=& 
\pdt{A^2_\rho}{x^\nu x^\mu} - \pdt{A^2_\nu}{x^\rho x^\mu} \\
\pdt{F_{\rho\mu}}{x^\nu} &=& 
\pdt{A^2_\mu}{x^\rho x^\nu} - \pdt{A^2_\rho}{x^\mu x^\nu}
\end{eqnarray*}
from which we get
\begin{equation}\label{c4e5}
\pdt{F_{\mu\nu}}{x^\rho} + \pdt{F_{\nu\rho}}{x^\mu} + 
\pdt{F_{\rho\mu}}{x^\nu} = 0.
\end{equation}

\item We will examine \eqref{c4e5} in greater details.
\begin{enumerate}
\item $\mu = 0, \nu = 1, \rho = 2$:
\[
\pdt{F_{01}}{x^2} + \pdt{F_{12}}{x^0} + \pdt{F_{20}}{x^1} = 0 \Rightarrow
\pdt{E_y}{x} - \pdt{E_x}{y} = - \frac{1}{c}\pdt{H_z}{t}.
\]

\item $\mu = 0, \nu = 1, \rho = 3$:
\[
\pdt{F_{01}}{x^3} + \pdt{F_{13}}{x^0} + \pdt{F_{30}}{x^1} = 0 \Rightarrow
\pdt{E_x}{z} - \pdt{E_z}{x} = - \frac{1}{c}\pdt{H_y}{t}.
\]

\item $\mu = 0, \nu = 2, \rho = 3$:
\[
\pdt{F_{02}}{x^3} + \pdt{F_{23}}{x^0} + \pdt{F_{30}}{x^2} = 0 \Rightarrow
\pdt{E_z}{y} - \pdt{E_y}{z} = - \frac{1}{c}\pdt{H_x}{t}.
\]

\item $\mu = 1, \nu = 2, \rho = 3$:
\[
\pdt{F_{12}}{x^3} + \pdt{F_{23}}{x^1} + \pdt{F_{31}}{x^2} = 0 \Rightarrow
-\pdt{H_z}{z} - \pdt{H_x}{x} - \pdt{H_y}{y} = 0.
\]
\end{enumerate}
Thus, \eqref{c4e5} encodes the two Maxwell equations \eqref{c4e1} and 
\eqref{c4e2}.
We also note that
\begin{enumerate}
\item If any two indices in \eqref{c4e5} are equal then we get the identity 
$0=0$.
\item If all three indices are equal then each term on lhs of \eqref{c4e5} is
zero.
\item Let us examine what happens when we consider other combinations of 
indices. If we swap the values of $\mu$ and $\nu$ in case (a), that is, if 
$\mu = 1, \nu = 0, \rho = 2$ then we have
\[
\pdt{F_{10}}{x^2} + \pdt{F_{02}}{x^1} + \pdt{F_{21}}{x^0} = 0 \Rightarrow
-\pdt{E_x}{y} + \pdt{E_y}{x} - \frac{1}{c}\pdt{H_z}{t} = 0,
\]
which is same as the conclusion of case (a). We can similarly show that all 
swaps give 
\[
\curl\vec{E} = -\frac{1}{c}\pdt{\vec{H}}{t}.
\]
\end{enumerate}

\item Now consider the expression,
\begin{equation}\label{c4e6}
C^\mu = \epsilon^{\mu\nu\rho\sigma}\pdt{F_{\rho\sigma}}{x^\nu}.
\end{equation}
Then
\[
C^0 = \pdt{F_{23}}{x^1} - \pdt{F_{32}}{x^1} - \pdt{F_{13}}{x^2} + 
\pdt{F_{31}}{x^2} + \pdt{F_{12}}{x^3} - \pdt{F_{21}}{x^3}
\]
which, owing to anti-symmetric nature of $F_{\mu\nu}$ is
\[
C^0 = 2\left(\pdt{F_{23}}{x^1} + \pdt{F_{31}}{x^2} + \pdt{F_{12}}{x^3}\right)
\]
From equation \eqref{c4e5}, it is $C^0 = 0$. Similarly we can show that the 
other components too vanish. Therefore, equation \eqref{c4e5} is equivalent to
\begin{equation}\label{c4e7}
\epsilon^{\mu\nu\rho\sigma}\pdt{F_{\rho\sigma}}{x^\nu} = 0.
\end{equation}

\item We next develop the action function for a system consisting of particles
in an electromagnetic field. From equations \eqref{c2e3} and \eqref{c2e6}, the
action for a free particle is
\begin{equation}\label{c4e8}
S_m = -mc\int ds.
\end{equation}
If there are many particles, we can write it as
\begin{equation}\label{c4e9}
S_m = -\sum_i m_ic\int ds.
\end{equation}
The subscript `m' indicates that the action depends only of the mass of the 
particles and therefore pertains to free particles alone. When the particles
interact with the field, we need an additional term,
\begin{equation}\label{c4e10}
S_{mf} = -\sum_i \frac{e_i}{c}\int A_\mu dx^\mu.
\end{equation}
It is a mild generalisation of the second term on the rhs of \eqref{c3e2}.
We need a term $S_f$ that determines the behaviour of the fields in absence of
matter so that the action of the system is
\begin{equation}\label{c4e11}
S = S_m + S_{mf} + S_f.
\end{equation}

It is an experimental fact that the electromagnetic fields obey the principle of
superposition. Therefore, the differential equations describing them must be 
linear.  Since the equations are derived from the variational principle, we 
require that the integrand of the action integral must be quadratic in the field
tensor. Furthermore, we require the action integral to be a scalar. Therefore, 
we also require the integrand to be a scalar. The only scalar, quadratic 
quantity that depends on the field tensor alone is $F_{\mu\nu}F^{\mu\nu}$. 
We propose that
\begin{equation}\label{c4e12}
S_f = a\int F_{\mu\nu}F^{\mu\nu} d\Omega.
\end{equation}

The quantity $\epsilon^{\mu\rho\nu\sigma}F^{\mu\rho}F^{\nu\sigma}$ is not 
considered to be part of $S_f$ because it is a pseudo-scalar. We showed in 
\eqref{c3e98} that it is $\vec{E}\cdot\vec{H}$. Since $\vec{H}$ is an axial 
vector, the dot product is a pseudo-scalar.

Equation \eqref{c3e97} gave us
\begin{equation}\label{c4e13}
F_{\mu\nu}F^{\mu\nu} = 2(H^2 - E^2).
\end{equation}
In terms of the potentials,
\begin{equation}\label{c4e14}
F_{\mu\nu}F^{\mu\nu} = |\curl\vec{A}|^2 - |\grad\varphi|^2 + 
2\grad\varphi\pdt{\vec{A}}{t} - \left|\pdt{\vec{A}}{t}\right|^2.
\end{equation}
This shows that $F_{\mu\nu}F^{\mu\nu}$ can be made arbitrarily negative if 
$\vec{A}$ can be made to vary arbitrarily quickly with $t$. In that case, we 
cannot minimise the action integral of \eqref{c4e12}. To prevent that from 
happening, we require the constant $a$ to be negative. Its value depends on 
our choice of units. In gaussian units, it is 
\[
a = -\frac{1}{16\pi c}
\]
so that the action for the field is
\begin{equation}\label{c4e15}
S_f = -\frac{1}{16\pi c}\int F_{\mu\nu}F^{\mu\nu} d\Omega.
\end{equation}
Using \eqref{c4e13},
\begin{equation}\label{c4e16}
S_f = \frac{1}{8\pi}\int (E^2 - H^2) dt dV,
\end{equation}
where we used the fact that $d\Omega = cdtdV$. We can also write it as
\begin{equation}\label{c4e17}
S_f = \int dt \left(\frac{1}{8\pi}\int dV (E^2 - H^2)\right) = \int dt L_f,
\end{equation}
where
\begin{equation}\label{c4e18}
L_f = \frac{1}{8\pi}\int dV (E^2 - H^2) = \int dV \mathcal{L}_f.
\end{equation}
The quantity,
\begin{equation}\label{c4e19}
\mathcal{L}_f = \frac{1}{8\pi}(E^2 - H^2)
\end{equation}
is called the Lagrangian density of the field and $L_f$ the Lagrangian of the 
field.

\item It is convenient to introduce charge density and treat discrete charges in
terms of it. For example,
\begin{equation}\label{c4e20}
\rho = \sum_a e_a\delta(\vec{r} - \vec{r}_a)
\end{equation}
and
\begin{equation}\label{c4e21}
de = \rho dV.
\end{equation}
From equation \eqref{c4e21} we have
\begin{equation}\label{c4e22}
de dx^\mu = \rho dV dx^\mu = \rho dV dt \td{x^\mu}{t}.
\end{equation}
We call the quantity,
\begin{equation}\label{c4e23}
j^\mu = \rho\td{x^\mu}{t}
\end{equation}
as the \emph{current density}. Clearly, as 
\[
\td{x^\mu}{t} = (c, \vec{v})
\]
we have
\begin{equation}\label{c4e24}
j^\mu = (\rho c, \rho\vec{v}) = (\rho c, \vec{j})
\end{equation}
where
\begin{equation}\label{c4e25}
\vec{j} = \rho\vec{v}
\end{equation}
is the usual current density 3-vector.

\item This is a bit tricky. Consider the surface integral,
\[
\int j^\mu dS_\mu.
\]
If the surface its normal along the $x^0$ axis then $j^\mu dS_\mu = j^0 dV$ and
\[
\int j^\mu dS_\mu = \int j^0 dV = c\int \rho dV = ce,
\]
where $e$ is the total charge enclosed by the hypersurface. From this equation
we can write
\begin{equation}\label{c4e26}
\int\rho dV = \frac{1}{c}\int j^\mu dS_\mu,
\end{equation}
where the surface integral is over a hypersurface whose normal is along the 
$x^0$ axis. For an arbitrary hypersurface,
\[
\frac{1}{c}\int j^\mu dS_\mu
\]
is just the sum of charges whose world lines pass through it.

\item From equations \eqref{c4e9}, \eqref{c4e10}, \eqref{c4e11} and 
\eqref{c4e15}, the action of the system of charged particles in 
electromagnetic fields is
\[
S = -\sum_i m_ic\int ds - 
\sum_i\frac{e_i}{c}\int A_\mu dx^\mu - \frac{1}{16\pi c}
\int F_{\mu\nu} F^{\mu\nu} d\Omega.
\]
The second term is
\[
S_{mf} = -\frac{1}{c}\sum_i e_i\left(\int A_0dx^0 + \cdots + 
\int A_3dx^3\right).
\]
For a continuous charge distribution, we replace $\sum_i e_i$ by $\int \rho dV$
so that
\begin{eqnarray*}
S_{mf} &=& -\frac{1}{c}\int dV \rho\left(\int A_0dx^0 + \cdots + 
\int A_3dx^3\right) \\
 &=& -\frac{1}{c}\left(\int dVdx^0 \frac{\rho c}{c}A_0 + \cdots + 
 \int dVdt \rho\td{x^3}{t}A_3\right) \\
 &=& -\frac{1}{c}\left(\int dVdx^0 \frac{\rho c}{c}A_0 + \cdots + 
 \int dVdx^0 \frac{\rho}{c}\td{x^3}{t}A_3\right) \\
 &=& -\frac{1}{c^2}\left(\int d\Omega j^0A_0 + \cdots + 
 	\int d\Omega j^3A_3\right)
\end{eqnarray*}
where we used equation \eqref{c4e23} in the last equation. Finally, we write it
as
\begin{equation}\label{c4e27}
S_{mf} = -\frac{1}{c^2}\int j^\mu A_\mu d\Omega.
\end{equation}
Thus, the action can be written as
\begin{equation}\label{c4e28}
S = -\sum_i m_ic\int ds - \frac{1}{c^2}\int j^\mu A_\mu d\Omega - 
\frac{1}{16\pi c}\int F_{\mu\nu} F^{\mu\nu} d\Omega.
\end{equation}

\item It is an experimental fact that electric charge is conserved. It is 
expressed as an equation of continuity,
\begin{equation}\label{c4e29}
\frac{\partial}{\partial t}\int\rho dV = -\oint \rho\vec{v}\cdot d\vec{f}.
\end{equation}
The left hand side, if it is positive, is the rate of increase of charge per 
unit time in the region of integration. The right hand side is the rate of 
inflow of charges in the region of integration. The negative sign is needed 
because $d\vec{f}$ always points outwards so that if $\vec{v}$ points inwards 
then the dot product is negative. To get an overall positive sign, we negate 
the surface integral. In differential form, \eqref{c4e29} is
\begin{equation}\label{c4e30}
\pdt{\rho}{t} + \dive\vec{j} = 0.
\end{equation}
We can also write it as
\[
\pdt{\rho}{t} + \pdt{j^1}{x^1} + \pdt{j^2}{x^2} + \pdt{j^3}{x^3} = 0.
\]
Since $j^\mu = (c\rho, \vec{j})$, this equation is equivalent to
\begin{equation}\label{c4e31}
\pdt{j^\mu}{x^\mu} = 0.
\end{equation}
This is the four-dimensional form of the equation of continuity.

\item Recall equation \eqref{c4e26} in which
\[
\frac{1}{c}\int j^\mu dS_\mu
\]
is the total charge at a time $x^0/c$ and the integral is over a hypersurface 
normal to the $x^0$ axis at that time. Consider the integral
\[
\frac{1}{c}\oint j^\mu dS_\mu,
\]
where the closed surface consists of the hypersurface of instant of time, a
similar hypersurface at a later instant of time and the two joined by surface
parallel to $x^0$ axis far away from the charge distribution. The integral over
this latter surface is zero because $j^\mu$ is zero on it. Therefore,
\begin{equation}\label{c4e32}
\frac{1}{c}\oint j^\mu dS_\mu = 0.
\end{equation}
Using the divergence theorem, we can transform the surface integral to a volume
integral so that
\[
\frac{1}{c}\int \pdt{j^\mu}{x^\mu} = 0.
\]
Since this is true for any two points on the $x^0$ axis, the divergence itself
is zero. This is another way of proving \eqref{c4e31}.

\item We showed in chapter \ref{c3} that the transformation described by 
equation
\eqref{c3e30},
\[
A_\mu^\op = A_\mu - \pdt{f}{x^\mu},
\]
also called the gauge transformation, leaves the fields unchanged. Let us make 
this transformation in the second term, \eqref{c4e27}, of the action 
\eqref{c4e28}.  Then,
\[
S_{mf} \rightarrow -\frac{1}{c^2}\int j^\mu A_\mu d\Omega + \frac{1}{c^2}
\int j^\mu\pdt{f}{x^\mu} d\Omega.
\]
Now,
\[
\frac{\partial}{\partial x^\mu}(fj^\mu) = f\pdt{j^\mu}{x^\mu} + 
j^\mu\pdt{f}{x^\mu}.
\]
If the equation of continuity, \eqref{c4e31}, is valid then
\begin{equation}\label{c4e33}
\frac{\partial}{\partial x^\mu}(fj^\mu) = j^\mu\pdt{f}{x^\mu}
\end{equation}
and
\[
S_{mf} \rightarrow -\frac{1}{c^2}\int j^\mu A_\mu d\Omega + \frac{1}{c^2}
\int \frac{\partial}{\partial x^\mu}(fj^\mu) d\Omega.
\]
Using divergence theorem on the second term,
\[
S_{mf} \rightarrow -\frac{1}{c^2}\int j^\mu A_\mu d\Omega + \frac{1}{c^2}
\int (fj^\mu) dS_\mu.
\]
We can always choose the surface to be large enough that $j^\mu$ is zero on it,
in which case, the integral vanishes. This manipulation critically depends on 
the validity of \eqref{c4e33}, which in turn depends on the equation of 
continuity.  This shows that the gauge invariance of the electromagnetic 
fields is closely related to the equation of continuity, in other words, to 
conservation of electric charge. Electric charge must be conserved if gauge 
invariance must be true.

\item In finding the equations of motion from variational principle we assumed
that the fields are fixed while varying the generalised coordinates. Now, we 
will assume that the generalised coordinates are fixed and will vary the fields
instead. The resulting equations are the second pair of Maxwell equations. The 
expression for total action in \eqref{c4e28} has fields only in the second and 
the third terms. Therefore,
\begin{equation}\label{c4e34}
\delta S = -\frac{1}{c^2}\int j^\mu \delta A_\mu d\Omega - \frac{1}{16\pi c}
\int\delta(F_{\mu\nu}F^{\mu\nu})d\Omega.
\end{equation}
Now 
\begin{eqnarray*}
\delta(F_{\mu\nu}F^{\mu\nu}) &=& (\delta F_{\mu\nu})F^{\mu\nu} + 
F_{\mu\nu} (\delta F^{\mu\nu}) \\
 &=& 2F^{\mu\nu}\delta F_{\mu\nu} \\
 &=& 2F^{\mu\nu}\left(\pdt{\delta A_\nu}{x^\mu} - 
 \pdt{\delta A_\mu}{x^\nu}\right) \\
 &=& 2F^{\mu\nu}\pdt{\delta A_\nu}{x^\mu} - 
 	2F^{\mu\nu}\pdt{\delta A_\mu}{x^\nu} \\
 &=& 2F^{\nu\mu}\pdt{\delta A_\mu}{x^\nu} - 
 	2F^{\mu\nu}\pdt{\delta A_\mu}{x^\nu} \\
 &=& -2F^{\mu\nu}\pdt{\delta A_\mu}{x^\nu} - 
 	2F^{\mu\nu}\pdt{\delta A_\mu}{x^\nu} \\
 &=& -4F^{\mu\nu}\pdt{\delta A_\mu}{x^\nu}
\end{eqnarray*}
so that equation \eqref{c4e34} becomes
\begin{equation}\label{c4e35}
\delta S = -\frac{1}{c^2}\int j^\mu \delta A_\mu d\Omega + \frac{1}{4\pi c}
\int F^{\mu\nu}\pdt{\delta A_\mu}{x^\nu} d\Omega
\end{equation}
Since 
\[
\frac{\partial}{\partial x^\nu}(F^{\mu\nu} \delta A_\mu) =
F^{\mu\nu}\pdt{\delta A_\mu}{x^\nu} + \delta A_\mu\pdt{F^{\mu\nu}}{x^\nu},
\]
\begin{eqnarray}
\delta S &=& -\frac{1}{c^2}\int j^\mu \delta A_\mu d\Omega + 
\frac{1}{4\pi c}\int \frac{\partial}{\partial x^\nu}(F^{\mu\nu} \delta A_\mu) 
d\Omega - \nonumber \\
 & & \frac{1}{4\pi c}\int\delta A_\mu\pdt{F^{\mu\nu}}{x^\nu}d\Omega\nonumber \\
 &=& -\frac{1}{c}\left(\frac{1}{c}\int j^\mu \delta A_\mu d\Omega + 
    \frac{1}{4\pi} \int\delta A_\mu\pdt{F^{\mu\nu}}{x^\nu}d\Omega\right) + 
	\nonumber \\
 & & \frac{1}{4\pi c}\int F^{\mu\nu}\delta A_\mu dS_\mu, \nonumber 
\end{eqnarray}
where we used the generalisation of Gauss' theorem to 4-dimensions to convert a
volume integral to a surface integral. We can always choose the surface to be so
large that there are no fields on it. As a result the last term can be ignored
and we are left with
\begin{equation}\label{c4e36}
\delta S = -\frac{1}{c}\int\left(\frac{j^\mu}{c} +
 \frac{1}{4\pi}\pdt{F^{\mu\nu}}{x^\nu}\right)\delta A_\mu d\Omega
\end{equation}
If $\delta S = 0$ for arbitrary variations of the fields then we must have
\begin{equation}\label{c4e37}
\pdt{F^{\mu\nu}}{x^\nu} = -\frac{4\pi}{c}j^\mu.
\end{equation}
This equation includes the remaining two Maxwell equations. For $\mu=0$, it is
\[
\pdt{F^{00}}{x^0} + \pdt{F^{01}}{x^1} + \pdt{F^{02}}{x^2} + \pdt{F^{03}}{x^3}
= -\frac{4\pi}{c}j^0 = -4\pi\rho.
\]
From \eqref{c3e77}, $F^{01} = -E_1, F^{02} = -E_2, F^{03} = -E_3$ so that the
above equation is
\begin{equation}\label{c4e38}
\dive\vec{E} = 4\pi\rho.
\end{equation}
For $\mu = 1$, \eqref{c4e37} becomes,
\[
\pdt{F^{10}}{x^0} + \pdt{F^{11}}{x^1} + \pdt{F^{12}}{x^2} + \pdt{F^{13}}{x^3}
= -\frac{4\pi}{c}j^1 = -\frac{4\pi}{c}j^1
\]
that is,
\[
\frac{1}{c}\pdt{E^1}{t} + 0 - \pdt{H^3}{x_2} + \pdt{H^2}{x_3} = 
-\frac{4\pi}{c}j^1
\]
or
\begin{equation}\label{c4e39}
(\curl\vec{H})_x = \frac{4\pi}{c}j_x + \frac{1}{c}\pdt{E_x}{t}.
\end{equation}
The other components can be derived similarly. We thus get,
\begin{equation}\label{c4e40}
\curl\vec{H} = \frac{4\pi}{c}\vec{j} + \frac{1}{c}\pdt{\vec{E}}{t}.
\end{equation}
Equations \eqref{c4e1}, \eqref{c4e2}, \eqref{c4e38} and \eqref{c4e40}, are the
four Maxwell equations. In terms of field vectors they are
\begin{eqnarray*}
\curl\vec{E} &=& -\frac{1}{c}\pdt{\vec{H}}{t} \\
\dive\vec{H} &=& 0 \\
\dive\vec{E} &=& 4\pi\rho \\
\curl\vec{H} &=& \frac{4\pi}{c}\vec{j} + \frac{1}{c}\pdt{\vec{E}}{t}.
\end{eqnarray*}
They can be equivalently written in terms of the electromagnetic field tensor as
equations \eqref{c4e5} and \eqref{c4e37}.
\begin{eqnarray*}
\pdt{F_{\mu\nu}}{x^\rho} + \pdt{F_{\nu\rho}}{x^\mu} + 
\pdt{F_{\rho\mu}}{x^\nu} &=& 0 \\
\pdt{F^{\mu\nu}}{x^\nu} &=& -\frac{4\pi}{c}j^\mu
\end{eqnarray*}

\item Equation \eqref{c4e38} in integral form is
\begin{equation}\label{c4e41}
\oint\vec{E}\cdot d\vec{f} = 4\pi\int\rho dV.
\end{equation}
The total electric flux over a closed surface is equal to $4\pi$ times the total
charge enclosed in it. This is \emph{Gauss law}.

\item Equation \eqref{c4e40} in integral form is
\begin{equation}\label{c4e42}
\oint\vec{H}\cdot d\vec{l} = \frac{4\pi}{c}\left(\int\vec{j}\cdot d\vec{f} + 
\frac{1}{4\pi}\int\pdt{\vec{E}}{t}\cdot d\vec{f}\right).
\end{equation}
The first term in the bracket on the rhs is the true current while the second 
one is called the `displacement current'. The term,
\[
\frac{1}{4\pi}\pdt{\vec{E}}{t}
\]
is called the \emph{displacement current density}.

\item From \eqref{c4e40}, one immediately gets, after taking divergence of both 
sides and using \eqref{c4e38},
\[
\dive\vec{j} + \pdt{\rho}{t} = 0.
\]
Thus the equation of continuity and charge conversation are encoded in 
\eqref{c4e37}. It can also be easily derived taking the derivative of 
\eqref{c4e37} with respect to $x^\mu$,
\[
\frac{\partial^2 F^{\mu\nu}}{\partial x^\mu \partial x^\nu} = 
-\frac{4\pi}{c}\pdt{j^\mu}{x^\mu}.
\]
The lhs can be written as
\[
\frac{\partial^2 F^{\mu\nu}}{\partial x^\mu \partial x^\nu} = 
\frac{\partial^2 F^{\nu\mu}}{\partial x^\nu \partial x^\mu}
= -\frac{\partial^2 F^{\mu\nu}}{\partial x^\nu \partial x^\mu} = 
-\frac{\partial^2 F^{\mu\nu}}{\partial x^\mu \partial x^\nu}.
\]
The last step follows from the equality of mixed partial derivatives and the 
second last from the anti-symmetry of $F^{\mu\nu}$. Thus the lhs has to be zero
so that, we have,
\[
\pdt{j^\mu}{x^\mu} = 0,
\]
which is indeed the equation of continuity, \eqref{c4e31}, in 4-vector form.

\item Taking dot product of \eqref{c4e1} with $\vec{H}$ and that of 
\eqref{c4e40} with $\vec{E}$, we get
\begin{eqnarray*}
\vec{H}\cdot\curl\vec{E} &=& -\frac{1}{2c}\pdt{H^2}{t} \\
\vec{E}\cdot\curl\vec{H} &=& \frac{4\pi}{c}\vec{j}\cdot\vec{E} + 
\frac{1}{2c}\pdt{E^2}{t}
\end{eqnarray*}
Subtracting the first equation from the second,
\begin{equation}\label{c4e43}
\vec{E}\cdot\curl\vec{H} - \vec{H}\cdot\curl\vec{E} = 
\frac{4\pi}{c}\vec{j}\cdot\vec{E}
+ \frac{1}{2c}\frac{\partial}{\partial t}(E^2 + H^2).
\end{equation}
We can rearrange it as
\begin{equation}\label{c4e44}
\frac{\partial}{\partial t}\left(\frac{E^2 + H^2}{8\pi}\right) = 
-\vec{j}\cdot\vec{E} - 
\frac{c}{4\pi}(\vec{H}\cdot\curl\vec{E} - \vec{E}\cdot\curl\vec{H}).
\end{equation}
The \emph{Poynting vector}, is defined as
\begin{equation}\label{c4e45}
\vec{S} = \frac{c}{4\pi}(\vec{E} \times \vec{H})
\end{equation}
so that \eqref{c4e44} can be written as
\begin{equation}\label{c4e46}
\frac{\partial}{\partial t}\left(\frac{E^2 + H^2}{8\pi}\right) = 
-\vec{j}\cdot\vec{E} - \dive\vec{S}.
\end{equation}

\item From equation \eqref{c3e27}, we have
\[
\td{\mathcal{E}}{t} = e\vec{v}\cdot\vec{E}
\]
so that for a collection of charges, we have
\[
\sum_a e_a\vec{v}_a\cdot\vec{E} = \td{\mathcal{E}}{t}
\]
or, in the continuum limit,
\begin{equation}\label{c4e47}
\int\vec{j}\cdot\vec{E}dV = \td{\mathcal{E}}{t}.
\end{equation}
If we integrate \eqref{c4e46} over a volume, $V$, we have
\begin{equation}\label{c4e48}
\frac{d}{dt}\int\frac{E^2 + H^2}{8\pi}dV + 
\int \vec{j}\cdot\vec{E}dV = -\oint d\vec{f}\cdot\vec{S}.
\end{equation}
If the volume of integration is so large that $\vec{S}$ is effectively zero on
its bounding surface then, using \eqref{c4e47},
\begin{equation}\label{c4e49}
\frac{d}{dt}\left(\int\frac{E^2 + H^2}{8\pi}dV + \mathcal{E}\right) = 0.
\end{equation}
This equation suggests that for a closed system of charges and currents,
the quantity
\[
\int\frac{E^2 + H^2}{8\pi}dV + \mathcal{E}
\]
is conserved. Since $\mathcal{E}$ is the kinetic energy of the charges,
the integral must be the energy of the fields. We all the quantity,
\begin{equation}\label{c4e50}
W = \frac{E^2 + H^2}{8\pi}
\end{equation}
the energy density of the field. For a small enough volume for which the rhs of
\eqref{c4e48} does not vanish, we interpret the rhs as the flux of energy 
escaping the bounding surface. The Poynting vector $\vec{S}$ is thus the flux 
density of energy passing through a surface.

\item Consider an arbitrary relativistic field with generalised coordinates 
$q_\alpha$ and described by a Lagrangian density,
\[
\Lambda\left(q_\alpha, \pdt{q_\alpha}{x^\mu}\right)
\]
so that its action is
\begin{equation}\label{c4e51}
S = \int \Lambda\left(q_\alpha, \pdt{q_\alpha}{x^\mu}\right)dV dt = \frac{1}{c}
\int\Lambda\left(q_\alpha, \pdt{q_\alpha}{x^\mu}\right)d\Omega.
\end{equation}
Let us introduce the notation
\begin{equation}\label{c4e52}
q_{\alpha; \mu} = \pdt{q_\alpha}{x^\mu}
\end{equation}
to simplify our expressions. A variation of the action is
\begin{equation}\label{c4e53}
\delta S = \frac{1}{c}\int\left(\pdt{\Lambda}{q_\alpha}\delta q_\alpha +
\pdt{\Lambda}{q_{\alpha;\mu}}\delta q_{\alpha;\mu}\right)d\Omega.
\end{equation}
Now,
\[
\frac{\partial}{\partial x^\mu}\left(\pdt{\Lambda}{q_{\alpha;\mu}} 
\delta q_\alpha\right) = 
\pdt{\Lambda}{q_{\alpha;\mu}}\delta q_{\alpha;\mu} + 
\delta q_\alpha\frac{\partial}
{\partial x^\mu}\pdt{\Lambda}{q_{\alpha;\mu}}
\]
so that \eqref{c4e53} becomes,
\begin{equation}\label{c4e54}
\delta S = \frac{1}{c}\int\left(\left(\pdt{\Lambda}{q_\alpha} - \frac{\partial}
{\partial x^\mu}\pdt{\Lambda}{q_{\alpha;\mu}}\right)\delta q_\alpha +
\frac{\partial}{\partial x^\mu}\left(\pdt{\Lambda}{q_{\alpha;\mu}} 
\delta q_\alpha \right)\right)d\Omega.
\end{equation}
The second term in the integrand is a 4-divergence. The volume integral of a 4-
divergence can be written as a surface integral. If we choose the surface large 
enough so that the field is zero on it, the surface integral vanishes and we are
left with
\begin{equation}\label{c4e55}
\delta S = \frac{1}{c}\int \left(\pdt{\Lambda}{q_\alpha} - \frac{\partial}
{\partial x^\mu}\pdt{\Lambda}{q_{\alpha;\mu}}\right)\delta q_\alpha d\Omega.
\end{equation}
Since this equation is true for all variations $\delta q_\alpha$, we have the 
Euler-Lagrange equations
\begin{equation}\label{c4e56}
\pdt{\Lambda}{q_\alpha} = \frac{\partial}{\partial x^\mu}
\pdt{\Lambda}{q_{\alpha;\mu}}
\end{equation}
Recall that $\Lambda$ is a function of $q_\alpha$ and $q_{\alpha;\mu}$ so that
\[
d\Lambda = \pdt{\Lambda}{q_\alpha}dq_\alpha + 
\pdt{\Lambda}{q_{\alpha;\mu}}dq_{\alpha;\mu}
\]
and
\begin{equation}\label{c4e57}
\pdt{\Lambda}{x^\nu} = \pdt{\Lambda}{q_\alpha}\pdt{q_\alpha}{x^\nu} + 
\pdt{\Lambda}{q_{\alpha;\mu}}\pdt{q_{\alpha;\mu}}{x^\nu}
\end{equation}
Since,
\[
\pdt{q_{\alpha;\mu}}{x^\nu} = 
\frac{\partial^2 q_\alpha}{\partial x^\mu \partial x^\nu}
= \frac{\partial^2 q_\alpha}{\partial x^\nu \partial x^\mu} =
\pdt{q_{\alpha;\nu}}{x^\mu},
\]
equation \eqref{c4e57} becomes
\[
\pdt{\Lambda}{x^\nu} = \pdt{\Lambda}{q_\alpha}\pdt{q_\alpha}{x^\nu} + 
\pdt{\Lambda}{q_{\alpha;\mu}}\pdt{q_{\alpha;\nu}}{x^\mu}.
\]
Using the Euler-Lagrange equation \eqref{c4e56} in the first term on the rhs,
\[
\pdt{\Lambda}{x^\nu} = \frac{\partial}{\partial x^\mu}
\pdt{\Lambda}{q_{\alpha;\mu}}
q_{\alpha;\nu} + \pdt{\Lambda}{q_{\alpha;\mu}}\pdt{q_{\alpha;\nu}}{x^\mu} =
\frac{\partial}{\partial x^\mu}
\left(\pdt{\Lambda}{q_{\alpha;\mu}}q_{\alpha;\nu}\right)
\]
We can write the lhs as
\[
\pdt{\Lambda}{x^\nu} = \delta^\mu_\nu\pdt{\Lambda}{x^\mu} = 
\frac{\partial}{\partial x^\mu}\left(\delta^\mu_\nu \Lambda\right)
\]
so that we have
\begin{equation}\label{c4e58}
\frac{\partial}{\partial x^\mu}
\left(\pdt{\Lambda}{q_{\alpha;\mu}}q_{\alpha;\nu} - 
\delta^\mu_\nu \Lambda\right) = 0.
\end{equation}
If
\begin{equation}\label{c4e59}
\tensor{T}{_\nu^\mu} = q_{\alpha;\nu}\pdt{\Lambda}{q_{\alpha;\mu}} - 
\tensor{\delta}{_\nu^\mu} \Lambda,
\end{equation}
then we can write \eqref{c4e58} as
\begin{equation}\label{c4e60}
\pdt{\tensor{T}{_\nu^\mu}}{x^\mu} = 0.
\end{equation}
Since $T^{\nu\mu} = \eta^{\mu\nu}\tensor{T}{_\nu^\mu}$ and $\eta^{\mu\nu}$ 
being a constant tensor, equation \eqref{c4e60} can also be written as
\begin{equation}\label{c4e61}
\pdt{T^{\nu\mu}}{x^\mu} = 0.
\end{equation}
Since
\[
\int \pdt{T^{\nu\mu}}{x^\mu} d\Omega = \int T^{\nu\mu} dS_\mu,
\]
equation \eqref{c4e61} also gives us
\begin{equation}\label{c4e62}
\oint T^{\nu\mu} dS_\mu = 0.
\end{equation}
If we choose the surface to be such that it is composed of a hyperplane 
perpendicular to $x^0$ at one instant of time, another hyperplane 
perpendicular to $x^0$ at another instant of time and an arbitrary surface
joining these hyperplanes at a great distance from the fields, then 
\eqref{c4e62} indicates that the vector
\begin{equation}\label{c4e63}
p^\nu = a\int T^{\nu\mu}dS_\mu,
\end{equation}
where $a$ is a constant, has the same value at the two instants of $x^0$ 
involved in constructing the closed surface. In other words, $p^\nu$ is a 
constant. For an appropriately chosen value of $a$, $p^\nu$ becomes the 
4-momentum density of the system.

\item If $\nu = 0$, \eqref{c4e63} becomes
\[
p^0 = a\int T^{0\mu}dS_\mu,
\]
If the integration is over the hyperplane $x^0$ then $T^{0\mu}dS_\mu = 
T^{00}dV$. Thus,
\[
p^0 = a\int T^{00}dV.
\]
Now, 
\[
T^{00} = \eta^{00}\tensor{T}{_0^0} = q_{\alpha;0}\pdt{\Lambda}{q_{\alpha;0}} - 
\tensor{\delta}{_0^0} \Lambda
\]
as $\eta^{00} = 1$. Furthermore, 
\[
q_{\alpha; 0} = \pdt{q_\alpha}{x^0} = \frac{1}{c}\dot{q}_\alpha
\]
so that
\[
T^{00} = \dot{q}_\alpha\pdt{\Lambda}{\dot{q}_\alpha} - 
\tensor{\delta}{_0^0} \Lambda.
\]
The rhs of this equation is a Legendre transformation. Therefore, it is the
Hamiltonian density or the energy density of the system. In analogy with 
\eqref{c2e26}, the constant $a$ can be chosen to be $1/c$. Thus, the correct 
expression for the 4-momentum density is
\begin{equation}\label{c4e64}
p^\nu = \frac{1}{c}\int T^{\nu\mu}dS_\mu.
\end{equation}

\item Consider the tensor
\begin{equation}\label{c4e65}
U^{\mu\nu} = T^{\mu\nu} + \pdt{\psi^{\mu\nu\sigma}}{x^\sigma}
\end{equation}
where $\psi^{\mu\nu\sigma}=-\psi^{\mu\sigma\nu}$, that is $\psi^{\mu\nu\sigma}$
is an anti-symmetric tensor of rank 3. Because of anti-symmetry,
\[
\frac{\partial^2\psi^{\mu\nu\sigma}}{\partial x^\mu \partial x^\sigma} = 
\frac{\partial^2\psi^{\mu\nu\sigma}}{\partial x^\sigma \partial x^\mu} = 
\frac{\partial^2\psi^{\mu\sigma\nu}}{\partial x^\sigma \partial x^\mu} = 
-\frac{\partial^2\psi^{\mu\nu\sigma}}{\partial x^\mu \partial x^\sigma}
\]
so that
\begin{equation}\label{c4e66}
\frac{\partial^2\psi^{\mu\nu\sigma}}{\partial x^\mu \partial x^\sigma} = 0.
\end{equation}
Now consider,
\begin{equation}\label{c4e67}
\int U^{\mu\nu} dS_\mu = \int T^{\mu\nu} dS_\mu + 
\int\pdt{\psi^{\mu\nu\sigma}}{x^\sigma} dS_\mu.
\end{equation}
We manipulate the second term on the rhs using equation \eqref{c1e77} for an 
anti-symmetric tensor to get
\begin{equation}\label{c4e68}
\int U^{\mu\nu} dS_\mu = \int T^{\mu\nu} dS_\mu + 
\frac{1}{2}\int df^\ast_{\mu\sigma}\psi^{\mu\nu\sigma}.
\end{equation}
The first two integrals in \eqref{c4e68} are over three dimensional 
hypersurfaces, that is ordinary volumes while the last integral is over a two 
dimensional surface. If we let the surface be large enough that the fields on 
it are zero then the last term will vanish and we are left with
\begin{equation}\label{c4e69}
\int U^{\mu\nu} dS_\mu = \int T^{\mu\nu} dS_\mu,
\end{equation}
that is, the 4-momentum density is the same for both energy-momentum tensors. 
Thus the energy-momentum tensor is defined to within the partial derivative of 
an anti-symmetric tensor of rank 3.

\item From equation \eqref{c2e86} the angular momentum of the field can be 
written as
\begin{equation}\label{c4e70}
L^{\mu\nu} = \int (x^\mu dp^\nu - x^\nu dp^\mu).
\end{equation}
From \eqref{c4e64}, we have $dp^\mu = T^{\mu\nu}dS_\nu$ so that we can write 
\eqref{c4e70} as 
\begin{equation}\label{c4e71}
L^{\mu\nu} = \frac{1}{c}\int (x^\mu T^{\nu\rho} - x^\nu T^{\mu\rho})dS_\rho.
\end{equation}
If the integration is taken such that $dS_\rho$ is normal to the hypersurface of
all space coordiates at at a certain value of $x^0$ then $L^{\mu\nu}$ is the
angular momentum of the system at time $x^0/c$. Now,
\[
L^{\mu\nu}(t_1) - L^{\mu\nu}(t_2) = 
\frac{1}{c}\oint(x^\mu T^{\nu\rho} - x^\nu T^{\mu\rho})dS_\rho,
\]
where the integral is over the closed hypersurface comprising of:
\begin{itemize}
\item The hypersurface normal to the $x^0$ axis at time $t_1$,
\item The hypersurface normal to the $x^0$ axis at time $t_2$ and
\item A hypersurface at a great distance from the fields joining these 
hyperplanes.
\end{itemize}
From the 4-dimensional Gauss theorem of \eqref{c1e74},
\begin{equation}\label{c4e72}
L^{\mu\nu}(t_1) - L^{\mu\nu}(t_2) = 
\frac{1}{c}\int\frac{\partial}{\partial x^\rho}(x^\mu T^{\nu\rho} - 
x^\nu T^{\mu\rho})d\Omega.
\end{equation}
The integrand is
\begin{eqnarray*}
\frac{\partial}{\partial x^\rho}(x^\mu T^{\nu\rho} - x^\nu T^{\mu\rho}) &=&
\tensor{\delta}{_\rho^\mu}T^{\nu\rho} + x^\mu\pdt{T^{\nu\rho}}{x^\rho} - 
\tensor{\delta}{_\rho^\nu}T^{\mu\rho} - x^\nu\pdt{T^{\mu\rho}}{x^\rho} \\
 &=& T^{\nu\mu} + x^\mu\pdt{T^{\nu\rho}}{x^\rho} - T^{\mu\nu} - 
 x^\nu\pdt{T^{\mu\rho}}{x^\rho}
\end{eqnarray*}
From equation \eqref{c4e61},
\[
\pdt{T^{\nu\rho}}{x^\rho} = 0; \;\; \pdt{T^{\mu\rho}}{x^\rho} = 0
\]
so that equation \eqref{c4e72} is
\begin{equation}\label{c4e73}
L^{\mu\nu}(t_1) - L^{\mu\nu}(t_2) = \frac{1}{c}\int (T^{\nu\mu} - 
T^{\mu\nu})d\Omega.
\end{equation}
If $L^{\mu\nu}(t_1) = L^{\mu\nu}(t_2)$ then we must have $T^{\mu\nu} = 
T^{\nu\mu}$. 

\item We once again choose to the surface of integration in \eqref{c4e64} as the
one whose normal is along the $x^0$ axis so that
\[
p^\nu = \frac{1}{c}\int T^{\nu\mu}dS_0 = \frac{1}{c}\int T^{\nu 0}dV.
\]
Since $p^\nu$ is the 4-momentum of the system, we call $W = T^{00}$ the energy
density and $T^{01}/c, T^{02}/c, T^{03}/c$ as the momentum density of the 
system.

\item Consider the conservation equation \eqref{c4e61} for $\nu = 0$. It is
\[
\frac{1}{c}\pdt{T^{00}}{t} + \pdt{T^{0i}}{x^i} = 0.
\]
If we integrate it over a volume $V$, 
\[
\frac{1}{c}\frac{\partial}{\partial t}\int_V T^{00}dV = -\int\pdt{T^{0i}}{x^i}dV
\]
Using the Gauss theorem in 3-dimensions on the rhs we get
\[
\frac{\partial}{\partial t}\int_V T^{00}dV = -c\oint T^{0i}df_i.
\]
Thus, the components $cT^{0i}$ are the also the flux of energy through the 
closed surface bounding the volume $V$. In the previous point we identified 
$T^{0i} = cp^i$. Thus, $c^2p^i$ can be identified as components of flux of 
energy.

\item Now consider the conservation equation \eqref{c4e62} for the space
components,
\[
\frac{1}{c}\pdt{T^{0i}}{t} + \pdt{T^{ij}}{x^j} = 0.
\]
Repeating the manipulations in the previous point, we get
\[
\frac{\partial}{\partial t}\int_V \frac{T^{0i}}{c} dt = -\oint T^{ij}df_j.
\]
However, $T^{0i}/c = p^i$, so that
\[
\frac{\partial}{\partial t}\int_V p^i dt = -\oint T^{ij}df_j.
\]
The lhs is the rate of change of momentum in the volume. Therefore, the rhs must
be the total force on it. The force is expressed as a surface integral. 
Therefore, we identify the integrand as the negative of the stress tensor 
$\sigma^{ij}$. The stress tensor is thus a flux of momentum. We can write the 
above equation as
\begin{equation}\label{c4e74}
\frac{\partial}{\partial t}\int_V p^i dt = \oint \sigma^{ij}df_j.
\end{equation}

$T^{ij}$ is the flux of $p^i$ through surface perpendicular to the $x^j$ axis. 
In similar vain, $\sigma^{ij}$ is the normal force on a surface perpendicular to
the $x^i$ axis and the $i$-th component of the shear force acting in the $x^j$ 
direction.

The physical nature of the energy-momentum tensor reveals itself if we write 
it as
\begin{equation}\label{c4e75}
T^{\mu\nu} = \begin{bmatrix}
W & p_x & p_y & p_z \\
p_x & -\sigma_{xx} & -\sigma_{xy} & -\sigma_{xz} \\
p_y & -\sigma_{yx} & -\sigma_{yy} & -\sigma_{yz} \\
p_z & -\sigma_{zx} & -\sigma_{yy} & -\sigma_{zz}
\end{bmatrix}.
\end{equation}
Note that, $p_x, p_y, p_z$ are components of the 3-momentum for which we do not
distinguish between covariant and contravariant components. Likewise for the
components of the stress tensor in three dimensions.

\item We will now apply the theory developed for a general Lagrangian density
in points 19 onwards to the electromagnetic field for which
\begin{equation}\label{c4e76}
\Lambda(A^\mu) = -\frac{1}{16\pi}F^{\mu\nu}F_{\mu\nu}.
\end{equation}
To get an expression for the stress tensor of \eqref{c4e59}, we need the 
derivative of $\Lambda$ with respect to 
\[
A_{\alpha; \mu} = \pdt{A_\alpha}{x^\mu}.
\]
A tedious and error prone method could be write down all $16$ terms on the rhs 
of \eqref{c4e78}. A more convenient method is to get a variation of $\Lambda$. 
That is,
\begin{equation}\label{c4e77}
\delta\Lambda = -\frac{1}{16\pi}\left((\delta F^{\mu\nu})F_{\mu\nu} + 
F^{\mu\nu}(\delta F_{\mu\nu})\right) = 
-\frac{1}{8\pi}F^{\mu\nu}\delta F_{\mu\nu}.
\end{equation}
However,
\begin{equation}\label{c4e78}
\delta F_{\mu\nu} = \delta(\partial_\mu A_\nu - \partial_\nu A_\mu)
= \delta\partial_\mu A_\nu - \delta\partial_\nu A_\mu.
\end{equation}
so that
\begin{eqnarray}
\delta\Lambda &=& -\frac{1}{8\pi}F^{\mu\nu}(\delta\partial_\mu A_\nu - 
	\delta\partial_\nu A_\mu) \nonumber \\
8\pi\delta\Lambda &=& -F^{\mu\nu}\delta\partial_\mu A_\nu + 
	F^{\mu\nu}\delta\partial_\nu A_\mu \nonumber \\
 &=& -F^{\mu\nu}\delta\partial_\mu A_\nu + F^{\nu\mu}\delta\partial_\mu 
 	A_\nu \nonumber \\
 &=& -F^{\mu\nu}\delta\partial_\mu A_\nu - F^{\mu\nu}\delta\partial_\mu A_\nu 
 \nonumber \\
 &=& -2F^{\mu\nu}\delta\partial_\mu A_\nu \label{c4e79}
\end{eqnarray}
so that
\begin{equation}\label{c4e80}
\pdt{\Lambda}{A_{\nu;\mu}} = -\frac{1}{4\pi}F^{\mu\nu}.
\end{equation}
Using this equation in \eqref{c4e59}, we get
\begin{equation}\label{c4e81}
\tensor{T}{_\nu^\mu} = -\frac{1}{4\pi}A_{\alpha;\nu}F^{\mu\alpha} +
\frac{1}{16\pi}\tensor{\delta}{_\nu^\mu} F_{\alpha\beta}F^{\alpha\beta}.
\end{equation}
Multiplying by $\eta^{\nu\rho}$,
\begin{eqnarray}
\eta^{\nu\rho}\tensor{T}{_\nu^\mu} &=& 
	-\frac{1}{4\pi}\eta^{\nu\rho}A_{\alpha;\nu} F^{\mu\alpha} +
	\frac{1}{16\pi}\eta^{\nu\rho}\tensor{\delta}{_\nu^\mu} F_{\alpha\beta}
	F^{\alpha\beta} \nonumber \\
T^{\rho\mu} &=& -\frac{1}{4\pi}\pdt{A_\alpha}{x_\rho}F^{\mu\alpha} + 
\frac{1}{16\pi}\tensor{g}{^\rho^\mu} F_{\alpha\beta}F^{\alpha\beta} \nonumber \\
 &=& -\frac{1}{4\pi}\pdt{A^\alpha}{x_\rho}\tensor{F}{^\mu_\alpha} + 
\frac{1}{16\pi}\tensor{g}{^\rho^\mu} F_{\alpha\beta}F^{\alpha\beta} 
\label{c4e82}
\end{eqnarray}
From equation \eqref{c4e37}, after multiplying by an appropriate metric tensor,
\[
\pdt{\tensor{F}{^\mu_\alpha}}{x_\alpha} = -\frac{4\pi}{c}j^\mu.
\]
In absence of charges, it becomes
\[
\pdt{\tensor{F}{^\mu_\alpha}}{x_\alpha} = 0
\]
so that
\begin{equation}\label{c4e83}
\frac{1}{4\pi}\frac{\partial}{\partial x_\alpha}(A^\rho\tensor{F}{^\mu_\alpha}) 
= \frac{1}{4\pi}\pdt{A^\rho}{x_\alpha}\tensor{F}{^\mu_\alpha}
\end{equation}
From equations \eqref{c4e65} and \eqref{c4e69}, we can always add a term like 
the lhs of above equation to the stress tensor without altering its physical 
content. We will add the term to make the stress tensor symmmetric. Thus, we 
write \eqref{c4e82} as
\begin{eqnarray*}
T^{\rho\mu} &=& -\frac{1}{4\pi}\pdt{A^\alpha}{x_\rho}\tensor{F}{^\mu_\alpha} + 
\frac{1}{16\pi}\tensor{g}{^\rho^\mu} F_{\alpha\beta}F^{\alpha\beta} +
\frac{1}{4\pi}\frac{\partial}{\partial x_\alpha}(A^\rho
\tensor{F}{^\mu_\alpha}) \\
 &=& -\frac{1}{4\pi}\pdt{A^\alpha}{x_\rho}\tensor{F}{^\mu_\alpha} + 
\frac{1}{16\pi}\tensor{g}{^\rho^\mu} F_{\alpha\beta}F^{\alpha\beta} +
\frac{1}{4\pi}\pdt{A^\rho}{x_\alpha}\tensor{F}{^\mu_\alpha} \\
 &=& \frac{1}{4\pi}\left(\pdt{A^\rho}{x_\alpha} - \pdt{A^\alpha}{x_\rho}\right)
 \tensor{F}{^\mu_\alpha} + 
 \frac{1}{16\pi}\tensor{g}{^\rho^\mu} F_{\alpha\beta}F^{\alpha\beta} \\
 &=& \frac{1}{4\pi}F^{\alpha\rho}\tensor{F}{^\mu_\alpha} +
  \frac{1}{16\pi}\tensor{g}{^\rho^\mu} F_{\alpha\beta}F^{\alpha\beta}
\end{eqnarray*}
so that
\begin{equation}\label{c4e84}
T^{\rho\mu} = \frac{1}{4\pi}\left(-F^{\rho\alpha}\tensor{F}{^\mu_\alpha} +
\frac{1}{4}\eta^{\rho\mu}F_{\alpha\beta}F^{\alpha\beta}\right)
\end{equation}
Equation \eqref{c4e84} is the appropriate form of the energy-momentum tensor for
the electromagnetic field.
From equations \eqref{c3e75} and \eqref{c3e77}
\begin{eqnarray*}
F_{\mu\nu} &=& \begin{bmatrix} 0 & E_x & E_y & E_z \\
-E_x & 0 & -H_z & H_y \\
-E_y & H_z & 0 & -H_x \\
-E_z & -H_y & H_x & 0
\end{bmatrix} \\
F^{\mu\nu} &=& \begin{bmatrix} 0 & -E_x & -E_y & -E_z \\
E_x & 0 & -H_z & H_y \\
E_y & H_z & 0 & -H_x \\
E_z & -H_y & H_x & 0
\end{bmatrix}
\end{eqnarray*}
Since $\tensor{F}{^\mu_\nu} = F^{\mu\rho}g_{\rho\nu}$,
\begin{eqnarray}
\tensor{F}{^\mu_\nu} &=& \begin{bmatrix} 0 & E_x & E_y & E_z \\
-E_x & 0 & -H_z & H_y \\
-E_y & H_z & 0 & -H_x \\
-E_z & -H_y & H_x & 0
\end{bmatrix}\begin{bmatrix}1 & 0 & 0 & 0 \\
0 & -1 & 0 & 0 \\
0 & 0 & -1 & 0 \\
0 & 0 & 0 & -1
\end{bmatrix} \nonumber \\
 &=& \begin{bmatrix}0 & -E_x & -E_y & -E_z \\
 -E_x & 0 & H_z & -H_y \\
 -E_y & -H_z & 0 & H_x \\
 -E_z & H_y & -H_x & 0 
 \end{bmatrix} \label{c4e85}
\end{eqnarray}
so that
\[
F^{\rho\alpha}\tensor{F}{^\mu_\alpha} = \begin{bmatrix}
0 & -E_x & -E_y & -E_z \\
E_x & 0 & -H_z & H_y \\
E_y & H_z & 0 & -H_x \\
E_z & -H_y & H_x & 0
\end{bmatrix}\begin{bmatrix}0 & -E_x & -E_y & -E_z \\
 -E_x & 0 & H_z & -H_y \\
 -E_y & -H_z & 0 & H_x \\
 -E_z & H_y & -H_x & 0 
 \end{bmatrix}
\]
or
\begin{equation}\label{c4e86}
F^{\rho\alpha}\tensor{F}{^\mu_\alpha} =
\begin{bmatrix} 
E^2 & E_yH_z - E_zH_y & -E_xH_z + E_zH_x & E_xH_y  - E_yH_x \\
E_yH_z - E_zH_y & -E_x^2 + H_z^2 + H_y^2 & -E_xE_y -H_xH_y & -E_xE_z -H_xH_z \\
E_zH_x-E_xH_z & -E_xE_y - H_xH_y & -E_y^2 + H_z^2 + H_x^2 & -E_yE_z - H_yH_z \\
E_xH_y - E_yH_x & -E_xE_z -H_xH_z & -E_yE_z - H_yH_z & -E_z^2 + H_y^2 + H_x^2
\end{bmatrix}
\end{equation}
Similarly,
\[
F_{\mu\nu}F^{\mu\nu} = \begin{bmatrix} 0 & E_x & E_y & E_z \\
-E_x & 0 & -H_z & H_y \\
-E_y & H_z & 0 & -H_x \\
-E_z & -H_y & H_x & 0
\end{bmatrix}\begin{bmatrix} 0 & -E_x & -E_y & -E_z \\
E_x & 0 & -H_z & H_y \\
E_y & H_z & 0 & -H_x \\
E_z & -H_y & H_x & 0
\end{bmatrix}
\]
or
\begin{equation}\label{c4e87}
F^{\mu\nu}F_{\mu\nu} = 2(-E^2 + H^2).
\end{equation}
and
\begin{equation}\label{c4e88}
\frac{1}{4}F^{\mu\nu}F_{\mu\nu} = \frac{H^2}{2} - \frac{E^2}{2}.
\end{equation}
and \eqref{c4e84} becomes (note that the second term in \eqref{c4e84} is a 
diagonal matrix)
\begin{eqnarray*}
T^{00} &=& \frac{E^2 + H^2}{8\pi} \\
T^{01} &=& \frac{E_yH_z - E_zH_y}{4\pi} \\
T^{02} &=& \frac{E_zH_x - E_xH_z}{4\pi} \\
T^{03} &=& \frac{E_xH_y - E_yH_x}{4\pi} \\
T^{11} &=& \frac{E_x^2 - E_y^2 - E_z^2 + H_x^2 - H_y^2 - H_z^2}{8\pi} \\
T^{12} &=& -\frac{E_xE_y + H_xH_y}{4\pi} \\
T^{13} &=& -\frac{E_xE_z + H_xH_z}{4\pi} \\
T^{22} &=& \frac{-E_x^2 + E_y^2 - E_z^2 - H_x^2 + H_y^2 - H_z^2}{8\pi} \\
T^{23} &=& -\frac{E_yE_z + H_yH_z}{4\pi} \\
T^{33} &=& \frac{-E_x^2 - E_y^2 + E_z^2 - H_x^2 - H_y^2 + H_z^2}{8\pi}
\end{eqnarray*}
The remaining terms need not be computed because of the symmetry of 
$T^{\mu\nu}$. From \eqref{c4e50}, we observe that $T^{00} = W$, the energy 
density and from \eqref{c4e45}, $T^{0i} = S_i/c$. Further, if
\begin{equation}\label{c4e89}
\sigma_{ij} = \frac{1}{4\pi}\left(E_iE_j + H_iH_j - 
\frac{\delta_{ij}}{2}(E^2+H^2)\right).
\end{equation}
is the Maxwell stress tensor then
\begin{equation}\label{c4e90}
T^{\mu\nu} = \begin{bmatrix}
W & S_x/c & S_y/c & S_z/c \\
S_x/c & -\sigma_{xx} & -\sigma_{xy} & -\sigma_{xz} \\
S_y/c & -\sigma_{yx} & -\sigma_{yy} & -\sigma_{yz} \\
S_z/c & -\sigma_{zx} & -\sigma_{zy} & -\sigma_{zz}
\end{bmatrix}
\end{equation}

\item Analogously to \eqref{c4e20}, we define the mass density as
\begin{equation}\label{c4e91}
\rho_m = \sum_a m_a\delta(\vec{r} - \vec{r}_a)
\end{equation}
so that the 4-momentum density is $\rho_m c u^\mu$. This is also $T^{0\mu}/c$,
where $T^{\mu\nu}$ is the energy-momentum tensor of a system of particles. Thus,
\begin{equation}\label{c4e92}
T^{0\mu} = \rho_m c^2u^\mu.
\end{equation}
Now, $dx^\nu = (cdt, dx^1, dx^2, dx^3)$ so that
\begin{equation}\label{c4e93}
c = \td{x^0}{t}
\end{equation} 
and hence, from \eqref{c4e92}
\begin{equation}\label{c4e94}
T^{0\mu} = \rho_m c\td{x^\mu}{s}\td{x^0}{t}.
\end{equation}
We generalise this to
\[
T^{\mu\nu} =  \rho_m c\td{x^\mu}{s}\td{x^\nu}{t} = 
\rho_m c\td{x^\mu}{s}\td{x^\nu}{s}\td{s}{t}
\]
or
\begin{equation}\label{c4e95}
T^{\mu\nu} = \rho_m cu^\mu u^\nu \td{s}{t}.
\end{equation}
This tensor is symmetric and it is the energy-momentum tensor for 
non-interacting particles.

\item What does energy/momentum conservation mean for the tensor of 
\eqref{c4e95}?
To understand that, we first write \eqref{c4e95} as
\begin{equation}\label{c4e96}
T^{\mu\nu} = \rho_m c u^\mu\td{x^\nu}{t}
\end{equation}
so that
\begin{eqnarray*}
\pdt{T^{\mu\nu}}{x^\nu} &=& cu^\mu\frac{\partial}{\partial x^\nu}
	\left(\rho_m\td{x^\nu}{t}\right)
+ \rho_m c\td{x^\nu}{t}\pdt{u^\mu}{x^\nu} \\
 &=& cu^\mu\frac{\partial}{\partial x^\nu}\left(\rho_m\td{x^\nu}{t}\right) + 
 \rho_m c\td{u^\mu}{t}
\end{eqnarray*}
The term
\[
\frac{\partial}{\partial x^\nu}\left(\rho_m\td{x^\nu}{t}\right)
\]
must vanish if the masses are conserved. In this case, since we assumed to be
non-interacting, they are indeed conserved so that
\begin{equation}\label{c4e97}
\pdt{T^{\mu\nu}}{x^\nu} = \rho_m c\td{u^\mu}{t}
\end{equation}

\item We will repeat this exercise for the tensor of \eqref{c4e84}.
\[
T^{\rho\mu} = \frac{1}{4\pi}\left(-F^{\rho\alpha}\tensor{F}{^\mu_\alpha} +
\frac{1}{4}\eta^{\rho\mu}F_{\alpha\beta}F^{\alpha\beta}\right)
\]
which is same as
\[
\tensor{T}{^\rho_\mu} = \frac{1}{4\pi}\left(-F^{\rho\alpha}F_{\mu\alpha} +
\frac{1}{4}\tensor{g}{^\rho_\mu}F_{\alpha\beta}F^{\alpha\beta}\right).
\]
Therefore,
\[
4\pi\pdt{\tensor{T}{^\rho_\mu}}{x^\rho} = -\pdt{F^{\rho\alpha}}{x^\rho}
F_{\mu\alpha} - F^{\rho\alpha}\pdt{F_{\mu\alpha}}{x^\rho} + 
\frac{1}{2}F^{\alpha\beta}\pdt{F_{\alpha\beta}}{x^\rho}
\]
Using \eqref{c4e5} in the last term,
\[
4\pi\pdt{\tensor{T}{^\rho_\mu}}{x^\rho} = -\pdt{F^{\rho\alpha}}{x^\rho}
F_{\mu\alpha}
- F^{\rho\alpha}\pdt{F_{\mu\alpha}}{x^\rho} - \frac{\tensor{g}{^\rho_\mu}}{2}
 F^{\alpha\beta}
 \left(\pdt{F_{\beta\rho}}{x^\alpha} + \pdt{F_{\rho\alpha}}{x^\beta}\right)
\]
Using \eqref{c4e37} in the first term,
\[
4\pi\pdt{\tensor{T}{^\rho_\mu}}{x^\rho} = -\frac{4\pi}{c}j^\alpha F_{\mu\alpha}
- F^{\rho\alpha}\pdt{F_{\mu\alpha}}{x^\rho} - \frac{\tensor{g}{^\rho_\mu}}{2}
F^{\alpha\beta}
\left(\pdt{F_{\beta\rho}}{x^\alpha} + \pdt{F_{\rho\alpha}}{x^\beta}\right)
\]
or
\[
4\pi\pdt{\tensor{T}{^\rho_\mu}}{x^\rho} = -\frac{4\pi}{c}j^\alpha F_{\mu\alpha}
- F^{\rho\alpha}\pdt{F_{\mu\alpha}}{x^\rho} + \frac{\tensor{g}{^\rho_\mu}}{2}
\left(F^{\alpha\beta}\pdt{F_{\rho\beta}}{x^\alpha} + F^{\beta\alpha}
\pdt{F_{\rho\alpha}}{x^\beta}\right).
\]
Interchange the bound indices $\alpha$ and $\beta$ in the last term,
\[
4\pi\pdt{\tensor{T}{^\rho_\mu}}{x^\rho} = -\frac{4\pi}{c}j^\alpha F_{\mu\alpha}
- F^{\rho\alpha}\pdt{F_{\mu\alpha}}{x^\rho} + \frac{\tensor{g}{^\rho_\mu}}{2}
\left(F^{\alpha\beta}\pdt{F_{\rho\beta}}{x^\alpha} + 
F^{\alpha\beta}\pdt{F_{\rho\beta}}{x^\alpha}\right).
\]
That is,
\[
4\pi\pdt{\tensor{T}{^\rho_\mu}}{x^\rho} = -\frac{4\pi}{c}j^\alpha F_{\mu\alpha}
- F^{\rho\alpha}\pdt{F_{\mu\alpha}}{x^\rho} + \tensor{g}{^\rho_\mu}
F^{\alpha\beta}\pdt{F_{\rho\beta}}{x^\alpha}
\]
In the second term on the rhs, rename $\rho \rightarrow \alpha, \alpha 
\rightarrow \beta$ so that
\[
4\pi\pdt{\tensor{T}{^\rho_\mu}}{x^\rho} = -\frac{4\pi}{c}j^\alpha F_{\mu\alpha}
- F^{\alpha\beta}\pdt{F_{\mu\beta}}{x^\alpha} + \tensor{g}{^\rho_\mu}
F^{\alpha\beta}\pdt{F_{\rho\beta}}{x^\alpha}
\]
Since $\tensor{g}{^\rho_\mu} F_{\rho\beta} = F_{\mu\beta}$,
\[
4\pi\pdt{\tensor{T}{^\rho_\mu}}{x^\rho} = -\frac{4\pi}{c}j^\alpha F_{\mu\alpha}
- F^{\alpha\beta}\pdt{F_{\mu\beta}}{x^\alpha} + F^{\alpha\beta}
\pdt{F_{\mu\beta}}{x^\alpha}
\]
or
\begin{equation}\label{c4e98}
\pdt{\tensor{T}{^\rho_\mu}}{x^\rho} = -\frac{1}{c}j^\alpha F_{\mu\alpha}
\end{equation}

\item We can generalise the equation of motion \eqref{c3e74} for continua as
\begin{equation}\label{c4e99}
\rho_m c\td{u^\mu}{s} = \frac{\rho}{c}F^{\mu\nu}u_\nu
\end{equation}
that is,
\[
\rho_m c\td{u^\mu}{t}\td{s}{t} = \frac{\rho}{c}F^{\mu\nu}u_\nu.
\]
Combining this with \eqref{c4e97}, we get
\begin{equation}\label{c4e100}
\pdt{T^{(m)\mu\nu}}{x^\nu}\td{s}{t} = \frac{\rho}{c}F^{\mu\nu}u_\nu,
\end{equation}
where we added the superscript $(m)$ to differentiate the energy momentum tensor
for a system of masses from that of the electromagnetic field. Recall that
\[
u^\nu = \td{x^\nu}{s} = \td{x^\nu}{t}\td{s}{t},
\]
so that \eqref{c4e100} becomes
\begin{equation}\label{c4e101}
\pdt{T^{(m)\mu\nu}}{x^\nu} = \frac{\rho}{c}F^{\mu\nu}\td{x_\nu}{t}.
\end{equation}
Using \eqref{c4e23},
\begin{equation}\label{c4e102}
\pdt{T^{(m)\mu\nu}}{x^\nu} = \frac{1}{c}F^{\mu\nu}j_\nu.
\end{equation}
An equivalent form of this equation is
\begin{equation}\label{c4e103}
\pdt{\tensor{T}{^{(m)\rho}_\mu}}{x^\rho} = \frac{1}{c}F_{\mu\alpha}j^\alpha.
\end{equation}
Adding equation \eqref{c4e98} and \eqref{c4e103} gives
\begin{equation}\label{c4e104}
\frac{\partial}{\partial x^\mu}\left(\tensor{T}{^{(f)\rho}_\mu} + 
\tensor{T}{^{(m)\rho}_\mu}\right) = 0.
\end{equation}
Thus the combined energy momentum tensor 
\begin{equation}\label{c4e105}
\tensor{T}{^\rho_\mu} = \tensor{T}{^{(f)\rho}_\mu} + \tensor{T}{^{(m)\rho}_\mu}
\end{equation}
satisfies the conservation equation \eqref{c4e61}. We used the superscript $(f)$
to identify the electromagnetic contribution of the tensor.

\item The trace of the energy momentum tensor for a system of charged particles
in an electromagnetic field is
\begin{equation}\label{c4e106}
\tensor{T}{^\mu_\mu} = \tensor{T}{^{(m)\mu}_\mu},
\end{equation}
because the electromagnetic field's energy momentum tensor is anti-symmetric and
hence traceless. From \eqref{c4e95},
\begin{equation}\label{c4e107}
\tensor{T}{^\mu_\mu} = \rho_m cu^\mu u_\mu \td{s}{t} = 
\rho_m c^2\sqrt{1 - \beta^2}.
\end{equation}
If, instead of a continuous mass distribution, we have discrete particles,
\begin{equation}\label{c4e108}
\tensor{T}{^{\mu}_\mu} = \sum_a m_ac^2\sqrt{1 - \beta_a^2}\delta(\vec{r} - 
\vec{r}_a).
\end{equation}
From this equation, it is evident that
\begin{equation}\label{c4e109}
\tensor{T}{^\mu_\mu} \ge 0.
\end{equation}

\item From equations \eqref{c4e104} and \eqref{c4e105}, we have
\begin{equation}\label{c4e110}
\pdt{T^{\mu\nu}}{x^\mu} = 0.
\end{equation}
For the space components alone, this means
\[
\pdt{T^{\mu i}}{x^\mu} = 0,
\]
or
\begin{equation}\label{c4e111}
\frac{1}{c}\pdt{T^{0i}}{t} + \pdt{T^{ji}}{x^j} = 0.
\end{equation}
Now let's assume that this system of particles always has a small range of
positions and momenta so that all components of $T^{\mu\nu}$ vary in a limited
range. If we define the average of a function $f$ as
\begin{equation}\label{c4e112}
\langle f \rangle = \frac{1}{T}\int_0^T fdt
\end{equation}
then
\begin{equation}\label{c4e113}
\left\langle\td{f}{t}\right\rangle = \frac{1}{T} \int_0^T \td{f}{t}dt = 
\frac{f(T) - f(0)}{T}.
\end{equation}
In the limit $T \rightarrow \infty$, since the function $f$ takes finite
values at all times,
\begin{equation}\label{c4e114}
\left\langle\td{f}{t}\right\rangle = 0.
\end{equation}
If we take the time-average of \eqref{c4e111}, in view of \eqref{c4e114}, we get
\[
\pdt{\langle T\rangle^{ji}}{x^j} = 0
\]
or equivalently,
\begin{equation}\label{c4e115}
\pdt{\tensor{\langle T \rangle}{_i^j}}{x^j} = 0.
\end{equation}
From this equation, we also get
\[
x^i\pdt{\tensor{\langle T \rangle}{_i^j}}{x^j} = 0.
\]
Integrating this equation over all space,
\[
\int x^i\pdt{\tensor{\langle T \rangle}{_i^j}}{x^j} dV = 0 \Rightarrow 
\int \left(\frac{\partial}{\partial x^j}(x^i\tensor{\langle T \rangle}{_i^j}) - 
\pdt{x^i}{x^j}\tensor{\langle T \rangle}{_i^j}\right)dV = 0
\]
The first term can be converted into a surface integral, so that
\[
\oint x^i\tensor{\langle T \rangle}{_i^j} df_j - 
\int \tensor{\delta}{^i_j}\tensor{\langle T \rangle}{_i^j} dV = 0.
\]
Since we assumed that the system's energy, momenta and positions vary over a 
limited range, if we extend the surface integral beyond it then the first term
vanishes and we are left with
\begin{equation}\label{c4e116}
\int \tensor{\langle T \rangle}{_i^i} dV = 0.
\end{equation}
From this equation we readily conclude that
\begin{equation}\label{c4e117}
\int \tensor{\langle T \rangle}{_\mu^\mu} dV = 
\int\tensor{\langle T \rangle}{_0^0} dV + 
\int\tensor{\langle T \rangle}{_i^i}dV = \int WdV = \mathcal{E}.
\end{equation}
From \eqref{c4e108}, we have
\[
\tensor{\langle T \rangle}{^{\mu}_\mu} = 
\sum_a m_ac^2\langle\sqrt{1 - \beta_a^2}\rangle\delta(\vec{r} - \vec{r}_a),
\]
so that substituting it on the extreme lhs of \eqref{c4e117} gives,
\[
\int \sum_a m_ac^2
\langle\sqrt{1 - \beta_a^2}\rangle\delta(\vec{r} - \vec{r}_a) dV = \mathcal{E}.
\]
or
\begin{equation}\label{c4e118}
\sum_a m_ac^2\langle(1 - \beta_a^2)^{1/2}\rangle = \mathcal{E}.
\end{equation}
This is the relativistic generalisation of the classical virial theorem.

\item The calculations in the previous point should be corrected by removing the
``self-energy'' terms. These are the terms arising from field contributions of 
the masses and are
\[
\sum_a \int\frac{E_a^2 + H_a^2}{8\pi}dV.
\]

\item We will now examine how the energy momentum tensor looks for a continuum.
Referring to the form in \eqref{c4e90}, in the frame of reference in which the
continuum is at rest,
\begin{equation}\label{c4e119}
\sigma_{ij} = -p\delta_{ij},
\end{equation}
where $p$ is the pressure on the body. This equation is always true for fluids
but is approximately true for solids. The energy momentum tensor then becomes
\begin{equation}\label{c4e120}
T^{\mu\nu} = \begin{bmatrix}\mathcal{E} & 0 & 0 & 0 \\
0 & p & 0 & 0 \\
0 & 0 & p & 0 \\
0 & 0 & 0 & p
\end{bmatrix}
\end{equation}
In the rest frame of reference, the 4-velocity of a particle is $u^\mu = 
(1, 0, 0, 0)$ so that an expression of the form
\begin{equation}\label{c4e121}
T^{\mu\nu} = (p + \mathcal{E})u^\mu u^\nu - p\delta^{\mu\nu}
\end{equation} 
reduces to \eqref{c4e120} in the rest frame. In mixed components, this equation
becomes
\begin{equation}\label{c4e122}
\tensor{T}{_\mu^\nu} = (p + \mathcal{E})u_\mu u^\nu - 
p\tensor{\delta}{_\mu^\nu}.
\end{equation} 

Since \eqref{c4e121} reduces to \eqref{c4e120} in the rest frame, we assume 
that it is always true and apply it when
\[
u^\mu = \gamma(1, \vec{v})
\]
so that
\[
W = T^{00} = (p + \mathcal{E})u^0 u^0 - p = \gamma^2(p + \mathcal{E}) - p
= \frac{p + \mathcal{E}}{1 - \beta^2} - p
\]
or
\begin{equation}\label{c4e123}
W = \frac{\mathcal{E} + p\beta^2}{1 - \beta^2} = \gamma(\mathcal{E} + p\beta^2).
\end{equation}
Similarly,
\[
\frac{\vec{S}}{c} = \frac{(p + \mathcal{E})\vec{\beta}}{1 - \beta^2}
\]
so that
\begin{equation}\label{c4e124}
\vec{S} = \frac{(p + \mathcal{E})\vec{v}}{1 - \beta^2}.
\end{equation}
The components of the stress tensor are
\begin{equation}\label{c4e125}
\sigma_{ij} = -\gamma(p + \mathcal{E})\beta_i\beta_j - p\delta_{ij}.
\end{equation}
If the velocity of the continuum is small, \eqref{c4e124} can be approximated by
\begin{equation}
\vec{S} = (p + \mathcal{E})\vec{v}.
\end{equation}

\item From equation \eqref{c4e122},
\[
\tensor{T}{_\mu^\mu} = (p + \mathcal{E})u_\mu u^\mu - 4p
\]
From equation \eqref{c1e84}, $u_\mu u^\mu = 1$ and since $\tensor{\delta}
{_\mu^\mu} = 4$, we get
\begin{equation}\label{c4e126}
\tensor{T}{_\mu^\mu} = \mathcal{E} - 3p.
\end{equation}
The inequality \eqref{c4e109} gives
\begin{equation}\label{c4e127}
p \le \frac{\mathcal{E}}{3}.
\end{equation}
Furtherm from \eqref{c4e108} and \eqref{c4e126} we have
\begin{equation}\label{c4e128}
\mathcal{E} - 3p = 
\sum_a m_ac^2\sqrt{1 - \beta_a^2}\delta(\vec{r} - \vec{r}_a).
\end{equation}
In the ultra-relativistic case, $\beta_a \rightarrow 1$ and we get the familiar
expression for the relation between energy density and pressure of the radiation
\begin{equation}\label{c4e129}
p = \frac{\mathcal{E}}{3}.
\end{equation}

\end{enumerate}


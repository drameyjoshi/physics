\chapter{The propagation of light}\label{c7}
\begin{enumerate}
\item A plane wave is the one for which the fields are a function only of $t - 
x/c$ if they are propagating along the $x$ axis. Its direction of propagation 
and amplitude are the same everywhere.

\item Oftentimes, one can treat arbitrary waves as plane waves in a small enough
region of space. This is possible only if the amplitude and direction remain 
practically the same over a single wavelength. If this is indeed true then one 
can introduce the idea of a \emph{wave surface} or a \emph{wave front}. It is an
imaginary surface at all points of which the wave has a constant phase. Wave 
fronts of plane waves are indeed planes.

In a tiny region of space a wave propagates along the normal to the wave front.
The normals of consecutive wave fronts create a \emph{ray}. Alternatively, a ray
is a curve whose tangent at every point coincides with the normal to the wave
front.

\item When the dimensions of the region of interest are large compared to the
wavelength, we can study the propagation of waves by paying attention only to
the rays. This is the domain of \emph{geometrical optics}. It is an approximation
in which we ignore the wave properties and consider only rays. Geometrical optics
works best for waves whose wavelengths are very small.

\item If $f$ is any quantity describing the wave then for a plane monochromatic 
wave,
\begin{equation}\label{c7e1}
f = a\exp(-i(k_\mu x^\mu + \alpha)),
\end{equation}
where $\alpha$ is the constant phase of the wave. The 4-vector $k^\mu$ is defined
by equation \eqref{c6e54}. Strictly speaking, we should have written \eqref{c7e1}
as
\[
f = a\re{\exp(-i(k_\mu x^\mu + \alpha))},
\]
but we will now assume that every equation with complex exponentials is 
interpreted by ignoring its imaginary part.

If 
\begin{equation}\label{c7e2}
\Psi = -k_\mu x^\mu + \alpha
\end{equation}
then $f = a\exp(i\Psi)$. It is called the \emph{eikonal}.

\item If we are dealing with waves whose wave fronts are not planes then their
amplitude is a function of $x^\mu$ and the expression for $\Psi$ is not as 
simple as in \eqref{c7e2}.

\item Whatever be the form of $\Psi$, the approximation of geometrical optics
is applicable only if $\Psi$ is practically constant over a single wave-length, 
or
\begin{equation}\label{c7e3}
\frac{\lambda}{\Psi} \rightarrow 0.
\end{equation} 
In regions whose extent is of the order of magnitude of $\lambda$, one can write
the eikonal as 
\begin{equation}\label{c7e4}
\Psi = \Psi(0) + x^\mu\pdt{\Psi}{x^\mu}.
\end{equation}
Comparing with \eqref{c7e2}, we see that
\begin{equation}\label{c7e5}
k_\mu = -\pdt{\Psi}{x^\mu}.
\end{equation}
From \eqref{c6e57}, $k_\mu k^\mu = 0$ so that we also have
\begin{equation}\label{c7e6}
\pdt{\Psi}{x^\mu}\pdt{\Psi}{x_\mu} = 0.
\end{equation}
Equation \eqref{c7e6} is called the \emph{eikonal equation}.

\item Since $f = a\exp(i\Psi)$ satisfies the wave equation, 
\[
\frac{\partial^2 f}{\partial x_\mu \partial x^\mu} = 0.
\]
Now,
\[
\pdt{f}{x_\mu} = \pdt{a}{x_\mu}\exp(i\Psi) + ia\exp(i\Psi)\pdt{\Psi}{x^\mu} =
\pdt{a}{x_\mu}\exp(i\Psi) + if\pdt{\Psi}{x_\mu}
\]
so that
\begin{eqnarray*}
\frac{\partial^2 f}{\partial x_\mu \partial x^\mu} &=& 
    \frac{\partial^2 a}{\partial x_\mu \partial x^\mu}e^{i\Psi} + 
 	i\pdt{a}{x_\mu}\exp(i\Psi)\pdt{\Psi}{x^\mu} + 
	i\pdt{f}{x^\mu}\pdt{\Psi}{x^\mu} + \\
 & & if\frac{\partial^2 \Psi}{\partial x_\mu \partial x^\mu} \\
 &=& \frac{\partial^2 a}{\partial x_\mu \partial x^\mu}e^{i\Psi} + 
    2i\pdt{a}{x_\mu}\exp(i\Psi)\pdt{\Psi}{x^\mu} -
	f\pdt{\Psi}{x^\mu}\pdt{\Psi}{x_\mu} + \\
 & &  if\frac{\partial^2 \Psi}{\partial x_\mu \partial x^\mu}
\end{eqnarray*}
The real and imaginary parts of this quantity should vanish independently. 
Therefore,
\begin{eqnarray}
\frac{1}{a}\frac{\partial^2 a}{\partial x_\mu \partial x^\mu} &=& 
  \pdt{\Psi}{x^\mu}\pdt{\Psi}{x_\mu} \label{c7e7} \\
\frac{2}{a}\pdt{a}{x_\mu}\exp(i\Psi)\pdt{\Psi}{x^\mu} &=& 
\frac{\partial^2 \Psi}{\partial x_\mu \partial x^\mu} \label{c7e8} 
\end{eqnarray}
In the domain of geometrical optics we also assume that the second derivative of
$a$ with respect to $x^\mu$ can be ignored so that the first of the previous pair
of equations gives the eikonal equation.

The book mentions that largeness of $\Psi$ leads one to that conclusion but I 
unable to see it that way.

Alternatively, since a small patch of an arbitrary wave front can be considered
to be a plane, and since for a plane wave $a$ is constant, the eikonal equation
follows immediately from \eqref{c7e7}.

\item We can expand the eikonal equation \eqref{c7e6} as
\[
\frac{1}{c^2}\left(\pdt{f}{t}\right)^2 - \left(\pdt{f}{x}\right)^2 - 
\left(\pdt{f}{y}\right)^2 - \left(\pdt{f}{z}\right)^2 = 0
\]
and compare it with the Hamilton-Jacobi equation \eqref{c2e35}
\[
\frac{1}{c^2}\left(\pdt{S}{t}\right)^2 - \left(\pdt{S}{x}\right)^2 
- \left(\pdt{S}{y}\right)^2 - \left(\pdt{S}{z}\right)^2 = m^2c^2
\]
to suspect an analogy between the action $S$ of point particle with the eikonal
$f$ of a ray and the ray itself being a massless point particle. We also recall
that the action is related to the Hamiltonian and the momemtum as (refer to
\eqref{c2e27}),
\begin{eqnarray}
\mathcal{H} &=& -\pdt{S}{t} \label{c7e9} \\
\vec{p} &=& \pdt{S}{\vec{r}} \label{c7e10}
\end{eqnarray}
If we expand \eqref{c7e4} and \eqref{c7e2} we get
\begin{eqnarray*}
\Psi &=& \Psi(0) + t\pdt{\Psi}{t} - \vec{r}\cdot\pdt{\Psi}{\vec{r}} \\
\Psi &=& \alpha - \omega t + \vec{r}\cdot\vec{k}
\end{eqnarray*}
Comparing the rhs of previous two equations we get
\begin{eqnarray}
\omega &=& -\pdt{\Psi}{t} \label{c7e11} \\
\vec{k} &=& \pdt{\Psi}{\vec{r}} \label{c7e12}
\end{eqnarray}
Comparing the pair of equations \eqref{c7e9} and \eqref{c7e10} with the pair
\eqref{c7e11} and \eqref{c7e12} we extend the analogy $\Psi \leftrightarrow S$
to $\omega \leftrightarrow \mathcal{H}$ and $\vec{k} \leftrightarrow \vec{p}$.
It was left to the development of quantum mechanics to transform this analogy
into fundamental equations of physics.

\item Hamilton equations for the particles are
\begin{eqnarray*}
\dot{\vec{p}} &=& -\pdt{\mathcal{H}}{\vec{r}} \\
\dot{\vec{r}} &=& +\pdt{\mathcal{H}}{\vec{p}}.
\end{eqnarray*}
Their analogues to geometrical optics are
\begin{eqnarray}
\dot{\vec{k}} &=& -\pdt{\omega}{\vec{r}} \label{c7e13} \\
\dot{\vec{r}} &=& +\pdt{\omega}{\vec{k}} \label{c7e14}
\end{eqnarray}
In vacuum, $\omega = ck$ so that the first equation gives $\dot{\vec{k}} = 0$ 
while the second one gives $\vec{v} = \dot{\vec{r}} = c\un$, $\un$ being a unit
vector in the direction of propagation of the ray. Neither of these results are 
surprising but the analogy is quite effective when extended to propagation of 
light in material media.

\item From the analogy $\omega \leftrightarrow \mathcal{H}$ and $\vec{k} 
\leftrightarrow \vec{p}$, we can write the Lagrangian for a ray as
\[
\mathcal{L} = \vec{k}\cdot\pdt{\omega}{\vec{k}} - \omega.
\]
However, $\omega = ck$ gives $\mathcal{L} = 0$ identically. Therefore, we cannot
extend the principle of least action for particles in the form
\[
\delta\int Ldt = 0
\]
to geometrical optics. However, when the energy of a particle is constant the
principle of least action is equivalent to Maupertuis principle
\[
\delta\int\vec{p}\cdot d\vec{q} = 0.
\]
The analogue of Maupertuis principle to geometrical optics is
\begin{equation}\label{c7e15}
\delta\int\vec{k}\cdot d\vec{l} = 0.
\end{equation}
In vacuum, $\omega\un = c\vec{k}$ so that the equation simplifies to
\begin{equation}\label{c7e16}
\delta\int dl = 0,
\end{equation}
which is Euler's principle of least time.

\item Geometrical optics deals only with rays. It considers the propagation of a
bundle or a pencil of rays. The rays determine the direction of propagation and 
give no information about the intensity. If rays start from a certain region then
after a certain duration $\delta t$ the points $c\delta t$ away from the starting
region define a surface called the wave-front. An infinitesimal portion of the
wave-front has two principal radii of curvatures, say $R_1$ and $R_2$. In general,
they are not the same. Therefore, the points on two principle circles of curvature
appear to emerge from different points. This is the cause of the aberration of
astigmatism.

The area of the wave-front is proportional to $R_1R_2$ while the intensity is
inversely proportional to it. Thus,
\begin{equation}\label{c7e17}
I \propto \frac{1}{R_1R_2}.
\end{equation}

Note that this formula, in particular the two radii of curvature, describe only 
the chosen infinitesimal surface. Therefore \eqref{c7e17} is not applicable to
areas illuminated by a different pencil of rays.

The intensity of light is proportional to the squared modulus of the field.
Therefore, the field itself is given by
\begin{equation}\label{c7e18}
f \propto \frac{1}{\sqrt{R_1R_2}} e^{ikR},
\end{equation}
where $R$ is any one of $R_1$ or $R_2$. Since $R_1$ and $R_2$ are fixed, the only
difference between the formulae using them is the constant phase $\exp(ik(R_1-R_2))$,
which can be readily absorbed in the constant of proportionality.

In the special case of the two radii of curvature being the same, the wavefront
is a sector of a sphere and two centres of curvature coincide. 

$I$ blows up with either $R_1$ or $R_2$ vanish. Therefore the centres of curvature
are points of infinite intensity. The locus of all centres of curvatures define a
caustic surface. Refer to \href{https://physics.stackexchange.com/a/256561/10236}
{a StackExchange answer} for an excellent explanation of the how caustics are 
formed and the rays are tangential to it.
\end{enumerate}

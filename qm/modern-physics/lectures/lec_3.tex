\documentclass{beamer}
\usefonttheme[onlymath]{serif}
\usepackage{graphicx,hyperref}
\usetheme{Boadilla}

\newcommand{\pvec}[1]{\vec{#1}\mkern2mu\vphantom{#1}}

\title{Special Theory of Relativity - 2}
\author{Amey Joshi}
\date{\today}
\begin{document}
\begin{frame}
\titlepage
\end{frame}

\begin{frame}
\frametitle{Classical Doppler effect}
\begin{itemize}
\item You can tell whether a siren is approaching you or receding from you.
\item Let a source $S$ emit sound at frequency $\nu_0$ and a detector $D$ 
approach it with speed $v$. $D$ sees the wave approaching at speed $v+c_s$, 
$c_s$ being the speed of sound. The distance between crests is still $\lambda$,
therefore the apparent frequency is
\begin{equation}\label{c1}
\nu = \frac{c_s+v}{\lambda} = \nu_0\left(1 + \frac{v}{c_s}\right).
\end{equation}
\item If $D$ is stationary and $S$ approaches $D$ with speed $v$ then the
crests which used to be a distance $\lambda$ apart are now closer by an amount
$v/\nu_0$. The apparent frequenct is
\begin{equation}\label{c2}
\nu = \frac{c_s}{\lambda - v/\nu_0} = \frac{nu_0}{1 - v/c_s}.
\end{equation}
\item Equations \eqref{c1} and \eqref{c2} are not identical. Is Galileo's 
relativity violated? (Read the notes to assure yourself that it isn't).
\end{itemize}
\end{frame}

\begin{frame}
\frametitle{Relativistic Doppler effect}
\begin{itemize}
\item Let a source $S$ emit radiation of frequency $\nu_0$. If a detector $D$
is travelling perpendicular to the direction of radiation, then it detects a 
crest every $\nu^{-1}$ s. In the frame of reference where $S$ is at rest, a 
crest is seen every $\nu_0^{-1}$ s. The two times are related as
\begin{equation}\label{e3}
\nu^{-1} = \frac{\nu_0^{-1}}{\sqrt{1 - \beta^2}} \Rightarrow
\nu = \nu_0\sqrt{1 - \beta^2}.
\end{equation}
\item $\nu < \nu_0$, the apparent frequency is lower than the actual. Appears
red-shifted.
\item One piece of evidence for the expanding universe.
\end{itemize}
\end{frame}

\begin{frame}
\frametitle{Relativistic Doppler effect - Longitudnal}
\begin{itemize}
\item If $D$ travels away from $S$ with speed $v$ in the direction of radiation,
the time between crests is $\nu_0^{-1}/\sqrt{1-\beta^2}$ and the crests lags
behind by a distance $\nu^{-1}/\sqrt{1-\beta^2}$. Apparent period of the wave is
\begin{equation}\label{e4}
\nu^{-1} = \frac{\nu_0^{-1}}{\sqrt{1-\beta^2}} + 
           \frac{\nu_0^{-1}}{\sqrt{1-\beta^2}}\frac{v}{c} = 
           \nu_0^{-1}\frac{1+\beta}{\sqrt{1-\beta^2}}.
\end{equation}
\item A simple rearrangement gives
\begin{equation}\label{e5}
\nu = \nu_0\sqrt{\frac{1 - \beta}{1 + \beta}}.
\end{equation}
\item Once again $\nu < \nu_0$, a red shift.
\end{itemize}
\end{frame}

\begin{frame}
\frametitle{Synchronization of clocks}
\begin{itemize}
\item There are two clocks $C_1$ and $C_2$ at either ends of a platform. Both
show the same time. You pass by in a uniformly moving train. You set your clock 
to $C_1$. When you reach the other end, will your clock agree with $C_2$?
\item Not so in the relativistic world. You travel a distance $L_0$ in your
frame at speed $v$ taking time $L_0/v$, which in the stationary frame is
\[
\frac{L_0}{v\sqrt{1-\beta^2}}.
\]
\item You measure the distance between the clocks to be $L_0\sqrt{1-\beta^2}$
and therefore the time elapsed is
\[
\frac{L_0\sqrt{1 - \beta^2}}{v}
\]
\item Your clock is no longer in sync with $C_2$.
\end{itemize}
\end{frame}

\begin{frame}
\frametitle{A pedagogical note}
\begin{itemize}
\item I don't want you to memorize derivations and formulae.
\item The mathematics is very simple; the physics is subtle.
\item So enjoy and take time to understand why they are true.
\item I am not going to ask you to reproduce these in your exams.
\item You will be given the relevant formulae and asked to use them to solve
problems. Make sure you understand what they mean.
\end{itemize}
\end{frame}

\begin{frame}
\frametitle{Galilean transformation}
\begin{itemize}
\item 
Recall the first lecture. If $x, y, z$ are the coordinates of a point in an
inertial frame of reference $S$ and $x^\prime, y^\prime, z^\prime$ are the
coordinates in another inertial frame $S^\prime$ moving at a velocity $\vec{v}
= v\hat{i}$ with respect to $S$ then
\begin{eqnarray}
x^\prime &=& x - vt \label{e6} \\
y^\prime &=& y \label{e7} \\
z^\prime &=& z. \label{e8}
\end{eqnarray}
It was implicitly accepted that $t^\prime = t$. We now know that this is not
true if $v$ is close to the speed of light.
\item What is the replacement?
\end{itemize}
\end{frame}

\begin{frame}
\frametitle{Lorentz transformation}
\begin{itemize}
\item $S$ and $S^\prime$ are the same as before. The transformation equations 
are
\begin{eqnarray}
x^\prime &=& \frac{x - vt}{\sqrt{1 - \beta^2}} \label{e9} \\
y^\prime &=& y \label{e10} \\
z^\prime &=& z \label{e11} \\
t^\prime &=& \frac{t - vx/c^2}{\sqrt{1 - \beta^2}} \label{e12}
\end{eqnarray}
\item Suppose $\tau = ct$. Then \eqref{e9} and \eqref{e12} can be written as
\begin{eqnarray}
x^\prime &=& \gamma(x - \beta\tau) \label{e13} \\
\tau^\prime &=& \gamma(\tau - \beta x) \label{e14}
\end{eqnarray}
where $\gamma = 1/\sqrt{1 - \beta^2}$.
\item Do you see the symmetry between space and time?
\end{itemize}
\end{frame}

\begin{frame}
\begin{itemize}
\item In the Galilean world, the velocities measured in the two frames are
\begin{eqnarray}
\frac{dx^\prime}{dt} &=& \frac{dx}{dt} - v \label{e15} \\
\frac{dy^\prime}{dt} &=& \frac{dy}{dt} \label{e16} \\
\frac{dz^\prime}{dt} &=& \frac{dz}{dt} \label{e17}
\end{eqnarray}
\item In Einstein's world, you would expect the $x$-component of the velocity
to change. What about the $y$ and $z$ components? Can you infer anything without
doing any derivation?
\end{itemize}
\end{frame}

\begin{frame}
\frametitle{Does mass get affected by motion? - 1}
\begin{enumerate}
\item Distance and time intervals depend on who measures them. What about mass?
\item Consider two inertial frames of reference $S$ and $S^\prime$ such that 
$S^\prime$ travels with respect to $S$ at a uniform velocity $\vec{v}=v\hat{i}$.
Let there be two identical bodies of mass $m$ and $m^\prime = m$ at `heights' 
$y_1$ and $y_2$ such that the first particle is at rest in $S$ and the other 
is at rest in $S^\prime$. Assume that $y_1 > y_2$.
\item How does the trajectory of the two particles appear in the two frames?
\item When their $x$ coordinates are such that $x_2 = x_1 - vt$, they are both 
given a velocity $u = 2(y_1 - y_2)/t$ so that they collide in time $t$ at a
height midway between $y_2$ and $y_1$.
\end{enumerate}
\end{frame}

\begin{frame}
\frametitle{Does mass get affected by motion? - 2}
\begin{enumerate}
\item An observer in $S$ sees a body initially at rest descend with a speed 
$u = 2(y_1 - y_2)/t$ before it collides with another identical body that has
ascended.
\item The ascending body covered the distance $(y_1 - y_2)/2$ in time 
$t/\sqrt{1 - \beta^2}$. 
\item Momentum of the ascending body is $m^\prime u\sqrt{1 - \beta^2}$. That of 
the descending body is $mu$. Since the system of the two bodies is isolated, 
momentum if conserved. That is $mu = m^\prime u\sqrt{1 - \beta^2}$.
\item Assume that $u \ll v$ so that the vertical velocities are very small the
bodies suffer a `grazing collision'. That way, $m$ is almost unaffected by 
motion and the effect on $m^\prime$ is only because of the horizontal motion.
\end{enumerate}
\end{frame}

\begin{frame}
\frametitle{Does mass get affected by motion? - 3}
\begin{enumerate}
\item Momentum is conserved if we define it as
\begin{equation}\label{e13}
p = \frac{mv}{\sqrt{1 - \beta^2}} = \frac{mv}{\sqrt{1 - v^2/c^2}} .
\end{equation}
Note that in a grazing collision, the velocity of the ascending body, as seen
in $S$ is approximately $\vec{v}$.
\item It is tempting to conclude that mass is affected by motion such that
\begin{equation}\label{e14}
m(v) = \frac{m(0)}{\sqrt{1 - v^2/c^2}}.
\end{equation}
However, we will resist that temptation and rather redefine momentum as in
equation \eqref{e13}
\item $p \rightarrow \infty$ as $v \rightarrow c$. Thus, it takes an infinite
energy to accelerate a material body to the speed of light. That is why we
always say that $v \le c$. 
\end{enumerate}
\end{frame}

\begin{frame}
\frametitle{Before we leave relativity}
\begin{enumerate}
\item The finite speed of light implies that when we see a far away object we 
see it the way it was when the light left it. 
\item That is why telescope that see farthest objects are also seeing the 
universe as it was a very long time ago.
\item The intimate relation between `farness' and `oldness' gives rise to 
space and time being on an equal footing.
\item How does a point charge at rest in your frame appear to your friend
in another intertial frame? Do you now see why we speak of an `electromagnetic'
field and not just an electric or a magentic field?
\item We had only a glimpse of relativity. But we must move on to other 
shocking discoveries.
\end{enumerate}
\end{frame}

\begin{frame}[fragile]
\frametitle{Homework}
\begin{itemize}
\item Problem sets 4 and 5 in chapter 1 of the accompanying notes.
\item Read section 1.3 in the notes to understand where $E = mc^2$ comes from.
\item Submit your assignments within one week of this lecture. Upload your 
files with the name \begin{verbatim}<your-name><hw2>.pdf\end{verbatim}
\end{itemize}
\end{frame}
\end{document}

\documentclass{article}
\usepackage{amsmath}
\newcommand{\un}{\hat{n}}
\newcommand{\uv}[1]{\hat{e}_{#1}}
\newcommand{\grad}[1]{\nabla{#1}}
\newcommand{\dive}[1]{\nabla\cdot\vec{#1}}
\newcommand{\curl}[1]{\nabla\times\vec{#1}}
\newcommand{\op}{\prime}
\newcommand{\pdt}[2]{\frac{\partial{#1}}{\partial{#2}}}
\newcommand{\ke}{\frac{1}{4\pi\epsilon_0}}
\title{Electrostatics}\label{c3}
\author{Amey Joshi}
\date{10-Mar-2024}
\begin{document}
\maketitle
\begin{enumerate}
\item[(1)] The potentials are $V_1$ and $V_2$ in the left and right half-planes
respectively. Therefore, the electric field is in the $x$ direction. It is non-
zero only along the $y$-axis where the potential changes. Since $\vec{E} = 
-\grad\varphi$,
\[
\vec{E} = -\uv{x}\lim_{\delta x \rightarrow 0}\frac{V_2 - V_1}{\delta}.
\]
As $V_2 < V_1$, the field is directed along the positive $x$ axis. A positively
charged particle will accelerate along the positive $x$-axis. Only the $x$-
component of the momentum changes, the $y$-component remains unchanged. If $\vec{p}$
and $\vec{p}^{\;\op}$ are the momenta in negative and positive half-planes, $p_x^\op = 
p_x + qE$, $p_y^\op = p_y$, $p = \sqrt{p_x^2 + p_y^2}, p^\op = \sqrt{{p_x^\op}^2
+ {p_y^\op}^2}$, $p_y = p\sin\theta_1, p_y^\op = p^\op\sin\theta_2$. Thus,
\[
(p_x^2 + p_y^2)\sin^2\theta_1 = ({p_x^\op}^2 + {p_y^\op}^2)\sin^2\theta_2,
\]
or
\[
(p_x^2 + p_y^2)\sin^2\theta_1 = ((p_x + qE)^2 + p_y^2)\sin^2\theta_2.
\]
Thus,
\begin{equation}\label{e1}
\frac{\sin\theta_1}{\sin\theta_2} = 
\sqrt{1 + \frac{2p_xqE + q^2E^2}{p_x^2 + p_y^2}} =
\sqrt{1 + \frac{2\cos\theta_1 qE}{p} + \frac{q^2E^2}{p^2}}
\end{equation}
The ratio of the sines of the two angles is not independent of the ``angle of
incidence'' because the amount of ``refraction'' depends on how much of the
total momentum is affected by the field.

A step field of this kind described in this problem shows up at the semiconductor
pn junction.

\item[(2)] If $E_k = C_k + D_{kj}r_j$,
\[
\dive{E} = \dive{C} + \frac{\partial}{\partial x_k}(D_{kj}x_j) = \dive{C} +
x_j\pdt{D_{kj}}{x_j} + D_{kj}\pdt{x_j}{x_k},
\]
that is,
\[
\dive{E} = \dive{C} + D_{kk}.
\]
If $\vec{C}$ is a constant vector, then $\dive{E} = D_{kk}$. In a charge free
region, $\dive{E} = 0$. Therefore, the tensor $\underline{D}$ is traceless.

We also have $\curl{E} = 0$. Thus,
\[
\epsilon_{ijk}\partial_jE_k = \epsilon_{ijk}\partial_jC_k + \epsilon_{ijk}\partial_j(D_{kl}x_l)
= 0 + \epsilon_{ijk}D_{kl}\partial_j(x_l) = \epsilon_{ijk}D_{kl}\delta_{jl}.
\]
Thus, we have $\epsilon_{ijk}D_{jk} = 0$, that is, $D_{jk} = D_{kj}$.

If $\varphi = -C_kx_k - x_kD_{kj}x_j$ then $E_k = \partial_k\varphi = C_k + D_{kj}r_j$.

\item[(3a)] I'll quickly derive an expression for the electric field due to a
uniformly charged ring at a point on its axis. Let the ring be in the $xy$ plane
and its centre coincident with the origin. At any point on the $z$ axis only the
$z$ component of the field will survive. Consider an element $dl$ of the ring
with charge $\lambda dl$. If $r$ if the radius of the ring then its contribution
to the field at a point $(0, 0, z)$ will be
\[
dE = \ke\frac{\lambda dl}{r^2 + z^2}\cos\theta,
\]
where $\theta$ is the polar angle. It is easy to see that 
\[
\cos\theta = \frac{z}{\sqrt{r^2 + z^2}}
\]
so that
\[
dE = \ke\frac{z\lambda dl}{(r^2 + z^2)^{3/2}}
\]
Since $dl = rd\phi$,
\begin{equation}\label{e2}
\vec{E} = \frac{1}{2\epsilon_0}\frac{\lambda rz}{(r^2 + z^2)^{3/2}}\uv{z}.
\end{equation}
If $q = 2\pi r \lambda$,
\begin{equation}\label{e3}
\vec{E} = \ke\frac{qz}{(r^2 + z^2)^{3/2}}\uv{z}.
\end{equation}
We can get to the same result by first finding the potential. The potential
$d\varphi$ due to an element $dl$ is
\[
d\varphi = \ke\frac{\lambda dl}{\sqrt{r^2 + z^2}} = \ke\frac{\lambda r d\phi}{\sqrt{r^2 + z^2}},
\]
so that
\begin{equation}\label{e4}
\phi = \ke \frac{2\pi r\lambda }{\sqrt{r^2 + z^2}} = \ke \frac{q}{\sqrt{r^2 + z^2}}
\end{equation}
so that
\[
\vec{E} = -\grad\varphi = \ke\frac{qz}{(r^2 + z^2)^{3/2}}\uv{z}.
\]

Calculating the potential is much simpler than the field. We will, therefore, use
equation \eqref{e4} to calculate the potential due to a spherical shell at a point 
outside it. We can always orient the $z$ axis along the centre of the shell and the
field point. A shell of 
radius $a$ can be considered to be made up of rings of infinitesimal thickness whose
radii $b$ increase from $0$ to $a$ and then decrease to $0$ as the distance from the
centre of the ring to the field point increases from $z - a$ to $z$ to $z + a$. The
potential due to the elementary ring as a position $z^\op$ is
\[
d\varphi = \ke \frac{dq}{\sqrt{b^2 + (z - z^\op)^2}}.
\]
Now $b^2 + {z^\op}^2 = a^2$ so that
\[
d\varphi = \ke \frac{dq}{\sqrt{a^2 + z^2 - 2z^\op z}}.
\]
We can also write $b = a\sin\psi, z^\op = a\cos\psi$ so that as $\psi$ varies from
$0$ to $\pi$ the entire shell is generated. Thus,
\[
d\varphi = \ke \frac{dq}{\sqrt{a^2 + z^2 - 2az\cos\psi}}.
\]
We also have $dq = \sigma (2\pi b ad\psi)$ so that
\[
d\varphi = \frac{2\pi a^2 \sigma}{4\pi\epsilon_0} \frac{\sin\psi d\psi}{\sqrt{a^2 + z^2 - 2az\cos\psi}}.
\]
Therefore,
\[
\varphi = \int_0^\pi \frac{2\pi a^2 \sigma}{4\pi\epsilon_0} \frac{\sin\psi d\psi}{\sqrt{a^2 + z^2 - 2az\cos\psi}}.
\]
Let $u = a^2 + z^2 - 2az\cos\psi$ so that $du = 2az\sin\psi d\psi$ and the 
limits of integration are from $(z - a)^2$ to $(z + a)^2$. Thus,
\[
\varphi = \frac{2\pi a^2 \sigma}{4\pi\epsilon_0}\int_{(z-a)^2}^{(z+a)^2}\frac{1}{2az} \frac{du}{\sqrt{u}}
= \frac{2\pi a^2 \sigma}{4\pi\epsilon_0}\frac{1}{2az} (2\sqrt{u})\Big|_{(z-a)^2}^{(z+a)^2} = 
\frac{2\pi a^2 \sigma}{4\pi\epsilon_0}\frac{4a}{2az}
\]
Thus,
\begin{equation}\label{e5}
\varphi = \ke \frac{4\pi a^2\sigma}{z} = \ke \frac{q}{z}.
\end{equation}
The electric field is
\begin{equation}\label{e6}
\vec{E} = \ke \frac{q}{z^2}\uv{z}.
\end{equation}

The electric field inside a spherical shell is zero. It is an immediate consequence
of Gauss' law
\[
\oint_S\vec{E}\cdot\un da = \frac{Q_{\text{enc}}}{\epsilon_0},
\]
where $Q_{\text{enc}}$ is the charge enclosed by $S$. We choose $S$ to be a small
spherical surface around the field point, entirely inside the shell.

\item[(3b)] We will use the expression for potential due to a ring to calculate 
that due to a charged disc at a field point on the disc's axis. If $dq$ is the 
charge on an infinitesimally thick ring of radius $r$ and thickness $dr$ then 
$dq = \sigma \times 2\pi rdr$. From equation \eqref{e5},
\[
d\varphi = \ke\frac{dq}{z} = \ke\frac{2\pi\sigma rdr}{z}
\]
so that for a disc of radius $a$,
\[
\varphi = \int_0^a \ke\frac{2\pi\sigma rdr}{z} = \ke\frac{2\pi\sigma}{z}\frac{a^2}{2}.
\]
Since $\sigma\pi a^2 = q$, the total charge on the disc, we have
\begin{equation}\label{e7}
\varphi = \ke\frac{q}{z}
\end{equation}
and hence 
\begin{equation}\label{e8}
\vec{E} = \ke \frac{q}{z^2}\uv{z}.
\end{equation}
The calculation of the electric field at a point outside the sphere proceeds in
exactly the same way as in the case of the shell. For the case of the field point 
inside the sphere, we can either use Gauss' law or ``summation method''. To
demonstrate the latter method, let the field point be at a distance $r_0$ from the
centre of the sphere. The potential at the field point will get a contribution
from shells of all radii from $0$ to $r_0$. If $dr$ is the thickness of the shell
then its volume is $4\pi r^2dr$. If $\rho$ is the charge density then the total
charge on the shell is $4\pi r^2\rho dr$. From equation \eqref{e5}, it contributes
a potential
\[
d\varphi = \ke \frac{4\pi r^2\rho dr}{r_0}
\]
Potential due to all shells will be
\[
\varphi = \ke\frac{4\pi\rho}{r_0}\frac{r_0^3}{3} = \ke\frac{Q_{\text{enc}}}{r_0}
\]
The field derived from this potential is the same as derived from Gauss' law.
\end{enumerate}
\end{document}
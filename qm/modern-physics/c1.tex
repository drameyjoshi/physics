\chapter{Special Theory of Relativity}\label{c1}
\section{Galilean relativity}\label{c1s1}
The idea of relative motion is as old as Newton's mechanics. Consider two 
observers, $O$ and $O^\prime$ with their frames of reference $S$ and $S^\prime$.
If the axes of $S$ and $S^\prime$ are parallel to each other but the origin of
$S^\prime$ is displaced by $\vec{R}$ from the origin of $S$ then a point 
$\vec{r}$ in $S$ appears as 
\begin{equation}\label{c1s1e1}
\pvec{r}^\prime = \vec{r} - \vec{R}
\end{equation}
in $S^\prime$. The two observers will disagree on the coordinates of a point. 
However, the disagreement will always be to the extent given in equation \eqref
{c1s1e1}. If the observer $O$ finds $O^\prime$ moving then their measurements 
of velocities of other bodies will not match. $O$ will measure $\vec{v} = 
\dot{\vec{r}}$, $O^\prime$ will measure $\pvec{v}^\prime = 
\dot{\pvec{r}}^\prime$ and the two will be related as
\begin{equation}\label{c1s1e2}
\pvec{v}^\prime = \vec{v} - \vec{V},
\end{equation}
where $\vec{V} = \dot{\vec{R}}$ is the velocity of $O^\prime$ as measured by
$O$. Likewise, if $O$ finds that $O^\prime$'s velocity changes with time then
their measurements of acceleration will be related as
\begin{equation}\label{c1s1e3}
\pvec{a}^\prime = \vec{a} - \vec{A},
\end{equation}
where $\vec{a} = \ddot{\vec{r}}, \pvec{a}^\prime = \ddot{\pvec{r}}^\prime$ 
and $\vec{A} = \ddot{\vec{R}}$. If $O$ observes a body moving in a straight 
line with a uniform velocity then $O^\prime$ sees it accelerating. This is the 
origin of pseudo-forces. Equations \eqref{c1s1e1} are called Galilean 
transformations and they describe the Galilean relativity.

Galilean transformation does not explicitly mention that the mass of a body
does not change across frames of references. It was assumed to be obvious and
not worth mentioning. Likewise, it was also assumed that the time remains
unaffected by relative motion. Therefore, we also have
\begin{eqnarray}
t^\prime &=& t \label{c1s1e4} \\
m^\prime &=& m. \label{c1s1e5}
\end{eqnarray}
Einstein's theory, described in the next sections, proved that these tacit
assumptions are untenable.

\subsection{Problem set 1}\label{c1s1s1}
\begin{enumerate}
\item Alice is standing on the platform and sees Bob on a train passing at a
uniform speed $V = 40$ kmph. Bob rolls a ball on the floor of the train and
measures its speed to be $v^\prime = 5$ kmph. What will be Alice's measurement
of the ball's speed if the ball moves (a) in the direction of the train, 
(b) against the direction of the train and (c) sideways?
\item Bob is siting on a flat wagon of a train. The floor of the wagon is 
smooth and there is a ball placed on it. The train begins to move. Both Alice
and Bob observe that the ball rolls in a direction opposite to that of the 
train. How will Alice and Bob explain the motion of the ball?
\item A fighter-pilot drops a bomb from an aircraft flying at a uniform speed. 
How will his co-pilot see it descending? How will an observer on ground see it
dropping? (Assume that the wind does not affect the motion of the bomb.)
\end{enumerate}

If the observer $O^\prime$ is moving at a uniform velocity with respect to $O$
then $\vec{A} = 0$ and $\pvec{a}^\prime = \vec{a}$. Therefore, if $O$ observes
a body at rest then $O^\prime$ observes it to be moving at a \emph{uniform}
velocity $\pvec{v}^\prime = -\vec{V}$.

Newton's first law states that every body continues to be in a state of rest
or of uniform motion in a straight line unless a force acts on it. A frame of
reference in which Newton's first law holds good is called an \emph{inertial
frame of reference}. 

\subsection{Problem set 2}\label{c1s1s2}
\begin{enumerate}
\item If $S$ is an inertial frame and if $S^\prime$ moves at a uniform velocity
$\vec{V}$ then show that $S^\prime$ is also an inertial frame of reference.
\item A tractor is attached to three trailers, each of mass $m$. The tractor 
itself has a mass $M$. The tractors starts to pull its consist with an 
acceleration $a$. Find the tension in each of the couplings. (A coupling is a 
device that connects one vehicle to another.)
\end{enumerate}

If a body does not feel a force in one inertial frame of reference then it
does not feel it in any other frame of reference moving uniformly with respect
to the first frame. All inertial frames of reference are equivalent. Observers
in inertial frames may disagree on position and velocities of bodies but never
on their accelerations. Newton's laws are valid in all inertial frames of 
reference. As an example, consider two space ships $S$ and $S^\prime$ observing
the planets. Let $\vec{r}_1$ and $\vec{r}_2$ be their positions with respect
to $S$. Then, the force between them is
\begin{equation}\label{c1s1e6}
\vec{F} = -\frac{Gm_1m_2}{|\vec{r}_1 - \vec{r}_2|^3}(\vec{r}_1 - \vec{r}_2).
\end{equation}
Here $m_1$ and $m_2$ are the masses of the planets at positions $\vec{r}_1$
and $\vec{r}_2$ with respect to the origin of $S$ and $G$ is the universal
gravitational constant. The acceleration of mass 
$m_1$ is
\begin{equation}\label{c1s1e7}
\vec{a}_1 = -\frac{Gm_2} {|\vec{r}_1 - \vec{r}_2|^3}(\vec{r}_1 - \vec{r}_2).
\end{equation}
Observer $O^\prime$ will express this force as
\begin{equation}\label{c1s1e8}
\pvec{F}^\prime = 
-\frac{Gm_1m_2}{|\pvec{r}_1^\prime - \pvec{r}_2^\prime|^3}
(\pvec{r}_1^\prime - \pvec{r}_2^\prime).
\end{equation}
and $m_1$'s acceleration as 
\begin{equation}\label{c1s1e9}
\pvec{a}_1^\prime = -\frac{Gm_2} {|\pvec{r}_1^\prime - \pvec{r}_2^\prime|^3}
(\pvec{r}_1^\prime - \pvec{r}_2^\prime).
\end{equation}
Equation \eqref{c1s1e1} assures that the right hand sides of equations 
\eqref{c1s1e6} and \eqref{c1s1e8} are the same. The inertial nature of the
frames $S$ and $S^\prime$ assures that the left hand sides are also the same.
The law of motion, Newton's second law, is valid in both frames of reference.
The governing equation are identical, albeit with different variables.

Continuing with the example of the two planets, if $S^\prime$ was accelerating
with respect to $S$ then $\pvec{a}_1^\prime$ would have been $\vec{a} - \vec{A}$
and equation \eqref{c1s1e9} would have been
\begin{equation}\label{c1s1e10}
\pvec{a}_1^\prime = -\frac{Gm_2} {|\pvec{r}_1^\prime - \pvec{r}_2^\prime|^3}
(\pvec{r}_1^\prime - \pvec{r}_2^\prime) + \vec{A}.
\end{equation}
The form of this equation is not the same as \eqref{c1s1e7}. 

Laws of physics are expressed in a mathematical form. When we say that Newton's
laws remain unchanged across inertial frames of reference we mean that their
mathematical form remains unchanged. The variables in one frame are replaced
with the variables in another frame but there are no extra or fewer terms. If
one were to do an experiment in any of these frames one will get identical
results.

When the laws of physics change, one can devise experiments to observe the 
change. For instance, in the accelerating frame in which we got \eqref{c1s1e10}
we can do an experiment whose results will differ from those in the inertial
frame $S$. The reason for the difference is the additional term on the right
hand side of \eqref{c1s1e10}.

We will illustrate how Newton's laws are affected in a non-inertial frame
of reference. We will analyze the circular motion of a person sitting in a 
merry-go-round. From the viewpoint of an observer $S$ on the ground - an 
inertial frame of reference - the person in the merry-go-round, $S^\prime$ is 
accelerating. Therefore, he expects that there is a net force acting on 
$S^\prime$. He lists the forces on $S^\prime$.
\begin{enumerate}
\item $S^\prime$'s weight,
\item The tension in the rod that bears $S^\prime$'s weight,
\item The tension in the rod that connects $S^\prime$ to the merry-go-round's
axis.
\end{enumerate}
$S$ concludes that the first two forces balance each other because $S^\prime$
does not undergo a vertical motion. The third force is responsible for 
$S^\prime$'s acceleration. He writes Newton's second law as
\begin{equation}\label{c1s1e11}
m\frac{v^2}{r} = T,
\end{equation}
where $m$ is $S^\prime$'s mass, $v$ is his speed, $r$ is the radius of the
merry-go-round and $T$ is the tension in the rod that connects $S^\prime$ to
the merry-go-round's axis.

We can also confirm that Newton's third law is valid in $S$'s frame of 
reference. For each of three forces on $S^\prime$ enumerated above, $S^\prime$
also exerts a force of reaction. $S^\prime$ pulls the earth towards him and he
also pulls the two rods, vertical and horizontal, towards him.

Let us now analyze the same experiment from the non-inertial frame of 
$S^\prime$. He includes the same forces acting on his body as $S$ and he
realizes that the net force on him is the horizontal tension $T$. However, he 
is at rest in his frame of reference. So concludes that Newton's first law has
failed. That is why, he further concludes, he is in a non-inertial frame of
reference. Newton's second law also fails because his acceleration is zero
in spite of a horizontal force acting on him. In order to save Newton's second
law he introduces a pseudo-force, the centrifugal force, that exactly balances
the horizontal tension $T$. The horizontal tension pulls him towards the
merry-go-round's axis and the centrifugal force pushes him away by an equal
amount, keeping him at rest.

\subsection{Problem set 3}\label{c1s1s3}
\begin{enumerate}
\item The introduction of centrifugal force allowed $S^\prime$ to extend
Newton's second law to his non-inertial frame. Nevertheless, show that Newton's
third law fails. (Hint: Find all action-reaction pairs.)
\end{enumerate}

Since Newton's laws are valid in all inertial frames of reference we cannot 
conduct a mechanical experiment in a frame of reference to find out if it is
at rest or in motion. It was believed that there is a frame of reference, with
respect to the fixed stars, that is at rest and everything else in the universe
moves. If there is a space ship floating in the space with its engines turned 
off and if it is away from the gravitational influence of the stars then it is 
an example of an inertial frame of reference. But nothing in the world is a true
inertial frame because nothing is completely isolated from its surroundings. 

We know that the earth goes around the sun and rotates about its axis. Therefore
it is \emph{not} an inertial frame of reference. One can detect its motion
using mechanical experiments alone. For example, Focault's pendulum shows the
accelerated motion of the earth. However, the acceleration is quite small for
experiments at the laboratory scale and one often approximates the earth to be
an inertial frame of reference.

For almost two hundred years, Newton's laws and the two types of frames of 
reference sufficed to explain all mechanical phenomena. In the later half of
the nineteenth century, Maxwell unified the fields of electricity and magnetism
in his theory of classical electrodynamics. He predicted the existence of 
electromagnetic waves and proposed that light is an electromagnetic wave. A
wave in three dimensions is the solution of the partial differential equation
of the form
\begin{equation}\label{c1s1e12}
\frac{\partial^2\phi}{\partial x^2} + \frac{\partial^2\phi}{\partial y^2} +
\frac{\partial^2\phi}{\partial z^2} = 
\frac{1}{c^2}\frac{\partial^2\phi}{\partial t^2},
\end{equation}
where the function $\phi = \phi(x, y, z, t)$ represents the oscillating 
quantity and $c$ is the speed of the wave. Consider two inertial frames of 
reference $S$ and $S^\prime$. If $S^\prime$ moves at a speed $v$ along the 
$x$-axis of $S$ then coordinates in the two frames are related as
\begin{eqnarray}
x^\prime &=& x - vt \label{c1s1e13} \\
y^\prime &=& y \label{c1s1e14} \\
z^\prime &=& z \label{c1s1e15} \\
t^\prime &=& t \label{c1s1e16}
\end{eqnarray}
We will show in the next problem set that the wave equation \eqref{c1s1e12}
does not retain its form under the transformation of \eqref{c1s1e13} to
\eqref{c1s1e15}. We can, therefore, devise an experiment that detects the
motion of $S^\prime$.

Several ingenious experiments were conducted to detect the earth's motion using
optical phenomena. None succeeded in its goals. We are, therefore, left with
two alternatives,
\begin{enumerate}
\item Galilean transformations are the correct but Maxwell's electrodynamics is
not.
\item Maxwell's electrodynamics is right but Galilean transformations and 
Newton's need a modification.
\end{enumerate}
There was no experimental evidence in support of the first alternative
\cite{panofsky2005classical}. The second alternative was proposed by Einstein 
\cite{einstein1905electrodynamics} in 1905.
\subsection{Problem set 4}
\begin{enumerate}
\item If equation \eqref{c1s1e16} is replaced by $t^\prime = t + \alpha$, where
$\alpha$ is a constant, then show that time intervals are remain invariant.
\item Show that distance between two points remains invariant under the 
transformation of equations \eqref{c1s1e13} to \eqref{c1s1e16}. (Hint: Choose
two points $(x_1, y_1, z_1)$ and $(x_2, y_2, z_2)$ in $S$. Find their 
coordinates $(x_1^\prime, y_1^\prime, z_1^\prime)$ and $(x_2^\prime, 
y_2^\prime, z_2^\prime)$ in $S^\prime$. Use the standard formula to compute the
Euclidean distance between the points in the two frames.)
\item We will now show that the wave equation \eqref{c1s1e12} is not invariant
under the transformation defined by equations \eqref{c1s1e13} and 
\eqref{c1s1e16}.
\begin{enumerate}
\item In the frame $S^\prime$, $\phi$ is a function of $x^\prime, y^\prime, 
z^\prime, t^\prime$. Recall the definition of the 
`differential' of $\phi$,
\[
d\phi = \frac{\partial\phi}{\partial x^\prime}dx^\prime + 
\frac{\partial\phi}{\partial y^\prime}dy^\prime + 
\frac{\partial\phi}{\partial z^\prime}dz^\prime + 
\frac{\partial\phi}{\partial t^\prime}dt^\prime 
\]
Find
\[
\frac{\partial\phi}{\partial x} = \frac{\partial\phi}{\partial x^\prime}
\frac{\partial x^\prime}{\partial x} + \frac{\partial\phi}{\partial y^\prime}
\frac{\partial y^\prime}{\partial x} + \frac{\partial\phi}{\partial z^\prime}
\frac{\partial z^\prime}{\partial x} + \frac{\partial\phi}{\partial t^\prime}
\frac{\partial t^\prime}{\partial x}
\]
Hence find
\[
\frac{\partial^2\phi}{\partial {x}^2}.
\]
\item Get the expressions for
\[
\frac{\partial^2\phi}{\partial {y}^2}, \frac{\partial^2\phi}{\partial {z}^2},
\frac{\partial^2\phi}{\partial {t}^2}
\]
using similar steps.
\item Substitute the second partial derivatives obtained in the previous two
steps in the wave equation \eqref{c1s1e12}.
\end{enumerate}
\end{enumerate}

\section{Postulates of the special theory of relativity}\label{c2s2}
Einstein proposed a new theory of relativity based on the two postulates
\begin{enumerate}
\item the laws of physics are the same in all frames of reference and 
\item the speed of light is the same in all frames of reference.
\end{enumerate}
The speed of light is thus a universal constant. The second postulate has a 
profund impact on our ideas of distances and time intervals. It asserts that
the speed of a pulse of light starting from a vehicle moving with speed $v$
and going in same direction as the vehicle is not $c + v$ but $c$. We will now
show that keeping $c$ constant makes time intervals to expand and lengths to 
contract.

\subsection{Time dilation}
To that end we consider a simple clock that has a light source and detector
at one end of a rigid rod of length $L_0$ and a mirror at the other end. A light
pulse emitted by the source travels up the length of the rod, gets reflected
by the mirror and is detected after a time 
\begin{equation}\label{c1s2e1}
t_0 = 2L_0/c.
\end{equation}
We will now observe the same clock in an inertial frame moving with velocity 
$\vec{v} = v\hat{i}$.  The clock itself is aligned along the $y$ axis, 
transverse to the motion.  In this frame, the pulse of light does not take the 
same path in its onward and return journey. If $t$ is the time taken for it to 
be detected, then $t/2$ is the time it takes to reach the mirror. However, in 
that time, the clock has moved a distance $vt/2$. The light pulse, therefore, 
travels along the hypotenuse of the triangle whose other two sides have lengths $vt/2$ along the horizontal direction and $L_0$ along the vertical one. 
Therefore, its onward path length is $\sqrt{L_0^2 + v^2t^2/4}$. It is easy to 
check that the path length of its return journey is also the same. As seen 
from a frame at rest, the pulse travels a distance $2\sqrt{L_0^2 + v^2t^2/4}$ 
in time $t$. Therefore,
\begin{equation}\label{c1s2e2}
c = \frac{2\sqrt{L_0^2 + v^2t^2/4}}{t}.
\end{equation}
Upon rearranging,
\[
c^2t^2 = 4L_0^2 + v^2t^2 \Rightarrow \left(1 - \frac{v^2}{c^2}\right)t^2 = 
\frac{4L_0^2}{c^2}.
\]
Using equation \eqref{c1s2e1} we get
\begin{equation}\label{c1s2e3}
t = \frac{t_0}{\sqrt{1 - \beta^2}},
\end{equation}
where
\begin{equation}\label{c1s2e4}
\beta = \frac{v}{c}.
\end{equation}
Equation \eqref{c1s2e3} tells that the time taken for a pulse to start from
one of the clock, get reflected at the other and return depends on the frame 
of reference. For an observer at rest with respect to the clock the time is 
$t_0$. An observer who sees the clock in motion measures it as $t$ and the two
are not equal.

Note that in this derivation we have assumed that the length $L_0$ of the 
moving clock did not change. We will justify this assumption at the end of the
next subsection.

\subsection{Length contraction}
Now suppose that we are given a rod of length $L_0$ as measured in a frame of
reference in which the rod is at rest. We will find out its length when it is
moving with a velocity $\vec{v} = v\hat{i}$. Once again we use a pulse of light
for our measurement. The rod is fitted with a source of light and a detector
at one end and a mirror at the other end. As seen by an observer at rest,
if $t_1$ is the time taken by a pulse of light in its onward journey then
\begin{equation}\label{c1s2e5}
t_1 = \frac{L + vt_1}{c}.
\end{equation}
If $L$ is the length of the rod measured by the observer at rest, then the pulse
has to travel an additional distance $vt_1$ in its onward journey because the
mirror moved by that distance. In its reverse journey, however, its has to
cover a path of length $L - vt_2$ because the detector approached it by the
distance $vt_2$. Thus,
\begin{equation}\label{c1s2e6}
t_2 = \frac{L - vt_2}{c}.
\end{equation}
From equations \eqref{c1s1e5} and \eqref{c1s2e6} we get
\begin{eqnarray*}
t_1 &=& \frac{L/c}{1 - \beta} \\
t_2 &=& \frac{L/c}{1 + \beta}
\end{eqnarray*}
and the total time 
\begin{equation}\label{c1s2e7}
t = t_1 + t_2 = \frac{2L/c}{1 - \beta^2}.
\end{equation}
Now $t$ is the time measured by the observer who sees the rod moving. It is 
related to the time measured by an oberver at rest with respect to the rod by
equation \eqref{c1s2e3}. We can write \eqref{c1s2e7} in terms of $t_0$ to get
\begin{equation}\label{c1s2e8}
t_0 = \frac{2L/c}{\sqrt{1 - \beta^2}}.
\end{equation}
In the frame in which the rod is at rest, $t_0 = 2L_0/c$ so that
\begin{equation}\label{c1s2e9}
L = L_0\sqrt{1 - \beta^2}.
\end{equation}
Since $\beta < 1$, $L < L_0$. Moving bodies appear shorter.
If $v < c, \beta < 1$ and hence $t > t_0$. Time intervals between events appear
to be dilated. Events in moving frame appear to be happening at a slower pace. 

We will now show that transverse length is unaffected by motion. Consider an 
experiment in which two 
identical rings, $R_1$ and $R_2$, of radius $L_0$ are travelling along their 
symmetry axis at a speed $v$ in opposite direction. With respect to $R_1$, 
$R_2$'s has shrunk and it passes through $R_1$ when the come together. The 
same thing happens from the point of view of $R_2$. From the point of view of 
an observer at rest, the rings collide. An experiment cannot give different 
results in different frames of reference. Therefore we rule out the possibility 
of transverse contraction. The idea of length contraction is a bit subtle as
Penrose \cite{penrose1959apparent} explained in the case of the observation of
a moving sphere and Micheal Weiss \cite{weiss1995} explained in an article.

\subsection{Classical Doppler effect}
We instinctively know whether a vehicle sounding a siren is approaching us or
receeding from us. The apparent change in the frequency of sound when there is
a relative motion between the source and receiver of sound is called Doppler
effect. 

Let a source $S$ of sound emit waves at frequency $\nu_0$ and a detector $D$
approach it with a uniform velocity $\vec{v}$. If $D$ were still, it would
see the compressions of the wave pass by with a speed $c_s$. However, because
it is approaching the source with a velocity $v$, it sees compressions passing
by with speed $c_s + v$. The wavelength of the wave is seen to be unchanged.
Therefore, the apparent frequency is
\[
\nu = \frac{c_s + v}{\lambda} = \frac{c_s}{\lambda}
\left(1 + \frac{v}{c_s}\right).
\]
Now, $c_s/\lambda - \nu_0$ so that
\begin{equation}\label{c1s2e10}
\nu = \nu_0\left(1 + \frac{v}{c_s}\right).
\end{equation}

Now consider the situation when the detector is stationary but the source
approaches it with a uniform velocity $\vec{v}$. If $S$ were still the detector
will notice a compression every $\lambda$ m apart. However, because $S$ is
approaching $D$ with a speed $v$ the distance is reduced by $v/\nu_0$. The 
apparent wavelength is
\[
\lambda = \lambda_0 - \frac{v}{\nu_0} = \frac{c_s}{\nu_0} - \frac{v}{\nu_0}.
\]
Since $\lambda = c_s/\nu$, $\nu$ being the apparent frequency, we have
\begin{equation}\label{c1s2e11}
\nu = \frac{\nu_0}{1 - v/c_s}.
\end{equation}



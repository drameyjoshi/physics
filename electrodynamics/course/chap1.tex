\documentclass{article}
\usepackage{amsmath, graphicx}
\newcommand{\op}{{\;\prime}}
\newcommand{\grad}{\nabla\;}
\newcommand{\dive}{\nabla\cdot}
\newcommand{\curl}{\nabla\times}
\newcommand{\td}[2]{\frac{d{#1}}{d{#2}}}
\newcommand{\pdt}[2]{\frac{\partial{#1}}{\partial{#2}}}
\newcommand{\uv}[1]{\hat{e}_{#1}}
\newcommand{\un}{\hat{n}}

\title{Electrostatics - Charges and Fields}
\author{Amey Joshi}
\begin{document}
\begin{enumerate}
\item The magnitude of electric and gravitational forces are
\begin{eqnarray}
F_e &=& \frac{q^2}{r^2} \label{e1} \\
F_g &=& \frac{Gm^2}{r^2} \label{e2}
\end{eqnarray}
where $m$ is the mass of the proton and $q$ is the charge. Therefore,
\[
\frac{F_e}{F_g} = \frac{q^2}{Gm^2}.
\]
The ratio depends only on the fundamental constants. In the case of a pair of
protons, it is $1.235 \times 10^{36}$. An electron is roughly $2000$ times lighter
than a proton. Therefore, the same ratio for a pair of electrons will be roughly
a million times more, that is, the electric repulsion will be $10^{42}$ times
stronger than the gravitational attraction.

\begin{itemize}
\item Will the ratio depend on units?
\end{itemize}

\item From the free body diagram of an electron, at equilibrium 
\begin{equation}\label{e3}
\frac{e^2}{r^2} = m_eg
\end{equation}
so that
\begin{equation}\label{e4}
r = \frac{e}{\sqrt{m_e g}} \approx 500 \text{ cm}.
\end{equation}
If a proton was to be balanced by an electron, then we would have had to place it
closer to the electron because a heavier object is to be balanced.

\item Using the free body diagram of one of the balls,
\begin{eqnarray}
mg &=& T\cos\theta \label{e5} \\
\frac{q^2}{r^2} &=& T\sin\theta \label{e6}
\end{eqnarray}
so that
\[
\tan\theta = \frac{q^2}{mgr^2}.
\]
From the geometry of the figure, we also have $\tan\theta = 0.1$. Therefore,
\begin{equation}\label{e7}
q = \sqrt{mg\tan\theta}\; r = 85.73 \text{ esu}. 
\end{equation}

\item Due to symmetry of the arrangement, the force on each $q$ is the same. The
force on any one of them is
\[
F = \frac{q^2}{a^2} + \frac{q^2}{a^2} + \frac{q^2}{2a^2} - \frac{2qQ}{a^2},
\]
so that equilibrium requires
\begin{equation}\label{e8}
Q = \frac{5}{4}q.
\end{equation}
The ratio $Q/q$ is independent of the size of the square.

\item Align the axes such that the origin coincides with the centre of the semi-
cirle and the semi-circle lies above the $x$ axis. The field due to an elementary
charge $dq$ at $\vec{r}$ on the ring is
\[
d\vec{E} = \frac{dq}{r^2}\uv{r} = \frac{dq}{r^3}\vec{r}.
\]

Before solving the problem analytically, from the geometry alone, can you
\begin{itemize}
\item Guess the $x$-component? The $z$-component?
\item Infer the field at the origin if the ring was full?
\end{itemize}
Now, $dq = \lambda Rd\theta$, $\vec{r} = R\cos\theta\uv{x} + R\sin\theta\uv{y}$ so
that
\[
d\vec{E} = (R\cos\theta\uv{x} + R\sin\theta\uv{y})\frac{\lambda R d\theta}{R^3}
\]
and
\begin{equation}\label{e9}
\vec{E} = \frac{2Q}{\pi R^2}\uv{y}.
\end{equation}

\item A numerical problem.

\item Align the axes such that one electron is at $(-a, 0)$ and the other at 
$(a, 0)$. Let the proton be at $(x, 0)$. The potential energy of the arrangement
is
\begin{equation}\label{e10}
U = -\frac{e^2}{x - a} - \frac{e^2}{x + a} + \frac{e^2}{2a},
\end{equation}
where $e$ is the electronic charge. The roots of the equation $U = 0$ are
\begin{equation}\label{e11}
x = (2 \pm \sqrt{5})a.
\end{equation}
Since distances must be positive, we drop the root $(2 - \sqrt{5})a$. Therefore,
the proton should be placed at $x = (2 + \sqrt{5})a$.

If the proton were at $(x, y)$, the locus of curve of zero potential energy is
\begin{equation}\label{e12}
\frac{1}{2a} = \frac{1}{\sqrt{(x - a)^2 + y^2}} + \frac{1}{\sqrt{(x + a)^2 + y^2}}.
\end{equation}

If
\begin{equation}\label{e13}
F(a, x, y) = \frac{1}{2a} - \frac{1}{\sqrt{(x - a)^2 + y^2}} + \frac{1}{\sqrt{(x + a)^2 + y^2}}
\end{equation}
then a the locus of points of zero potential energy looks like can be generated using
\begin{verbatim}
import numpy as np
import matplotlib.pyplot as plt

from functools import partial
from scipy.optimize import fsolve

def F(a, x, y):
    return 1/(2 * a) - 1/np.sqrt((x - a)**2 + y**2) \
     - 1/np.sqrt((x + a)**2 + y**2)

a_curve = partial(F, 1)

a = 1
X = np.linspace(-4, 4, 1000)
Y = []

for x in X:
    roots = fsolve(a_curve, a, x)
    Y.append(roots[0])

Xs = X.tolist()
Xs.extend(Xs)
Ys = Y + [-y for y in Y]

plt.scatter(Xs, Ys, s = 1)
plt.xlabel(r'$x$')
plt.ylabel(r'$y$')
plt.title(r'$F(a=1, x, y) = 0$')
plt.savefig('purcell_c1p7.png')
plt.show()
\end{verbatim}

For $a = 1$, it looks like the one shown in figure \ref{c1f1}.
\begin{figure}
\center
\includegraphics[scale=0.6]{purcell_c1p7}
\caption{Locus of points for which $U=0$.}
\label{c1f1}
\end{figure}

\item Let $a$ be the distance between the ions. Starting from any ion, the potential
energy is
\[
U = \left(-\frac{e^2}{a} - \frac{e^2}{a}\right) + \left(\frac{e^2}{2a} + \frac{e^2}{2a}\right) + \cdots
\]
or,
\begin{equation}\label{e14}
U = -2\frac{e^2}{a}\left(1 - \frac{1}{2} + \frac{1}{3} - \frac{1}{4} + \cdots\right) 
= -2\ln 2\frac{e^2}{a}
\end{equation}

\item By Gauss' law the electric field on a closed surface is only due to the 
charges enclosed by the surface. Consider a portion of the sphere of radius $r$
and let $q$ be the charge in it. Let $dq$ be the change on a thin shell with 
radii $r$ and $r + dr$. It has a charge $dq = \rho 4\pi r^2dr$. This combination
of charges gives rise to an energy
\[
dU = \frac{qdq}{r}.
\]
Since $q = \rho(4\pi r^3)/3$,
\[
dU = \frac{(4\pi\rho)^2}{3}r^4dr
\]
so that, the energy of a uniformly charged sphere of radius $R$ is
\begin{equation}\label{e15}
U = \int_0^a dU = \frac{16\pi^2\rho^2}{3}\frac{a^5}{5}.
\end{equation}
Since
\[
\rho = \frac{3Q}{4\pi a^3},
\]
equation \eqref{e13} becomes
\begin{equation}\label{e16}
U = \frac{3}{5}\frac{Q^2}{a}.
\end{equation}

\item If $m$ is the mass of an electron and if an electron is considered to be a
sphere of radius $a$, then
\[
mc^2 =  \frac{3}{5}\frac{Q^2}{a}
\]
so that
\begin{equation}\label{e17}
a = \frac{3}{5}\frac{Q^2}{mc^2}
\end{equation}
is called the `classical electron radius'. This picture of an electron does not 
tell what holds so the negative charge together. This question is left unanswered
in classical physics.

\item A charge of $-1$ esu is at the origin and another charge of $2$ esu is at
$(1, 0)$. At a point $(x, 0)$, the electric field is
\[
\vec{E} = -\frac{1}{x^2}\uv{x} + \frac{2}{(x - 1)^2}\uv{x}
\]
so that $E = 0$ when
\[
\frac{1}{x^2} = \frac{2}{(x - 1)^2} \Rightarrow x = -1 \pm \sqrt{2}.
\]
For a point on the $y$ axis,
\[
\vec{E}(y) = -\frac{1}{y^2}\uv{y} + 2\frac{-\uv{x} + y\uv{y}}{(1 + y^2)^{3/2}} =
\frac{2}{(1 + y^2)^{3/2}}\uv{x} + \left(\frac{2y}{(1 + y^2)^{3/2}} - \frac{1}{y^2}\right)\uv{y}.
\]
$\vec{E}$ will be parallel to $x$ axis if the $y$ component vanishes. That will
happen if $2y^3 = (1 + y^2)^{3/2}$ or $4y^6 = (1 + y^2)^3 = 1 + 3y^2 + 3y^4 + y^6$,
or
\[
3y^6 - 3y^4 - 3y^2 - 1 = 0.
\]
The only real roots of this equation are $y = \pm 1.30476603$. The roots can be
found using
\begin{verbatim}
import numpy as np

coeffs = [3, 0, -3, 0, -3, 0, -1]
[r for r in np.roots(coeffs) if np.isreal(r)]
\end{verbatim}

\item Let the equilateral triangle have vertices at $(\pm a, 0)$ with positive
ions and $(0, a\sqrt{3})$ with a negative ion. The electric field at a point
$(x, y)$ is
\begin{eqnarray*}
\vec{E} &=& \uv{x}\left[\frac{x-a}{((x-a)^2+y^2)^{3/2}}+
            \frac{x+a}{((x+a)^2+y^2)^{3/2}}-\frac{x}{(x^2+(y-a\sqrt{3})^2)^{3/2}}\right]+\\
 & &\uv{y}\left[\frac{y}{((x-a)^2+y^2)^{3/2}}+\frac{y}{((x+a)^2+y^2)^{3/2}}-
    \frac{y - a\sqrt{3}}{(x^2+(y-a\sqrt{3})^2)^{3/2}}\right]
\end{eqnarray*}
Symmetry suggests that the equilibrium point is likely to be on the $y$-axis. The
$x$ component of the field is zero at all points on the $y$-axis. The $y$-component
is
\[
E_y = \frac{2y}{(a^2+y^2)^{3/2}} - \frac{1}{(y-a\sqrt{3})^2}
\]    
$E_y = 0$ if $4y^2(y - a\sqrt{3})^4 = (a^2 + y^2)^3$, that is,
\begin{equation}\label{e18}
3y^6 - 24a\sqrt{3}y^5 + 117 a^2y^4 - 72\sqrt{3}a^3y^3 - 3a^4y^2 + (36a^4 - a^6) = 0.
\end{equation}
If we choose $a = 1$ then the roots of this equation are found using
\begin{verbatim}
import numpy as np

coeffs = [3, -24*np.sqrt(3), 117, -72*np.sqrt(3), -3, 0, 35]
[np.round(r, 4) for r in np.roots(coeffs) if np.isreal(r)]
\end{verbatim}
They are $y = 10.5277, 0.8501$.

\item Assume that the cloud is like an infinite plane. Then $E = 2\pi\sigma$ so
that the surface charge density on it is
\[
\sigma = \frac{0.1}{2\pi}\text{ esu/cm}^{2}.
\]
If $A$ is the area of the cloud then the volume of water in $0.25$ cm rain is
$V = 0.25A$. If $r$ is the radius of a rain drop, then the number of rain drops
in this volume is
\begin{equation}\label{e19}
N\times \frac{4\pi}{3}r^3 = 0.25A
\end{equation}
that is,
\begin{equation}\label{e20}
N = \frac{0.75A}{4\pi r^3}
\end{equation}
Let $q$ be the charge on each of them. Then, $Nq = \sigma A$ so that
\begin{equation}\label{e21}
\frac{0.75q}{4\pi r^3} = \frac{0.1}{2\pi}
\end{equation}
that is,
\begin{equation}\label{e22}
q = \frac{4}{15}r^3.
\end{equation}
The electric field strength at the surface of the drop has magnitude $q/r^2$, that
is,
\begin{equation}\label{e23}
E = \frac{4}{15}r.
\end{equation}
For $r = 0.1$ cm, it is $E = 0.0267$ esu/cm$^2$.

\item Let the ring be of radius $R$. Align the axes such that the origin coincides
with the centre, the ring is in the $xy$ plane so that the $z$ axis is perpendicular
to the ring. Before proceeding, 
\begin{itemize}
\item What is the field at $z = 0$?
\item What is it at $z \rightarrow \infty$?
\item Can you guess where the maximum will be?
\end{itemize}

Now let's do the maths. Fix a point $(0, 0, z)$ on the $z$ axis. Let a line joining
it and a point on the ring make an angle $\alpha$ with the $z$ axis. The electric
field due to an element of charge at that point will be along this line and make 
and angle $\alpha$ with the $z$ axis. Its sine component will be cancelled by the
field due to a diametrically opposite element. Therefore, it's only contribution
will be a cosine component, aligned along the $z$ axis. Thus,
\[
d\vec{E} = \frac{dq}{r^2}\cos\alpha\uv{z}.
\]
$r^2 = z^2 + R^2$, $dq = \lambda d\theta$, where $\lambda = Q/(2\pi R)$ is the
charge density. Thus,
\begin{equation}\label{e24}
\vec{E} = \frac{2\pi R\lambda}{R^2 + z^2}\cos\alpha\uv{z} = \frac{Qz}{(R^2 + z^2)^{3/2}}\uv{z}
\end{equation}
Therefore,
\[
\td{E}{z} = \frac{Q}{(R^2 + z^2)^{3/2}} - \frac{3}{2}\frac{Qz(2z)}{(R^2 + z^2)^{5/2}}
= Q\frac{R^2 - 2z^2}{(R^2 + z^2)^{5/2}}.
\]
The extremum is at $z = R/\sqrt{2}$ or $\cos\alpha = z/R = 1/\sqrt{2}$ or $\alpha=
\pi/4$.

\item For a point at distance $r \le a$, consider a spherical surface whose centre
coincides with that of the charge distribution and whose radius is $r$. If $\rho$
is the constant charge density then the surface encloses a charge
\[
q(r) = \frac{4\pi}{3}r^3 \rho
\]
By Gauss law, 
\[
E \times 4\pi r^2 = 4\pi \times \frac{4\pi}{3}r^3 \rho
\]
so that
\begin{equation}\label{e25}
\vec{E} = \frac{4\pi\rho}{3}r\uv{r} = \frac{4\pi\rho}{3}\vec{r}.
\end{equation}

For a point at a distance $r > a$, the charge enclosed is a constant
\[
Q = \frac{4\pi\rho}{3}a^3
\]
and
\begin{equation}\label{e26}
\vec{E} = \frac{Q}{r^2}\uv{r}.
\end{equation}
At $r = a$, \eqref{e26} is
\[
\vec{E}(a) = \frac{Q}{a^3}\vec{r} = \frac{4\pi\rho}{3}\vec{r},
\]
which is same as \eqref{e25}.

\item The field at $B$ will be radially outwards and will have a magnitude
\[
E = \frac{Q}{a^2},
\]
where
\[
Q = \frac{4\pi\rho}{3}a^3 - \frac{4\pi\rho}{3}\frac{a^3}{8} = 
\frac{7}{8}\frac{4\pi\rho}{3}a^3,
\]
Before carving out the cavity, the field at $A$ was zero. We can consider it to
be a sum of field in the matter in the carved out ball plus the field due to the
rest. The field due to the carved out ball is radially outwards from the centre
of the cavity and has magnitude 
\[
E_c = \frac{Q_c}{(a/2)^2} = \frac{4\pi\rho}{3}\frac{a}{2}.
\]
If $\vec{R}$ is the centre of the cavity then the radially outward direction at
a point $\vec{r}$ is $\vec{r} - \vec{R}$. Thus,
\[
\vec{E}_c = \frac{4\pi\rho}{3}\frac{a}{2}\frac{\vec{r} - \vec{R}}{|\vec{r} - \vec{R}|}.
\]
If the origin is chosen to coincide with the point $A$ then
\[
\vec{E}_c(A) = -\frac{4\pi\rho}{3}\frac{a}{2}\hat{R}.
\]

\item When the charge is at the centre of the cube, by Gauss law,
\[
\oint\vec{E}\cdot d\vec{a} = 4\pi q.
\]
The symmetry of the cube suggests that there is an equal flux through each face.
Therefore, through any one of them, it is
\[
\int\vec{E}\cdot d\vec{a} = \frac{2\pi q}{3}.
\]

When the charge is placed at a corner of a cube, assemble eight similar cubes so
that the charge is now at the centre of them. The flux through any one of them is
$4\pi q/8 = \pi q/2$. The field will be parallel to the faces at whose intersection
$q$ is. Therefore, there will be no flux out of them. The remaining three faces
are symmetric and will carry the flux equally. Therefore, the flux through any
one of them is $\pi q/6$.
\end{enumerate}
\end{document}
\chapter{Electromagnetic Waves}\label{c6}
\begin{enumerate}
\item Maxwell equations in vacuum are:
\begin{eqnarray}
\dive\vec{E} &=& 0 \label{c6e1} \\
\curl\vec{E} &=& -\frac{1}{c}\pdt{\vec{H}}{t} \label{c6e2} \\
\dive\vec{H} &=& 0 \label{c6e3} \\
\curl\vec{H} &=& \frac{1}{c}\pdt{\vec{E}}{t} \label{c6e4}
\end{eqnarray}
If the fields are also time independent, then we will have $\dive\vec{E} = 0,
\curl\vec{E} = 0, \dive\vec{H} = 0$ and $\curl\vec{H} = 0$. From equations
\eqref{c5e16} (Coulomb's law) and \eqref{c5e140} (Biot-Savart's law), we get
$\vec{E} = 0$ and $\tav{H} = 0$. Therefore, we assume that the fields depend
on time.

\item We will soon show that equations \eqref{c6e1} to \eqref{c6e4} have a
non-zero solution in vacuum. That is, electromagnetic fields can exist in vacuum.
We will also show that they satisfy the wave equation. Solutions of Maxwell
equations in vacuum are called electromagnetic waves.

\item As usual, it is easier to analyse the potentials. Let us choose them such
that $\varphi = 0$ and $\dive\vec{A} = 0$. Equation \eqref{c6e2} becomes
\[
-\frac{1}{c}\curl\pdt{\vec{A}}{t} = 
-\frac{1}{c}\frac{\partial}{\partial t}\curl\vec{A},
\]
an identity. While, equation \eqref{c6e4} becomes
\[
\curl\curl\vec{A} = -\frac{1}{c^2}\spdt{\vec{A}}{t}.
\]
The lhs of this equation is $\grad\dive\vec{A} - \nabla^2\vec{A}$. Since we chose
$\dive{\vec{A}} = 0$, we get
\begin{equation}\label{c6e5}
\nabla^2\vec{A} - \frac{1}{c^2}\spdt{\vec{A}}{t} = 0.
\end{equation}
The choice $\dive{\vec{A}} = 0$ is called the Coulomb gauge and that it can be
exercised without violating physics is explained in the point leading to 
\eqref{c5e134}. Taking the curl of this equation gives
\begin{equation}\label{c6e6}
\nabla^2\vec{H} - \frac{1}{c^2}\spdt{\vec{H}}{t} = 0.
\end{equation}
Taking a partial derivative of \eqref{c6e5} and recalling that we chose $\varphi
= 0$, gives
\begin{equation}\label{c6e7}
\nabla^2\vec{E} - \frac{1}{c^2}\spdt{\vec{E}}{t} = 0.
\end{equation}

\item We can repeat this analysis using the Maxwell equations written in terms of
the electromagnetic field tensor. From \eqref{c4e37},
\[
\pdt{F^{\mu\nu}}{x^\nu} = -\frac{4\pi}{c}j^\mu.
\]
In the absence of sources, it becomes
\[
\pdt{F^{\mu\nu}}{x^\nu} = 0.
\]
We now use the definition 
\[
F^{\mu\nu} = \pdt{A^\nu}{x_\mu} - \pdt{A^\mu}{x_\nu}
\]
to get
\[
\frac{\partial}{\partial x^\nu}\pdt{A^\nu}{x_\mu} - \frac{\partial}{\partial x^\mu}\pdt{A^\nu}{x_\nu} = 0
\Rightarrow
\frac{\partial}{\partial x_\mu}\pdt{A^\nu}{x^\nu} - \frac{\partial}{\partial x^\mu}\pdt{A^\nu}{x_\nu} = 0.
\]
We now choose $A^\mu$ such that
\begin{equation}\label{c6e8}
\pdt{A^\nu}{x^\nu} = 0.
\end{equation}
In terms of coordinates, it is
\begin{equation}\label{c6e9}
\frac{1}{c}\pdt{\varphi}{t} - \dive\vec{A} = 0.
\end{equation}
Equations \eqref{c6e8} and \eqref{c6e9} are called the \emph{Lorenz gauge}. Unlike
the condition of Coulomb gauge, that of the Lorenz gauge is invariant under Lorentz 
transformation. We are thus left with
\begin{equation}\label{c6e10}
\frac{\partial}{\partial x^\nu}\pdt{A^\mu}{x_\nu} = 
g^{\nu\rho}\frac{\partial}{\partial x^\nu}\pdt{A^\mu}{x^\rho} = 0.
\end{equation}
It is equivalent to \eqref{c6e5}.

\item The nine equations \eqref{c6e5}, \eqref{c6e6} and \eqref{c6e7} are all of
the form
\[
\nabla^2 f - \frac{1}{c^2}\spdt{f}{t} = 0 \Rightarrow 
\spdt{f}{t} - c^2\nabla^2 f = 0,
\]
where $f$ is one of the components of $\vec{A}, \vec{E}$ or $\vec{H}$. If the 
function $f$ depends only on $x$ and $t$, that is, if it is independent of $y$ and
$z$, then the solution $f$ is called a plane wave. In this case, the equation 
simplifies to
\begin{equation}\label{c6e11}
\spdt{f}{t} - c^2\spdt{f}{x} = 0.
\end{equation}
To solve this equation, write it as
\begin{equation}\label{c6e12}
\left(\frac{\partial}{\partial t} - c\frac{\partial}{\partial x}\right)
\left(\frac{\partial}{\partial t} + c\frac{\partial}{\partial x}\right)f = 0.
\end{equation}
Introduce the variables
\begin{eqnarray}
\xi &=& t - \frac{x}{c} \label{c6e13} \\
\eta &=& t + \frac{x}{c} \label{c6e14}
\end{eqnarray}
so that
\begin{eqnarray}
\frac{\partial}{\partial t} &=& \frac{\partial}{\partial\xi} = 
 \frac{\partial}{\partial\eta} \label{c6e15} \\
\frac{\partial}{\partial x} &=& -\frac{1}{c}\frac{\partial}{\partial\xi} =
\frac{1}{c}\frac{\partial}{\partial\eta} \label{c6e16}
\end{eqnarray}
so that
\begin{eqnarray}
\frac{\partial}{\partial\eta} &=& \frac{1}{2}
	\left(\frac{\partial}{\partial t} + c\frac{\partial}{\partial x}\right) 
	\label{c6e17} \\
\frac{\partial}{\partial\xi} &=& \frac{1}{2}
	\left(\frac{\partial}{\partial t} - c\frac{\partial}{\partial x}\right) 
	\label{c6e18}
\end{eqnarray}
This pair of relations allows us to write \eqref{c6e12} as
\begin{equation}\label{c6e19}
\frac{\partial^2 f}{\partial\xi\partial\eta} = 0.
\end{equation}
Its solution can be written as
\begin{equation}\label{c6e20}
f(\xi, \eta) = f_1(\xi) + f_2(\eta).
\end{equation}
In terms of the original variables,
\begin{equation}\label{c6e21}
f(x, t) = f_1\left(t - \frac{x}{c}\right) + f_2\left(t + \frac{x}{c}\right).
\end{equation}

\item If $f_2 = 0$ then
\[
f(x, t) = f_1\left(t - \frac{x}{c}\right)
\]
Any reasonably function of $t - x/c$ is a solution of \eqref{c6e11}. For a fixed
$x$, the value of the function changes with $t$. Likewise, for a fixed $t$, the
function changes value with $x$. Further the value of the function is the same
for a given value of its argument, namely, $t - x/c$ or $ct - x$. If we focus 
our attention on one such value, we will see it travelling along the $x$ axis 
with a speed $x/t = c$, the speed of light.

If $f_1 = 0$ and $f = f_2$ then a similar interpretation is applicable except
that a point with a certain value of $f$ travels along the negative $x$ axis
with the speed $c$.

\item How do we use this information to derive the forms of $\vec{E}$ and $\vec{H}$
for plane waves? We will start by deriving an expression for the vector potential.
Since the components of $\vec{A}$ individually satisfy \eqref{c6e11}, none of them
will be functions of $y$ and $z$. Further, $\dive\vec{A} = 0$ implies that
\[
\pdt{A_x}{x} = 0.
\]
Therefore,
\[
\spdt{A_x}{t} - c^2\spdt{A_x}{x} = 0 \Rightarrow \spdt{A_x}{t} = 0
\Rightarrow \pdt{A_x}{t} = \text{const.}
\]
This, in turn, implies that $E_x$ is a constant. If this constant is non-zero
then it results in a constant electric field in the direction of propagation
of the wave, also called the longitudnal direction. Being a constant, it is
not of the form $f(t \pm x/c)$ and therefore unrelated to the electromagnetic
wave. We can ignore it by setting $A_x = 0$. We are thus left with
\begin{equation}\label{c6e22}
\vec{A} = A_y(x)\uv{y} + A_z(x)\uv{z}
\end{equation}
and hence
\begin{eqnarray}
\vec{E} &=& -\frac{1}{c}\pdt{\vec{A}}{t} \label{c6e23} \\
\vec{H} &=& \curl\vec{A} \label{c6e24}
\end{eqnarray}
From equation \eqref{c6e15},
\[
\pdt{\vec{A}}{t} = \pdt{\vec{A}}{\xi}
\]
so that if we denote rhs of the above equation as $\vec{A}^\op$ then equation 
\eqref{c6e23} becomes
\begin{equation}\label{c6e25}
\vec{E} = -\frac{1}{c}\vec{A}^\op.
\end{equation}
Now,
\begin{equation}\label{c6e26}
\curl\vec{A} = -\uv{y}\pdt{A_z}{x} + \uv{z}\pdt{A_y}{x} = 
\frac{1}{c}\uv{y}\pdt{A_z}{\xi} - \frac{1}{c}\uv{z}\pdt{A_y}{\xi}
\end{equation}
where we used \eqref{c6e16}. If $\un$ is the direction of wave propagation then
$\un = \uv{x}$ and 
\begin{equation}\label{c6e27}
\un\times\vec{A}^\op = \uv{z}A^\op_y - \uv{y}A^\op_z
\end{equation}
and hence from \eqref{c6e26} and \eqref{c6e27} we get
\begin{equation}\label{c6e28}
\vec{H} = -\frac{1}{c}\un\times\vec{A}^\op.
\end{equation}
From \eqref{c6e25} and \eqref{c6e28} it immediately follows that
\begin{equation}\label{c6e29}
\vec{H} = \un\times\vec{E}.
\end{equation}
Thus both fields are perpendicular to $\un$, the direction of propagation. As a 
result, the plane electromagnetic waves in vacuum are transverse in nature. 
Equation \eqref{c6e29} also assures that magnitude of the two fields is the same.

\item The energy flux is
\begin{equation}\label{c6e30}
\vec{S} = \frac{c}{4\pi}\vec{E}\times\vec{H} = 
\frac{c}{4\pi}\vec{E}\times(\un\times\vec{E}) = 
\frac{c}{4\pi}E^2\un = \frac{c}{4\pi}H^2\un.
\end{equation}
We can as well write it as 
\begin{equation}\label{c6e31}
2W = \frac{c}{4\pi}(E^2 + H^2)\un
\end{equation}
so that
\begin{equation}\label{c6e32}
\vec{S} = cW\un
\end{equation}
where we used \eqref{c4e50}, where $W$ where is the energy density of the field.
The components of $\vec{S}/c$ are the components 
\[
T^{01}, T^{02}, T^{03}
\]
of the energy-momentum tensor \eqref{c4e90}. We also argued in point 23 of chapter
\ref{c4} that $T^{01}/c, T^{02}/c, T^{03}/c$ are the components of momentum 
density of the system. Thus, we can interpret $\vec{S}/c^2$ as the momentum flux
in the system. From \eqref{c6e32} we see that electromagnetic waves carry momentum
in the direction of propagation.

The relation between energy density and momentum density of electromagentic 
waves is similar to that of relativistic particle of zero mass.

\item If the $x$-axis is chosen to be along the direction of propagation of
the wave then $\vec{E} = E\uv{y}$ and $\vec{H} = \uv{z}$. The components of the
energy momentum tensor evaluated below \eqref{c4e88} are
\begin{eqnarray}
T^{00} &=& \frac{E^2 + H^2}{8\pi} \label{c6e33} \\
T^{11} &=& -\frac{E^2 + H^2}{8\pi} = -\sigma_{xx} \label{c6e34},
\end{eqnarray}

\item Lorentz transformation for $W$ follows from \eqref{c1e88}.
\begin{equation}\label{c6e35}
W = \frac{W^\op + 2\beta S^\op_x/c + \beta^2\sigma_{xx}^\op}{1 - \beta^2}.
\end{equation}
If $\alpha^\op$ is the angle between $\bm{\beta}$ and the direction of 
propagation of the wave then
\[
S_x^\op = cW^\op\cos\alpha^\op
\]
and 
\begin{equation}\label{c6e36}
\sigma_{xx}^\op = -W^\op\cos^2\alpha^\op,
\end{equation}
{\color{red}To do: justify \eqref{c6e36}} so that
\begin{equation}\label{c6e37}
W = \frac{W^\op(1 - \beta\cos\alpha^\op)^2}{1 - \beta^2}.
\end{equation}
\end{enumerate}
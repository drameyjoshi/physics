\documentclass{beamer}
\usefonttheme[onlymath]{serif}
\usepackage{graphicx, hyperref}
\hypersetup{colorlinks, urlcolor=blue}
\usepackage{subcaption}
\usetheme{Boadilla}

\newcommand{\pvec}[1]{\vec{#1}\mkern2mu\vphantom{#1}}

\title{1-d problems}
\author{Amey Joshi}
\date{\today}

\begin{document}
\begin{frame}
\titlepage
\end{frame}

\begin{frame}
\frametitle{Schr\"{o}dinger's time independent equation}
\begin{enumerate}
\item Schr\"{o}dinger's equation is
\begin{equation}\label{e1}
-\frac{\hslash^2}{2m}\nabla^2\Psi + V(\vec{r},t)\Psi = i\hslash
\frac{\partial\Psi}{\partial t}.
\end{equation}
\item When $V$ is independent of $t$, one can write $ \Psi(\vec{r}, t) = 
\psi(\vec{r})e^{iEt/\hslash}$ where the function $\psi$ satisfies the 
Schr\"{o}dinger's time independent equation
\begin{equation}\label{e2}
-\frac{\hslash^2}{2m}\nabla^2\psi + V(\vec{r})\psi = E\psi.
\end{equation}
\item In terms of the Hamiltonian, we can write \eqref{e3} as
\begin{equation}\label{e3}
\hat{H}\psi = E\psi.
\end{equation}
\item Equation \eqref{e3} is an eigenvalue equation. $E$ is the eigenvalue of
the Hamiltonian. From now on, we shall mean \eqref{e3} when we refer to 
Schr\"{o}dinger equation.
\end{enumerate}
\end{frame}

\begin{frame}
\frametitle{What are 1-d problems}
\begin{enumerate}
\item $\psi$ and $V$  depend on only $x$. Equation \eqref{e2} simplifies to
\begin{equation}\label{e4}
-\frac{\hslash^2}{2m}\frac{d^2\psi}{dx^2} + V(x)\psi = E\psi.
\end{equation}
\item This is a second order ordinary differential equation. It is easier to
solve than the partial differential equation \eqref{e2}.
\item $\psi$ is required to satisfy:
\begin{enumerate}
\item Continuity and continnuity of the first derivative. Technically, a member
of the class $C^1$.
\item $\psi \rightarrow 0$ as $x \rightarrow \pm\infty$. We expect the particle
to be in a finite region.
\end{enumerate}
Usually, these conditions suffice to solve the problem.
\item We will solve \eqref{e4} for a few simple cases.
\end{enumerate}
\end{frame}

\begin{frame}
\frametitle{Particle in a 1-d box}
\begin{enumerate}
\item The potential is defined by
\begin{equation}\label{e5}
V(x) = \begin{cases} 0 & \text{ if } -L/2 \le x \le L/2 \\
\infty & \text{ otherwise.}
\end{cases}
\end{equation}
\item Particle cannot escape from the box.
\item In the box's interior, the Schr\"{o}dinger's equation takes a simple
form
\begin{equation}\label{e6}
\psi^{\prime\prime} + k^2\psi = 0,
\end{equation}
where
\begin{equation}\label{e7}
k^2 = \frac{2mE}{\hslash^2}.
\end{equation}
\item Can you quickly solve this equation?
\end{enumerate}
\end{frame}

\begin{frame}
\frametitle{Particle in a 1-d box}
\begin{enumerate}
\item General solution is $\psi(x) = \alpha e^{-ikx} + \beta e^{ikx}$.
\item Boundary conditions $\psi(-L/2) = \psi(L/2) - 0$. Why?
\item Boundary conditions require that $2i\beta e^{ikL/2}\sin(kL) = 0$, which is
true if and only if
\begin{equation}\label{e8}
kL = n\pi, n = 0, \pm 1, \pm 2, \ldots.
\end{equation}
\item If $n = 0, k = 0$ and $\psi(x) = \alpha + \beta$. What does this result 
in?
\item Exclude $k = 0$. That gives us
\begin{equation}\label{e9}
p = \frac{nh}{2L}, n = \pm 1, \pm 2, \ldots.
\end{equation}
Momentum is quantised. So is energy $E = p^2/2m$ or
\begin{equation}\label{e10}
E = \frac{n^2h^2}{8mL^2}, n = \pm 1, \pm 2, \ldots.
\end{equation}
\end{enumerate}
\end{frame}

\begin{frame}
\frametitle{Particle in a 1-d box}
\begin{enumerate}
\item This is not an artificial problem. Semiconductor quantum wells are 
modelled as particles in a box.
\item Electrons in long chained molecules are sometimes treated as particles in
a box, although the model is not great because there is never a single electron.
\item Contrast the quantum mechanical situation with the classical one. In
the classical case, all energies and momenta a permitted.
\item Note that in equations \eqref{e9} and \eqref{e10}, $n \ne 0$. The energy
and momentum of a particle is never zero. Can you interpret this fact from the
point of view of the uncertainty principle?
\end{enumerate}
\end{frame}

\begin{frame}
\frametitle{Particle in a finite potential well}
\begin{enumerate}
\item Let's make the problem a little more realistic and express the potential
as
\begin{equation}\label{e11}
V(x) = \begin{cases} 0 \text{ if } -L/2 \le x \le L/2 \\
 V \text{ otherwise.}
\end{cases}
\end{equation}
\item The Schr\"{o}dinger equation is
\begin{equation}\label{e12}
-\frac{\hslash^2}{2m}\frac{d^2\psi}{dx^2} = E\psi
\end{equation}
inside the well and
\begin{equation}\label{e13}
-\frac{\hslash^2}{2m}\frac{d^2\psi}{dx^2} + V\psi = E\psi.
\end{equation}
outside the well.
\end{enumerate}
\end{frame}

\begin{frame}
\frametitle{Particle in a finite potential well}
\begin{enumerate}
\item The solution in the well is
\begin{equation}\label{e14}
\psi(x) = \alpha e^{-ikx} + \beta e^{ikx} 
\end{equation}
where $\alpha$ and $\beta$ are constants of integration and 
\begin{equation}\label{e15}
k^2 = \frac{2mE}{\hslash^2}.
\end{equation}
\item Assume $E < V$, that is, the particle does not have enough energy to cross
the barrier. Then, the equation outside the box is
\begin{equation}\label{e16}
k_o^2 = \frac{2m(V - E)}{\hslash^2} > 0
\end{equation}
and Schr\"{o}dinger equation is
\begin{equation}\label{e17}
\psi_o(x) = \alpha_o e^{-k_ox} + \beta_o e^{k_ox},
\end{equation}
$\alpha_o, \beta_o$ being constants of integration.
\end{enumerate}
\end{frame}

\begin{frame}
\frametitle{Particle in a finite potential well}
\begin{enumerate}
\item The wave function is \emph{not} zero outside the box. 
\item The particle has a non-zero probability of being found in a region where
$E < V$.
\item If the potential barrier were a narrow region, the particle would pierce
through the barrier and emerge out of the box. This is the phenomenon of
quantum tunnelling.
\item The tunnel diode exploits the tunneling phenomenon in the lower voltage
region when forward biased. Beyond a certain point, as the forward voltage
increases, tunneling ceases to happen and the current drops despite a rise
in voltage giving rise to a `negative resistance'.
\item Also used in scanning tunnelling microscope which can produce images of
atomic sized features.
\end{enumerate}
\end{frame}

\begin{frame}
\frametitle{Particle in a finite potential well}
A few points to ponder upon
\begin{enumerate}
\item In equation \eqref{e16}, we assumed $E < V$. What will happen if $V > E$.
\item Do you expect the energy and momentum to be quantized?
\item How will you ensure the continuity of the wave function?
\item How will you solve the problem of a particle in a three dimensional box?
\end{enumerate}
\end{frame}

\end{document}

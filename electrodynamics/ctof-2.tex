\chapter{Relativistic Mechanics}\label{c2}
\begin{enumerate}
\item In the previous chapter, we argued that the expression
\begin{equation}\label{c2e1}
\delta t = \frac{1}{c}\int_A^B ds
\end{equation}
has a maximum for a particle if it is stationary. In this expression, $A$ and 
$B$ are world points and 
\[
ds = \sqrt{dx^\nu dx_\nu}.
\]
It follows immediately that, among all particles that start their journey at 
$A$ and end it at $B$, the expression
\begin{equation}\label{c2e2}
-\frac{1}{c}\int_A^B ds,
\end{equation}
has a minimum for the particle that is stationary.

\item We also know that for a classical system, there exists an action integral
$S$ whose value is an extremum for a path followed by the system in the 
configuration space. It is a minimum over an infinitesiml length of the path.

\item If we have to extend this idea to relativistic mechanics then the integral
must be invariant under Lorentz transformation. Therefore, it must be a true scalar.
One example of such an integral is given by \eqref{c2e2}. We can mildly generalise
it to
\begin{equation}\label{c2e3}
S = -\alpha \int_A^B ds,
\end{equation}
where $\alpha$ is a constant. From the discussion around equations \eqref{c2e1}
and \eqref{c2e3}, we infer that $\alpha > 0$.

\item We would like to write the action integral for relativistic systems as
\begin{equation}\label{c2e4}
S = \int_{t_1}^{t_2}Ldt,
\end{equation}
in analogy for the classical systems. From equations \eqref{c2e3} and 
\eqref{c2e4}, we get
\begin{equation}\label{c2e5}
L = -\alpha c\sqrt{1 - \frac{v^2}{c^2}},
\end{equation}
where
\[
v^i = d{x^i}{t}
\]
is the 3-velocity of the particle. Can we guess $\alpha$?

\item As $v/c \rightarrow 0$, $L$ of \eqref{c2e5} should go over to the classical
Lagrangian for a free particle, which is just $mv^2/2$. For small $v/c$, we can
write \eqref{c2e5} as
\[
L = -\alpha c\left(1 - \frac{1}{2}\frac{v^2}{c^2}\right) = 
-\alpha c + \frac{\alpha}{2}\frac{v^2}{c}.
\]
A constant term in a Lagrangian can always be ignored and we infer that
\begin{equation}\label{c2e6}
\alpha = mc.
\end{equation}
Therefore, we guess the correct form of the Lagrangian for a free relativistic
particle to be
\begin{equation}\label{c2e7}
L = -mc^2\sqrt{1 - \frac{v^2}{c^2}}.
\end{equation}

\item The generalised momentum corresponding to this Lagrangian is
\begin{equation}\label{c2e8}
p^i = \pdt{L}{v^i} = (-mc^2)\frac{1}{2}\left(1 - \frac{v^2}{c^2}\right)^{-1/2}\frac{-2v^i}{c^2}
= \frac{mv^i}{\sqrt{1 - v^2/c^2}}.
\end{equation}
If
\begin{eqnarray}
\beta &=& \frac{v}{c} \\ \label{c2e9}
\gamma &=& \frac{1}{\sqrt{1 - \beta^2}} \label{c2e10}
\end{eqnarray}
then
\begin{equation}\label{c2e11}
\td{p^i}{t} = m\gamma \td{v^i}{t} + m\td{\gamma}{t}v^i.
\end{equation}
This equation is quite different from the one used in classical mechanics.

\item The energy of the particle is
\[
E = p^i v_i - L = mv^2\gamma + \frac{mc^2}{\gamma} = m\gamma\left(v^2 + \frac{c^2}{\gamma^2}\right)
\]
so that
\begin{equation}\label{c2e12}
E = m\gamma\left(v^2 + c^2\left(1 - \frac{v^2}{c^2}\right)\right) 
= mc^2\gamma
\end{equation}

\item From \eqref{c2e12} we see that the energy of the particle is not zero in
the relativistic framework even if the particle is resting. When $v = 0$, $\gamma
= 1$ and the particle's energy is
\begin{equation}\label{c2e13}
E = mc^2.
\end{equation}
For small velocities, we can approximate \eqref{c2e12} to
\begin{equation}\label{c2e14}
E = mc^2\left(1 + \frac{1}{2}\frac{v^2}{c^2} + O(\beta^4)\right) = 
mc^2 + \frac{1}{2}mv^2 + O(\beta^4).
\end{equation}

\item Since $\beta < 1$ for material bodies, $\gamma > 1$ and hence, from 
\eqref{c2e12} $E \ge mc^2 > 0$. Thus energy of a free particle is always positive.
This is also true in classical mechanics for an elementary particle, that is a
particle without internal degrees of freedom. For a composite body, in classical
mechanics, energy can be negative and is determined to within a constant. Howwever,
in the relativistic regime this is not so, $E > 0$ is always true.

\item For a composite body at rest, the energy $E = Mc^2$ is not the same as 
\[
\sum_{i=1}^N m_ic^2,
\]
where $m_i, i = 1, \ldots, N$ are the rest masses of the constituent particles.
The rest energy will also include the energy of interaction between the particles.
Since 
\[
M \ne \sum_{i=1}^N m_ic
\]
conservation of mass is not true in relativistic mechanics.

\item From \eqref{c2e8},
\[
p^2 = p_ip^i = mv^2\gamma^2
\]
and from \eqref{c2e12}
\[
E^2 = m^2c^4\gamma^2 \Rightarrow E^2 - E^2\beta^2 = m^2c^4 \Rightarrow 
E^2 - p^2c^2  = m^2c^4,
\]
so that
\begin{equation}\label{c2e15}
E^2 = p^2c^2 + m^2c^4.
\end{equation}
Since $E$ is expressed in terms of $p$, we can as well write the Hamiltonian as
\begin{equation}\label{c2e16}
H = c\sqrt{p^2 + m^2c^2}
\end{equation}

\item Since $p^i = mv^i\gamma$ and $E = mc^2\gamma$, we also have
\begin{equation}\label{c2e17}
p^i = \frac{E}{c^2}v^i.
\end{equation}
If $v \rightarrow c$, $\gamma \rightarrow \infty$. In this limit, both $E$ and
$p^i$ blow up unless $m = 0$. In that case, the two are related by
\begin{equation}\label{c2e18}
E = pc.
\end{equation}

\item From equations \eqref{c2e4} and \eqref{c2e6}, we can write the principle of
least action as
\begin{equation}\label{c2e19}
\delta S = -mc\delta\int_{t_1}^{t_2} ds = 0.
\end{equation}
Since $ds = \sqrt{dx^\mu dx_\mu}$, 
\begin{eqnarray*}
\delta ds &=& \frac{1}{2}\frac{(\delta dx^\mu)dx_\mu + dx^\mu (\delta dx_\mu)}{ds} \\
 &=& \frac{1}{2}\frac{(\delta dx^\mu)dx_\mu + dx_\mu (\delta dx^\mu)}{ds} \\
 &=& \frac{dx_\mu(\delta dx^\mu)}{ds} \\
 &=& u_\mu \delta(dx^\mu) \\
 &=& u_\mu d(\delta x^\mu),
\end{eqnarray*}
where we used \eqref{c1e78}. Thus, equation \eqref{c2e19} becomes
\begin{equation}\label{c2e20}
\delta S = -mc\int_{t_1}^{t_2}u_\mu d(\delta x^\mu)
\end{equation}
Integrating by parts,
\begin{equation}\label{c2e21}
\delta S = -mc u_\mu \delta x^\mu\Big|_{t_1}^{t_2} + 
mc\int_{t_1}^{t_2}\delta x^\mu \td{u_\mu}{s}ds.
\end{equation}
Since variations in trajectories vanish at the end points, the first term on the
right hand side is zero and hence
\begin{equation}\label{c2e22}
\delta S = mc\int_{t_1}^{t_2}\delta x^\mu \td{u_\mu}{s}ds.
\end{equation}
$\delta S = 0$ for all possible variations $\delta x^\mu$ therefore implies that 
for a free particle,
\begin{equation}\label{c2e23}
\td{u_\mu}{s} = 0.
\end{equation}
The particle travels with a uniform velocity, unsurprisingly.

\item Now fix $t_1$ and let $t_2$ be varied for true trajectories, that the ones
which occur in reality and for which \eqref{c2e23} is valid. Then we have from
\eqref{c2e21},
\begin{equation}\label{c2e24}
\delta S = -mcu_\mu \delta (x^\mu)_{t_2},
\end{equation}
so that
\begin{equation}\label{c2e25}
mcu_\mu = -\pdt{S}{x^\mu}.
\end{equation}
The 4-vector $mc^\mu$, from equations \eqref{c1e79} to \eqref{c1e82} is
\[
u^\mu = \left(\gamma, \frac{v^i}{c}\gamma\right)
\]
so that
\[
mcu^\mu = \left(mc\gamma, mv^i\gamma\right) = \left(\frac{E}{c}, p^i\right)
\]
We define the 4-momemtum $p^\mu$ as
\begin{equation}\label{c2e26}
p^\mu = \left(\frac{E}{c}, p^i\right)
\end{equation}
so that \eqref{c2e25} becomes
\begin{equation}\label{c2e27}
p_\mu = -\pdt{S}{x^\mu}.
\end{equation}
Note that if $p^\mu$ is given by \eqref{c2e26} then
\begin{equation}\label{c2e28}
p_\mu = \left(\frac{E}{c}, -p^i\right).
\end{equation}
The transformation equations for $p^\mu$ are given by
\begin{equation}\label{c2e29}
\frac{E}{c} = \gamma\left(\frac{\bar{E}}{c} + \beta\bar{p}^1\right);
p^1 = \gamma\left(\bar{p}^1 + \beta\frac{\bar{E}}{c}\right); p^2 = \bar{p}^2;
p^3 = \bar{p}^3.
\end{equation}
We also have
\begin{equation}\label{c2e30}
p^\mu p_\mu = \frac{E^2}{c^2} - p^2 = m^2c^2.
\end{equation}
We now define the 4-force as 
\begin{equation}\label{c2e31}
g^\mu = \td{p^\mu}{s}.
\end{equation}
Since $ds = \sqrt{c^2(dt)^2 - dx_i dx^i}$,
\begin{equation}\label{c2e32}
ds = cdt\sqrt{1 - \frac{v^2}{c^2}} = \frac{c}{\gamma}dt.
\end{equation}
Therefore,
\[
g^\mu = \frac{\gamma}{c}\left(\frac{1}{c}\td{E}{t}, \td{\vec{p}}{t}\right).
\]
Since $E = \vec{f}\cdot\vec{v}$, we can write this as
\begin{equation}\label{c2e33}
g^\mu = \left(\frac{\gamma}{c^2}\vec{f}\cdot\vec{v}, \frac{\gamma}{c}\td{\vec{p}}{t}\right).
\end{equation}

\item From equations \eqref{c2e27} and \eqref{c2e30},
\[
\left(-\pdt{S}{x^\mu}\right)\left(-\pdt{S}{x_\mu}\right) = m^2c^2
\]
or
\begin{equation}\label{c2e34}
\pdt{S}{x^\mu}\pdt{S}{x_\mu} = g^{\mu\nu}\pdt{S}{x^\mu}\pdt{S}{x^\nu} = m^2c^2
\end{equation}
is the relativistic Hamilton-Jacobi equation. Expanding the sum
\begin{equation}\label{c2e35}
\frac{1}{c^2}\left(\pdt{S}{t}\right)^2 - \left(\pdt{S}{x}\right)^2 
- \left(\pdt{S}{y}\right)^2 - \left(\pdt{S}{z}\right)^2 = m^2c^2.
\end{equation}
\end{enumerate}
\documentclass{beamer}
\usefonttheme[onlymath]{serif}
\usepackage{graphicx, hyperref}
\hypersetup{colorlinks, urlcolor=blue}
\usepackage{subcaption}
\usetheme{Boadilla}

\newcommand{\pvec}[1]{\vec{#1}\mkern2mu\vphantom{#1}}

\title{The Hydrogen Atom}
\author{Amey Joshi}
\date{\today}

\begin{document}
\begin{frame}
\titlepage
\end{frame}

\begin{frame}
\frametitle{The Hamiltonian}
\begin{enumerate}
\item We consider the simplest Hamiltonian by ignoring
\begin{enumerate}
\item the nuclear motion
\item the magnetic field of the nucleus
\item relativistic effects
\end{enumerate}
and include only the electrostatic potential between the proton and the 
electron
\begin{equation}\label{e1}
V(r) = -\frac{1}{4\pi\epsilon_0}\frac{e^2}{r}.
\end{equation}
\item Therefore, the Hamiltonian is
\begin{equation}\label{e2}
\hat{H} = -\frac{\hslash^2}{2m}\nabla^2 - 
\frac{1}{4\pi\epsilon_0}\frac{e^2}{r}
\end{equation}
\item The Schr\"{o}dinger equation is
\begin{equation}\label{e3}
-\frac{\hslash^2}{2m}\nabla^2\psi - 
\frac{1}{4\pi\epsilon_0}\frac{e^2}{r}\psi = E\psi
\end{equation}
where $\psi$ is a function of $x, y, z$.
\end{enumerate}
\end{frame}

\begin{frame}
\frametitle{How to solve it?}
\begin{enumerate}
\item The problem has a spherical symmetry. The potential $V$ depens on $r$ and
not the angles.
\item Easier to solve in spherical polar coordinates.
\item The boundary condition now becomes $\psi \rightarrow 0$ as $r \rightarrow 
\infty$. It depends on only the distance, not the angles.
\item The differential operator now becomes
\begin{equation}\label{e4}
\nabla^2 = \frac{1}{r^2}
\frac{\partial}{\partial r}\left(r^2\frac{\partial}{\partial r}\right) +
\frac{1}{r^2\sin\theta}\frac{\partial}{\partial\theta}\left(\sin\theta\frac{
\partial}{\partial\theta}\right) + \frac{1}{r^2\sin\theta}\frac{\partial^2}
{\partial\phi^2}.
\end{equation}
\item Looks far more complicated than in Cartesian coordinates. It is 
deceptively complicated.
\item Allows the technique of `separation of variables'.
\end{enumerate}
\end{frame}

\begin{frame}
\frametitle{How to solve it?}
\begin{enumerate}
\item $\psi$ is now a function of $r, \theta, \phi$.
\item Write $\psi(r, \theta, \phi) = R(r)\Theta(\theta)\Phi(\phi)$.
\item The partial differential equations gives three ordinary differential 
equations
\begin{eqnarray}
\frac{d^2\Phi}{d\phi^2} + m_l^2\Phi &=& 0 \label{e5} \\
-\frac{1}{\Theta\sin\theta}\frac{d}{d\theta}
\left(\sin\theta\frac{d\Theta}{d\theta}\right) +\frac{m_l^2}{\sin^2\theta} &=& 
\beta \label{e6} \\
\frac{1}{R}\frac{d}{dr}\left(r^2\frac{dR}{dr}\right) +
\frac{2m}{\hslash^2}\left(E + \frac{1}{4\pi\epsilon_0}\frac{e^2}{r}\right)r^2&=&
\beta, \label{e7}
\end{eqnarray}
where $m_l^2$ and $\beta$ are constants.
\item For the solution to be finite, $\beta = l(l+1)$ where $l$ is a non-negative
integer.
\end{enumerate}
\end{frame}

\begin{frame}
\frametitle{The solution}
\begin{enumerate}
\item The $\theta$ and $\phi$ equations are independent of the potential energy.
\item $\Theta$ is an associated Legendre polynomial, $R$ is an associated 
Laguerre polynomial and $\Phi = e^{im_l\phi}/\sqrt{2\pi}$.
\item $E$ is a part of the $r$ equation alone. All spherically symmetric 
potentials have the same structure.
\item A solution exists for all $E \ge 0$ but if $E < 0$ then it exists only if
\begin{equation}\label{e8}
E = E_n = -\frac{me^4}{8n^2h^2\epsilon_0^2},
\end{equation}
where the positive integer $n$ is called the principal quantum number. This is 
same as the Bohr model.
\item The probability density is independent of $\phi$.

\end{enumerate}
\end{frame}

\begin{frame}
\frametitle{The quantum numbers}
\begin{enumerate}
\item The number $l$ is called the orbital quantum number and it is restricted
to be less then $n$.
\item The number $m_l$ is called the magnetic quantum number and it is restricted
to $-l \le m_l \le l$.
\item This equation does not give us information about the spin quantum number.
\end{enumerate}
\end{frame}
\end{document}
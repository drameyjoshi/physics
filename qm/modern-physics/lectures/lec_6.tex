\documentclass{beamer}
\usefonttheme[onlymath]{serif}
\usepackage{graphicx, hyperref}
\hypersetup{colorlinks, urlcolor=blue}
\usepackage{subcaption}
\usetheme{Boadilla}

\newcommand{\pvec}[1]{\vec{#1}\mkern2mu\vphantom{#1}}

\title{Early Models of an Atom}
\author{Amey Joshi}
\date{\today}

\begin{document}
\begin{frame}
\titlepage
\end{frame}

\begin{frame}
\frametitle{Why bother to know them?}
\begin{enumerate}
\item They are chronicles of human's imagination to understand stuff beyond the
senses.
\item The imagination was subject to theoretical analysis and experimental 
confirmation. This is what sets science apart from religion.
\item As a bonus, we will get a glimpse of those parts of classical physics that 
are too advanced to be studied in depth.
\item Studying how models were proposed and discarded helps us understand how
great science progresses.
\item In the modern times, Sir J. J. Thomson was the first one to describe an 
atom. He imagined it to be a continuous distribution of positive charge in which
electrons were studded like plum pieces in a pudding.
\item The ultimate arbiter of a theory/model is an experiment. Geiger and 
Marsden performed a series of experiments between 1908 and 1913 which gave 
strong evidence against Thomson's model. 
\end{enumerate}
\end{frame}

\begin{frame}
\frametitle{How to test Thomson's model?}
\begin{enumerate}
\item Thomson's atom was a continuous distribution of positive charge. 
Therefore, if it were bombarded with another positive charge, that charge would
be strongly repulsed.
\item A collimated beam of $\alpha$-particles was let in an evacuated glass tube 
at one end and detected at the opposite one. A sharp image of the slit was seen.
\item Admission of a little air blurred the image; indicating scattering of 
$\alpha$-particles.
\item The tube was evacuated again and a very thin gold foil was fitted midway.
The expectation was that no image should be formed. But a blurred image of the
slit was seen on the detector. Some $\alpha$-particles were deflected at large
angles.
\item Qualitatively, a solid gold foil is no different from little air in the 
tube.
\end{enumerate}
\end{frame}

\begin{frame}
\frametitle{Can that happen?}
\begin{enumerate}
\item An incident $\alpha$ particle can be scattered either by the electrons
or the continuously smeared positive charge density.
\item It was known that an $\alpha$ particle was much heavier than the electron.
Therefore, by a rather elementary analysis of collisions, it can be shown that
the angle of deflection of an $\alpha$ particle is close to zero. (Study the
derivation of equation (3.1.8) in the notes.)
\item If the number of electrons in an atom is $Z$ then the nucleus has a charge
$Ze$. If the atom's radius is $R$ and the atom is spherical then using Gauss' 
law we can show that the electric field at a point at a distance $r \le R$ from
the atom's centre is (equation (3.1.11) in then notes)
\begin{equation}\label{e1}
\vec{E} = \frac{Ze}{4\pi\epsilon_0}\frac{r}{R^3}\hat{r}.
\end{equation}
\end{enumerate}
\end{frame}

\begin{frame}
\frametitle{Can that happen? Continued ...}
\begin{enumerate}
\item The angle of deflection of an $\alpha$-particle with mass $m_1$ and 
initial speed $v_1$ is (equation (3.1.14) in then notes)
\begin{equation}\label{e2}
\tan\theta = \frac{Ze^2}{\pi\epsilon_0 Rm_1v_1^2}.
\end{equation}
\item The $\alpha$-particles were let in at $v_1 = 2 \times 10^7$ ms${}^{-1}$.
$Z = 79$ for gold. After plugging the rest of the constants we once again get
a rather small value of $\theta$.
\item If so many $\alpha$-particles made it to the other end then the gold 
foil was largely porous. If so, what held the atoms together and why did not
come closer still?
\item If we ignore this objection and let the pores be, how come some $\alpha$
particles were so strongly deflected?
\end{enumerate}
\end{frame}

\begin{frame}
\frametitle{Rutherford's model}
\begin{enumerate}
\item Lord Rutherford proposed that an atom's positive charge is not smeared
in a continuous distribution but is concentrated in a tiny nucleus.
\item The electrons were thought to be moving around the nucleus like planets
around the sun.
\item Two very strong objections against this idea:
\begin{enumerate}
\item What holds so many positive charges together in such a tiny nucleus? 
\item Why don't the electrons drop in the nucleus when they lose energy 
continuously and quickly by radiation?
\end{enumerate}
\item Perhaps there is just a single positive charge $Ze$ that makes up the 
nucleus. So we can ignore the first objection.
\item Can we explain the large deflection of $\alpha$-particles by assuming that
the entire positive charge is concentrated in a tiny nucleus?
\end{enumerate}
\end{frame}

\begin{frame}
\frametitle{Rutherford's scattering formula}
\begin{enumerate}
\item Rutherford derived a formula to estimate the number $N$ of $\alpha$-
particles scattered in a small interval around an angle $\phi$ when the flux
of particles is $F$. The particles are detected a distance $s$ from incident 
pencil of $\alpha$-particles. It is (equation (3.2.20) of the notes)
\begin{equation}\label{e3}
N(\phi) = \frac{FntZ^2e^4}{m^2v^4s^2}\csc^4\left(\frac{\phi}{2}\right).
\end{equation}
\item Here $t$ is the thickness of the gold foil, $n$ is the number of atoms
per unit volume, $m$ is the mass of $\alpha$ particle, $v$ is the speed of an
$\alpha$ particle and $Z = 79$ is the number of electron in a gold atom.
\item Geiger and Marsden's experiments could be explained perfectly by equation
\eqref{e3}.
\item The experiments proved that an atom is largely void with almost all of its
mass concentrated in a tiny nucleus.
\end{enumerate}
\end{frame}

\begin{frame}
\frametitle{What about the electrons?}
\begin{enumerate}
\item An accelerated charge $e$ emits radiation at the rate
\begin{equation}\label{e4}
P_r = \frac{2}{3}\frac{e^2}{4\pi\epsilon_0}\frac{a^2}{c^3},
\end{equation}
where $a$ is the acceleration and $c$ is the speed of light in vacuum.
\item Equation \eqref{e4} follows from classical electrodynamics and is verified
by experiments.
\item Electrons circling around the nucleus will soon lose energy and will be
drawn to the positively charge nucleus, once again bringing back Thomson's model!
\item Equation \eqref{e4} is a serious hurdle and there is absolutely nothing in
classical physics hinting a way around.
\end{enumerate}
\end{frame}

\begin{frame}
\frametitle{Bohr model}
Originally proposed for the simplest of all atoms, H. Based on two postulates:
\begin{enumerate}
\item An electron moves in circular orbits aroud the nucleus with the 
centripetal acceleration caused by the electrostatic of the nucleus.
\begin{equation}\label{e5}
\frac{mv^2}{r_n} = \frac{1}{4\pi\epsilon_0}\frac{e^2}{r_n^2}.
\end{equation} 
\item An electron \emph{does not} radiate if its angular momentum is an integral
multiple of $\hslash = h/(2\pi)$. That is,
\begin{equation}\label{e6}
mvr_n = n\frac{h}{2\pi}, n = 1, 2, \ldots
\end{equation}
\end{enumerate}
The second postulate is outright arbitrary, awe-inspiring and fully justified by
quantum mechanics to be developed 13 years later. Not only energy, but even 
angular momentum is quantised.
\end{frame}

\begin{frame}
\frametitle{Consequences of Bohr model}
From equations \eqref{e5} and \eqref{e6} it is immediately evident that
\begin{equation}\label{e7}
r_n = \frac{n^2h^2\epsilon}{\pi me^2}.
\end{equation}
One can now compute the potential energy of the electron in the $n$th orbit and 
hence get the total energy as
\begin{equation}\label{e8}
E_n = -\frac{me^4}{8n^2h^2\epsilon_0^2}.
\end{equation}
When an electron `falls' to an orbit $n_1$ from another orbit $n_2 > n_1$, it 
emits a photon of energy $E = E_{n_2} - E_{n_1}$. The frequency of the emitted
photon can be found using Planck's relation. Perfect explanation of the H-atom
spectrum.
\end{frame}

\begin{frame}
\frametitle{But ...}
\begin{enumerate}
\item Does not explain Zeeman and Stark effects.
\item Does not tell anything about atoms with more than one electron.
\item More disturbingly, the quantisation of angular momentum is hard to 
understand.
\item Arnold Sommerfeld tried to extend the Bohr model by allowing elliptical
orbits and introducing newer conditions like quantisation of angular momentum
but it was getting clear that the models were getting increasingly arbitrary.
\item The final breakthrough came in 1925 and 1926. Heisenberg published the
first paper on quantum mechanics in 1925. His techniques are now called `matrix
mechanics'. Schr\"{o}dinger proposed yet another formulation in 1926, now called
`wave mechanics'.
\end{enumerate}
\end{frame}
\end{document}